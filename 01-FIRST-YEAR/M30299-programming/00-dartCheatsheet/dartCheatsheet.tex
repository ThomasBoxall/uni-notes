\documentclass[a4paper,11pt]{article}
\usepackage{xcolor}
\usepackage{geometry}
\geometry{
a4paper,
total={170mm,257mm},
left=20mm,
top=20mm,
marginparsep=0mm,
}
\setlength\parindent{0pt} % get rid of the stupid indent

\title{Dart Syntax Cheatsheet}
\author{Thomas Boxall\\ \texttt{up2108121@myport.ac.uk}}
\date{April 2023}

\usepackage{fancyhdr}
\pagestyle{fancy}
\fancyhead{} % clear all header fields
\renewcommand{\headrulewidth}{0pt} % no line in header area
\fancyfoot{} % clear all footer fields
\renewcommand{\footrulewidth}{0.4pt}
\fancyfoot[C]{\thepage} % page number in centre of the page
\fancyfoot[R]{\footnotesize Thomas Boxall\\ \texttt{up2108121@myport.ac.uk}} % right hand footer has author name on top line and author contact on bottom line
\fancyfoot[L]{\footnotesize Dart Syntax Cheatsheet \\ April 2023} % left hand footer has title of document on top line and date on bottom line


\begin{document}

\maketitle
\thispagestyle{fancy}

\section{Basics of Dart}
entry point to a Dart program is the \verb|main| function:
\begin{verbatim}
void main(){

}
\end{verbatim}
print to the console: \verb|print("hello world");|\\
comments start with: \verb|//|

\section{Types}
declare an integer variable: \verb|int imAnInt = 12;|\\
declare a double variable: \verb|double imADouble = 6.9;|\\
declare a string variable: \verb|String imAString = "Cheese";|\\
declare a constant: \verb|const pi = 3.14;|\\
declare a boolean: \verb|bool dartIsBetterThanPython = true;|\\
declare a num: \verb|num iCanBeDoubleOrInt = 4;|
\subsection{Casting}
cast to a string: \verb|String wasANumber = numberVariable.toString();|\\
cast a double to string with fixed decimal places:\\ \verb|String wasADouble = doubleVariable.toStringAsFixed(numbDecimalPlaces);|\\
cast a string to an integer: \verb|int wasAString = int.parse(stringVariable);|\\
cast a string to a double: \verb|double wasAString = double.parse(stringVariable);|\\
cast an int or num to double: \verb|double wasAnInt = intVariable.toDouble();|\\
cast a double or num to int: \verb|int wasADouble = doubleVariable;|\\ or \verb|int wasADouble = doubleVariable.toInt();|\\
cast a double or num to int and round = \verb|int wasADouble = doubleVariable.round();|\\
cast a double or num to int and round to upper bound: \verb|int wasADouble = doubleVariable.ceil();|\\
cast a double or num to int and round to lower bound: \verb|int wasADouble = doubleVariable.floor();|

\section{Math}
import math library: \verb|import 'dart:math';|\\
calculate power of number: \verb|pow(base, exponent)|\\
calculate square root: \verb|sqrt(numberToSquareRoot)|

\section{Strings}
concatenate two strings: \verb|print("String One" + "String Two");|\\
interpolate a non-string into a string: \verb|print("I'm a string, $andImAndIntVariable");|\\
interpolate a non-string into a string and apply an operation to it: \verb|print("String: ${intVariable * 2}");|\\
use string indexing to access the 1st character in a string: \verb|print(stringVariable[0]);|\\
use substring to access particular characters in a string:\\ \verb|print(stringVariable.substring(start, endOptional));|\\
get the length of a string: \verb|print(stringVariable.length);|\\
get the index of a particular character in a string: \verb|print(stringVariable.indexOf(queryChar));|\\
split a string and remove the pattern: \verb|stringVariable.split(pattern)|\\
convert all characters in the string to lower case: \verb|stringVariable.toLowerCase()|\\
convert character to ASCII value: \verb|stringVariable.codeUnitAt(position)|

\section{Functions}
general structure to a Dart function:
\begin{verbatim}
returnType functionName(param1Type paramOne, ...){
    
}
\end{verbatim}
a function made of an expression can be simplified:\\
\verb|returnType functionName (param1Type, param1) => expressionToReturn;|\\
a function can be passed as a parameter to another function using \verb|returnType Function(paramsType) name| which can then be called using \verb|name()|, passing any parameters into it.

\section{Program Flow Control}
if statements have the following structure:
\begin{verbatim}
if (statement){

} else if (anotherStatement){

} else{

}
\end{verbatim}
for loops have the following structure:
\begin{verbatim}
for (loopVarStartPoint; loopVarEndPoint; loopVarIteration){

}
\end{verbatim}
while statements have the following structure:
\begin{verbatim}
while(statement){

}
\end{verbatim}


\end{document}