\lecture{Further Object Oriented Progrmaming}{2023-01-30}{15:00}{Nadim}{RB LT1}

\section*{Composition}
Composition is one of the fundamental concepts in object oriented programming. Composition describes a class that references one or more objects of other classes in instance variables.
\begin{link}{Composition Example}
An example of composition can be found in Practical 11. The \texttt{ShoppingCart} class holds a list of items which are `in' the cart. 
\end{link}

\section*{Inheritance}
When programming using the object oriented paradigm, we wil often find that two classes are similar, however one may have additional attributes required. 

Subclasses can access all methods and attributes of superclasses. 

A simple example of this is to have a \texttt{Laptop} class which stores information about laptops. Gaming Laptops are a special type of laptop as they also have a GPU. This means that the gaming laptop class should \textit{inherit} attributes from \texttt{Laptop} and should have some additional methods.

In this example, the \texttt{GamingLaptop} inherits from \texttt{Laptop}; this can also be said as \texttt{GamingLaptop} extends \texttt{Laptop}. \texttt{Laptop} is the superclass of \texttt{GamingLaptop} or \texttt{GamingLaptop} is the subclass of \texttt{Laptop}. 

In the example below, it can be assumed that the \texttt{Laptop} class is defined with a constructor and string method.
\begin{python}
class GamingLaptop(Laptop):
    def __init__(self, brand, model, basePrice):
        super().__init__(brand, model, basePrice)
        self.gpu = "NVIDIA GeForce 4GB"

    def __str__(self):
        output = super().__str__()
        output += " {}".format(self.gpu)
        return output
\end{python}