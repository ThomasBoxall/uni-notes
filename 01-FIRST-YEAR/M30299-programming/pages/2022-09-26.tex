\lecture{Writing Simple Programs}{26-09-2022}{15:00}{Nadim}{RB LT1}

This lecture introduces the basic steps involved in programming and provides some additional information about each stage.

\section*{Stages of Algorithm Design}
When presented a problem to solve programmatically, the first stage to doing so is to understand the problem and to ensure that this understanding is correct.
To aid this, it can be useful to work out how the user interacts with the system, through listing the user inputs and outputs to screen. At this stage, it can also be beneficial to make a note of some inputs and their expected outputs as this can be used to test the program at the end of development.

The next stage is to design an algorithm that accomplishes the task.
\define{Algorithm}{A detailed sequence of actions which acomplish a task. Cna be written in plain English or any other language.}

The next stage is to implement the algorithm. This is where the plain English algorithm is converted into programming statements which can be executed by the machine.

The final stage is to test the program. This can be done ith the data noted down in stage one.

\section*{Key Program Concepts}
In programming, there are a number of key concepts. These will be illustrated using examples written in Python 3.

\subsection*{Statements}
Every line of a program is called a command or statement. These are executed (carried out) one after the other (there are ways in which the flow of the program can be altered, but this will be covered at a later date). Program execution ends after the last statement is executed.
\subsection*{Variables}
A variable is a name for a part of the computer memory where a value is stored. The variables have names in the programs.

Statements in the program may create a new variable, use the value of a variable or change the value of a variable.

\subsection*{Assignment Statements}
Assignment statements are used to assign a value to a variable. The syntax is as follows:
\begin{itemize}
    \item The variable appears on the left hand size of the \verb|=|
    \item The right hand side of is an expression, which has a value
\end{itemize}
\begin{python}
variableName = expressionWhichHasAValue
\end{python}

Assignment statements are executed in two steps. First they evaluate the expression on the right hand side then second, assign the value to the variable on the left hand side.

If the variable on the lft hand side doesn't already exist, then it is created. If the variable exists already, its old value is replaced.

\subsection*{Numeric and String Values}
Numeric values are numbers. They do not need any demarcation. For example, \verb|2.2| is a numeric value.\\
String values are strings of characters. These can be any character. Strings need to be encased in single quotes or double quotes. Both are valid, however they can't be mixed. Lines 1 and 2 in the following example are valid, however line 3 is not.
\begin{python}
validStringOne = "I'm in double quotes, notice I can use single quotes where I like!"
validStringTwo = 'Im in single quotes, notice I cant use single quotes in the string.'
invalidString = "Im not valid'
\end{python}

\subsection*{Arithmetic Expressions}
Standard arithmetic expressions can be formed using \verb|+|, \verb|-|, \verb|*|, \verb|/| and \verb|()|. Expressions are evaluated to give a value, this is commonly stored in a variable or outputted directly to the user.

\subsection*{Built-In Functions}
Python has a number of built-in functions. These are algorithms which are part of the Python language. They can be accessed by using its name. Sometimes they have parameters, sometimes they return a value and sometimes they do both. Common examples of built-in functions are shown below.
\begin{python}
print("I display information to the user")
variable = input("I allow the user to enter text, then I store it in the variable")
\end{python}

\begin{link} {Example Execution}
See Week 1, lecture 01c slides on Moodle for a detailed look at how programs execute and how the variable contents change. 
\end{link}

\section*{Example programs from Lecture}
\subsection*{Program 01}
This program introduces a count-controlled loop (for loop) and the print statement.
\begin{python}
total = 0
for i in range(34):
    #print("banana")
    #print(i)
    total = total + i

print("The total is: ", total)
\end{python}
This program should output the following
\begin{pseudo*}
The total is:  561
\end{pseudo*}
The two commented out lines (lines which begin with the \verb|#|) symbol can be un-commented so that they run.
\subsection*{Program 02}
This program introduces the concept of \verb|input()|, \verb|int()| and subroutines.
\begin{python}
def simpleProgram():
    
    value = int(input("Please enter a whole number: "))

    for loopCount in range (value):
        print(loopCount)

#################

simpleProgram()
\end{python}
The program should output the following.
\begin{pseudo*}
Please enter a whole number: 12
0
1
2
3
4
5
6
7
8
9
10
11
\end{pseudo*}
The number \verb|12| on line one is entered by the user.