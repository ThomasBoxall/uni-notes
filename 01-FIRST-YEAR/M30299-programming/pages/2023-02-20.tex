\lecture{Debugging \& Exception Handling}{2023-02-20}{15:00}{Nadim}{RB LT1}

\section*{Debugging}
Debugging refers to the process of finding bugs in software and fixing them. 

When a programme crashes, it will usually give an error which we can use to identify the problem.

Editors contain debugging tools, including breakpoints, which can be used to `step' through the code and identify which line exactly is throwing an error. This can be used to inspect variables on each line as well.

\section*{Exception Handling}
A piece of software which has been well written and debugged fully may still crash due to run-time exceptions. Exception handling is the process of preparing your code to avoid failure from certain exceptions. 

This can be done using \texttt{try catch} statements, as seen below.
\begin{python}
def getSize():
    try: 
        size = int(input("Enter the size of the patchwork:"))
        return size 
    except ValueError:
        print("Size must be an integer, the default size of 5 will be used") 
        return 4
\end{python}
