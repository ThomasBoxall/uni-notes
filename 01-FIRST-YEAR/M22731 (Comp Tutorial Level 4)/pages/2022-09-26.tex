\lecture{Introduction}{26-09-22}{14:00}{Nadim}{RB LT1}

Being good at computing is not about being scientific but being good at a bunch of arts.

\section*{Basic Concepts}
\subsection*{Human Practices}
Human practices are anything humans do, this includes and is not limited to: working, studying and having parties.

\subsection*{Complex Situations}
Other than the nervousness, the thing which is different to a student delivering a lecture and Nadim delivering the lecture is his experience and knowledge.

Experience and knowledge mean you get the Concepts.

Everything is complex, having experience and knowledge means we can understand situations and work out what to do in them.

Will get lots of knowledge from University but won't get as much experience unless you do a placement year.

\subsection*{Goal Directed}
Means something has a purpose. This can be applied to computing in that everything to do with computing is purpose driven. 

\subsection*{Complex Problems}
Studying at university involves lots of information being given to students, some of which isn't as useful as other parts. A complex problem is to take on board that information and work out what is worth spending time to understand and what can be forgotten about immediately.

\section*{Situational Complexity}
Situations can be used to analyze problems as well as solve them.

A situation is something which you are experiencing as well as problems. They can include things like: being in a group of friends, being in a lecture, going to talk to a bank manager.
\subsection*{Step 1}
All human activity, behavior occurs in contexts. 

When you understand someone, you always understand what they are doing and thinking in relation to a context.


\subsection*{Step 2}
All human activity, behavior is a response to context. This is true for all living beings.

\subsection*{Development}
All developments arise out of problems generated by contexts. Understand these contexts and you gain an insight into the subject.

Everything studied has a historical development, to arrive at where it is now. To understand something, you have to understand why it came about, where it came from, why it was developed and who developed. 

Survival means something is able to respond to a context correctly.

Learning is then a process of working out how context works (using concepts, theories and rules) and how those who work in them respond ot the demands of those contexts.