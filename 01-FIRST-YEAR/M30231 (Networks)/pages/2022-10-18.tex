\lecture{LECTURE: Protocols Continued}{18-10-22}{09:00}{Amanda}{Zoom}
\textit{NB: Connection Type notes moved \& merged with previous lecture (2022-10-11), for simplicity and clarity.}
\section*{Protocols Recap}
Protocols are sets of rules which are used for sending and receiving data across networks. They can provide addressing as well as management and verification of transmission. Often protocols are used together to form a suite of protocols, for example TCP/IP.

\section*{TCP/IP}
TCP/IP stands for Transmission Control Protocol/ Internet Protocols. It is a collection of protocols that govern the way that data travels from one machine to another across networks. Commonly it is found in the core of networks. In this term, networks could be a small LAN, enterprise environment networks, metropolitan networks or wide area networks. There are two major components of TCP/IP.

At a high level, TCP/IP protocols work together to break the data into small pieces that can be efficiently handled by the network; communicates the destination of the data to the network; acknowledges receipt of data; reconstructs the data into its original form; and checks that the data is not corrupt. These individual tasks are completed between the different protocols which make up the suite of protocols called TCP/IP.
\subsection*{Transmission Control Protocol}
\marginNote{reviewed 06-01-22} TCP, a connection-oriented subprotocol, ensures reliable data delivery through sequencing a checksums as well as providing flow control to the transmission. At the sending device, TCP breaks up the data into packets which the network can handle effectively. During transmission TCP does nothing. At the recipient node, TCP ensures all the packets have arrived and are in a fit state; TCP then reports the condition of the packets back to the sender node so it knows if it needs to re-transmit any of them. TCP will then reassemble the data into its sequence.
\subsection*{Internet Protocol}
IP is used to envelope the data, this provides a location for the sender and destination IP addresses to be added to the packet.

\section*{Packets}
A packet is a single unit of data that is sent across a network. Data to be transmitted is broken into a number of packets before it is transmitted across the internet. Packets have multiple parts, one part is the \textit{header} in which, the sender and recipient IP addresses are stored as well as the code which is used to handle transmission errors and keep packets flowing.
\subsection*{Packet Routes}
As the packet "hops" from node to node on its journey across the network, it across routers. These are devices which are dedicated to reading the header information and determining which route the packet should take to the next router. Packets move from router to router until they reach their final destination. All the packets going from one sender to one recipient may not all take the same route, there are a number of variables which influence this including the network traffic at that particular moment and the size of the packet being sent. 
\subsection*{Packets and TCP/IP}
TCP sends the packets in sequence; ensures the integrity of the packets and where needed requests new packets to be sent if on receipt a packet is damaged; and acknowledges receipt of packets. 

IP breaks the data in to packets; places header information into packets; and determines how much data can fit into a single packet, this can include fragmenting the packet further if there is lots of congestion on the network. 

\subsection*{Example Packet Transmission}
The example below shows how an email message would be transmitted across a network. 
\begin{enumerate}
    \item The data that makes up an email message is split into packets by the IP portion of TCP/IP. IP also adds header information to each packet.
    \item Using the header information in the packets, routers determine the best path for each packet to take to its final destination.
    \item The TCP portion of TCP/IP reassembles the packets in the correct order and ensures that ll packets have arrived undamaged.
\end{enumerate}
\begin{figure}[H]
    \centering
    \includegraphics*[width=0.8\textwidth]{assets/packet-transmission.png}
    \caption{Packet transmission across a network where packets travel from router to router}
\end{figure}