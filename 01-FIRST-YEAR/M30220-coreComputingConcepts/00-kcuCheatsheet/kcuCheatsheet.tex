\documentclass[a4paper,11pt]{article}
\usepackage{xcolor}
\usepackage{geometry}
\usepackage{hyperref}
\hypersetup{
    colorlinks=false,
    linkcolor=blue,
    linkbordercolor=blue,
    pdfborderstyle={/S/U/W 1}
}
\usepackage{graphicx}
\usepackage{float}
\usepackage[strict]{changepage} %allows indented blocks
\geometry{
a4paper,
total={170mm,257mm},
left=20mm,
top=20mm,
marginparsep=0mm,
}
\setlength\parindent{0pt} % get rid of the stupid indent

\title{CCC Item IV: Key Concepts of Usability (KCU) Cheatsheet}
\author{Thomas Boxall\\ \texttt{up2108121@myport.ac.uk}}
\date{May 2023}

\usepackage{fancyhdr}
\pagestyle{fancy}
\fancyhead{} % clear all header fields
\renewcommand{\headrulewidth}{0pt} % no line in header area
\fancyfoot{} % clear all footer fields
\renewcommand{\footrulewidth}{0.4pt}
\fancyfoot[C]{\thepage} % page number in centre of the page
\fancyfoot[R]{\footnotesize Thomas Boxall\\ \texttt{up2108121@myport.ac.uk}} % right hand footer has author name on top line and author contact on bottom line
\fancyfoot[L]{\footnotesize CCC Item IV: KCU Cheatsheet \\ May 2023} % left hand footer has title of document on top line and date on bottom line


\begin{document}

\maketitle
\thispagestyle{fancy}

\section{Human Computer Interaction}
\textbf{Human Computer Interaction} (HCI) is the study of interaction between people and machines.\\
\textbf{Goals} of HCI are to improve the interactions between people and computers and to make computers more usable \& receptive to the user's needs. Longer term, HCI aims to design systems that minimise the barriers between the human cognitive perception of the task and the computers understanding of the user's task.\\
\textbf{Scope} of HCI is very broad. HCI encompasses many fields: AI, Psychology, Sociology, Art, Design, and many more.\\
\textbf{Sub-Domains} each have their own scope and focus within the broader focus of design \& evaluation of the User Interface.
\begin{adjustwidth}{2em}{1em}
\textit{Usability} refers to how successfully a user can use a system to accomplish a specific goal and looks at how `easy to use' a system is and `limiting errors' within systems.\\
\textit{User Experience} encompasses the user's entire experience with the interface, including how well it worked, how the user expected it to work and how users feel about the system overall.\\
\textit{User Centred Design} is a design process where designs are iterated upon to refine meeting the needs of the user.
\end{adjustwidth}
\textbf{Why HCI?} Badly designed interfaces can waste the user's time, cause frustration and lead to errors. We have understood the principles of good design for we \& desktop apps for 30 years however application of these principles is often neglected. We are still working on understanding principles of good design for up-and-coming paradigms such as NUI.
\subsection{HCI Paradigms}
\textbf{Command Line Interface} allows the user to enter text commands to control the computer.\\
\textbf{Graphical User Interfaces} allows the user to `point-and-click' pictures which control the computers. NUI will often simulate this where there is not a complete NUI environment.\\
\textbf{Natural User Interface} allows the user to interact with the computer in natural methods, by talking, touching, gesturing or the computer tracking eye movements or through brain-machine interfaces.\\
\textbf{Metaverse} allows the user to interact with others in a `virtual world' where people can socialise, play and work together. It can either be immersion in AR or VR.
\subsection{Interface Design}
\textbf{Interface Design} is extremely important. Designers must consider: what people expect, what people find enjoyable and attractive, how information processing systems work, and technical characteristics and limitations of the computer hardware.\\
\textbf{User Interface} is the part of the computer that people can interact with. Good UI design will provide a mix of input \& output mechanisms. The best UI design will not be noticed as it will allow the user to just focus on the task at hand.\\
\textbf{Bad Interfaces} can be categorised as bad for a number of reasons. Windows 8 is categorised as bad as there were two versions of everything - one for touchscreen input and one for traditional mouse input.

\section{Usability Heuristics}
\begin{enumerate}
    \item Match between systems and the real world
    \item Consistency of Standards
    \item Visibility of System Status
    \item User control and freedom
    \item Error prevention
    \item Help users recognise, diagnose and recover from errors
    \item Recognition rather than recall
    \item Flexibility and efficiency of use
    \item Aesthetic and minimalist design
    \item Help and documentation
\end{enumerate}

\section{Usability Testing}
\textbf{Usability Testing} is evaluating the usability of a web-page, app or other software by testing it with real users. Often it is left too-late in the development cycle to properly integrate feedback from testing into the product. The testing shouldn't be carried out by the developers as they know the app too well, it should be outsiders or specialist testers.\\
\textbf{When to test?} Testing should be done early in the SDLC (Software Development LifeCycle) and should be done often, with the same tests repeated. \textit{What} is tested will be different in each test (early may just be sketches, middle may be comparisons of dumb UIs and final may be the final version of the UI).\\
\textbf{Formative Testing} will feed into the design and is typically done with a prototype.\\
\textbf{Summative Testing} shows a complete system and takes measurements from it (e.g. how long it takes for a user to complete a task).\\
\textbf{Validation Testing} will show all the changes made so far and compare them to evaluate the end result of the overall usability of the system.
\subsection{Testing Approaches}
\textbf{Testing Approaches} are the different methods which can be used to test usability. There are two we look at.\\
\textbf{Traditional Testing} takes place with 8 or more participants who have been carefully selected. The test takes place in a specialist usability lab and is conducted by usability professionals. The date for the test is decided well in advance as well as lots of discussions about the test protocol. Generally only conducted once as it can cost between £5k and £15k for one test. The output is a single 20-page report which takes a few weeks to arrive.\\
\textbf{Lost our Lease Testing} takes place with 3 or 4 participants, anyone can take part. The test takes place in any office/ conference room and can be conducted by anyone. They can happen anytime and the only preparation needed is to decide what to show. There are lots of small tests continually throughout development. Each test costs about £300 and a debrief as well as short notes are delivered to the developers the same day as the test.\\
\textbf{Usability Room} is a specialist room containing a computer and a number of chairs. The participant will sit at the computer and perform tasks with the interface. The moderator will sit over the shoulder of the participant. An observation team will either observe through a one-way mirror or webcam. \\
\textbf{Participants} will ideally not know much about the app before testing. The participants don't necessarily be the end user target group unless the app required specific expert knowledge to use.\\
\textbf{Facilitators} are present to support the user without giving hints on using the software, they encourage the user to think out loud and probe for more of the users thoughts when possible. Facilitators shouldn't be concerned with note taking. \\
\textbf{Types of Testing} include \textit{Get It} (exploring the users understanding of the apps basic purpose); \textit{key task} (asking the user to do something specific and observing them); \textit{exploratory/ formative} (explores the high level design concepts - can a user walk up and use it); \textit{assessment/ summative} (explores lower level operations - can a user perform a specific task); \textit{comparison} (explores differences between different prototypes or designs); and \textit{verification} (verifies that a UI is okay or a fix works). \\
\textbf{Data to gather} during a test should include users actions and explanations but not opinions. Numerical data is also extremely useful to gather (e.g. how long does it take to complete a task).\\
\textbf{Performance Goals} are goals which are set for a test. These should be narrow (e.g. ``every user will be able to find/ use the navigation bar'') and not broad.\\
\textbf{Limitations} of tests are many. The tests are set in artificial environments, some won't prove that a UI `works'. It also may be that a UI has a learning curve which the user can't get very far up in a short test. 

\section{Accessibility}
\textbf{Accessibility} concerns removing the barriers that would otherwise exclude some people from using a service, product or system at all.\\
\textbf{Software Accessibility} means that websites, apps, tools and technologies are designed and developed so that people with disabilities can use them.\\
\textbf{Web Accessibility} is a term used to identify the extent to which information on web pages can be successfully accessed by persons with disabilities including the ageing.\\
\textbf{Standards} for accessibility do exists (published by UN and W3C) however compliance is low.\\
Users of a system are very diverse, systems must be able to be accessed by anyone regardless of their circumstances (including economic, location, communication infrastructure or disabilities.)

\subsection{Impairments}
\textbf{Impairment} is something that has an adverse effect on people's ability to carry out normal day-to-day activities.\\
\textbf{Limited Impairments} are some people with conditions that described would not consider themselves to have disabilities. They may, however, have limitations of sensory, physical or cognitive functioning, which can affect access to the system.\\
\textbf{Types of Impairments} include: visual, motor, cognitive, auditory, and speech.

\subsection{Design for All}
\textbf{Design for All} is an approach for designing for accessibility and is known as universal design. It applies to all user endeavours and includes: equitable use, flexibility in use, simple intuitive use, perceptible information, tolerance for error, low physical effort, and size \& space for approach \& use.\\
\textbf{Inclusive design} is based on four premises: varying ability is part of being human, we change throughout our lives; designs that work for people with disabilities work better for everyone; our self-esteem, identity and well-being are affected by our ability to function independently; and usability \& aesthetics are mutually compatible.\\
\textbf{Web Accessibility} has been standardized by the \textit{Web Content Accessibility Guidelines} (WCAG) and can be categorised into four criteria: perceivable, operable, understandable, robust.

\section{Standards}
\textbf{Standards} are agreed ways of doing something which has been created by a recognised body, they assure a level or quality or attainment and are specific \& measurable. They are officially documented.\\
\textbf{Needed} for a number of reasons: they promote best practice; they enforce consistency; they are independent of any one companies thoughts; they cannot be ignored by businesses.\\
\textbf{ISO} \textit{(International Organisation for Standardisation)} gives world-class specifications for services and systems to ensure quality, safety and efficiency. 

\section{Guidelines}
\textbf{Guidelines} are general rules which it is \textit{advisable} to follow, a statement, policy or procedure, a piece of advice and a piece of evidence-based practice. \\
\textbf{Benefits} of guidelines: provide clear instruction; help usability specialists to evaluate the design of their products; provides a good overview of the usability issues that designers may encounter when creating apps; and when guidelines are applied they can help to reduce the negative impacts of `opinion-driven' design and referring to evidence-based guidance can reduce clashes of opinions.

\section{Evaluation}
\textbf{Evaluation} the process by which the interface is tested against the needs and practices of the user. It allows UX designers to learn about what the user thinks and what makes a good system as well as finding out how effective and efficient the software being studied is; how much the users enjoy using it; how much it annoys and frustrates them; and where they get stuck.\\
\textbf{Summative Evaluation} is undertaken at the end of a development process, it provides an evaluation or summary of the end product.\\
\textbf{Formative Evaluation} is undertaken during the development process. This means it can be a design, a mock up or a working interface which can be tested.\\
\textbf{Analytic Evaluation} is undertaken to understand existing solutions not to envisage new systems. It will include GOMS and Cognitive Walkthroughs
\begin{adjustwidth}{2em} {1em}
\textit{GOMS} (Goals Operations Methods Selection Rules) looks at what the task is, the actions need to complete the goal, sequences of operators, and the decisions which the user has to make.\\
\textit{Cognitive Walkthroughs} is a technique used to evaluate the learnability of a system. It uses experts (not users) to evaluate the interface. It involves: identifying the goal, identifying the tasks to complete the goal, documentation of the experience. It also looks at how clear starting and stopping the task is to the user.
\end{adjustwidth}
\textbf{Empirical Evaluation} is undertaken with users and prototype interfaces to understand users opinions and to gather benchmark tasks.\\
\textbf{Lab Based Usability Evaluation} contains representative tasks which have been planned out well in advance. As this is conducted in a `lab' environment, the test results may be skewed slightly.

\end{document}