\documentclass[a4paper,11pt]{article}
\usepackage{xcolor}
\usepackage{geometry}
\usepackage{hyperref}
\hypersetup{
    colorlinks=false,
    linkcolor=blue,
    linkbordercolor=blue,
    pdfborderstyle={/S/U/W 1}
}
\usepackage{graphicx}
\usepackage{float}
\usepackage[strict]{changepage} %allows indented blocks
\geometry{
a4paper,
total={170mm,257mm},
left=20mm,
top=20mm,
marginparsep=0mm,
}
\setlength\parindent{0pt} % get rid of the stupid indent

\title{CCC Item IV: Key Concepts of Usability (KCU) Cheatsheet}
\author{Thomas Boxall\\ \texttt{up2108121@myport.ac.uk}}
\date{May 2023}

\usepackage{fancyhdr}
\pagestyle{fancy}
\fancyhead{} % clear all header fields
\renewcommand{\headrulewidth}{0pt} % no line in header area
\fancyfoot{} % clear all footer fields
\renewcommand{\footrulewidth}{0.4pt}
\fancyfoot[C]{\thepage} % page number in centre of the page
\fancyfoot[R]{\footnotesize Thomas Boxall\\ \texttt{up2108121@myport.ac.uk}} % right hand footer has author name on top line and author contact on bottom line
\fancyfoot[L]{\footnotesize CCC Item IV: KCU Cheatsheet \\ May 2023} % left hand footer has title of document on top line and date on bottom line


\begin{document}

\maketitle
\thispagestyle{fancy}

\section{Human Computer Interaction}
\textbf{Human Computer Interaction} (HCI) is the study of interaction between people and machines.\\
\textbf{Goals} of HCI are to improve the interactions between people and computers and to make computers more usable \& receptive to the user's needs. Longer term, HCI aims to design systems that minimise the barriers between the human cognitive perception of the task and the computers understanding of the user's task.\\
\textbf{Scope} of HCI is very broad. HCI encompasses many fields: AI, Psychology, Sociology, Art, Design, and many more.\\
\textbf{Sub-Domains} each have their own scope and focus within the broader focus of design \& evaluation of the User Interface.
\begin{adjustwidth}{2em}{1em}
\textit{Usability} refers to how successfully a user can use a system to accomplish a specific goal and looks at how `easy to use' a system is and `limiting errors' within systems.\\
\textit{User Experience} encompasses the user's entire experience with the interface, including how well it worked, how the user expected it to work and how users feel about the system overall.\\
\textit{User Centred Design} is a design process where designs are iterated upon to refine meeting the needs of the user.
\end{adjustwidth}
\textbf{Why HCI?} Badly designed interfaces can waste the user's time, cause frustration and lead to errors. We have understood the principles of good design for we \& desktop apps for 30 years however application of these principles is often neglected. We are still working on understanding principles of good design for up-and-coming paradigms such as NUI.
\subsection{HCI Paradigms}
\textbf{Command Line Interface} allows the user to enter text commands to control the computer.\\
\textbf{Graphical User Interfaces} allows the user to `point-and-click' pictures which control the computers. NUI will often simulate this where there is not a complete NUI environment.\\
\textbf{Natural User Interface} allows the user to interact with the computer in natural methods, by talking, touching, gesturing or the computer tracking eye movements or through brain-machine interfaces.\\
\textbf{Metaverse} allows the user to interact with others in a `virtual world' where people can socialise, play and work together. It can either be immersion in AR or VR.
\subsection{Interface Design}
\textbf{Interface Design} is extremely important. Designers must consider: what people expect, what people find enjoyable and attractive, how information processing systems work, and technical characteristics and limitations of the computer hardware.\\
\textbf{User Interface} is the part of the computer that people can interact with. Good UI design will provide a mix of input \& output mechanisms. The best UI design will not be noticed as it will allow the user to just focus on the task at hand.\\
\textbf{Bad Interfaces} can be categorised as bad for a number of reasons. Windows 8 is categorised as bad as there were two versions of everything - one for touchscreen input and one for traditional mouse input.

\section{Usability Heuristics}
\begin{enumerate}
    \item Match between systems and the real world
    \item Consistency of Standards
    \item Visibility of System Status
    \item User control and freedom
    \item Error prevention
    \item Help users recognise, diagnose and recover from errors
    \item Recognition rather than recall
    \item Flexibility and efficiency of use
    \item Aesthetic and minimalist design
    \item Help and documentation
\end{enumerate}

\section{Usability Testing}
\textbf{Usability Testing} is evaluating the usability of a web-page, app or other software by testing it with real users.


\end{document}