\lecture{Designing for Accessibility}{2023-04-25}{13:00}{John}{RB LT1}

\section{What is Accessibility}
\define{Accessibility}{Accessibility concerns removing the barriers that would otherwise exclude some people from using a service, product or system at all. It can be measured by the ease in which people with disabilities can perceive, understand, navigate and interact with the software technology.}
There are a number of different types of accessibility which will be covered in this lecture:
\begin{description}
    \item[Software Accessibility] means that websites, apps, tools and technologies are designed and developed so that people with disabilities can use them.
    \item[Web Accessibility] is a term used to identify the extent to which information on web pages can be successfully accessed by persons with disabilities including the aging\footnote{(W3C, 2000)}.
\end{description}

The United Nations (UN) and World Wide Web Consortium (W3C) have published declarations and guidelines on ensuring that everyone can get access to information that is delivered through software technologies. Despite these standards existing, there is still a significant shortfall of compliance. This, in part, is the result of so many different people creating software technologies and now that using off-the-shelf frameworks such as \textit{WordPress} are becoming more standardised, compliance levels are increasing. The shortfall in compliance prevents people with disabilities equal access of information. 

A major challenge is to create a good user experience for people with disabilities that is both accessible and usable.

\section{Impairments}
\define{Impairment}{Something that has an adverse effect on people's ability to carry out normal day-to-day activites.}
Impairments may prevent people from accessing web based information.

\define{Limited Impairments}{``Some people with conditions descrived would not consider themselves to have disabilities. They may, however, have limitations of sensory, physical or cognitive functioning, which can affect access to the system.'' (\textit{W3C, 2001})}

Users are very diverse and for a website to be accessible to anyone it must be able to be accessed by any user regardless of their circumstances; which may include economics, location, communication infrastructure or a disability.

\subsection{Types of Impairments}
There are lots of different types of impairments.
\begin{description}
    \item[Visual] blindness, low vision, colour blindness
    \item[Motor] cerebal palsy, Parkinson's disease, arthritis
    \item[Cognitive] dyslexia, attention deficit disorder
    \item[Auditory] sound, hearing impair audience
    \item[Speech] stutter, speech impediment
\end{description}