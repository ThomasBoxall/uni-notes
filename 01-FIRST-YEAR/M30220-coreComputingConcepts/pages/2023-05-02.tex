\lecture{Standards}{2023-05-02}{13:00}{John}{RB LT1}

\section{Standards}
A standard is an agreed way of doing something which has been created by a recognised body. They assure a level or quality or attainment and are specific and measurable. Standards are officially documented. 

We need standards for a number of reasons:
\begin{description}
    \item[Best Practice] is promoted through the use of standards. This is important in areas, such as web usability, which are still young  and contains many conflicting opinions on what makes a website usable.
    \item[Consistency] is an important factor in creating websites that are simple to use. Where standards are followed correctly, you should be able to transfer from one organisation to another with relative ease.
    \item[Independent] The guidance in standards does not represent the opinion of one company or one usability expert but presents a balanced, authoritative view. 
    \item[Business] Companies can ignore your research findings but they can't ignore standards since compliance is a mandatory requirement in many contracts (especially in the EU).
\end{description}

\subsection{International Organisation for Standardisation}
The International Organisation for Standardisation (ISO) gives world-class specifications for services and systems, to ensure quality, safety and efficiency. They have a wide range of standards for a wide range of applications.

They help companies to access new markets, level the playing field for developing countries and facilitate free and fair global trade. 

\section{Guidelines}
A guideline is a general rule which is advisable to follow, a statement or policy or procedure, a piece of advice and a piece of evidence based practice. Guidelines can be \textit{principles} (an abstract concept) or the steps required to achieve the principle.

\subsection{Benefits of Guidelines}
\begin{itemize}
    \item Guidelines provide a clear instruction on a range of issues that designers will encounter
    \item Guidelines help usability specialists to evaluate the designs of their products
    \item It provides a good overview of the wide range of usability and Web design issues that designers may encounter when creating websites
    \item Applying the guidelines helps to reduce the negative impacts of `opinion-driven' design and referring to evidence-based guidance can reduce the clashes resulting from differences of opinion between design team members. 
\end{itemize}

\section{Evaluation}
Evaluation is the process by which the interface is tested against the needs and practices of the user. It is hoped that evaluation will get rid of any problems that might be presented in the system.

Evaluation allows user experience designers to learn about what users think and what makes a good system as well as finding out how effective and efficient the software being studied is; how much the users enjoy using it; how much it annoys and frustrates them; and where they get stuck. 

\subsection{Research Questions}
There are seven key research questions for evaluation of usability:
\begin{description}
    \item[Who] is using the product?
    \item[What] is it they need and are they comfortable with?
    \item[When] are they using it?
    \item[Where] are they using it?
    \item[Why] are they using it?
    \item[How] do people see a product in the context of what it is they are trying to accomplish in their life?
    \item[How many] people are you creating the experience for or how many people do you need to test with?
\end{description} 

\subsection{Summative Evaluation}
Summative evaluation is often undertaken at the end of the development process. It provides an evaluation or summary of the end product which allows the system to be matched to the requirements specification.

\subsection{Formative Evaluation}
Formative evaluations are undertaken during the development process. This means it doesn't have to have a complete build which can be used, a diagram or mock-up can be used.\\
Used to inform the development process which is why a prototype is the best thing to evaluate rather than a complete system. It takes account of users (knowledge, skills, gender, age, disability) and takes account of users tasks or goals.

\subsection{Analytic Evaluation}
Analytic evaluation consists of formal methods for analysing interfaces against Heuristic evaluation. This might be completed via a cognitive walk through or Goals Operations Methods Selection Rules (GOMS), both of which are task related. Aims to investigate existing situation, not envision new systems. 
\subsubsection{Goals Operations Methods Selection Rules (GOMS)}
\begin{description}
    \item[Goals] is the task to do (eg send an email)
    \item[Operators] are all the actions needed to achieve the goal (eg amount of mouse clicks to send email)
    \item[Methods] are the group of operators (eg move mouse to send button, click the button) 
    \item[Selection] are the decisions which users have to make (eg `move mouse to send button, click button' or `move mouse to send button and press enter') 
\end{description}
\subsubsection{Cognitive Walkthrough}
This is a technique used to evaluate the learnability of a system. Unlike user testing, it doesn't involve users (which means it can be relatively cheap to implement) as, like heuristic evaluations, it relies on experts to assess the interface.

At its core, a cognitive walkthrough has three parts:
\begin{enumerate}
    \item Identify the user goal you want to examine
    \item Identify the tasks you must complete to accomplish that goal
    \item Document the experience while completing the tasks
\end{enumerate}
The final stage asks questions: will users understand how to start the task; are the controls conspicuous; will users know the control is the correct one; was there feedback to indicate you completed (or did not complete) the task.

\subsection{Empirical Evaluation}
In this method, the users participate in trials of the prototype interfaces. It requires careful design of the trial's content and conduct which may involve benchmark tasks, may involve collecting and processing subjective opinions, and will involve evaluating with user participation.

Empirical evaluation may be done through field studies, lab-based usability evaluations or controlled experiments.

\subsection{Lab-Based Usability Evaluation}
Typically this contains representative tasks. Before the testing starts, the things to measure \& observe have to be decided - these may be ease of learning or ease of performance.

As this testing is conducted in a `lab' setting, the participants of the tests may behave in different ways to the way they are expected to which can skew results. However, this should not discourage from using this method.