\lecture{Introduction \& Markup}{08-11-22}{13:00}{Rich and Co.}{RB LT1}

\section*{Introduction to Item II}
There are a number of different lecturers on this module: Rich, Matt and Kirsten.

The exam will be computer based however not all computer marked. It will comprise mostly of multiple choice questions which will test knowledge of modern HTML and CSS. The multiple choice answers will be evil. The best practice is preparation. Exam date and time will be in January and will be announced on timetable at some point.

There is a Google Doc linked from Moodle which contains all the information and resources about this item. This document contains, pre session, during session and post session work.

There are drop ins on Thursdays in the FTC, these are compulsory. 

There is a channel on the Discord Server (\texttt{\#ccc-web}) where support can be sought.

The recommended book is available electronically through the library. One of the authors, Remy, has delivered guest lectures at the University.

\subsection*{Online Resources}
Look at Mozilla Developer Network, add \texttt{MDN} to the end of any Google query about web development and their resources will come up.

Do not use W3Schools. It is bad.

\section*{Markup}
Markup comes from the days of editors hand writing articles to be printed then annotating that with styles. This document then goes to a Typesetter who would design the content based off of the editors markings, hence markup.

HyperText Markup Language (HTML) is a form of markup, which is non-linear. It is a series of opening and closing tags, which together make an element. Elements can have attributes which provide more information on them or the way in which they should behave. 

\section*{HTML Introduction}
HTML5, the latest and most up to date version, should always start with the line \texttt{<!doctype html>}. This will tell the browser that the page is to be rendered as a HTML5 document.

A HTML document is comprised of two sections, a \texttt{<head>} and a \texttt{<body>}. In HTML5, the two sections do not need to be marked out as different sections, once you have specified that the document is HTML5 then the renderer is able to infer the difference.
\subsection*{\texttt{<head>}}
This contains information about the document. Elements which you might see include \texttt{<title>} which defines the title of the page and \texttt{<meta>} which provides additional information about the webpage. Nothing in the head element is rendered.

\subsection*{\texttt{<body>}}
This contains the content of the page. Numerous different tags are available within this to define the style of the content.

\section*{Markup}
There are two types of Markup.
\subsection*{Procedural}
This defines what to do and how it looks. It does not define why to do it.
\subsection*{Descriptive}
This says what it means, not how it looks or what to do.

This is stratified (separates content from presentation), dynamic (different presentation to suit circumstances) and semantic (enables machine processing).

This means that we use descriptive markup, with semantic value, improving information quality and consequently styling of our pages must be achieved outside HTML.