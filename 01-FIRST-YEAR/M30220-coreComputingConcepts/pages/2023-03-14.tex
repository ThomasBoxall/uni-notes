\lecture{Introduction to HCI \& Usability}{2023-03-14}{13:00}{John}{RB LT1}

The format for this item will be a lecture on Tuesday where new content is delivered then a research task issued to be completed during the Thursday drop-in session. The research task is to be submitted on Moodle.

\section{What is HCI?}
Human Computer Interaction (HCI) has historically been known as the study of interaction between people and computers. As a result of the internet of things HCI is changing, it is shifting to become the study of interaction between people and machines. 

The \textit{Association for Computing Machinery} has defined HCI as "a discipline concerned with the design, evaluation and implementation of interactive computing systems for human use and with the study of major phenomena surrounding them".

\subsection{Goals}
HCI's major goal is to improve the interactions between people and computers, by extension machines; and to make computers more usable and receptive to the user's needs.

HCI's long term goal is to design systems that minimise the barrier between the human's cognitive model of what they want to accomplish and the computer's understanding of the user's task. 

\subsection{Scope of HCI}
HCI is very broad and encompasses many different fields of Computer Science, including: AI, Psychology, Sociology, Art, Design, etc. 

\subsection{Sub-Domains of HCI}
There are three different subdomains of HCI, each has a different scope and focus however they are all concerned with the design and evaluation of the user interface.
\subsubsection{Usability}
Usability is concerned with making systems easy to learn, easy to use, limiting errors, and the severity of errors. Usability refers to how successfully a user can use a system to accomplish a specific goal and uses terms like: error rates, time to complete tasks, task failures, number of lookups made.
\subsubsection{User Experience}
User Experience encompasses the user's entire experience with an interface. This includes how well the user interface worked, how the user \textit{expected} it to work, how the user felt about using it, and how the user feels about the system overall. User Experience uses terms like: satisfaction, intuitive, frustration, good experience, difficult, confusing.
\subsubsection{User Centred Design}
User-Centred Design (UCD) is an iterative design process in which designers focus on the users and their needs in each phase of the design process. Design teams involve users throughout the design process through research abd design techniques; this enables them to create highly usable and accessible products for them. 

\subsection{Why Is HCI Important?}
Badly designed interfaces can waste the user's time, cause frustration and lead to errors. Users will often leave websites or apps with bad interfaces in frustration. 

For web apps and desktop apps, the principles of good user interface design have been well understood since the 1990's. However, applying these design principles is often neglected, which could be due to: prioritizing functionality over usability or user experience; budgetary and time constraints; or not having a clear part of the software development life cycle to consider usability and user experience. 

\subsection{HCI: A Solved Problem?}
No. The rise in popularity of NUI methods (and the metaverse) is changing everything. Application of HCI to NUI and Metaverse is far from a solved problem. The popularity of NUI methods has now crossed a threshold for a paradigm shift, which could lead to a redesign of interfaces to suit NUI methods more (currently NUI methods often simulate mouse point and click). NUI methods are the foundations of the Metaverse. 

\section{Paradigms In HCI}
There are four key paradigms in HCI, each different from one another.
\subsection{Command Line Interfaces}
Command Line Interfaces (CLIs) are text based. Users control the computer by typing in commands; this requires little processing power and can be extremely powerful, however they can take longer to learn than a GUI. Originally, most interfaces were CLIs and most modern devices will still have one. Internet of Things devices will often use CLIs. An example command execution in Linux is shown below.
\begin{verbatim}
user@bash: pwd
/home/myCSdata
user@bash: ls
file.txt
\end{verbatim}

\subsection{Graphical User Interfaces}
Graphical User Interfaces (GUIs) are the desktop metaphor. They are based on point-and-click interaction which means they are not adapted for NUI input modalities, therefore NUI will often simulate point-and-click. Not much has changed with GUIs since 1984. 

\subsection{Natural User Interfaces}
Natural User Interfaces (NUIs) are ``a system for human-computer interaction that a user operates via intuitive actions related to natural, everyday human behaviour leveraging modalities like touch, gestures, or voice''. 

A NUI mimics real-world interaction, however they are not fully developed so will often simulate point-and-click in the GUI. There are potential redesigns needed for the NUI method, which won't restrict to simulating point and click. 

There are already a number of uses of NUI
\begin{description}
    \item[Speech Recognition] used in voice recognition in smart assistants. Users can ask questions, control home automation devices and media playback through their voice. 
    \item[Brain-Machine Interface] reads brain signals and translates them into actions within the computer systems. These have lots of possible applications in the health sector.
    \item[Touch Screen] allows user to interact with device through touch of a finger. This is the most common form of NUI application.
    \item[Gesture Recognition] tracking user motions and physical actions then using these as the input to computing devices. Eg, Nintendo Wii's controller.
    \item[Gaze Tracking] uses user eye movements to estimate a position on screen for a mouse pointer.
\end{description}

\subsection{Metaverse}
\define{Metaverse}{Generally refers to the concept of a highly immersive virtual world where people can gather to socialise, play and work\\ - \textit{Merriam-Webster}}

A metaverse can either be immersion in VR or AR. 

\section{Interface Design}
User Interface design is a subset of HCI. Designers must consider a variety of factors, including
\begin{itemize}
    \item What people want and expect, physical limitations and abilities people posses;
    \item What people find enjoyable and attractive;
    \item How information processing systems work;
    \item Technical characteristics and limitations of the computer hardware and software must also be considered/
\end{itemize}

The \textit{user interface} is the part of a computer and its software that people can see, hear, touch, talk to, or otherwise understand or direct. This includes the input and output of the machine: input is some form of communication of requirements (predominantly through point-and-click); output is the results of processing user's requirements (predominantly through the display screen).

Good user interface design will provide a mix of well-designed input ad output mechanisms that satisfy the user's needs, capabilities and limitations in the most effective way possible. The best interface is one that is not noticed, rather it allows the user to focus on the information and task at hand, not the mechanisms used to present the information and perform the task.

The use of other human senses as part of the user interface is currently being developed. 

\section{Examples of Bad Interfaces}
Windows 8 is an example of a bad interface. The designers tried to incorporate two operating systems into one. This lead to it having double the complexity and requiring double the time to learn. 

The Apple Watch interface has also been classified as a bad interface - this is down to its tiny targets which can be difficult to tap on, leading to miss-taps. 