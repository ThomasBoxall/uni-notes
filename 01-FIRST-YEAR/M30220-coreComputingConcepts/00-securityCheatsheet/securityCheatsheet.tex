\documentclass[a4paper,11pt]{article}
\usepackage{xcolor}
\usepackage{geometry}
\usepackage{hyperref}
\hypersetup{
    colorlinks=false,
    linkcolor=blue,
    linkbordercolor=blue,
    pdfborderstyle={/S/U/W 1}
}
\usepackage{graphicx}
\usepackage{float}
\usepackage[strict]{changepage} %allows indented blocks
\geometry{
a4paper,
total={170mm,257mm},
left=20mm,
top=20mm,
marginparsep=0mm,
}
\setlength\parindent{0pt} % get rid of the stupid indent

\title{CCC Item III: Security Cheatsheet}
\author{Thomas Boxall\\ \texttt{up2108121@myport.ac.uk}}
\date{May 2023}

\usepackage{fancyhdr}
\pagestyle{fancy}
\fancyhead{} % clear all header fields
\renewcommand{\headrulewidth}{0pt} % no line in header area
\fancyfoot{} % clear all footer fields
\renewcommand{\footrulewidth}{0.4pt}
\fancyfoot[C]{\thepage} % page number in centre of the page
\fancyfoot[R]{\footnotesize Thomas Boxall\\ \texttt{up2108121@myport.ac.uk}} % right hand footer has author name on top line and author contact on bottom line
\fancyfoot[L]{\footnotesize CCC Item III: Security Cheatsheet \\ May 2023} % left hand footer has title of document on top line and date on bottom line


\begin{document}

\maketitle
\thispagestyle{fancy}

\section{Definitions}
\textbf{Information Security} is the preservation of confidentiality, integrity and availability of information. There is a balance to be found between the three, as often when you change one the others will change negatively.\\
\textbf{Confidentiality} is the property of information which means it is not disclosed to unauthorised individuals, entities or processes.\\
\textbf{Integrity} is the property of safeguarding the accuracy and completeness of assets.\\
\textbf{Availability} is the property of being accessible and usable upon demand by an authorised entity.\\
\textbf{Assets} are anything of value to the organisation, its business operations, and its continuity.\\
\textbf{Threats} are potential causes of incidents that may result in harm to a system or organisation (can be internal or external).\\
\textbf{Vulnerabilities} are weaknesses of assets or groups of assets that can be exploited by \textit{threats}.\\
\textbf{Impact} is the result of an information security incident which has been caused by a threat and affects assets. (e.g. monetary loss, fines, loss of reputation)\\
\textbf{Risk Management} is the process organisations go through to identify, assess and control risks. \\
\textbf{Object} of an attack is the entity which is being attacked, the target.\\
\textbf{Subject} of an attack is the entity carrying out the attack against the target. The subject attacks the object. 

\section{Assets}
\textbf{Primary Assets} are information or business procedures. These are the most valuable things to the organisation.\\
\textbf{Supporting Assets} are other assets who, if compromised, could adversely impact primary assets.
\subsection{Primary Assets}
\textbf{Information assets} are typically of the highest value to an organisation. Especially the case for \textit{business critical information} (without which, business could not operate), \textit{personal information} (data of employees and customers, must be protected in accordance with the law), \textit{strategic information} (gives the business an advantage in the market), and \textit{high-cost information} (gathering, processing, storage, transmission is expensive - if lost then business has to spend lots of money again).\\
\textbf{Business Processes} are processes which: contain secret processes; involve proprietary technology; if modified can affect the outcome of the organisation's mission; are necessary for the organisation to comply with contractual, legal or regulatory requirements. If a business fails to document or protect these procedures significant adverse effects can be had, often this will come to light when an employee is off sick or on holiday, or leaves the company all together and it is discovered that the processes they oversaw are not sufficiently documented.

\subsection{Supporting Assets}
\textbf{Hardware Assets} are the physically technology which is used to: house and execute the software; store and carry the data; or provide the interface for data entry/ removal from the system. Hardware assets cover most physical technology (e.g. computer, laptop, keyboards). Hardware devices should also have adequate physical protection.\\
\textbf{Software} covers all applications, operating systems and assorted command utilities. Development of software is often under-resourced which leads to security being an afterthought not implemented throughout, this is bad and ideally security would be implemented throughout the software development cycle.\\
\textbf{Networks} are responsible for the effective transmission of information between interconnected computing devices. They are also a very good vector for attacks. The likelyhood of an attack succeeding can be reduced by implementing policies and technical responses as well as examining ports and packets at the perimeter of the network to ensure only the data which is necessary for business function is being exchanged. The later is done using a firewall. The internal network should also be protected through segregation of critical systems, access controls and monitoring software.\\
\textbf{Personnel} are the people who are interacting with the information systems and are the subject of numerous vulnerabilities. Anyone interacting with an information system should be given appropriate training. Organisations should also develop a comprehensive set of policies which should be written using plain business terminology with minimal use of technical jargon. The effectiveness and awareness of information security should be monitored.

\section{Information Security Governance}
\textbf{Information Security Governance} is how organisations control, direct and communicate their cyber risk management activities. This will include a collection of policies which must be continually reviewed and revised to keep up-to-date with the business needs and continually changing threats/ vulnerabilities.\\
\textbf{Policies} are a principle or rule to guide decisions and achieve rational outcomes.\\
\textbf{Procedures} are a list of steps that constitute instructions for performing some action or accomplishing some task.\\
\textbf{Standards} are detailed statements which quantify what must be done to comply with policies.\\
\textbf{Guidelines} are a set of recommended actions to assist in complying with policies.\\
\textbf{SETA} \textit{Security Education, Training and Awareness} is a programme that helps employees do their jobs securely. 

\section{Identification \& Authentication}
\textbf{Authentication} of claimed identities is the first line of defence for the system and safeguards against unauthorised use.\\
\textbf{Passwords} are the most common means of authentication, conceptually simple however they are weak: badly selected, written down, infrequently or never changed, reused across multiple systems, forgotten, shared.\\
Users must be able to identify user-specific access controls and be held individually accountable for their actions.

\subsection{Passwords}
\textbf{Lots of vulnerabilities} with use of passwords including: easy to select a bad one, get written down, infrequently or never changed, same password used for multiple systems, only needed at the start of a session.\\
\textbf{Defence against password guessing} is traditionally to lock the user out after a number of failed attempts is a form of \textit{denial of service}. 

\section{Access Control}
\textbf{Identity} is the properties of an individual or resource that can be used to identify uniquely one individual or resource.\\
\textbf{Authentication} is the process of ensuring that the identity of a subject or resource is the one claimed.\\
\textbf{Authorisation} is the process of checking the authentication of an individual or resource to establish their authorised use of, or access to information or other assets.\\
\textbf{Accounting} ensures that user activities can be tracked back to them\\
\textbf{Auditing} is the process of either a formal or informal review of actions, processes, policies and procedures\\
\textbf{Compliance} is working in accordance with the actions, processes, policies and procedures laid down. 

\subsection{Access Control Policies}
\textbf{Discretionary Access Control} Policy (DAC) - controls access based on identity of individuals. Each access controlled object must be set individually for each user.
\begin{adjustwidth}{2em}{1em}
\textit{Access Control Matrix} permissions are shown as a matrix with a subject having a single entry for each object containing all their permissions.\\
\textit{Access Control List (ACL)} stores all the permissions for a single object in a `linked-list' style structure, with each index representing a subject's access levels.\\
\textit{Capability List} stores all the access rights for a single subject in a `linked-list' style structure, with each index representing a different object that subject has access to.
\end{adjustwidth}
\textbf{Mandatory Access Control} Policy (MAC) - controls access based upon security labels. Users are assigned under a clearance level which defines what they have access to. Labels may include: top secret, secret, classified, unclassified. A number of models are available for access control.
\begin{adjustwidth}{2em}{1em}
\textit{Bell LaPadula (BLP) Model} provides subject with no access to anything above their level but read access to everything below.\\
\textit{Biba Integrity Model} provides a ``no write up, no read down'' model\\
\textit{Clark-Wilson Integrity Model} provides a model where no changes may be made by unauthorised subjects, no unauthorised changes can be made by authorised subjects, and maintains internal \& external consistency.\\
\textit{Graham-Denning Access Control Model} uses objects, subjects and rights.\\
\textit{Brewer-Nash Model} allows subjects to only access one of two conflicting sets of data, preventing conflicts of interest.
\end{adjustwidth}
\textbf{Role-Based Access Control} Policy (RBAC) - controls access based on roles. Users are assigned to one or many roles. Roles come with permissions. Users inherit permissions of the role.\\
\textbf{Attribute-Based Access Control} Policy (ABAC) - controls access based on attributes of users. It uses various attributes of the user including their environment and information assets to determine permissions.\\

\section{Authentication}
\textbf{Factors of Authentication} are mechanisms by which an individual or resource can be authenticated. The three common factors are: \textit{something the supplicant knows} (PIN number, password), \textit{something the supplicant has} (security token, bank card), and \textit{something the supplicant is} (fingerprint, retina/ iris scan).\\
\textbf{Strong Authentication} is a procedure based on the use of two or more \textit{different} factors. The factors used should be mutually independent (which means if one is compromised, the other isn't).\\
\subsection{Biometrics}
\textbf{Biometrics} is the use of a body measurement (e.g. fingerprint) as a factor of authentication.\\
\textbf{False Rejection Rate} (FRR) is the percentage of identification instances in which authorised users are denied access (Type I error)\\
\textbf{False Accept Rate} (FAR) is the percentage of identification instances in which unauthorised users are allowed access (Type II error)\\
\textbf{Crossover Error Rate} (CER) is the level at which the number of false rejections equals the false acceptances\\
\textbf{Requirements} of a biometric system: universality, distinctiveness, persistence, collectability, performance, acceptability, circumvention.

\section{Cryptography}
\textbf{Cryptography} is a way of turning plaintext into cyphertext (which can later be turned back into plaintext).\\
\textbf{Encryption} is the process of altering the plaintext so that it becomes unreadable to the normal human.\\
\textbf{Decryption} is the process of altering the cyphertext so that it returns to plaintext.\\
Encryption links four things together (plaintext $m$, ciphertext $c$, key $k$ and the algorithm $E$) in the relationship $c = E_k(m)$\\
\textbf{Cryptographic Algorithm} is the algorithm used to encrypt/ decrypt the data. Modern algorithms should be: large enough key to resist brute force search; resistant to frequency analysis; setup so that a small change in the plaintext results in a large change in the ciphertext; and that the security depends on  the secrecy of the key \& not on the secrecy of the algorithm used.

\subsection{Cryptographic Algorithms}
\textbf{Symmetric} is where encryption and decryption both use the same key. This results in both the sender and receiver using the same key which means if the key is compromised, an intermediate can decrypt and read the messages. Examples include DES, 3DES, AES.\\
\textbf{Asymmetric} is where the encryption and decryption use different but related keys. This is complex so it is often used to encrypt the symmetric secret key which has been used for encryption for the main data transmission.\\
\textbf{Diffie-Hellman Key Exchange} is where asymmetric encryption is used to exchange session keys. This then allows quick and efficient communication of symmetrically encrypted messages where the session key is the symmetric encryption key.\\
\textbf{Caesar Cipher} is a simple cipher where letters are substituted for other letters in a circular motion.\\
\textbf{Substitution} is where one value is substituted for another.\\
\textbf{Transposition} is where values are rearranged within a block.\\
\textbf{RSA} is a asymmetric cryptographic algorithm which uses prime numbers and maths to encrypt each character in the transmission. 

\section{Risk Management}
\textbf{Risk Management} the process of identifying and reviewing risks so that they can be minimised/ removed.\\
\textbf{Risk} has three elements: threat, vulnerability and impact. The higher likelihood and impact of a risk, the higher risk the activity is.\\
\textbf{Risk Assessment Steps} identify risks, analyse risks, treat risks, monitor \& review.\\
\textbf{Qualitative Risk Analysis} uses a scale of attributes to determine the magnitude of the consequences/ likelihood. Can be used as initial screening to identify risks requiring detailed analysis.\\
\textbf{Quantitative Risk Analysis} uses a sale of numerical values for consequences/ likelihood values. Typically these numbers are derived from historical incidents.\\
\textbf{Critical Appraisal of Risk Methods and Frameworks} was produced by the NCSC so risks can be better understood and approaches available can be supported. Context is important when using a framework.
\subsection{Cost Benefit Analysis}
\textbf{Cost Benefit Analysis} may be used to get quantitative figures which can be analysed to see what is worth investing in to prevent risks.\\
CBE is calculated using a formula where $CBA$ is Cost Benefit Analysis, $ACS$ is Annualized Cost of Safeguard, $ALE$ is Annualized Loss Expectancy, $SLE$ is Single Loss Expectancy, and $ARO$ is Annualized Rate of Occurrence:
\[CBE = ALE_{prior} - (ALE_{post} + ACS)\]
where $ALE$ is calculated as $ALE = SLE \times ARO$
\subsection{Risk Treatment Options}
\textbf{Retain / Accept} where an organisation may tolerate (but not ignore) the risk.\\
\textbf{Avoid / Terminate} where an organisation may decide not to do the thing that incurs risk.\\
\textbf{Share / Transfer} where the risk is transferred or shared via an insurance policy or third party.\\
\textbf{Modify / Reduce} where controls are adopted to lower the current levels of risk.

\section{Common Software Errors}
Software is highly-prone to causing security compromises. This is mostly due to the fact that software development is under-resourced which leads to security being added as an afterthought.\\
\textbf{Web Applications} may have attack vectors however finding the path from threat agents through attack vectors, security weaknesses, security controls, technical impact, to business impact can be very difficult.\\
\textbf{Security Requirements Specification} should be part of the overall statement of requirements document from which the design is generated.\\
Security doesn't just mean prevent improper access and misuse, it also means defensive programming (to make sure only valid and accurate data is processed by the system); proper functional testing (to ensure it behaves as expected); methods to back up secure data against loss or damage; and compliance with any legal and regulatory requirements.

\section{Typical Web Errors}
\textbf{Cross Site Scripting} (XSS) is where the input on a webpage is not neutralised correctly.\\
\textbf{Reflected XSS} is where the victim's browser executes code based off on an input which has been made in the browser.\\
\textbf{Stores XSS} is where the app stores the bad data in a database which is later read into the app and included in dynamic content.\\
\textbf{SQL Injection} is where a piece of software doesn't properly sanitise an input which is directly fed into a SQL query. Malicious code can be inserted which can display content from the database.

\section{Secure Development \& Deployment}
All developers are responsible for security during development. It is important to secure the code, build \& deployment pipeline and continually test the security.\\
It is generally regarded as good practice to fully isolate the development and production environments as well as treating the development environment as if it is compromised.\\
\textbf{Change Control} should be strict for production software. At least two people should be involved in pushing code, preferably one writes it and another approves it.\\
\textbf{Patching} of software should be done as soon as bugs are found. Before releasing into a production environment, patches should be tested in development environments.

\section{Accreditation and Certification}
\textbf{Accreditation} is formal recognition by an independent body (the accreditation body) that a certificate body operates according to international standard.\\
\textbf{Certification} is provision by an independent body that the product, system or service in question meets specific requirement.


\end{document}