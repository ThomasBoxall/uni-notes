\lecture{Module Introduction}{20-09-2022}{14:00}{Nadim \& Matthew}{PK2.23}

\begin{keyconcept}{Module Aims}
This module will build up programming skills either from scratch or from where you are currently. 

It will give you the basic knowledge; guidance, help and feedback to help develop programming skills.
\end{keyconcept}
Importantly, this module is 40 credits. It spans across the entire year.

\section*{Programming}
Programming is the process of constructing computer programs, this encompasses analysing the problem, designing the algorithm, implementing the algorithm and testing the algorithm.

We write the programs in a programming language.

For the first $\frac{3}{4}$ of the year, we'll use Python 3 and for the final $\frac{1}{4}$ of the year, we'll use Dart. Dart is similar to Java. We will be the first cohort to use Dart.

Programming is a skill, which can only be developed through practice and should be fun!
Having a good understanding and ability to program is important later during in the course and for careers.

\section*{Module Organisation}
For this module, there will be content shared on Moodle (notes for lectures and videos complementing the notes) and timetabled sessions (in some, fundamental ideas will be covered which will make it possible to complete the weekly worksheets). Worksheets will be released weekly onto Moodle, these should be completed before the practical class of the following week.

Monday at 3pm in RB LT1 is the tutorial class. You need to go through the notes on Moodle before the sessions.

Practical classes are 1 hour 50 minute sessions in a computer lab. The main purpose of these is to get feedback on the worksheets.

\section*{Support}
The academic tutors (Xia and Eleni) can be booked on Moodle.

There are drop-in sessions on Monday in the FTC. This session is optional and is designed for targeted questions or issues which can't be resolved in the tutorial/ practical classes. 

\section*{Out Of Class Work}
Should be spending about 8 hours per week outside of timetabled sessions working on this module. This includes working through the worksheets. 

\section*{Assessments}
There are three types of assessment used throughout the year
\begin{itemize}
    \item 5x 30min programming tests (held in class, weighted 5\% each)
    \item 2x 60min Computer based multiple choice tests (weighted 15 \% each, one in January and one in May/June)
    \item 2x large programming assignments (weighted 20\% and 25\% respectively)
\end{itemize}
The programming tests will be based off of the previous weeks worksheets. There will be a practice test in week 3 (so we can understand how they work)

Each of the programming assignments will have a few weeks in which they can be worked on.

\section*{Resources}
To write and execute Python programs, the recommended IDE is Pyzo. Other IDEs can be used however no support for configuration will be provided.

We will be using Python 3.x NOT Python 2.x.

The recommended book is called `Python programming: an Introduction to Computer Science 3rd Edition'. There are a number of copies available in the library. Its ISBN number is `9781590282755'.