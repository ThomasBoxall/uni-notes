\lecture{LECTURE: ERD, Attributes \& Datatypes}{27-10-22}{13:00}{RB LT1}{Mark}

\section*{Attributes}
An entity is a thing. The attributes, of an entity, are the things which describe the thing. We need to be able to identify individual entities.
\subsection*{Example: People}
If we are having a person as an entity, the attributes we will probably need are: date of birth; given name; family name. There are attributes which we don't need to store (for example: weight, height).
\subsubsection*{Addresses}
When we store people, we will usually store their address in their record. This will be explored when do normalisation after consolidation week.

\subsection*{GDPR}
When we store data, we have to be sure we are being GDPR compliant and storing what what you need to store.

GDPR states that you must ensure the personal data you are processing is:
\begin{itemize}
    \item adequate - sufficient to properly fulfil your stated purpose;
    \item relevant - has a rational link to that purpose; and
    \item limited to what is necessary - you do not hold more than you need for that purpose.
\end{itemize}

\section*{Data Types}
Now we know what attributes we need to store about the attribute, we need to think about types of data that is. 
\subsection*{Names}
Names are made up from characters, these could include apostrophes and hyphens. There is a question here as to how long names can be. A rule of thumb would be to use 20 characters for first name and 25 for surnames.
\subsection*{Numeric}
There are a number of different numeric data types.
\begin{itemize}
    \item \texttt{smallint} - holds an integer range -32768 to +32767
    \item \texttt{integer} - holds an integer range -2147483648 to +2147483647
    \item \texttt{bigint} - holds an integer range -9223372036854775808 to +9223372036854775807
    \item \texttt{decimal} - holds a decimal number with up to 131072 digits before the decimal point; up to 16383 digits after the decimal point
    \item \texttt{real} - similar to decimal but provides 6 decimal digits precision
    \item \texttt{double} - similar to real but provides 15 decimal digits precision
    \item \texttt{serial} - holds an integer range 1 to 2147483647
    \item \texttt{bigserial} - holds an integer range 1 to 9223372036854775807
\end{itemize}
\subsection*{Characters}
There are a number of different character data types.\\
Phone numbers should be stored as a character not as a numeric data type as they will often have leading zeros. 
\begin{itemize}
    \item \texttt{text} - variable `unlimited' length
    \item \texttt{character/ char} - fixed length (blank padding is added if less than given size)
    \item \texttt{varying character / varchar} - variable length with limit
\end{itemize}
\subsection*{Dates and Times}
There are a number of different date/ time data types.
\begin{itemize}
    \item \texttt{timestamp without timezone} - both date and time (no time zone) range 4713 BC to 294276 AD with 1 microsecond resolution
    \item \texttt{timestamp with timezone} - both date and time (with time zone) range 4713 BC to 294276 AD with 1 microsecond resolution
    \item \texttt{date} - date without time range 4713 BC to 5874897 AD with 1 day resolution
    \item \texttt{time without timezone} - time of day (no date) range 00:00:00 to 24:00:00 with 1 microsecond resolution
    \item \texttt{time with timezone} - time of day (no date), with time zone range 00:00:00 to 24:00:00 with 1 microsecond resolution and adjustment for time zone
\end{itemize}

\section*{Example of drawing up an entity}
If we have a draft entity with the following attributes \texttt{cust\_id}, \texttt{cust\_name}, \texttt{addeess}, \texttt{email}. This presents a number of problems.

If we want to search for a specific name, this is more complicated because the customer name is stored as a single attribute where it should be multiple attributes.

Addresses should not be stored as a single attribute.
\subsection*{Break down data}
We should break down information into usable data. For example, addresses should be broken down into: \texttt{address1}, \texttt{address2}, \texttt{town}, \texttt{county}, \texttt{postcode}, \texttt{country}.

Names should be broken down into \texttt{firstName}, \texttt{lastName}. It could also be argued that a single middle name could also be included.

\subsection*{Adding data types}
\begin{itemize}
    \item \texttt{cust\_id - int}
    \item \texttt{cust\_fname - varchar}
    \item \texttt{cust\_mname - varchar}
    \item \texttt{cust\_lname - varchar}
    \item \texttt{addr1 - varchar}
    \item \texttt{addr2 - varchar}
    \item \texttt{town - varchar}
    \item \texttt{postcode - char} (could be a \texttt{varchar})
    \item \texttt{email - varchar}
\end{itemize}

\subsection*{Sizes of data types}
Now we have worked out what data types we want to use, we need to think about the sizes of those data types.

