\lecture{LECTURE: Network Policies, Procedures and Standards}{2023-03-28}{09:00}{Amanda}{Zoom}

\section{The Internet Society}
The Internet Society aims to oversee the standardisation of protocols to promote interoperability. It coordinates internet design, engineering and management. The society has three main sections
\begin{description}
    \item[Internet Architecture Board (IAB)] defines the architecture of the internet
    \item[Internet Engineering Task Force (IETF)] reviews architecture definitions from IAB and works on developing protocols
    \item[Internet Engineering Steering Group (IESG)] manages the technical side of IEFT and defines the internet standard. 
\end{description}

When a proposed standard or protocol gets to the Internet Society, a working group initiated who: makes a draft version is developed and this is made available for consultation. Then the IESG gives final approval before the standard or protocol is published as a \textit{Request for Comments} (RFC), which if it doesn't reach within six months then the proposed standard or protocol is withdrawn.\\ The criteria which these proposed standard or protocols have to stick to is that: they have to be clear; be technically competent; have multiple independent and interoperable implementations with substantial operational experience; gain significant public support; and be useful within the internet domain. \\
To gain standard status there must be at least two independent and interoperable implementations which both have to be rigorously tested.\\
When a proposed standard becomes an internet standard it is also assigned a STD number and RFC number. 

\section{International Standards Organisation (ISO)}
The ISO aims to promote standardisation and related activities to facilitate international exchanges of goods and services. The ISO collaborates with other organisations including the International Electronic Commission (IEC) - who focus on the hardware side of the electrical and electronic engineering standards. The ISO and IEC collaborate through the Joint Technical Committee (JTC 1) for IT Standards.
\subsection{6-Stage Standard Development Process}
The ISO has a 6-stage development process for new standards.
\subsubsection{Stage 1: Proposal}
A new proposal is assigned to a technician and working group.
\subsubsection{Stage 2: Preparatory}
The working group prepares a draft of the standard. Once this is complete, it is passed to the committee for the consensus building phase. 
\subsubsection{Stage 3: Committee}
The proposal is registered at the ISO central Secretariat who then distribute it for balloting and consensus. Once it achieves consensus then it becomes a Draft International Standard (DIS). 
\subsubsection{Stage 4: Enquiry}
The DIS is circulated to all ISO member bodies. A time limit of five months is in place for comment and vote on approval. If successful, it becomes a Final Draft International Standard (FDIS) and if unsuccessful, it returns to the working group.
\subsubsection{Stage 5: Approval}
The FDIS is redistributed for final acceptance, which has a two month time limit on it. Technical comments are no longer considered at this stage however if it is not accepted then the FDIS returns to the working group.
\subsubsection{Stage 6: Final}
Once agreed in stage 5, then the FDIS becomes an International Standard. Some minor editorial changes are permitted prior to publication of the standard.

\section{ITU Telecommunications Standardization Sector (ITU-T)}
The Telecommunications Standardization Sector is a United Nations specialised agency whose members are governments. It ``is responsible for studying technical, operating and tariff questions and issuing recommendations on them with a view to standardisation of telecommunications on a worldwide basis''.\\
The ITU-T works across a number of different `sectors' including: network and service operations; tariff and accounting principles; network and network maintenance; data networks and open system communications; and general networks aspects.
\subsection{Broadband}
The ITU-T is responsible for broadband standards based on ATM technology. The ATM forum contributes to ATM standards.
\subsection{ITU-T Process}
The ITU-T works in four year cycles and has meetings at world telecommunications standardisation conferences. During meetings the work program for the next cycle is established, study groups are created and abolished. They also decide questions to submit to groups.\\
There is an accelerated procedure allowing recommendations to be approved when they are ready. \\
There can be significant delays in decision making as wide participation by governments, users and industrial representatives make decisions through consensus. A new way of voting - majority rule, is now being introduced which also enables more end users and vendors to take part. 

\section{Institute Of Electrical and Electronic Engineers (IEEE)}
The IEEE works in cycles.
\subsection{Cycle Part 1: Initiating the Project}
IEEE are driven by sponsor input into a collaborative group who produce a Project Authorisation Request (PAR). This includes why the project is required and what it is going to do.
\subsection{Cycle Part 2: Working Group}
Once the PAR is approved a working group is formed which works to create and write the standards. Anyone can be included in the working group.
\subsection{Cycle Part 3: Drafting The Standard}
The first draft is developed. Then the mandatory editorial coordinator checks the draft, at this stage it is only the layout of the specification which is checked. The draft then goes to ballot.
\subsection{Cycle Part 4: Balloting}
This happens once the sponsor decides the standard is stable. The sponsor will create a balloting group byt anyone can comment. The balloting group aims to eliminate bias and as such includes: producers, users, governments, and those with general interest. Balloting lasts 30-60 days and voters have the option to approve, disapprove or abstain. Consensus requires 75\% of the balloting group to respond with 75\% of the respondents approving the standard.
\subsection{Cycle Part 5: Gaining Approval}
The IEEE-SA Standards Board has the final approval of a standard. The decision they make is based on recommendations from the review committee.Standards are valid for 5 years, after which they can either be reaffirmed, revised or withdrawn. 
\subsection{Cycle Part 6: Maintaining Standards}
Formal interpretations highlight areas for consideration in future revisions. 