\lecture{Boolean Algebra Simplification I}{07-11-22}{16:00}{Farzad}{RB LT1}

Now we have covered all the different gates and how they interact, we need to look at ways to make the circuits simpler. There are a number of ways this can be done, either through Boolean Algebra Simplification or through Karnaugh Maps. We'll be looking at the former in this lecture.

As a general principle of simplification, we need to get rid of the brackets to see exactly what we are dealing with then simplify (which may involve returning some things to brackets).

\section*{Laws of Boolean Algebra}
\subsection*{Law 1: Commutative}
This law states that the order of terms is interchangable without changing the overall expression permitting the gates between them do not change.
\[A + B = B + A\]
\[A \cdot B = B \cdot A\]

\subsection*{Law 2: Associative}
This law states that where there are multiple of the same operator used brackets, and therefore gates, can be placed anywhere.
\begin{align*}
A + (B  + C) &= (A + B) + C\\
&= A + B + C\\ \\
A \cdot (B \cdot C) &= (A \cdot B) \cdot C\\
&= A \cdot B \cdot C
\end{align*}

\subsection*{Law 3: Distributive (Associative)}
This rule acts much like simplification in standard maths where a common term is `factored' out so that it only appears once in the equation.
\[ A(B+C) = A\cdot B + A \cdot C \]

\section*{Rules of Boolean Algebra}
\subsection*{Operations with 0 and 1}
\subsubsection*{Rule 1}
\begin{figure}[H]
    \begin{minipage}[t]{0.65\textwidth}
        In this rule, the 0 bit means that the output will always be A.
        \[A + 0 = A\]
    \end{minipage}\hfill
    \begin{minipage}[t]{0.25\textwidth}
        \begin{table}[H]
            \begin{tabularx}{0.5\textwidth}{XX|X}
                A & 0 & Q\\
                \hline
                0 & 0 & 0\\
                1 & 0 & 1\\
            \end{tabularx}
        \end{table}
    \end{minipage}\hfill
\end{figure}

\subsubsection*{Rule 2}
\begin{figure}[H]
    \begin{minipage}[t]{0.65\textwidth}
        In this rule, the 1 bit means that the output will always be 1.
        \[A + 1 = 1\]
    \end{minipage}\hfill
    \begin{minipage}[t]{0.25\textwidth}
        \begin{table}[H]
            \begin{tabularx}{0.5\textwidth}{XX|X}
                A & 1 & Q\\
                \hline
                0 & 1 & 1\\
                1 & 1 & 1\\
            \end{tabularx}
        \end{table}
    \end{minipage}\hfill
\end{figure}

\subsubsection*{Rule 3}
\begin{figure}[H]
    \begin{minipage}[t]{0.65\textwidth}
        In this rule, the 0 bit means that the output will always be 0.
        \[A \cdot 0 = 0\]
    \end{minipage}\hfill
    \begin{minipage}[t]{0.25\textwidth}
        \begin{table}[H]
            \begin{tabularx}{0.5\textwidth}{XX|X}
                A & 0 & Q\\
                \hline
                0 & 0 & 0\\
                1 & 0 & 0\\
            \end{tabularx}
        \end{table}
    \end{minipage}\hfill
\end{figure}

\subsubsection*{Rule 4}
\begin{figure}[H]
    \begin{minipage}[t]{0.65\textwidth}
        In this rule, the 1 bit means that the output will always be A.
        \[A \cdot 1 = A\]
    \end{minipage}\hfill
    \begin{minipage}[t]{0.25\textwidth}
        \begin{table}[H]
            \begin{tabularx}{0.5\textwidth}{XX|X}
                A & 0 & Q\\
                \hline
                0 & 0 & 0\\
                1 & 0 & 1\\
            \end{tabularx}
        \end{table}
    \end{minipage}\hfill
\end{figure}

\subsection*{Idempotent Rules}
\subsubsection*{Rule 5}
\begin{figure}[H]
    \begin{minipage}[t]{0.65\textwidth}
        \[A + A = A\]
    \end{minipage}\hfill
    \begin{minipage}[t]{0.25\textwidth}
        \begin{table}[H]
            \begin{tabularx}{0.5\textwidth}{XX|X}
                A & A & Q\\
                \hline
                0 & 0 & 0\\
                1 & 1 & 1\\
            \end{tabularx}
        \end{table}
    \end{minipage}\hfill
\end{figure}

\subsubsection*{Rule 6}
\begin{figure}[H]
    \begin{minipage}[t]{0.65\textwidth}
        \[A \cdot A = A\]
    \end{minipage}\hfill
    \begin{minipage}[t]{0.25\textwidth}
        \begin{table}[H]
            \begin{tabularx}{0.5\textwidth}{XX|X}
                A & 0 & Q\\
                \hline
                0 & 0 & 0\\
                1 & 1 & 1\\
            \end{tabularx}
        \end{table}
    \end{minipage}\hfill
\end{figure}

\subsection*{Laws Of Complementarity}
\subsubsection*{Rule 7}
\begin{figure}[H]
    \begin{minipage}[t]{0.65\textwidth}
        \[A + \overline{A} = 1\]
    \end{minipage}\hfill
    \begin{minipage}[t]{0.25\textwidth}
        \begin{table}[H]
            \begin{tabularx}{0.5\textwidth}{XX|X}
                A & $\overline{A}$ & Q\\
                \hline
                0 & 1 & 1\\
                1 & 0 & 1\\
            \end{tabularx}
        \end{table}
    \end{minipage}\hfill
\end{figure}

\subsubsection*{Rule 8}
\begin{figure}[H]
    \begin{minipage}[t]{0.65\textwidth}
        \[A \cdot \overline{A} = 0\]
    \end{minipage}\hfill
    \begin{minipage}[t]{0.25\textwidth}
        \begin{table}[H]
            \begin{tabularx}{0.5\textwidth}{XX|X}
                A & $\overline{A}$ & Q\\
                \hline
                0 & 1 & 0\\
                1 & 0 & 0\\
            \end{tabularx}
        \end{table}
    \end{minipage}\hfill
\end{figure}

\subsection*{Other Laws}
\subsubsection*{Rule 9: Double Inversion}
\begin{figure}[H]
    \begin{minipage}[t]{0.65\textwidth}
        In this rule, the 0 bit means that the output will always be A.
        \[\overline{\overline{A}} = A\]
    \end{minipage}\hfill
    \begin{minipage}[t]{0.25\textwidth}
        \begin{table}[H]
            \begin{tabularx}{0.5\textwidth}{X|X}
                $\overline{\overline{A}}$ & Q\\
                \hline
                0 & 0\\
                1 & 1\\
            \end{tabularx}
        \end{table}
    \end{minipage}\hfill
\end{figure}

\subsubsection*{Rule 10}
This rule is applicable in both directions. The bits removed can also be added back where needed. This is useful for some simplifications.
\[A + A\cdot B = A\]
\[A = A \cdot B  + A \]
\begin{example}{Derivation of Rule 10}
\begin{align*}
A &= A && rule\ 10\\
A &= A + A \cdot B\\
A &= A + (A+A\cdot B)\cdot B && rule\ 10\\
A &= A + A\cdot B + A \cdot B \cdot B && rule\ 6\\
A &= A + A \cdot B + A \cdot B\\
A &= A + nA \cdot B +\\
A &= A + A \cdot B +
\end{align*}
\end{example}

\subsubsection*{Rule 11}
\[A + \overline{A}\cdot B = A + B\]
\begin{example}{Derivation of Rule 11}
\begin{align*}
A + \overline{A}\cdot B &=\\
&= (A+A\cdot B)+\overline{A}\cdot B && rule\ 10\\
&= A\cdot A + A \cdot B + \overline{A} \cdot B && rule\ 7\\
&= A\cdot A + A \cdot B + A \cdot \overline{A} + \overline{A} \cdot B && rule\ 8\\
&= A\cdot (A + B) + \overline{A}\cdot (A + B)\\
&= (A + \overline{A})\cdot (A + B) && rule\ 6\\
&= 1\cdot (A + B)\\
&= A + B
\end{align*}
\end{example}

\subsubsection*{Rule 12}
\[(A+B)\cdot (A+C) = A+ B \cdot C\]
\begin{example}{Derivation of Rule 12}
\begin{align*}
(A+B) \cdot (A+C) &=\\
&= A\cdot A + A \cdot C + A\cdot B + B\cdot C && rule\ 7\\
&= A + A\cdot C + A\cdot B + B \cdot C\\
&= A\cdot (1+C) + A\cdot B + B \cdot C && rule\ 2\\
&= A + A \cdot B + B \cdot C\\
&= A \cdot (1 + B) + B\cdot C && rule\ 2\\
&= A+ B\cdot C
\end{align*}
\end{example}