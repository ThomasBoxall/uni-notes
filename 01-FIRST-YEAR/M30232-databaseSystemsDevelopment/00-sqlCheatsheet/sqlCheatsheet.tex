\documentclass[a4paper,11pt]{article}
\usepackage{xcolor}
\usepackage{geometry}
\geometry{
a4paper,
total={170mm,257mm},
left=20mm,
top=20mm,
marginparsep=0mm,
}
\setlength\parindent{0pt} % get rid of the stupid indent

\title{SQL Syntax Cheatsheet}
\author{Thomas Boxall\\ \texttt{up2108121@myport.ac.uk}}
\date{April 2023}

\usepackage{fancyhdr}
\pagestyle{fancy}
\fancyhead{} % clear all header fields
\renewcommand{\headrulewidth}{0pt} % no line in header area
\fancyfoot{} % clear all footer fields
\renewcommand{\footrulewidth}{0.4pt}
\fancyfoot[C]{\thepage} % page number in centre of the page
\fancyfoot[R]{\footnotesize Thomas Boxall\\ \texttt{up2108121@myport.ac.uk}} % right hand footer has author name on top line and images reference on bottom line
\fancyfoot[L]{\footnotesize SQL Syntax \\ Cheatsheet} % left hand footer has title of document on top line and 'Revision Sheet' on bottom line


\begin{document}

\maketitle
\thispagestyle{fancy}

\section{PostegreSQL `Slash' Commands}
\verb|\l| list all databases in the server\\
\verb|\c databaseName| connect to database\\
\verb|\d| list everything in the database\\
\verb|\dt| list just the tables

\section{\texttt{CREATE}}
a database: \verb|CREATE DATABASE databaseName;|\\
a table: 
\begin{verbatim}
CREATE TABLE tableName(
    attribute dataType constraints,
    ...
);
\end{verbatim}

\section{\texttt{DROP} \& \texttt{DELETE}}
a whole database: \verb|DROP DATABASE databaseName;|\\
a single table: \verb|DROP TABLE tableName;|\\
a single record: \verb|DELETE FROM tableName WHERE attribute = value;|

\section{\texttt{ALTER}}
a table to add new column: \verb|ALTER TABLE tableName ADD COLUMN colName dataType;|

\section{\texttt{INSERT}}
data to a table: \verb|INSERT INTO tableName (attr1, attr2) VALUES(val1, val2);|

\section{\texttt{UPDATE}}
a record in a table: \verb|UPDATE tableName SET attributeName = val WHERE idAttribute = val;|

\section{\texttt{SELECT}}
everything from a table: \verb|SELECT * FROM tableName;|\\
only certain records from a table: \verb|SELECT * FROM tableName WHERE attributeName = desiredValue;|\\
only certain attributes from a table: \verb|SELECT attr1, attr2 FROM tableName;|

\section{Functions}
number of values: \verb|COUNT(toCount)|\\
maximum value: \verb|MAX()|\\
minimum value: \verb|MIN()|\\
average value: \verb|AVG()|\\
generate a random number: \verb|RANDOM()|\\
current date and time: \verb|NOW()|

\section{\texttt{JOIN}}
two tables together using an inner join
\begin{verbatim}
SELECT tableName.attributeName, ... FROM tableName
JOIN tableToJoinName ON tableToJoinName.attribute = currentTableName.attribute;
\end{verbatim}
two tables together not using inner join but achieve the same results:
\begin{verbatim}
SELECT attributeName, attributeName FROM tableOneName, tableTwoName
WHERE tableOneName.attributeName = tableTwoName.attributeName;
\end{verbatim}

two tables together using a left join:
\begin{verbatim}
SELECT currentTableName.attributeName, ... FROM currentTableName
LEFT JOIN tableTwo ON tableToJoinName.attribute = currentTableName.attribute;
\end{verbatim}

two tables together using an outer join:
\begin{verbatim}
SELECT currentTableName.attributeName, ... FROM currentTableName
RIGHT JOIN tableTwo ON tableToJoinName.attribute = currentTableName.attribute;
\end{verbatim}

two tables together using a full outer join:
\begin{verbatim}
SELECT currentTableName.attributeName, ... FROM currentTableName
FULL OUTER JOIN tableTwo ON tableToJoinName.attribute = currentTableName.attribute;
\end{verbatim}

\section{\texttt{GROUP BY}}
an attribute in the table:
\begin{verbatim}
SELECT attributeOne, attributeTwo FROM tableName
GROUP BY attributeOne;
\end{verbatim}
an attribute in the table and only show certain records:
\begin{verbatim}
SELECT attributeOne, attributeTwo FROM tableName
GROUP BY attributeOne
HAVING attributeName = value;
\end{verbatim}

\section{\texttt{ORDER BY}}
an attribute in ascending order: 
\begin{verbatim}
SELECT attributeName FROM tableName ORDER BY attributeName;
\end{verbatim}

an attribute in ascending order:
\begin{verbatim}
SELECT attributeName FROM tableName ORDER BY attributeName ASC;
\end{verbatim}

an attribute in descending order:
\begin{verbatim}
SELECT attributeName FROM tableName ORDER BY attributeName DESC;
\end{verbatim}

\section{Wildcards}
any number of characters: \verb|%|

\section{Views}
create a view: \verb|CREATE viewName AS queryString;|\\
executeView \verb|SELECT * FROM viewName;|

\end{document}