\documentclass[a4paper,11pt]{article}
\usepackage{xcolor}
\usepackage{geometry}
\geometry{
a4paper,
total={170mm,257mm},
left=20mm,
top=20mm,
marginparsep=0mm,
}
\setlength\parindent{0pt} % get rid of the stupid indent

\title{SQL Syntax Cheatsheet}
\author{Thomas Boxall\\ \texttt{up2108121@myport.ac.uk}}
\date{April 2023}

\usepackage{fancyhdr}
\pagestyle{fancy}
\fancyhead{} % clear all header fields
\renewcommand{\headrulewidth}{0pt} % no line in header area
\fancyfoot{} % clear all footer fields
\renewcommand{\footrulewidth}{0.4pt}
\fancyfoot[C]{\thepage} % page number in centre of the page
\fancyfoot[R]{\footnotesize Thomas Boxall\\ \texttt{up2108121@myport.ac.uk}} % right hand footer has author name on top line and author contact on bottom line
\fancyfoot[L]{\footnotesize SQL Syntax Cheatsheet \\ April 2023} % left hand footer has title of document on top line and date on bottom line


\begin{document}

\maketitle
\thispagestyle{fancy}

\section{PostegreSQL `Slash' Commands}
\verb|\l| list all databases in the server\\
\verb|\c databaseName| connect to database\\
\verb|\d| list everything in the database\\
\verb|\dt| list just the tables

\section{\texttt{CREATE}}
a database: \verb|CREATE DATABASE databaseName;|\\
a table: 
\begin{verbatim}
CREATE TABLE tableName(
    attribute dataType constraints,
    ...
);
\end{verbatim}
a database owned by another user: \verb|CREATE DATABASE userName OWNER userName;|

\section{\texttt{DROP} \& \texttt{DELETE}}
a whole database: \verb|DROP DATABASE databaseName;|\\
a single table: \verb|DROP TABLE tableName;|\\
a single record: \verb|DELETE FROM tableName WHERE attribute = value;|

\section{\texttt{ALTER}}
a table to add new column: \verb|ALTER TABLE tableName ADD COLUMN colName dataType;|\\
add a unique constraint to an attribute:\\ \verb|ALTER TABLE tableName ADD CONSTRAINT constraintName UNIQUE (attribute);|\\
add a length constraint to an attribute:\\ \verb|ALTER TABLE tableName ADD CONSTRAINT constraintName CHECK(length(postcode)>5);|\\

\section{\texttt{INSERT}}
data to a table: \verb|INSERT INTO tableName (attr1, attr2) VALUES(val1, val2);|

\section{\texttt{UPDATE}}
a record in a table: \verb|UPDATE tableName SET attributeName = val WHERE idAttribute = val;|

\section{\texttt{SELECT}}
everything from a table: \verb|SELECT * FROM tableName;|\\
only certain records from a table: \verb|SELECT * FROM tableName WHERE attributeName = desiredValue;|\\
only certain attributes from a table: \verb|SELECT attr1, attr2 FROM tableName;|\\
only distinct attributes from a table: \verb|SELECT DISTINCT attributeName FROM tableName;|\\
only 5 records from a table: \verb|SELECT * FROM tableName LIMIT 5;|\\
an attribute and give it an alias: \verb|SELECT attributeName as "Alias" FROM tableName;|

\section{Functions}
number of values: \verb|COUNT(toCount)|
\subsection{Text Functions}
convert character to ASCII: \verb|ASCII('char')|\\
convert ASCII to character: \verb|CHR(intVal)|\\
convert the first letter after a space to a capital: \verb|INTCAP('string to convert')|\\
get the position of a substring in a string: \verb|POSITION('substring' IN 'string to search')| (NB: string index starts at 1)\\
concatenate strings with a space between: \verb|CONCAT(stringOne, ' ', stringThree)|\\
concatenate strings with the same delimiter between each: \verb|CONCAT_WS(' ', stringOne, stringTwo)|\\
convert binary to text: \verb|CONVERT_FROM(toConvert, 'utf-8)|\\
convert all characters to uppercase: \verb|UPPER(toConvert)|

\subsection{Date Functions}
current date and time: \verb|NOW()|\\
return the current time: \verb|CURRENT_TIME|\\
extract part of a date: \verb|DATE_PART('target_part', dateToExtractFrom)|\\
truncate date to specific level \& fill the rest with \verb|0|: \verb|DATE_TRUNC('truncation-level', dateToTruncate)|\\
get the difference between now and a date as a formatted string: \verb|AGE(dateToGetDifferenceOf)|\\
get the difference between two dates as a formatted string: \verb|AGE(laterDate, earlierDate)|

\subsection{Numerical Functions}
maximum value: \verb|MAX()|\\
minimum value: \verb|MIN()|\\
average value: \verb|AVG()|\\
generate a random number: \verb|RANDOM()|\\


\section{\texttt{JOIN}}
two tables together using an inner join
\begin{verbatim}
SELECT tableName.attributeName, ... FROM tableName
JOIN tableToJoinName ON tableToJoinName.attribute = currentTableName.attribute;
\end{verbatim}
two tables together not using inner join but achieve the same results:
\begin{verbatim}
SELECT attributeName, attributeName FROM tableOneName, tableTwoName
WHERE tableOneName.attributeName = tableTwoName.attributeName;
\end{verbatim}

two tables together using a left join:
\begin{verbatim}
SELECT currentTableName.attributeName, ... FROM currentTableName
LEFT JOIN tableTwo ON tableToJoinName.attribute = currentTableName.attribute;
\end{verbatim}

two tables together using an outer join:
\begin{verbatim}
SELECT currentTableName.attributeName, ... FROM currentTableName
RIGHT JOIN tableTwo ON tableToJoinName.attribute = currentTableName.attribute;
\end{verbatim}

two tables together using a full outer join:
\begin{verbatim}
SELECT currentTableName.attributeName, ... FROM currentTableName
FULL OUTER JOIN tableTwo ON tableToJoinName.attribute = currentTableName.attribute;
\end{verbatim}

\section{\texttt{GROUP BY}}
an attribute in the table:
\begin{verbatim}
SELECT attributeOne, attributeTwo FROM tableName
GROUP BY attributeOne;
\end{verbatim}
an attribute in the table and only show certain records:
\begin{verbatim}
SELECT attributeOne, attributeTwo FROM tableName
GROUP BY attributeOne
HAVING attributeName = value;
\end{verbatim}

\section{\texttt{ORDER BY}}
an attribute in ascending order: 
\begin{verbatim}
SELECT attributeName FROM tableName ORDER BY attributeName;
\end{verbatim}

an attribute in ascending order:
\begin{verbatim}
SELECT attributeName FROM tableName ORDER BY attributeName ASC;
\end{verbatim}

an attribute in descending order:
\begin{verbatim}
SELECT attributeName FROM tableName ORDER BY attributeName DESC;
\end{verbatim}

\section{Wildcards}
any number of characters: \verb|%|

\section{Views}
create a view: \verb|CREATE viewName AS queryString;|\\
execute view: \verb|SELECT * FROM viewName;|

\section{Roles}
create a role: \verb|CREATE ROLE username WITH LOGIN PASSWORD 'password';|\\
give role superuser permission: \verb|ALTER ROLE username WITH SUPERUSER;|\\
give role read access to table: \verb|GRANT select ON tableName TO username;|\\
give two roles write access to a table: \verb|GRANS update ON tableName TO username, username;|\\
revoke role read access to a table: \verb|REVOKE select ON tableName FROM username|\\
delete a role:
\begin{verbatim}
-- need to revoke all permissions first
DROP DATABASE username;
DROP ROLE username;
\end{verbatim}

\section{Encryption}
Turn on encryption (off by default): \verb|CREATE EXTENSION IF NOT EXISTS pgcrypto;|\\
Encrypt values (probably used as part of \verb|INSERT|): \verb|ENCRYPT('toEncrypt', 'key', 'aes')|\\
Decrypt values (provably used as part of \verb|SELECT|): \verb|DECRYPT(toDecrypt, 'key', 'aes')|

\section{Dates}
display data when between two dates:\\
\verb|SELECT * FROM tableName WHERE dateAttr BETWEEN 'firstDate' AND 'lastDate';|\\
change date style: \verb|SET DATESTYLE TO newDateStyle;|

\section{JSON}
return the raw value: \verb|SELECT data -> 'key' FROM jsonTable;|\\
return the text value: \verb|SELECT data ->> 'key' FROM jsonTable;|

\end{document}