\lecture{LECTURE: Types of Joins}{01-12-22}{13:00}{Mark}{RB LT1}

The joins we have looked at so far are \texttt{inner} joins. This displays the data where the tables overlap. For example
\begin{sql}
SELECT CUSTOMER.CUST_ID, CUST_ORDER.CUST_ORD_ID FROM CUSTOMER 
JOIN CUST_ORDER ON CUSTOMER.CUST_ID=CUST_ORDER.CUST_ID;
\end{sql}
Will probably use this the most.

\section*{Left Join}
This will produce everything form the left table (\texttt{customer}) and the overlapping data from the right hand table (\texttt{cust\_order}) where there is a match on the common attribute to both (\texttt{cust\_id})
\begin{sql}
SELECT CUSTOMER.CUST_ID, CUST_ORDER.CUST_ORD_ID FROM CUSTOMER
LEFT JOIN CUST_ORDER ON CUSTOMER.CUST_ID= CUST_ORDER.CUST_ID;
\end{sql}

\section*{Right Join}
This will return everything from the right table (\texttt{cust\_order}) and common data where it is there. 
\begin{sql}
SELECT CUSTOMER.CUST_ID, CUST_ORDER.CUST_ORD_ID FROM CUSTOMER
RIGHT JOIN CUST_ORDER ON CUSTOMER.CUST_ID= CUST_ORDER.CUST_ID;
\end{sql}
It is important to use the correct join for the situation as when used incorrectly as you won't get the data returned which you are expecting.

\section*{Outer Joins}
This gives everything from all the tables mentioned in the query.
\begin{sql}
SELECT role_name, staff_lname, staff_fname FROM staff FULL OUTER JOIN
ROLE ON ROLE=role_id;
\end{sql}
Will probably use this the least.

\section*{Things To Remember}
\begin{itemize}
    \item Use the correct type of join for the job
    \item Match like for like
\end{itemize}