\lecture{PRACTICAL: SQL and Entities}{20-10-22}{14:00}{Mark \& Team}{FTC 3rd Floor}

\section*{Task 1: Run the provided code and observe the outputs.}
Run the following DDL code.
\begin{sql}
CREATE DATABASE customer_db;

\c customer_db

CREATE TABLE customer1 (cust_id SERIAL PRIMARY KEY, cust_fname VARCHAR(20) NOT NULL, cust_lname VARCHAR(20) NOT NULL);

\d customer1

ALTER TABLE customer1 ADD COLUMN cust_email varchar(100) NOT NULL UNIQUE;

\d customer1

DROP TABLE customer1;

-- check that the table is gone now

-- anything in the line after the two dashes is a comment by the way

\l
\d customer1
\end{sql}

Run the following DML code.

Creating a new table and populating it with some dummy data.
\begin{sql}
CREATE TABLE customer (cust_id SERIAL PRIMARY KEY, cust_fname VARCHAR(20) NOT NULL, cust_lname VARCHAR(20) NOT NULL, cust_email varchar(60) NOT NULL);

INSERT INTO customer (cust_id, cust_fname, cust_lname, cust_email) VALUES (22,'Kamil', 'Novak','kamnovak@gmail.com');

INSERT INTO customer (cust_id, cust_fname, cust_lname, cust_email) VALUES (66,'Aarav', 'Anand','aanand98@gmail.com');

INSERT INTO customer (cust_id, cust_fname, cust_lname, cust_email) VALUES (67,'Alia', 'Anand','aanand98@gmail.com');
\end{sql}

Viewing what is in the table
\begin{sql}
SELECT * FROM customer;

SELECT cust_fname, cust_email from customer;
\end{sql}
Selecting only the attributes which we need, so we don't have to retrieve all of the data from a table.
\begin{sql}
SELECT cust_email, cust_id, cust_fname, cust_lname from customer;
\end{sql}

Insert more records, some of these return errors.
\begin{sql}
UPDATE customer SET cust_email = 'i_love_elephants_1@gmail.com' where cust_id = 66;
SELECT * FROM customer;

INSERT INTO customer (cust_id, cust_fname, cust_lname, cust_email) VALUES (67,'Alia', 'Anand','aanand98@gmail.com');

INSERT INTO customer (cust_id, cust_fname, cust_lname, cust_email) VALUES (70,'Connor', 'Murphy','connormurphy99@gmail.com');

INSERT INTO customer (cust_id, cust_fname, cust_lname, cust_email) VALUES (71,'Connor', 'Murphy','connormurphy199@gmail.com');

INSERT INTO customer (cust_id, cust_fname, cust_lname, cust_email) VALUES (1,'Tim', 'Nice-but-Dimm','alongemailaddresswillfit@quitealongdomainnametoo.com');

INSERT INTO customer ( cust_fname, cust_lname, cust_email) VALUES ('Tim', 'Nice-but-Dimm','averylongemailaddresswillnotfit11111111111111111111111111111111111111111a@quitealongdomainnametoo.com');
\end{sql}


\section*{Task 2: Write SQL code for the following questions.}
\begin{enumerate}
\item Create a new database called \verb|code_test|
\begin{sql}
CREATE DATABASE code_test;
\end{sql}
\item Connect to this new database
\begin{sql}
\c code_test
\end{sql}
\item Create a new table called \verb|table_one|, which has the following attributes
\begin{enumerate}
    \item \verb|Record_id| an integer
    \item \verb|Att_1| a varchar that will hold upto 30 characters
    \item \verb|Att_2| a char that will hold 10 characters
    \item \verb|Att_3| a decimal that can hold the value of 9.99.
\end{enumerate}
\begin{sql}
CREATE TABLE table_one(Record_id INT PRIMARY KEY, Att_1 VARCHAR(30), Att_2 CHAR(10), Att_3 DECIMAL(3,2));
\end{sql}
\item Look at the structure of this table once you have created it. Show the output below.
\begin{sql}
\d table_one
\end{sql}
\begin{pseudo*}
                      Table "public.table_one"
  Column   |         Type          | Collation | Nullable | Default
-----------+-----------------------+-----------+----------+---------
 record_id | integer               |           | not null |
 att_1     | character varying(30) |           |          |
 att_2     | character(10)         |           |          |
 att_3     | numeric(3,2)          |           |          |
Indexes:
    "table_one_pkey" PRIMARY KEY, btree (record_id)
\end{pseudo*}
\item Alter the table by adding a new column called \verb|Att_4| that will hold another integer.
\begin{sql}
ALTER TABLE table_one ADD COLUMN Att_4 INT;
\end{sql}
\item Look at the structure of this table again once you have added this new column. Show the output below.
\begin{sql}
\d table_one
\end{sql}
\begin{pseudo*}
                      Table "public.table_one"
  Column   |         Type          | Collation | Nullable | Default
-----------+-----------------------+-----------+----------+---------
 record_id | integer               |           | not null |
 att_1     | character varying(30) |           |          |
 att_2     | character(10)         |           |          |
 att_3     | numeric(3,2)          |           |          |
 att_4     | integer               |           |          |
Indexes:
    "table_one_pkey" PRIMARY KEY, btree (record_id)
\end{pseudo*}

\item Insert two records into the table called \verb|table_one|
\begin{enumerate}
    \item Record\_id = 1, Att\_1 = continent , Att2 = 0olP\$fguj , Att\_3 = 9.99 , Att\_4 = 42
    \item Record\_id = 2, Att\_1 = Portsmouth University , Att2 = Violet , Att\_3 = 9.99 , Att\_4 = 99999
\end{enumerate}
\begin{sql}
INSERT INTO table_one (Record_id, Att_1, Att_2, Att_3, Att_4) VALUES (1, 'continent', '0olP[dollarSign]fguj', 9.99, 42);
INSERT INTO table_one (Record_id, Att_1, Att_2, Att_3, Att_4) VALUES (2, 'Portsmouth University', 'Violet', 9.99, 9999);
\end{sql}
\item Get all fo the data from the table
\begin{sql}
SELECT * FROM table_one;
\end{sql}
\item Get a screenshot of the data
\begin{pseudo*}
 record_id |         att_1         |   att_2    | att_3 | att_4
-----------+-----------------------+------------+-------+-------
         1 | continent             | 0olP[dollarSign]fguj  |  9.99 |    42
         2 | Portsmouth University | Violet     |  9.99 |  9999
(2 rows)
\end{pseudo*}
\item Change the value of \verb|Att_4| in record 1 from \verb|44| to \verb|66|
\begin{sql}
UPDATE table_one SET Att_4 = 66 WHERE record_id = 1;
\end{sql}
\item Get the data from the table for only record 1
\begin{sql}
SELECT * FROM table_one WHERE record_id = 1;
\end{sql}
\item Get a screenshot of the results.
\begin{pseudo*}
 record_id |   att_1   |   att_2    | att_3 | att_4
-----------+-----------+------------+-------+-------
         1 | continent | 0olP[dollarSign]fguj  |  9.99 |    66
(1 row)
\end{pseudo*}
\end{enumerate}