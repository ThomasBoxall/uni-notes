\lecture{PRACTICAL: Count()}{13-10-22}{14:00}{Mark and team}{FTC Floor 3}

Q1. using the \verb|count()| function demonstrated by your tutor, how many records are there in each of the tables in the dsd\_22 database. (Remember to use \verb|\dt| to give you a list of tables in the database.) Copy the outputs below.

\begin{pseudo*}
dsd_22=# select count(*) from category;
count
-------
    6
(1 row)
dsd_22=# select count(*) from cust_order;
count
-------
150
(1 row)
dsd_22=# select count(*) from customer;
count
-------
11
(1 row)
dsd_22=# select count(*) from manifest;
count
-------
150
(1 row)

dsd_22=# select count(*) from product;
count 
-------
100
(1 row)

dsd_22=# select count(*) from role;
count
-------
    5
(1 row)

dsd_22=# select count(*) from staff;
count 
-------
10
(1 row)
\end{pseudo*}

Q2. Use the max() function to find the highest value of the role\_id attribute in the role table. Copy the output below
\begin{sql}
select max(role_id) from role;
\end{sql}
\begin{pseudo*}
max 
-----
    5
(1 row)
\end{pseudo*}

Q3. Insert a new row of data into the role table with 
\begin{sql}
INSERT INTO ROLE (ROLE_NAME) VALUES ('Pre Sales'); 
\end{sql}

Q4. How many rows of data are now in the role table? Copy it below.
\begin{sql}
select count(*) from role;
\end{sql}
\begin{pseudo*}
count
-------
    6
(1 row)
\end{pseudo*}

Q5. What is the maximum value of the role\_id now? Copy it below.
\begin{sql}
select max(role_id) from role;
\end{sql}
\begin{pseudo*}
max 
-----
    6
(1 row)
\end{pseudo*}

Q6. Delete this new row with  
\begin{sql}
DELETE FROM ROLE WHERE ROLE_NAME = 'Pre Sales'; 
\end{sql}
\begin{pseudo*}
DELETE 1
\end{pseudo*}

Q7. How many rows of data are now in the role table? Copy it below.
\begin{pseudo*}
count
-------
      5
(1 row)
\end{pseudo*}

Q8. What is the maximum value of the role\_id now? Copy it below.
\begin{pseudo*}
max
-----
    5
(1 row)
\end{pseudo*}

Q9. Reinsert the row of data into the role table again with   
\begin{sql}
INSERT INTO ROLE (ROLE_NAME) VALUES ('Cleaning Team');
\end{sql}
\begin{pseudo*}
INSERT 0 1
\end{pseudo*}

Q10. How many rows of data are now in the role table? Copy it below.
\begin{pseudo*}
count
-------
        6
(1 row)
\end{pseudo*}

Q11. What is the maximum value of the role\_id now? Copy it below.
\begin{pseudo*}
max
-----
    6
(1 row)
\end{pseudo*}

Q12. Create a random value using the random function. Copy the value below
\begin{sql}
SELECT RANDOM(); 
\end{sql}
\begin{pseudo*}
      random       
-------------------
 0.175315219908953
(1 row)
\end{pseudo*}

Q12a.  Create another random number. Copy the value below
\begin{sql}
SELECT RANDOM(); 
\end{sql}
\begin{pseudo*}
      random
-------------------
 0.272884896956384
(1 row)
\end{pseudo*}

Q13. Create one more random value but now multiply it by 11. Remember that to multiply you do not use x but use the \verb|*| symbol. Run this code 5 times and copy the values below.
\begin{pseudo*}
dsd_22=# select random()*11;
     ?column?
------------------
 9.60335403773934
(1 row)
dsd_22=# select random()*11;
     ?column?
-------------------
 0.160588529892266
(1 row)

dsd_22=# select random()*11;
     ?column?
------------------
 5.25661591161042
(1 row)

dsd_22=# select random()*11;
     ?column?
------------------
 7.78145408304408
(1 row)
dsd_22=# select random()*11;
     ?column?     
------------------
 10.1819118564017
(1 row)
\end{pseudo*}

Q14. Connect to your home database, upxxxxxxx and run the following code to create a new table and insert some random numbers into it. 
\begin{sql}
create table numb1(numb_id int primary key, ran_val decimal(17,15));

insert into numb1(numb_id, ran_val) values
(1,random()),(2,random()),(3,random()),(4,random()),(5,random()),(6,random()),(7,random()),(8,random()),(9,random()),(10,random());
\end{sql}
\begin{pseudo*}
    INSERT 0 10
\end{pseudo*}

Q14a. Check that there are 10 rows of data with \verb|SELECT COUNT(*) FROM NUMB1;|  If not, check your output for any error messages. You should get responses below except the prompt will be your student id number.
\begin{pseudo*}
 count 
-------
    10
(1 row)
\end{pseudo*}

Q15. Run a \verb|SELECT * FROM NUMB1;|  Copy the output below.
\begin{pseudo*}
 numb_id |      ran_val
---------+-------------------
       1 | 0.481754711363465
       2 | 0.020102311857045
       3 | 0.541421711910516
       4 | 0.046512784436345
       5 | 0.842869907151908
       6 | 0.137599688488990
       7 | 0.925696460530162
       8 | 0.765472991392016
       9 | 0.712954005226493
      10 | 0.161490791942924
(10 rows)
\end{pseudo*}


Q15a. Compare the values that you get with the values below. They should be different. This is because the code used inserts a fixed value, the numb\_id and a completely random value into the ran\_val attribute for each row. 
\begin{pseudo*}
test_num=# SELECT * FROM NUMB1;
 numb_id |  	ran_val
---------+-------------------
   	1 | 0.477631121408194
   	2 | 0.978080025874078
   	3 | 0.516494689509273
   	4 | 0.849129045847803
   	5 | 0.484937957022339
   	6 | 0.895700289402157
   	7 | 0.852438564877957
   	8 | 0.727535046637058
   	9 | 0.062769805546850
  	10 | 0.594313766807318
(10 rows)
\end{pseudo*}

Q16. Find the highest value of ran\_val using the max() function. Copy it below.
\begin{sql}
select max(ran_val) from numb1;
\end{sql}
\begin{pseudo*}
        max        
-------------------
 0.925696460530162
(1 row)
\end{pseudo*}

Q17. Find the lowest value of ran\_val using the min() function. Copy it below.
\begin{sql}
select min(ran_val) from numb1;
\end{sql}
\begin{pseudo*}
        min
-------------------
 0.020102311857045
(1 row)
\end{pseudo*}

Q18. What is the average value of ran\_val. Reminder: look at the basic functions document for ideas.
\begin{sql}
select avg(ran_val) from numb1;
\end{sql}

\begin{pseudo*}
          avg
------------------------
 0.46358753642998640000
(1 row)
\end{pseudo*}

Q19. What is the  current timestamp on your server? Copy it below
\begin{sql}
select now();
\end{sql}
\begin{pseudo*}
              now
-------------------------------
 2022-10-13 13:43:49.196518+00
(1 row)
\end{pseudo*}

Q20. What is the first name of the customer with the ID number of 3?
\begin{sql}
select cust_fname from customer where cust_id=3;
\end{sql}
\begin{pseudo*}
 cust_fname
------------
 Penelope
(1 row)
\end{pseudo*}

Q21. What is the category id number of the outdoor category? Copy below.
\begin{sql}
select cat_id from category where cat_name='Outdoor';
\end{sql}
\begin{pseudo*}
 cat_id
--------
      4
(1 row)
\end{pseudo*}

Q22. How many orders in the cust\_order table are for cust\_id 15? Copy below.
\begin{sql}
select count(*) from cust_order where cust_id=15;
\end{sql}
\begin{pseudo*}
 count
-------
     0
(1 row)
\end{pseudo*}

Q23. List the first and last names of the staff members who live in Portsmouth. Copy below.
\begin{sql}
select staff_fname, staff_lname from staff where town='Portsmouth';
\end{sql}
\begin{pseudo*}
 staff_fname | staff_lname
-------------+-------------
 Niel        | Welsby
 Janeva      | Gillicuddy
(2 rows)
\end{pseudo*}

Q24. What values does addr1 and addr2 have for the staff member whose id = 4?  Copy below.
\begin{sql}
    select addr1 , addr2 from staff where staff_id=4;
\end{sql}
\begin{pseudo*}
      addr1       | addr2
------------------+-------
 959 Algoma Plaza |
(1 row)
\end{pseudo*}

Q25. How many members of staff have the role value of 3?  Copy below.
\begin{sql}
select count(*) from staff where role=3;
\end{sql}
\begin{pseudo*}
 count
-------
     3
(1 row)
\end{pseudo*}

Q26. How many products are in the product category = 2?
\begin{sql}
select count(*) from product where prod_id=2;
\end{sql}
\begin{pseudo*}
 count
-------
     1
(1 row)
\end{pseudo*}