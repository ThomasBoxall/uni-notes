\lecture{Object-Oriented Programming}{2023-01-23}{15:00}{Nadim}{RB LT1}

We have been using pre-written classes lots so far The graphics.py module provides a number of classes which we can use to create simple graphical displays on GUIs. 

We will now look at creating our own classes. 

\section*{More Information On Classes}
Like variable and function names, class names have to be unique. It is regarded common practice for class names to start with a capital letter (for example \texttt{Animal}). 

There are three important components to all classes.
\begin{description}
    \item[Constructor] This instantiates an object based on the class. The constructor of a class is a function with the name \texttt{\_\_init\_\_}. 
    \item[Instance Variables] These are variables which each instance of the class (an object) will have.
    \item[Methods] These are things which the instance of a class can do. Methods are functions in a class. 
\end{description}

\section*{Classes In Python}
A class is defined using the \texttt{class} keyword in python. 

\subsection*{Constructor}
The constructor is the code which is run whenever a new instance of the class is needed. An example of a constructor is shown below. Note the parameter \texttt{self} is passed into it, this is the same as for all methods within a class and allows you to access the instance variables.
\begin{python}
class Pet:
    def __init__(self, providedName):
        self.name = providedName
\end{python}

\subsection*{Instance Variables}
To be able to access instance variables, every method within the class must have its first parameter of \texttt{self}. Other parameters (for example, a value from the user that is being stored in an object) can be passed in after \texttt{self}.

When using the methods of a class, you don't need to pass any values for the \texttt{self} parameter, this is automatically passed to the method. 

\subsection*{Methods}
A class can have as many or as few methods as required. They are defined the same way as a function would be, using the \texttt{def} keyword. 

\subsection*{Printing A Class}
The \texttt{\_\_str\_\_} method can be used to return a printable representation of the class. See below for an example (continued from the \texttt{Pet} example above).
\begin{python}
...
    def __str__(self):
        returnString = "Pets name: {}".format(self.name)
        return returnString
\end{python}
Note that you don't actually print the formatted statement in the method, you return it. 