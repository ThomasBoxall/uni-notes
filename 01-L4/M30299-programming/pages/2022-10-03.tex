\lecture{Computing with Data and Numbers}{03-10-22}{1500}{Nadim}{RB LT1}

\section*{Data and Data Types}
There is a lot of data which programs have to process. Different types of data are stored as different `Data Types', this allows them to be processed differently; an example of this is numerical data. In programming, we commonly distinguish between two different types of numerical data: integers (whole numbers, eg 55, 77, 88, -5) and fractional number (or floating point numbers, eg 4.6, 7.00956, -9.89). Words and other multi-character statements can be written within strings and truth values are stored as booleans.

All data values belong to one single data type and in some contexts, we use Class rather than data type.

\subsection*{Python Data Types}
Python has all of the common data types within it. Each of the data types have a specific keyword:
\begin{table}[H]
    \centering
    \begin{tabularx}{0.45\textwidth}{c|c|c}
        Type & Python Keyword & Example\\
        \hline
        Integer & \verb|int| & \verb|33|\\
        Fractional & \verb|float| & \verb|2.3|\\
        String & \verb|str| & \verb|"Spam"|\\
        Boolean & \verb|bool| & \verb|True|        
    \end{tabularx}
\end{table}

\subsection*{Operations on Data Types}
Data types have operations associated with them, some of these are language specific functions however the majority are universal across most programming languages.

For example, \verb|int| and \verb|float| both have the operations \verb|+|, \verb|-|, \verb|*| and \verb|/|. 

Numeric data types follow the operator precedence rules, as a human would with mathematical equations. They follow BIDMAS. Where two operators have equal precedence, the calculations are carried out from left to right.

\subsection*{Type Conversions}
It is important to be able to convert between different data types. The following example code shows the different functions.
\begin{python}
# convert 5 (int) to float, equals 5.0
floatVariable = float(5)

# convert 4.5 to int, this truncates, so will equal 4
intVariable = int(4.8)

# convert 6.8 to a string, equals "6.8"
strVariable = str(6.8)
\end{python}

It can be useful to find out what data type a variable or value is. To do this, use the \verb|type()| function as seen below.
\begin{python}
print(type(44)) # outputs <class 'int'>
print(type("Banana")) # outputs <class 'str'>
print(type(4.67)) # outputs <class 'float'>
\end{python}

\subsection*{User Input}
The \verb|input()| function, returns a string. This can be really useful if we want to do something with a string. However, if we want to do something with a number, this is less useful. We can use the \verb|float()| or \verb|int()| functions to convert into floats or integers respectively. Examples of this can be seen below.

There is another function which can be used. The \verb|eval()| function returns either a float or integer depending on the value passed into it. It can be useful in situations where the value entered by the user could be either floating point or integer.

\begin{python}
# convert to float
floatInput = float(input("Enter a float here: "))

# convert to an integer
intInput = int(input("Enter an integer here: "))
\end{python}

\section*{Arithmetic Operations}
Where an arithmetic operation involves both a float and integer, the integer is automatically converted to a float then the operation is carried out. For example, in the operation \verb|7+1.5|, the \verb|7| would be converted to \verb|7.0|. Therefore, the calculation would then be \verb|7.0 + 1.5 = 8.5|. 

\subsection*{Division}
The \verb|/| operator always performs floating point division, hence \verb|11 / 4 = 2.75|.

The \verb|//| operator  performs integer division, where it is given two integers as inputs, the result will be a truncated integer; as seen in the following example \verb|11 // 4 = 2|

The \verb|%| operator gives the remainder of an integer division, hence \verb|11 % 4 = 3|.

\subsection*{Issues with Floating Point Arithmetic}
Floating point numbers are represented within the computer using a fixed number of space (64 bits), this means that there is a limit to the range and accuracy of the number which is able to be stored.

There are some numbers, 0.1 for example, which are unable to be represented within this size limit in binary, this can lead to issues with the value of a float number after performing mathematical operations on it.

This problem is true of all programming languages that use floating point numbers. 

\section*{Python's Numeric Functions}
There are a number of useful built-in functions in Python which help with maths.

The \verb|round()| function takes a float as a parameter and returns the rounded value to the nearest int. It takes a second optional parameter which allows you to specify the number of digits after the decimal point to round to, as seen below.
\begin{python}
intRound = round(5.6) # equals 6

floatRound = round(6.3345742, 3) # equals 6.335
\end{python}

The \verb|abs()| function returns the absolute value of a number which is passed in as a parameter.

The \verb|pow()| function takes two parameters, the first being the number and the second being the power of it we want to calculate, as seen below
\begin{python}
powerTwo = pow(2, 3) # equals 8
powerThree = pow(3, 2) # equals 9
\end{python}
This function is the same as the \verb|**| operator.

\subsection*{Math Module}
Sometimes things we want to do mathematical things in Python which the base language can't do. To be able to do this, we import a library. This is some pre-written code which we can use in our programs.

To be able to use the math library, we first have to import it
\begin{python}
import math
# or alternatively, if the line above doesn't work, use line below
from math import *
\end{python}
The math module provides a number of useful things including some constants (eg, \verb|math.pi|) and mathematical functions (eg, \verb|math.sqrt()|). 