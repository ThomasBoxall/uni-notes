\chapter{Week 02}
\section{FreeCodeCamp}
\subsection{Registration Form}
\begin{html}
<!DOCTYPE html>
<html lang="en">
  <head>
    <meta charset="UTF-8">
    <title>Registration Form</title>
    <link rel="stylesheet" href="styles.css" />
  </head>
  <body>
    <h1>Registration Form</h1>
    <p>Please fill out this form with the required information</p>
    <form method="post" action='https://register-demo.freecodecamp.org'>
      <fieldset>
        <label for="first-name">Enter Your First Name: <input id="first-name" name="first-name" type="text" required /></label>
        <label for="last-name">Enter Your Last Name: <input id="last-name" name="last-name" type="text" required /></label>
        <label for="email">Enter Your Email: <input id="email" name="email" type="email" required /></label>
        <label for="new-password">Create a New Password: <input id="new-password" name="new-password" type="password" pattern="[a-z0-5]{8,}" required /></label>
      </fieldset>
      <fieldset>
        <label for="personal-account"><input id="personal-account" type="radio" name="account-type" class="inline" /> Personal Account</label>
        <label for="business-account"><input id="business-account" type="radio" name="account-type" class="inline" /> Business Account</label>
        <label for="terms-and-conditions" name="terms-and-conditions">
          <input id="terms-and-conditions" type="checkbox" required name="terms-and-conditions" class="inline" /> I accept the <a href="https://www.freecodecamp.org/news/terms-of-service/">terms and conditions</a>
        </label>
      </fieldset>
      <fieldset>
        <label for="profile-picture">Upload a profile picture: <input id="profile-picture" type="file" name="file" /></label>
        <label for="age">Input your age (years): <input id="age" type="number" name="age" min="13" max="120" /></label>
        <label for="referrer">How did you hear about us?
          <select id="referrer" name="referrer">
            <option value="">(select one)</option>
            <option value="1">freeCodeCamp News</option>
            <option value="2">freeCodeCamp YouTube Channel</option>
            <option value="3">freeCodeCamp Forum</option>
            <option value="4">Other</option>
          </select>
        </label>
        <label for="bio">Provide a bio:
          <textarea id="bio" name="bio" rows="3" cols="30" placeholder="I like coding on the beach..."></textarea>
        </label>
      </fieldset>
      <input type="submit" value="Submit" />
    </form>
  </body>
</html>
\end{html}
\begin{css}
body {
    width: 100%;
    height: 100vh;
    margin: 0;
    background-color: #1b1b32;
    color: #f5f6f7;
    font-family: Tahoma;
    font-size: 16px;
  }
  
  h1, p {
    margin: 1em auto;
    text-align: center;
  }
  
  form {
    width: 60vw;
    max-width: 500px;
    min-width: 300px;
    margin: 0 auto;
    padding-bottom: 2em;
  }
  
  fieldset {
    border: none;
    padding: 2rem 0;
    border-bottom: 3px solid #3b3b4f;
  }
  
  fieldset:last-of-type {
    border-bottom: none;
  }
  
  label {
    display: block;
    margin: 0.5rem 0;
  }
  
  input,
  textarea,
  select {
    margin: 10px 0 0 0;
    width: 100%;
    min-height: 2em;
  }
  
  input, textarea {
    background-color: #0a0a23;
    border: 1px solid #0a0a23;
    color: #ffffff;
  }
  
  .inline {
    width: unset;
    margin: 0 0.5em 0 0;
    vertical-align: middle;
  }
  
  input[type="submit"] {
    display: block;
    width: 60%;
    margin: 1em auto;
    height: 2em;
    font-size: 1.1rem;
    background-color: #3b3b4f;
    border-color: white;
    min-width: 300px;
  }
  
  input[type="file"] {
    padding: 1px 2px;
  }
  
  a{
    color: #dfdfe2
  }
  
\end{css}
To remove the scrollbar present in some browsers, add the line \texttt{margin:0;} to the \texttt{body} CSS styling.

Sometimes you will need to specify the last of a type, to do this use \texttt{p:last-of-type {  }}. This would select the last \texttt{<p>} tag and allow that to have different styling, for example a different amount of padding or a different bottom border.

The \texttt{unset} property applied to width will reset the value so that its inherited from its parent.

\subsection{Learnn The CSS Box Model By Building A Rothko Painting}

\begin{html}
<!DOCTYPE html>
<html lang="en">
  <head>
    <meta charset="UTF-8">
    <title>Rothko Painting</title>
    <link href="./styles.css" rel="stylesheet">
  </head>
  <body>
    <div class="frame">
      <div class="canvas">
        <div class="one"></div>
        <div class="two"></div>
        <div class="three"></div>
      </div>
    </div>
  </body>
</html>
\end{html}
\begin{css}
.canvas {
    width: 500px;
    height: 600px;
    background-color: #4d0f00;
    overflow: hidden;
    filter: blur(2px);
    }
    
    .frame {
    border: 50px solid black;
    width: 500px;
    padding: 50px;
    margin: 20px auto;
    }
    
    .one {
    width: 425px;
    height: 150px;
    background-color: #efb762;
    margin: 20px auto;
    box-shadow: 0 0 3px 3px #efb762;
    border-radius: 9px;
    transform: rotate(-0.6deg);
    }
    
    .two {
    width: 475px;
    height: 200px;
    background-color: #8f0401;
    margin: 0 auto 20px;
    box-shadow: 0 0 3px 3px #8f0401;
    border-radius: 8px 10px;
    transform: rotate(0.4deg);
    }
    
    .one, .two {
    filter: blur(1px);
    }
    
    .three {
    width: 91%;
    height: 28%;
    background-color: #b20403;
    margin: auto;
    filter: blur(2px);
    box-shadow: 0 0 5px 5px #b20403;
    border-radius: 30px 25px 60px 12px;
    transform: rotate(-0.2deg)
    }
\end{css}

\subsubsection{Box Model}
The content is the HTML elements which we are looking at on the page.

The padding is space which wraps around the content. It can be used to control whitespace around elements

The border encapsulates the padding, this has properties which can be used to show or hide it in various states.

The Margin is the area outside of the box and can be used to control the space between bother boxes or elements.

\subsubsection{CSS Properties}
The \texttt{filer: blur([val])} property allows a blur to be placed over a element. This gives a visual blur appearance. As the CSS styles stack on top of each other, the \texttt{blur} effects add. This means that if you had \texttt{blur(1px)} on your body and you wanted a blur value of 3px on a particular element within the body, you would need to give that element a value of \texttt{blue(2px)}.

\texttt{border-radius} can either accept a single value which it will use for all for corners or, alternatively, you can feed it 4 values where each corresponds to a different corner. Their order is: top-left, top-right, bottom-right, bottom-left.

\texttt{transform: rotate([val])} can be used to rotate an element by the specified value. Degrees can be represented with the unit \texttt{deg}.


\subsection{CSS Flexbox Photo Gallery}
\begin{html}
<!DOCTYPE html>
<html lang="en">
  <head>
    <meta charset="utf-8">
    <meta name="viewport" content="width=device-width, initial-scale=1.0">
    <title>Photo Gallery</title>
    <link rel="stylesheet" href="styles.css">
  </head>
  <body>
    <header class="header">
      <h1>css flexbox photo gallery</h1>
    </header>
    <div class="gallery">
      <img src="https://cdn.freecodecamp.org/curriculum/css-photo-gallery/1.jpg" alt="sleeping cat">
      <img src="https://cdn.freecodecamp.org/curriculum/css-photo-gallery/2.jpg" alt="cat on back">
      <img src="https://cdn.freecodecamp.org/curriculum/css-photo-gallery/3.jpg" alt="staring cat">
      <img src="https://cdn.freecodecamp.org/curriculum/css-photo-gallery/4.jpg" alt="cat in bed">
      <img src="https://cdn.freecodecamp.org/curriculum/css-photo-gallery/5.jpg" alt="white cat">
      <img src="https://cdn.freecodecamp.org/curriculum/css-photo-gallery/6.jpg" alt="two tabby kittens">
      <img src="https://cdn.freecodecamp.org/curriculum/css-photo-gallery/7.jpg" alt="cat under blanket">
      <img src="https://cdn.freecodecamp.org/curriculum/css-photo-gallery/8.jpg" alt="orange cat">
      <img src="https://cdn.freecodecamp.org/curriculum/css-photo-gallery/9.jpg" alt="black cat and white cat">
    </div>
  </body>
</html>
\end{html}
\begin{css}
* {
    box-sizing: border-box;
  }
  
  body {
    margin: 0;
    font-family: sans-serif;
    background: #f5f6f7;
  }
  
  .header {
    text-align: center;
    text-transform: uppercase;
    padding: 32px;
    background-color: #0a0a23;
    color: #fff;
    border-bottom: 4px solid #fdb347;
  }
  
  .gallery {
    display: flex;
    flex-direction: row;
    flex-wrap: wrap;
    justify-content: center;
    align-items: center;
    gap: 16px;
    max-width: 1400px;
    margin: 0 auto;
    padding: 20px 10px;
  }
  
  .gallery img {
    width: 100%;
    max-width: 350px;
    height: 300px;
    object-fit: cover;
    border-radius: 10px;
  }
  
  .gallery::after {
    content: "";
    width: 350px;
  }
\end{css}

\subsubsection{Flexbox}
\textit{Unless otherwise specified, all properties shown below are added to the container which will become the flexbox (e.g. \texttt{<div>}).}

Flexbox is a one dimensional CSS layout that can control the way items are spaced out and aligned within the container. To use it, add \texttt{display: flex;} to the element which you wish to be a flexbox.

The main axis of the flexbox is defined by the \texttt{flex-direction} property. It has 4 possible values (\texttt{row}, \texttt{row-reverse}, \texttt{column}, \texttt{column-reverse})

The \texttt{flex-wrap} property will determine how the items within the flex container will behave when the container is too smaller. \texttt{wrap} allows the items to wrap onto a new line; \texttt{no-wrap} prevents the items from wrapping and shrinks them if needed, this is the default.

The \texttt{justify-content} property determines how the items inside the flexbox align.

The \texttt{align-items} property positions the items in the flex box along the cross axis (the opposite axis to the to that specified with \texttt{flex-direction}). 

The \texttt{object-fit} property determines how images should behave. The value \texttt{cover} will specify that images should fit the specified width and height and should be cropped to fit so they can be left in their original aspect ratio. NB: this gets added to the image CSS, not the Flexbox css.

The \texttt{gap} property allows the gap sizes between the rows and columns to be specified. \texttt{gap} is shorthand for \texttt{row-gap} and \texttt{column-gap}. 

The \texttt{::after} pseudo-element creates an element that is the last child of the selected element. They can be used to add an empty element after the last.

\subsection{Typography Nutrition Label}
\begin{html}
<!DOCTYPE html>
<html lang="en">
  <head>
    <meta charset="UTF-8">
    <title>Nutrition Label</title>
    <link href="https://fonts.googleapis.com/css?family=Open+Sans:400,700,800" rel="stylesheet">
    <link href="./styles.css" rel="stylesheet">
  </head>
  <body>
    <div class="label">
      <header>
        <h1 class="bold">Nutrition Facts</h1>
        <div class="divider"></div>
        <p>8 servings per container</p>
        <p class="bold">Serving size <span class="right">2/3 cup (55g)</span></p>
      </header>
      <div class="divider lg"></div>
      <div class="calories-info">
        <p class="bold sm-text">Amount per serving</p>
        <h1>Calories <span class="right">230</span></h1>
      </div>
      <div class="divider md"></div>
      <div class="daily-value sm-text">
        <p class="right bold no-divider">% Daily Value *</p>
        <div class="divider"></div>
        <p><span class="bold">Total Fat</span> 8g <span class="bold right">10%</span></p>
        <p class="indent no-divider">Saturated Fat 1g <span class="bold right">5%</span></p>
        <div class="divider"></div>
        <p class="indent no-divider"><i>Trans</i> Fat 0g</p>
        <div class="divider"></div>
        <p><span class="bold">Cholesterol</span> 0mg <span class="right bold">0%</span></p>
        <p><span class="bold">Sodium</span> 160mg <span class="right bold">7%</span></p>
        <p><span class="bold">Total Carbohydrate</span> 37g <span class="right bold">13%</span></p>
        <p class="indent no-divider">Dietary Fiber 4g</p>
        <div class="divider"></div>
        <p class="indent no-divider">Total Sugars 12g</p>
        <div class="divider dbl-indent"></div>
        <p class="dbl-indent no-divider">Includes 10g Added Sugars <span class="right bold">20%</span>
        <div class="divider"></div>
        <p class="no-divider"><span class="bold">Protein</span> 3g</p>
        <div class="divider lg"></div>
        <p>Vitamin D 2mcg <span class="right">10%</span></p>
        <p>Calcium 260mg <span class="right">20%</span></p>
        <p>Iron 8mg <span class="right">45%</span></p>
        <p class="no-divider">Potassium 235mg <span class="right">6%</span></p>
      </div>
      <div class="divider md"></div>
      <p class="note">* The % Daily Value (DV) tells you how much a nutrient in a serving of food contributes to a daily diet. 2,000 calories a day is used for general nutrition advice.</p>
    </div>
  </body>
</html>
\end{html}
\begin{css}
* {
    box-sizing: border-box;
  }
  
  html {
    font-size: 16px;
  }
  
  body {
    font-family: 'Open Sans', sans-serif;
  }
  
  .label {
    border: 2px solid black;
    width: 270px;
    margin: 20px auto;
    padding: 0 7px;
  }
  
  header h1 {
    text-align: center;
    margin: -4px 0;
    letter-spacing: 0.15px
  }
  
  p {
    margin: 0;
  }
  
  .divider {
    border-bottom: 1px solid #888989;
    margin: 2px 0;
    clear: right;
  }
  
  .bold {
    font-weight: 800;
  }
  
  .right {
    float: right;
  }
  
  .lg {
    height: 10px;
  }
  
  .lg, .md {
    background-color: black;
    border: 0;
  }
  
  .md {
    height: 5px;
  }
  
  .sm-text {
    font-size: 0.85rem;
  }
  
  .calories-info h1 {
    margin: -5px -2px;
    overflow: hidden;
  }
  
  .calories-info span {
    font-size: 1.2em;
    margin-top: -7px;
  }
  
  .indent {
    margin-left: 1em;
  }
  
  .dbl-indent {
    margin-left: 2em;
  }
  
  .daily-value p:not(.no-divider) {
    border-bottom: 1px solid #888989;
  }
  
  .note {
    font-size: 0.6rem;
    margin: 5px 0;
    padding: 0 8px 0 8px;
    text-indent: -8px;
  }
\end{css}

\texttt{float} is used to place an element on the left or right of the container, allowing the other content (e.g. text) to wrap around it.

The \texttt{:not} pseudo-selector can be used to select all elements that do not match the given CSS rule.