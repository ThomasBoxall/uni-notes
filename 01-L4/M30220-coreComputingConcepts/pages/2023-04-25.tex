\lecture{Designing for Accessibility}{2023-04-25}{13:00}{John}{RB LT1}

\section{What is Accessibility}
\define{Accessibility}{Accessibility concerns removing the barriers that would otherwise exclude some people from using a service, product or system at all. It can be measured by the ease in which people with disabilities can perceive, understand, navigate and interact with the software technology.}
There are a number of different types of accessibility which will be covered in this lecture:
\begin{description}
    \item[Software Accessibility] means that websites, apps, tools and technologies are designed and developed so that people with disabilities can use them.
    \item[Web Accessibility] is a term used to identify the extent to which information on web pages can be successfully accessed by persons with disabilities including the aging\footnote{(W3C, 2000)}.
\end{description}

The United Nations (UN) and World Wide Web Consortium (W3C) have published declarations and guidelines on ensuring that everyone can get access to information that is delivered through software technologies. Despite these standards existing, there is still a significant shortfall of compliance. This, in part, is the result of so many different people creating software technologies and now that using off-the-shelf frameworks such as \textit{WordPress} are becoming more standardised, compliance levels are increasing. The shortfall in compliance prevents people with disabilities equal access of information. 

A major challenge is to create a good user experience for people with disabilities that is both accessible and usable.

\section{Impairments}
\define{Impairment}{Something that has an adverse effect on people's ability to carry out normal day-to-day activites.}
Impairments may prevent people from accessing web based information.

\define{Limited Impairments}{``Some people with conditions descrived would not consider themselves to have disabilities. They may, however, have limitations of sensory, physical or cognitive functioning, which can affect access to the system.'' (\textit{W3C, 2001})}

Users are very diverse and for a website to be accessible to anyone it must be able to be accessed by any user regardless of their circumstances; which may include economics, location, communication infrastructure or a disability.

\subsection{Types of Impairments}
There are lots of different types of impairments.
\begin{description}
    \item[Visual] blindness, low vision, colour blindness
    \item[Motor] cerebal palsy, Parkinson's disease, arthritis
    \item[Cognitive] dyslexia, attention deficit disorder
    \item[Auditory] sound, hearing impair audience
    \item[Speech] stutter, speech impediment
\end{description}


\section{Why is Accessibility Important?}
Use of the internet and digital systems are spreading rapidly into all areas of society. It is becoming more important to have a connection to these systems and for those with impairments they may not be able to use them. Limited accessibility reduces the internet's potential a an effective tool.

Assistive/ adaptive software will not work if webpages are incorrectly coded. 

The W3C have setup a working group (Web Accessibility Initiative - WAI) who work in conjunction with organisations around the world, and works to make the web more accessible. In 2004, an extension to the Disability Discrimination Act stated that all E-Commerce sites must be accessible.

\section{Design for All}
\textit{Design for All} is an approach for designing for accessibility and is known as \textit{universal design}. It goes beyond the design of user experience and applies to all user endeavours. The principles of universal design include
\begin{itemize}
    \item Equitable Use
    \item Flexibility in use
    \item Simple, intuitive use
    \item Perceptible information
    \item Tolerance for error
    \item Low physical effort
    \item Size and space for approach and use
\end{itemize}

\subsection{Inclusive Design}
Designing for accessibility is based on four premises.
\begin{enumerate}
    \item Varying ability is not a special condition of the few but a common characteristics of being human and we change physically and intellectually throughout out lives.
    \item If a design works well for people with disabilities, it works better for everyone.
    \item At any point in our lives, our self-esteem, identity and well-being are deeply affected by our ability to function in our physical surroundings with a sense of comfort, independence and control.
    \item Usability and aesthetics are mutually compatible. 
\end{enumerate}

\subsection{Web Accessibility}
The \textit{Web Content Accessibility Guidelines} (WCAG) can be categorized according to success criteria.
\begin{description}
    \item[Perceivable] users must be able to perceive the information being presented (it can't be invisible to all their senses).
    \item[Operable] users must be able to operate the interface (the interface cannot require interaction that a user cannot perform)
    \item[Understandable] user must be able to understand the information presented to them and the operation of the interface (the content of the operation cannot be beyond their understanding). 
    \item[Robust] users must be able to access the content as technologies advance (as technologies and user agents evolve, the content should remain accessible).
\end{description}

There are a number of Web Accessibility Checkers available online which can be used to check webpages for conformance to W3C standards. 