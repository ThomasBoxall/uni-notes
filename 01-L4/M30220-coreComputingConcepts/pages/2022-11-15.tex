\lecture{Style}{15-11-22}{13:00}{Rich \& Co}{RB LT1}

\section*{Cascading Style Sheets}
Cascading Style Sheets (CSS) have been around since about 1997. They are a W3C standard for styling HTML and take the form of text files. The files contain rules which users define. 

CSS is comprised of a number of rules.
\begin{css}
p{
    background: red;
    color: white;
    padding: 1em;
}
\end{css}
The rule above will turn the background colour to red, the text colour to white and give a padding on all sides of 1em to every \verb|p| element in the page.

\subsection*{Selectors}
There are a number of different ways in which we can define what elements in a HTML document we want to target with a given CSS rule.
\begin{itemize}
    \item \verb|p{}| will target all \verb|p| elements within the document. This is the same for any other element when the rule is written this way.
    \item \verb|*{}| will target \textit{everything} in that HTML document.
    \item \verb|#myid{}| will target the elements with the id of \verb|myid|. This is the same for any other ids used in the same way.
    \item \verb|.myclass{}| will target all the elements with the class of \verb|myclass|. This is the same for any other classes used in the same way. 
    \item \verb|classOne, classTwo| will target both \verb|classOne| and \verb|classTwo|. This is useful for when multiple components on a HTML page need styling in the same way.
\end{itemize}
