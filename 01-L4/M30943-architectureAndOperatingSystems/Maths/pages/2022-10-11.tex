\lecture{Powers and Logarithms}{11-10-22}{10:00}{Zhaojie}{Zoom}

\section*{Powers}
$2^4$ rads as 2 to the power of 4 and it means
\[2^4=2\times 2 \times 2\times 2=16\]
In the example above, 2 is the base and 4 is the index (or power).
\subsection*{Special Cases of Powers}
$x^0 = 1$ This will be true for all cases except for where $x=0$, in this case $x^0=undefined$.

$x^1 = 1$ This is true for all values of $x$.

\subsection*{Laws Of Indices}
There are three laws of indices.
\begin{enumerate}
    \item $a^n \times a^m = a^{n+m}$ (when multiplying, add the indices)
    \item $\displaystyle \frac{a^n}{a^m} = a^{n-m}$ (when dividing, subtract the indices)
    \item $\displaystyle (a^n)^m = a^{n\times m}$ (when raising one power to another, multiply the indices).
\end{enumerate}

\subsection*{Negative Powers}
With negative powers, there is a general rule
\[a^{-n} = \frac{1}{a^n}\]

\subsection*{Fractional Powers}
Where $a$ and $n$ are positive numbers, the general rule indices
\[a^{\frac{1}{n}} = \sqrt[n]{a}\]

\begin{example}{Simplify the following expression}
    \begin{align*}
        \sqrt{\frac{72a^{12}b^7c^2}{2a^2b^3c^{-10}}} &=\\
        &= \sqrt{36a^{10}b^4c^{12}}\\
        &= (36a^{10}b^4c^{12})^{\frac{1}{2}}\\
        &= 6a^5b^2c^6
    \end{align*}
\end{example}


\section*{Logarithms}
\define{Logarithm}{A logarithm determines how many times a certain number must be multiplied by itself to reach another number.}
The general rule for logarithms is shown below, this is applicable where $a>1$.
\[y=a^x\]
\[\log_{a}y = x\]
\subsection*{Base of a logarithm}
The most commonly used bases are
\begin{itemize}
    \item 10 ($\log_{10}$)
    \item 2 ($\log_2$)
    \item natural logarithm $e$ ($\log_e$ or $\ln$)
\end{itemize}


\subsection*{First Law Of Logs}
\[\log_ax + \log_ay = \log_axy\]
All bases must be the same.

\subsection*{Second Law Of Logs}
\[\log_ax - \log_ay = \log_a\frac{x}{y}\]
All bases must be the same.

\subsection*{Third Law Of Logs}
\[n\log_ax=\log_ax^n\]
This law applies of $n$ is an integer, fractional, positive or negative.


\subsection*{Example}
\begin{example}{Simplify}
    \begin{align*}
        \log_2 y -3\log_22y+2\log_24y &=\\
        &= \log_2y-\log_2(2y)^3 +\log_2(4y)^2\\
        &= \log_2y - \log_28y^3 + \log_216y^2\\
        &= \log_2 \left(\frac{y\times 16y^2}{8y^3} \right)\\
        &= \log_2 2\\
        &=1
    \end{align*}
\end{example}