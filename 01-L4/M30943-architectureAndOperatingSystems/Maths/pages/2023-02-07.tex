\lecture{Basics of Logic}{2023-02-07}{10:00}{Zhaojie}{Zoom}

Symbolic logic has provided the theoretical basis for many areas of computer science such as: digital logic circuit designs; relational database theory; programming; algorithms; and expert systems. Symbolic logic is also seen in everyday life.

\section*{Propositions}
\define{Proposition}{A statement (declarative sentence) that is either \textit{true} or \textit{false}, but not both.}
The true/ false value is called the \textit{truth value} of the proposition.

`London is the capital of the UK'; `2 + 3 = 7'; and `There are 8 days in a week' are all examples of propositions however `Come here'; `take two aspirins'; and `Do you speak German?' are not.

\subsection*{Notation}
A proposition is represented by a single letter, a proposition variable (similarly to sets and variables in algebra, etc). 

\section*{Logical Connectives}
Proposition can be combined with logical connectives to obtain compound statements (logical expressions). Three of the most important connectives are shown below.
\begin{table}[H]
    \centering
    \begin{tabular}{p{0.2\textwidth} p{0.2\textwidth} p{0.2\textwidth}}
        \textbf{Name} & \textbf{Connective} & \textbf{Symbol} \\
        \hline
        \hline
        negation & not & ¬\\
        \hline
        conjunction & and & \wedge \\
        \hline
        disjunction & or  & \vee \\
        \hline
    \end{tabular}
\end{table}

The truth value of a compound statement (for example p or q) depends only on
\begin{itemize}
    \item the truth values of the statements being combined; and
    \item on the types of connectives being used.
\end{itemize}

\subsection*{Negation}
If $p$ is a statement, then the negation of $p$ is the statement not $p$, denoted by $¬p$. 
\begin{table}[H]
    \centering
    \begin{tabular}{C{0.1\textwidth} C{0.1\textwidth}}
        \textbf{$p$} & \textbf{$¬p$}\\
        \hline
        \hline
        T & F \\
        \hline
        F & T \\
        \hline
    \end{tabular}
\end{table}

\subsection*{Conjunction}
If $p$ and $q$ are statements, the conjunction of $p$ and $q$ is the compound statement `p and q', denoted by $p \wedge q$. 
\begin{table}[H]
    \centering
    \begin{tabular}{C{0.1\textwidth} C{0.1\textwidth} C{0.1\textwidth}}
        $p$ & $q$ & $p \wedge q$\\
        \hline
        \hline
        T & T & T\\
        \hline
        T & F & F \\
        \hline
        F & T & F \\
        \hline
        F & F & F\\
        \hline
    \end{tabular}
\end{table}

\subsection*{Disjunction}
If $p$ and $q$ are statements, the disjunction of $p$ and $q$ is the compound statement `either $p$ or $q$ or both' or simply $p$ or $q$, denoted by $p \vee q$.
\begin{table}[H]
    \centering
    \begin{tabular}{C{0.1\textwidth} C{0.1\textwidth} C{0.1\textwidth}}
        $p$ & $q$ & $p \wedge q$\\
        \hline
        \hline
        T & T & T\\
        \hline
        T & F & T \\
        \hline
        F & T & T \\
        \hline
        F & F & F\\
        \hline
    \end{tabular}
\end{table}

\subsection*{Hierarchy of Evaluation}
The truth value of more complicated compound statements can be evaluated step-by-step.
\begin{enumerate}
    \item Brackets
    \item ¬
    \item \vee, \wedge (equal priority, work left to right)
\end{enumerate}

\section*{Logical Equivalence}
Two logical expressions are said to be logically equivalent, \equiv, if they have identical truth values for each possible value of their statement variables.

\section*{Laws}
These expressions can be simplified in the same way as seen in the Computer side of Architecture and Operating Systems. See those notes for a breakdown of the different methods. NB. they use different symbols than that used in this set of notes. 