\lecture{Basic Numeracy and Basic Algebra}{27-09-22}{10:20}{Zhaojie}{Zoom}

\section*{Negative Numbers}
Subtracting a negative number is equivalent to adding a positive number. This can be seen in the following example.
\begin{align*}
    2-(-5)&=\\
    2+5&=\\
    &=7
\end{align*}

The result of multiplying or dividing two numbers of the same sign is always positive.

The result of multiplying of dividing two numbers of opposing signs is always negative.

\section*{BIDMAS}
The order in which to carry out operations in complex mathematical expressions is defined by the following priority list
\begin{itemize}
    \item[1] Brackets
    \item[2] Indices
    \item[3] Division
    \item[3] Multiplication
    \item[4] Addition
    \item[4] Subtraction 
\end{itemize}

\section*{Fractions}
The names of different components of a fraction are as follows:
$ \displaystyle fraction = \frac{numerator}{denominator} = \frac{p}{q}$
\subsection*{Addition \& Subtraction of Fractions}
To add or subtract two fractions, their denominator needs to be the same. Then the addition/subtraction is performed just to the numerator. The fraction is usually then simplified.
\subsection*{Multiplication of Fractions}
To multiply two fractions together: first, multiply the numerators together then multiply the denominators together.
\subsection*{Division of Fractions}
To divide one fraction by another, multiply the first fraction by the inverse (reciprocal) of the second fraction. Simplify where necessary.
\subsection*{Simplification Of Fractions}
A fraction is in its simplest form where there are no factors other than one to both the numerator and the denominator. 

\section*{Algebra}
The use of letters in maths is called Algebra. It defines the rules of how to manipulate with symbols.

\subsection*{Addition \& Subtraction of like terms}
\define{Term}{Either a single umber or variable, or the product of several and/or variables, for example $3y$.}
\define{Constant}{A term without a symbol, for example, $2$.}

Like terms are multiples of the same variables; they can be added/ subtracted.
\begin{example}{Multi-variable simplification}
\begin{align*}
24y^2 + 7x + 12xy - 4x - 5y^2 + 3xy &=\\
19y^2 + 15xy + 3x &=
\end{align*}
\end{example}

\subsection*{Multiplication algebraic expressions}
The fundamental concept behind multiplication of terms is to multiply the numbers and multiply the variables (using the rules for multiplication of indices if possible), taking into account the sign rules where multiplying terms with different signs. 
\begin{example}{Multiplying algebra example}
    $\displaystyle (2a)(6ab^2) = 12a^2b^2$
\end{example}

\section*{Expressions}
\subsection*{Removing Brackets}
In the expression $a(b+c)$, $a$ is multiplies by all the bracketed terms to give $ab+ac$.

In the expression $(a+b) (c+d)$, $(a+b)$ is multiplied by the other pair of brackets as individual terms. Giving the answer as $ac+ad+bc+bd$

This principle along with the principle of simplifying algebra can be used to remove brackets fromm more complex expressions.
\begin{example}{Removing brackets from a more-complex expression}
\begin{align*}
    (x+6)(x-3) & =  x(x+6) + (-3)(x+6)\\
    & =  x^2 + 6x -3x - 18\\
    & =  x^2 + 3x -18
\end{align*}
\end{example}

\define{Substitution}{Where letters are replaced by actual numerical values.}

\section*{Simple Linear Equations}
Equations state that two quantities, usually one is known and one is not, are equal. We can use this information to solve the equation - to work out what the unknown quantity is.

A linear equation comes in the form of $ax+b=c$ where $a$, $b$ and $c$ are given numbers and $x$ is an unknown quantity.
\begin{example}{Solve $4x+8=0$ for $x$}
We can start by removing one of the known values, by subtracting 8 from both sides. This results in 
$4x = -8$
We can then divide both sides by 4 to get $ x = -2$, which is our solution.
\end{example}