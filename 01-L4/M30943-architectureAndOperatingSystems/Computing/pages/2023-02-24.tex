\research{WORKSHEET 14 Computer Memory Systems}{2023-02-24}{Worksheet}

\begin{enumerate}
    \item \textbf{What is computer memory?}\\
    The part of the computer where the computer stores data and instructions, can be stored either temporarily or permanently.
    \item \textbf{Why do computers need memory?}\\
    To be able to store data and instructions, either temporarily or permanently.
    \item \textbf{How is memory capacity usually measured?}\\
    In terms or words, bits or bytes (or multiples of them, eg terabytes).
    \item \textbf{What is the memory unit of transfer? What are the units?}\\
    Bits are used when referring to main memory, words are used in registers and blocks are used when moving from one memory location to another.
    \item \textbf{What does memory access mean?}\\
    The computer retrieving data from or sending data to main memory. 
    \item \textbf{What is the sequential access method in memory? Give an example.}\\
    The memory starts being read from the first location and all locations are examined until the desired location is found, this is used in tape units.
    \item \textbf{What is the direct access method in memory? Give an example.}\\
    The read/write head moves directly to the address which needs to be accessed. Disk units generally use this method.
    \item \textbf{What is the random access method in memory? Give an example.}\\
    The memory will transfer the desired word as well as some words from either side of the desired word. This is often found in main memory or cache systems.
    \item \textbf{What is the associative access method in memory? Give an example.}\\
    Hybrid between direct and random access. It searches through the entire memory and returns all the words similar to what you have requested. Often found in cache systems.
    \item \textbf{How is memory performance measured?}\\
    Through a number of metrics: access time, cycle time and transfer rate.
    \item \textbf{What is memory access time?}\\
    The time it takes to access the memory.
    \item \textbf{What is the memory cycle time?}\\
    The time it takes for one cycle of the Fetch-Decode-Execute cycle to complete
    \item \textbf{What is the transfer rate in memory?}\\
    The maximum frequency of memory access.
    \item \textbf{What are different physical types of memory?}\\
    Semiconductor, magnetic, optical and magneto-optical.
    \item \textbf{What is a volatile memory? What is non-volatile memory?}\\
    Memory is volatile when it empties when the power is cut. Memory is non-volatile when its contents stays when the power cuts. 
    \item \textbf{What is ROM? Is it volatile memory? Is it erasable?}\\
    Read Only Memory (ROM) is non-volatile and is non-erasable.
    \item \textbf{What are the relationships between cost, capacity, and access time in memory design?}\\
    As cost decreases, the capacity increases however access time increases which decreases the frequency of access of the memory.
    \item \textbf{What is a good computer memory system? How it should be in terms of a combination of different types of memory?}\\
    Smaller amounts of more expensive, fast memory (eg registers, cache) are combined with larger, cheaper slower memory (eg magnetic disk).
    \item \textbf{How do you explain memory hierarchy?}
    \begin{table}[H]
        \centering
        \begin{tabular}{C{0.3\textwidth} C{0.3\textwidth}}
            \textbf{Location} & \textbf{Memory Type}\\
            \hline
            \hline
            \multirow{3}{*}{Inboard Memory} & Registers\\
            & Cache\\
            & Main Memory\\
            \hline
            \multirow{3}{*}{Outboard Storage} & Magnetic Disk\\
            & CD-ROM/ CD-RW\\
            & DVD-RW/ DVD-RAM\\
            \hline
            Off-line storage & Magnetic tape\\
            \hline
        \end{tabular}
    \end{table}
    \item \textbf{What is cache memory? Why is it important?}\\
    Small amounts of very fast memory which is used between the CPU and registers. This allows the cache memory to contain a copy of a portion of the memory which decreases access time of the CPU getting data.
    \item \textbf{How does multi-level cache organisation work?}\\
    There are three levels of cache memory, the fastest speed of transfer is between the CPU and Level 1 cache (which is the smallest); the slowest speed of transfer is between Level 3 cache and Main Memory (L3 cache is the biggest). Level 2 cache is found in the middle of L1 and L3.
    \item \textbf{How does word and block transfer work in cache memory?}\\
    Whole blocks of data are transferred from main memory to level 3 cache. Sub blocks are transferred onto level 2. Smaller sub-parts of blocks are transferred onto level 1 where words are then transferred onto the CPU as needed. 
    \end{enumerate}