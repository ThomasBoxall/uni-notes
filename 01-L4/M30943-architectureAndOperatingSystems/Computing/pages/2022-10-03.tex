\lecture{Negative Numbers}{03-10-22}{16:00}{Farzad}{RB LT1}
In computers, subtraction is not possible. We must convert the calculation to be an addition. For example \verb|5-3| is not possible, so it becomes \verb|5+(-3)|. This means we need to be able to represent negative numbers in binary; there are three methods we can use to do this.

\section*{Sign and Magnitude}
In this method, the Most Significant Bit (MSB) is replaced to show the sign rather than a number. A 0 represents a positive number and a 1 represents a negative number. The other bits behave the same.

Converting to and from sign and magnitude binary and decimal is the same as unsigned binary.

\begin{table}[H]
    \centering
    \begin{tabularx}{0.7\textwidth}{X|XXXXXXXX}
         & $+/-$ & 64 & 32 & 16 & 8 & 4 & 2 & 1\\
         \hline
        27 & 0 & 0 & 0 & 1 & 1 & 0 & 1 & 1\\
        -27 & 1 & 0 & 0 & 1 & 1 & 0 & 1 & 1\\
        +13 & 0 & 0 & 0 & 0 & 1 & 1 & 0 & 1 \\
        -34 & 1 & 0 & 1 & 0 & 0 & 0 & 1 & 0        
    \end{tabularx}
\end{table}


\section*{1's Complement}
To convert to 1's complement, first you need to convert to unsigned binary. You then invert the bits so that 0s become 1s and 1s become 0s.

When doing a 1s complement addition, its important that any overflow bits are carried around to the least significant bit and added on there.

\begin{example}{1's Complement subtraction example}
Perform the calculation 10-6 = 1010 - 0110.\\
First, convert the second value to 1s complement = 1010 + 1001. Then draw out the addition grid and perform the addition

\begin{table}[H]
    \begin{tabularx}{0.3\textwidth}{XXXXX}
        & 1 & 0 & 1 & 0\\
        + & 1 & 0 & 0 & 1\\
        \hline
        & 0 & 0 & 1 & 1\\
        \hline
        1 &  &  &  & 
    \end{tabularx}
\end{table}
As we have an overflowing carry, we have to add this to the least significant bit of the answer.
\begin{table}[H]
    \begin{tabularx}{0.3\textwidth}{XXXXX}
        & 1 & 0 & 1 & 0\\
        + & 1 & 0 & 0 & 1\\
        \hline
        & 0 & 0 & 1 & 1\\
        + & & & & 1\\
        \hline
        & 0 & 1 & 0 & 0\\
        \hline
        & & 1 & 1 & 
    \end{tabularx}
\end{table}
And here we have our final answer, 4.
\end{example}

\section*{2's Complement}
To convert to decimal to 2's complement binary, first convert to unsigned binary. Then work from right to left, inverting the bits so that 0 becomes 1 and 1 becomes 0. However, don't flip any bits to the right of or including the first 1. All bits to the left of should be flipped.

\begin{example}{2's Complement subtraction example}
Perform the calculation 6-1 = 110-001.\\
First, convert the second value to 2's complement = 111. Then draw out the addition grids and perform the addition.
\begin{table}[H]
    \begin{tabularx}{0.3\textwidth}{XXXX}
        & 1 & 1 & 0\\
        + & 1 & 1 & 1\\
        \hline
        & 1 & 0 & 1\\
        \hline
        1 & 1 & 
    \end{tabularx}
\end{table}
We have an overflow carry, we discard this. This gives us our final answer of 5.
\end{example}