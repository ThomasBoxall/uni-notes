\lecture{Fetch Execute Cycle}{2023-02-06}{16:00}{Farzad}{RB LT1}

\section*{Computer Function}
One of a computers main functions is to execute a program (this is a pre-written set of instructions). The processors execute the instructions in two stages, the first stage is to `fetch' the instruction from memory and the second stage is to execute the instruction. Program execution is these two stages repeated until a \verb|halt| command is executed.

\section*{Fetch \& Execute Example}
This example will perform the operation \verb|2 + 3| then store the result in memory. This uses a single data register, the memory has 16-bit words of which the opcodes are 4-bits. 

As the Opcodes are 4 bits in length, there are 16 possible opcodes ($2^4$).

At the start of the example, the program counter is set to 300. 
\begin{link}{More detailed example}
The example shown below is available as a step-by-step walk though through the slides linked on Moodle.
\end{link}
\subsection*{Step 0}
The data stored in memory is represented as hexadecimal as we can store one nibble with one character.
\begin{figure}[H]
    \begin{minipage}[t]{0.45\textwidth}
        \centering
        \begin{tabular}[H]{p{0.15\linewidth} p{0.12\linewidth} p{0.12\linewidth} p{0.12\linewidth} p{0.12\linewidth}}
            \multicolumn{5}{c}{\textbf{CPU Registers}}\\
            \hline
            \hline
            PC & 3 & 0 & 0 & 0 \\
            \hline
            MAR &  &  &  &  \\
            \hline
            MBR &  &  &  &  \\
            \hline
            IR &  &  &  &  \\
            \hline
            AC &  &  &  &  \\
            \hline
        \end{tabular}
    \end{minipage}\hfill
    \begin{minipage}[t]{0.45\textwidth}
        \centering
        \begin{tabular}[H]{p{0.15\linewidth} p{0.12\linewidth} p{0.12\linewidth} p{0.12\linewidth} p{0.12\linewidth}}
            \multicolumn{5}{c}{\textbf{Main Memory}}\\
            \hline
            \hline
            300 & 1 & 9 & 4 & 0 \\
            \hline
            301 & 5 & 9 & 4 & 1 \\
            \hline
            302 & 2 & 9 & 4 & 1 \\
            \hline
            \ldots & \ldots & \ldots & \ldots & \ldots \\
            \hline
            940 & 0 & 0 & 0 & 2 \\
            \hline
            941 & 0 & 0 & 0 & 3 \\
            \hline
        \end{tabular}
    \end{minipage}\hfill
\end{figure}

\subsection*{Step 1 - Fetch I}
\begin{enumerate}
    \item The contents of the PC is copied to the MAR
    \item The contents of the memory address stored in the MAR is copied into the MBR
    \item The contents of the MBR is copied into the IR
\end{enumerate}
\begin{figure}[H]
    \begin{minipage}[t]{0.45\textwidth}
        \centering
        \begin{tabular}[H]{p{0.15\linewidth} p{0.12\linewidth} p{0.12\linewidth} p{0.12\linewidth} p{0.12\linewidth}}
            \multicolumn{5}{c}{\textbf{CPU Registers}}\\
            \hline
            \hline
            PC & 3 & 0 & 0 & 0 \\
            \hline
            MAR & 3 & 0 & 0 & 0 \\
            \hline
            MBR & 1 & 9 & 4 & 0 \\
            \hline
            IR & 1 & 9 & 4 & 0 \\
            \hline
            AC &  &  &  &  \\
            \hline
        \end{tabular}
    \end{minipage}\hfill
    \begin{minipage}[t]{0.45\textwidth}
        \centering
        \begin{tabular}[H]{p{0.15\linewidth} p{0.12\linewidth} p{0.12\linewidth} p{0.12\linewidth} p{0.12\linewidth}}
            \multicolumn{5}{c}{\textbf{Main Memory}}\\
            \hline
            \hline
            300 & 1 & 9 & 4 & 0 \\
            \hline
            301 & 5 & 9 & 4 & 1 \\
            \hline
            302 & 2 & 9 & 4 & 1 \\
            \hline
            \ldots & \ldots & \ldots & \ldots & \ldots \\
            \hline
            940 & 0 & 0 & 0 & 3 \\
            \hline
            941 & 0 & 0 & 0 & 2 \\
            \hline
        \end{tabular}
    \end{minipage}\hfill
\end{figure}

\subsection*{Step 2 - Execute I}
\begin{enumerate}
    \item The PC is incremented
    \item The opcode is analysed and the instruction is performed. In this case, the opcode is \verb|1| which means \verb|load|. The address is \verb|940| so the contents of memory address 940 is copied into the AC. 
\end{enumerate}
\begin{figure}[H]
    \begin{minipage}[t]{0.45\textwidth}
        \centering
        \begin{tabular}[H]{p{0.15\linewidth} p{0.12\linewidth} p{0.12\linewidth} p{0.12\linewidth} p{0.12\linewidth}}
            \multicolumn{5}{c}{\textbf{CPU Registers}}\\
            \hline
            \hline
            PC & 3 & 0 & 0 & 1 \\
            \hline
            MAR & 3 & 0 & 0 & 0 \\
            \hline
            MBR & 1 & 9 & 4 & 0 \\
            \hline
            IR & 1 & 9 & 4 & 0 \\
            \hline
            AC & 0 & 0 & 0 & 3 \\
            \hline
        \end{tabular}
    \end{minipage}\hfill
    \begin{minipage}[t]{0.45\textwidth}
        \centering
        \begin{tabular}[H]{p{0.15\linewidth} p{0.12\linewidth} p{0.12\linewidth} p{0.12\linewidth} p{0.12\linewidth}}
            \multicolumn{5}{c}{\textbf{Main Memory}}\\
            \hline
            \hline
            300 & 1 & 9 & 4 & 0 \\
            \hline
            301 & 5 & 9 & 4 & 1 \\
            \hline
            302 & 2 & 9 & 4 & 1 \\
            \hline
            \ldots & \ldots & \ldots & \ldots & \ldots \\
            \hline
            940 & 0 & 0 & 0 & 3 \\
            \hline
            941 & 0 & 0 & 0 & 2 \\
            \hline
        \end{tabular}
    \end{minipage}\hfill
\end{figure}

\subsection*{Step 3 - Fetch II}
\begin{enumerate}
    \item The contents of the PC is copied to the MAR
    \item The contents of the memory address stored in the MAR is copied into the MBR
    \item The contents of the MBR is copied into the IR
\end{enumerate}

\begin{figure}[H]
    \begin{minipage}[t]{0.45\textwidth}
        \centering
        \begin{tabular}[H]{p{0.15\linewidth} p{0.12\linewidth} p{0.12\linewidth} p{0.12\linewidth} p{0.12\linewidth}}
            \multicolumn{5}{c}{\textbf{CPU Registers}}\\
            \hline
            \hline
            PC & 3 & 0 & 0 & 1 \\
            \hline
            MAR & 3 & 0 & 0 & 1 \\
            \hline
            MBR & 5 & 9 & 4 & 1 \\
            \hline
            IR & 5 & 9 & 4 & 1 \\
            \hline
            AC & 0 & 0 & 0 & 3 \\
            \hline
        \end{tabular}
    \end{minipage}\hfill
    \begin{minipage}[t]{0.45\textwidth}
        \centering
        \begin{tabular}[H]{p{0.15\linewidth} p{0.12\linewidth} p{0.12\linewidth} p{0.12\linewidth} p{0.12\linewidth}}
            \multicolumn{5}{c}{\textbf{Main Memory}}\\
            \hline
            \hline
            300 & 1 & 9 & 4 & 0 \\
            \hline
            301 & 5 & 9 & 4 & 1 \\
            \hline
            302 & 2 & 9 & 4 & 1 \\
            \hline
            \ldots & \ldots & \ldots & \ldots & \ldots \\
            \hline
            940 & 0 & 0 & 0 & 3 \\
            \hline
            941 & 0 & 0 & 0 & 2 \\
            \hline
        \end{tabular}
    \end{minipage}\hfill
\end{figure}

\subsection*{Step 4 - Execute II}
\begin{enumerate}
    \item The PC is incremented
    \item The opcode is analysed and the instruction is performed. In this case, the opcode is \verb|5| which means \verb|add to AC from memory|. The address is \verb|941| so the contents of memory address 941 is added to the contents of the AC. 
\end{enumerate}

\begin{figure}[H]
    \begin{minipage}[t]{0.45\textwidth}
        \centering
        \begin{tabular}[H]{p{0.15\linewidth} p{0.12\linewidth} p{0.12\linewidth} p{0.12\linewidth} p{0.12\linewidth}}
            \multicolumn{5}{c}{\textbf{CPU Registers}}\\
            \hline
            \hline
            PC & 3 & 0 & 0 & 2 \\
            \hline
            MAR & 3 & 0 & 0 & 1 \\
            \hline
            MBR & 5 & 9 & 4 & 1 \\
            \hline
            IR & 5 & 9 & 4 & 1 \\
            \hline
            AC & 0 & 0 & 0 & 5 \\
            \hline
        \end{tabular}
    \end{minipage}\hfill
    \begin{minipage}[t]{0.45\textwidth}
        \centering
        \begin{tabular}[H]{p{0.15\linewidth} p{0.12\linewidth} p{0.12\linewidth} p{0.12\linewidth} p{0.12\linewidth}}
            \multicolumn{5}{c}{\textbf{Main Memory}}\\
            \hline
            \hline
            300 & 1 & 9 & 4 & 0 \\
            \hline
            301 & 5 & 9 & 4 & 1 \\
            \hline
            302 & 2 & 9 & 4 & 1 \\
            \hline
            \ldots & \ldots & \ldots & \ldots & \ldots \\
            \hline
            940 & 0 & 0 & 0 & 3 \\
            \hline
            941 & 0 & 0 & 0 & 2 \\
            \hline
        \end{tabular}
    \end{minipage}\hfill
\end{figure}

\subsection*{Step 5 - Fetch III}
\begin{enumerate}
    \item The contents of the PC is copied to the MAR
    \item The contents of the memory address stored in the MAR is copied into the MBR
    \item The contents of the MBR is copied into the IR
\end{enumerate}

\begin{figure}[H]
    \begin{minipage}[t]{0.45\textwidth}
        \centering
        \begin{tabular}[H]{p{0.15\linewidth} p{0.12\linewidth} p{0.12\linewidth} p{0.12\linewidth} p{0.12\linewidth}}
            \multicolumn{5}{c}{\textbf{CPU Registers}}\\
            \hline
            \hline
            PC & 3 & 0 & 0 & 2 \\
            \hline
            MAR & 3 & 0 & 0 & 2 \\
            \hline
            MBR & 2 & 9 & 4 & 1 \\
            \hline
            IR & 2 & 9 & 4 & 1 \\
            \hline
            AC & 0 & 0 & 0 & 5 \\
            \hline
        \end{tabular}
    \end{minipage}\hfill
    \begin{minipage}[t]{0.45\textwidth}
        \centering
        \begin{tabular}[H]{p{0.15\linewidth} p{0.12\linewidth} p{0.12\linewidth} p{0.12\linewidth} p{0.12\linewidth}}
            \multicolumn{5}{c}{\textbf{Main Memory}}\\
            \hline
            \hline
            300 & 1 & 9 & 4 & 0 \\
            \hline
            301 & 5 & 9 & 4 & 1 \\
            \hline
            302 & 2 & 9 & 4 & 1 \\
            \hline
            \ldots & \ldots & \ldots & \ldots & \ldots \\
            \hline
            940 & 0 & 0 & 0 & 3 \\
            \hline
            941 & 0 & 0 & 0 & 2 \\
            \hline
        \end{tabular}
    \end{minipage}\hfill
\end{figure}

\subsection*{Step 6 - Execute III}
\begin{enumerate}
    \item The PC is incremented
    \item The opcode is analysed and the instruction is performed. In this case, the opcode is \verb|2| which means \verb|store to AC to memory|. The address is \verb|941| so the contents of the AC is written to memory address 941. 
\end{enumerate}

\begin{figure}[H]
    \begin{minipage}[t]{0.45\textwidth}
        \centering
        \begin{tabular}[H]{p{0.15\linewidth} p{0.12\linewidth} p{0.12\linewidth} p{0.12\linewidth} p{0.12\linewidth}}
            \multicolumn{5}{c}{\textbf{CPU Registers}}\\
            \hline
            \hline
            PC & 3 & 0 & 0 & 3 \\
            \hline
            MAR & 3 & 0 & 0 & 2 \\
            \hline
            MBR & 2 & 9 & 4 & 1 \\
            \hline
            IR & 2 & 9 & 4 & 1 \\
            \hline
            AC & 0 & 0 & 0 & 5 \\
            \hline
        \end{tabular}
    \end{minipage}\hfill
    \begin{minipage}[t]{0.45\textwidth}
        \centering
        \begin{tabular}[H]{p{0.15\linewidth} p{0.12\linewidth} p{0.12\linewidth} p{0.12\linewidth} p{0.12\linewidth}}
            \multicolumn{5}{c}{\textbf{Main Memory}}\\
            \hline
            \hline
            300 & 1 & 9 & 4 & 0 \\
            \hline
            301 & 5 & 9 & 4 & 1 \\
            \hline
            302 & 2 & 9 & 4 & 1 \\
            \hline
            \ldots & \ldots & \ldots & \ldots & \ldots \\
            \hline
            940 & 0 & 0 & 0 & 3 \\
            \hline
            941 & 0 & 0 & 0 & 5 \\
            \hline
        \end{tabular}
    \end{minipage}\hfill
\end{figure}

This cycle will then continue on to the next instruction and so-on.