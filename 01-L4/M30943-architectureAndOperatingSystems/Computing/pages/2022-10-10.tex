\lecture{Digital Gates}{10-10-22}{16:00}{Farzad}{RB LT1}

Digital gates are used as the building block of digital systems. They manipulate inputs to provide outputs. 

Throughout this lecture, its important to remember that a signal of 1 represents on (or high voltage) and a signal of 0 represents off (or low voltage). 

Digital gates can be represented by a circuit symbol. Their inputs and outputs can be mapped onto a Truth Table and they have symbols which can be used in expressions to describe the circuit.

\section*{NOT Gaate}
This can also be called an inverter.

As the name inverter suggests, the NOT gate inverts the bits. This means a 0 inputted, will output a 1 and a 1 inputted will output a 0.

NOT gate can only have one input.

\begin{figure}[H]
    \begin{minipage}[t]{0.45\textwidth}
        \centering
        \begin{circuit}
            \node[not port] (not) at (0,0){};
            \draw (not.out) -- ++(0.5,0) node[right]{Q};
            \draw (not.in) -- ++(-0.5,0) node[left]{A};
        \end{circuit}
        \[Q=\overline{A}\]
        \[Q=A'\]
    \end{minipage}\hfill
    \begin{minipage}[H]{0.45\textwidth}
        \centering
        \begin{table}[H]
            \centering
            \begin{tabularx}{0.2\textwidth}{X|X}
                A & Q\\
                \hline
                0 & 1\\
                1 & 0\\
            \end{tabularx}
        \end{table}
        Truth table for the NOT gate.
    \end{minipage}\hfill
\end{figure}


\section*{AND Gate}
This gate compares two or more inputs. If all its inputs are 1, then it outputs 1; otherwise it outputs 0.

The rules of binary multiplication are the same of the AND gate.

\begin{figure}[H]
    \begin{minipage}[t]{0.45\textwidth}
        \centering
        \begin{circuit}
            \node[and port] (and) at (0,0){};
            \draw (and.out) -- ++(0.5,0) node[right]{Q};
            \draw (and.in 1) -- ++(-0.5,0) node[left]{A};
            \draw (and.in 2) -- ++(-0.5,0) node[left]{B};
        \end{circuit}
        \[Q = A\bigwedge B\]
        \[Q = A \cdot B\]
        \[Q=AB\]
    \end{minipage}\hfill
    \begin{minipage}[H]{0.45\textwidth}
        \centering
        \begin{table}[H]
            \centering
            \begin{tabularx}{0.3\textwidth}{XX|X}
                B & A & Q\\
                \hline
                0 & 0 & 0\\
                0 & 1 & 0\\
                1 & 0 & 0 \\
                1 & 1 & 1\\
            \end{tabularx}
        \end{table}
        Truth table for the AND gate.
    \end{minipage}\hfill
\end{figure}

\section*{OR Gate}
This gate compares two or more inputs. If one or more input is 1, then it outputs 1; otherwise it outputs 0.

\begin{figure}[H]
    \begin{minipage}[t]{0.45\textwidth}
        \centering
        \begin{circuit}
            \node[or port] (or) at (0,0){};
            \draw (or.out) -- ++(0.5,0) node[right]{Q};
            \draw (or.in 1) -- ++(-0.5,0) node[left]{A};
            \draw (or.in 2) -- ++(-0.5,0) node[left]{B};
        \end{circuit}
        \[Q = A\bigvee B\]
        \[Q = A + B\]
    \end{minipage}\hfill
    \begin{minipage}[H]{0.45\textwidth}
        \centering
        \begin{table}[H]
            \centering
            \begin{tabularx}{0.3\textwidth}{XX|X}
                B & A & Q\\
                \hline
                0 & 0 & 0\\
                0 & 1 & 1\\
                1 & 0 & 1 \\
                1 & 1 & 1\\
            \end{tabularx}
        \end{table}
        Truth table for the OR gate.
    \end{minipage}\hfill
\end{figure}

\section*{NAND Gate}
\textit{Not AND}\\
This gate compares two or more inputs. It outputs 1 where at least one of the inputs are not 1 and 0 where all the inputs are 1.

Through combinations of this gate, all the other logic gates can be made, due to this, it is called a Universal Gate.

\begin{figure}[H]
    \begin{minipage}[t]{0.45\textwidth}
        \centering
        \begin{circuit}
            \node[nand port] (nand) at (0,0){};
            \draw (nand.out) -- ++(0.5,0) node[right]{Q};
            \draw (nand.in 1) -- ++(-0.5,0) node[left]{A};
            \draw (nand.in 2) -- ++(-0.5,0) node[left]{B};
        \end{circuit}
        \[Q = \overline{AB} = (AB)'\]
        \[Q = \overline{A\bigwedge B} = (A\bigwedge B)'\]
        \[Q = \overline{A\cdot B} = (A\cdot B)'\]

    \end{minipage}\hfill
    \begin{minipage}[H]{0.45\textwidth}
        \centering
        \begin{table}[H]
            \centering
            \begin{tabularx}{0.3\textwidth}{XX|X}
                B & A & Q\\
                \hline
                0 & 0 & 1\\
                0 & 1 & 1\\
                1 & 0 & 1 \\
                1 & 1 & 0\\
            \end{tabularx}
        \end{table}
        Truth table for the NAND gate.
    \end{minipage}\hfill
\end{figure}

\section*{NOR Gate}
\textit{Not OR}\\
This gate compares two or more inputs. Where all the inputs are 0, it outputs 1; otherwise it outputs 0. 

Through combinations of this gate, all other logic gates can be constructed, due to this it can be called a Universal Gate.

\begin{figure}[H]
    \begin{minipage}[t]{0.45\textwidth}
        \centering
        \begin{circuit}
            \node[nor port] (nor) at (0,0){};
            \draw (nor.out) -- ++(0.5,0) node[right]{Q};
            \draw (nor.in 1) -- ++(-0.5,0) node[left]{A};
            \draw (nor.in 2) -- ++(-0.5,0) node[left]{B};
        \end{circuit}
        \[Q = \overline{A+B} = (A+B)'\]
        \[Q = \overline{A\bigvee B} = (A\bigvee B)'\]
    \end{minipage}\hfill
    \begin{minipage}[H]{0.45\textwidth}
        \centering
        \begin{table}[H]
            \centering
            \begin{tabularx}{0.3\textwidth}{XX|X}
                B & A & Q\\
                \hline
                0 & 0 & 1\\
                0 & 1 & 0\\
                1 & 0 & 0 \\
                1 & 1 & 0\\
            \end{tabularx}
        \end{table}
        Truth table for the NOR gate.
    \end{minipage}\hfill
\end{figure}

\section*{XOR Gate}
\textit{eXclusive OR}\\
This gate compares two or more inputs. For a 2-input XOR gate, where the two inputs are different, it outputs 1; otherwise it outputs 0.

This gate is used for adder circuits.
\begin{figure}[H]
    \begin{minipage}[t]{0.45\textwidth}
        \centering
        \begin{circuit}
            \node[xor port] (xor) at (0,0){};
            \draw (xor.out) -- ++(0.5,0) node[right]{Q};
            \draw (xor.in 1) -- ++(-0.5,0) node[left]{A};
            \draw (xor.in 2) -- ++(-0.5,0) node[left]{B};
        \end{circuit}
        \[Q=A \oplus B\]
        \[Q = AB' + A'B\]
        \[Q = A\cdot \overline{B} + \overline{A} \cdot B\]
    \end{minipage}\hfill
    \begin{minipage}[H]{0.45\textwidth}
        \centering
        \begin{table}[H]
            \centering
            \begin{tabularx}{0.3\textwidth}{XX|X}
                B & A & Q\\
                \hline
                0 & 0 & 0\\
                0 & 1 & 1\\
                1 & 0 & 1 \\
                1 & 1 & 0\\
            \end{tabularx}
        \end{table}
        Truth table for the XOR gate.
    \end{minipage}\hfill
\end{figure}

\section*{XNOR Gate}
\textit{eXclusive Not OR}\\
This gate compares two or more inputs and where the inputs are the same, it outputs 1 and where the inputs are different, it outputs 0.

\begin{figure}[H]
    \begin{minipage}[t]{0.45\textwidth}
        \centering
        \begin{circuit}
            \node[xnor port] (xnor) at (0,0){};
            \draw (xnor.out) -- ++(0.5,0) node[right]{Q};
            \draw (xnor.in 1) -- ++(-0.5,0) node[left]{A};
            \draw (xnor.in 2) -- ++(-0.5,0) node[left]{B};
        \end{circuit}
        \[Q=\overline{A \oplus B}\]
        \[Q = \left(AB' + A'B\right)'\]
    \end{minipage}\hfill
    \begin{minipage}[H]{0.45\textwidth}
        \centering
        \begin{table}[H]
            \centering
            \begin{tabularx}{0.3\textwidth}{XX|X}
                B & A & Q\\
                \hline
                0 & 0 & 1\\
                0 & 1 & 0\\
                1 & 0 & 0 \\
                1 & 1 & 1\\
            \end{tabularx}
        \end{table}
        Truth table for the XNOR gate.
    \end{minipage}\hfill
\end{figure}

\section*{Practice Questions}
\subsection*{Question 1}
Find the value of X where the inputs have the following values.
\[A=0\ \ B=1\ \ C=1\]
\begin{circuit}
    \node[and port] (and) at (0,2){};
    \node[or port] (or) at (3,1){};
    \node[not port] (not) at (5,1){};

    \draw(not.out) -- ++(0.5,0) node[right]{X};
    \draw(or.out) -| (not.in);
    \draw (and.out) -| (or.in 1);
    \draw(and.in 1) -- ++(-0.5,0) node[left]{C};
    \draw(and.in 2) -- ++(-0.5,0) node[left]{B};
    \draw(or.in 2) -- ++(-3.5,0) node[left]{A};
\end{circuit}

X=0

\subsection*{Question 2}
Find the value of B where the logic system is as follows and has the following values.
\[X=0 \ \ A = 0\]
\begin{circuit}
    \node[not port] (not1) at (1,2){};
    \node[and port] (and) at (3,1){};
    \node[not port] (not2) at (5,1){};

    \draw (not1.in) -- (0,2) node[left]{A};
    \draw (not1.out) -| (and.in 1);
    \draw (and.in 2) -- (0,0.75) node[left]{B};
    \draw (and.out) -- (not2.in);
    \draw(not2.out) -- ++(0.5,0) node[right] {X};
\end{circuit}\\
B=1

\subsection*{Question 3}
Convert the following boolean logic expression to a circuit diagram.
\[\overline{A\cdot B + C}\]

\begin{circuit}
    \node[and port] (and) at (1,2){};
    \node[or port] (or) at (4,1){};
    \node[not port] (not) at (6,1){};

    \draw(not.out) -- ++(0.5,0) node[right]{Q};
    \draw(or.out) -| (not.in);
    \draw(and.out) -| (or.in 1);
    \draw(or.in 2) -- ++(-3.5,0) node[left]{C};
    \draw(and.in 1) -- ++(-0.5,0) node[left]{A};
    \draw(and.in 2) -- ++(-0.5,0) node[left]{B};
\end{circuit}