\lecture{LECTURE: Introduction To Module}{27-09-2022}{09:00}{Amanda}{Zoom}

\section*{Module Overview}
The module coordinator for this module is Amanda. Thanos also teaches on the module. Taiwo and Uchenna are practical tutors. Amanda and Thanos' offices can be found on the first floor of BK building. Taiwo and Uchenna can be found on the ground floor of BK building. In general, they all operate an open door policy. 

The module runs through both teaching blocks. 

The practical sessions are held in Portland 2.27. This is on the second floor, in the far right hand corner of the building. If you are going to this for the first time, its advised to allow extra time to find the room.

This module covers the fundamental building blocks of computer networks. It introduces computer networks, focusing on: data connections; current and legacy technologies; network protocols; computer network terminology.

There is a lot of terminology used in this module, some students have found it helpful to create a glossary. 

\section*{Module Learning Outcomes}
\begin{enumerate}
    \item Recognize computer systems network terminology and use it appropriately. \textit{(Terminology will be used in every lecture, the key to this outcome is the appropriate use of the terminology.)}
    \item Define the fundamental principles of computer networking topologies and professional standards, utilizing simulation software. \textit{(This will encompass the IEEE standard. This term, we will use simulation software to build networks and see how they work.)}
    \item Describe the 7-layer OSI model and discuss its application.
    \item Describe the fundamental operational aspects of Network Protocol Architecture. \textit{(For the most part, networks are plug and play however they have lots of software and protocols that interface with different components to allow them to communicate with each other. Lots of this module is about the protocols and how they interface which makes things work. The end goal for networking is that the user has a seamless experience when using technology.)}
    \item Examine the fundamental requirements of systems management. \textit{(Management and maintenance of a network is often overlooked. Networks have to be seamless but also available 99.9\% of the time, this limited downtime is the responsibility of the network administrators.)}
    \item Identify network security and the impact of network vulnerabilities. \textit{(Looking at how networks are secured, this is the fundamentals only.)}
\end{enumerate}

\section*{Assessments}
There are three components to the assessment for this module.
\subsection*{Exam 1}
This will be a computer based, 45 minute exam held in the January 2023 exam period. It wil be closed book and have a variety of question styles. It will examine content taught in teaching block 1 and will be worth 30\%. There will be revision sessions and revision questions made available closer to the time.
\subsection*{Coursework}
This will be completed during teaching block 2 as part of a group. It will be in the area of Network Design and specification. The basic premise is that a group works together to cerate a company and deliver a pitch for a contract in a Dragons Den style presentation. This is worth 50\%. 
\subsection*{Exam 2}
This will be a computer based, 60 minute exam held in the May/June 2023 exam period. It will be closed book and have a variety of question styles. As with exam 1, it will be worth 30\% and revision questions and revision sessions will be made available closer to the time. 

\section*{Hours}
The lectures will be delivered online, most will be live with some pre-recorded. For live Zoom lectures, attendance is automatically recorded through Zoom.

The practical sessions will be held in Portland 2.27, in groups of about 20 people.

Outside of timetabled sessions, you should spend about 6 to 7 hours working on this module (university expects about 200 hours per 20 credit module). If you have lots of experience in this subject, then it may not need to be this much however if you are new to the subject, they you may require longer.

There will be quizzes provided throughout the year to test knowledge. 

\section*{Resources}
There are a number of options for the textbooks, each with varying degrees of detail. 
\begin{itemize}
    \item Stallings, W., 2013, Data and Computer Communications 10th Ed, Pearson Prentice-Hall (ISBN: 1292014385) - this covers all of the first year networking module and some of the second year networks module.
    \item Tanenbaum, A., 2010, Computer Networks 5th Ed, Upper Saddle River NJ, Prentice Hall (ISBN: 0132553171) - this covers all modules until the final year networking module, it can be hard to read.
    \item Kurose and Ross, 2011, 5thEd Computer Networking: A Top-Down Approach: International Edition (ISBN 978-0131365483) - this covers all modules until the final year module, it has a looser style making it easier to read than Tanenbaum.
    \item West, Dean and Andrews 2019, Network+ Guide to Networks - this covers the first year module only and is quite easy to read.
    \item Peterson and Davie, 2011, Computer Networks 5TH ED (ISBN: 0123851386) - this can be quite technical and covers quite a lot of the three years.
\end{itemize}
All the books listed above are available in the university library. It is recommended to have a look through them before purchasing so that you get the one which works for you. 

If using Google to find information, be sure to use a reputable source. 

We will be directed to internet resources when we need. If we are really keen, could do Linkedin Learning Courses. Any Cisco accreditation already completed are useful however there will be a difference in some terminology between Cisco and this course - we will be taught generic terms, Cisco uses Cisco-specific terms.