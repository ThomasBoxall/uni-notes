\lecture{A system of learning by thinking}{03-10-22}{14:00}{Nadim}{RB LT1}

This lecture introduces a system of learning by thinking (meaning construction), which Nadim has used for a long while and he feels it helps him to study and understand a variety of subjects. Most of these subjects are not computing related subjects.

The system is outlined below
\begin{enumerate}
    \item Start with the first thing which you need to learn about the subject the find as much information on this which comes from different perspectives as possible. It is a good idea to start with introductory texts to the subjects (if they are good, then they will cover the key concepts well) so that you can build a framework. This framework can then be filled in at a later date.
    \item Don't leave the first piece of knowledge until you are sure what is going on with it and understand it. Part of the understanding process can involve adding more knowledge and application of the knowledge to a number of different settings and to be able to vary it. You will initially get everything wrong but over time this improves and you will gain an understanding. If you do not begin to understand, return to the beginning and start again with the same content.
    \item Start the process again with different content to learn.
\end{enumerate}