\lecture{Exams}{12-12-22}{14:00}{Nadim}{RB LT1}

Most of the exams we will sit will be computer based, however there are still some paper based.

Computer based tests (CBT) will be done through Moodle. The order of questions is randomised for each candidate.

There are multiple types of question in the paper, easy and difficult. Approximately 1/3 of the paper should be easy, 1/3 should be in the middle and 1/3 should be more difficult. Difficult questions will often yield more marks.

\section*{How To Sequence CBT Questions}
It is recommended to take a Two Pass System approach to exams. There are two approaches which can be taken.
\subsection*{Approach 1}
This approach picks out the difficult questions and completes them first then completes the remaining questions. The remaining questions should be the easier questions. An issue with this is timing; in that if you stick on the difficult questions, you will run out of time.
\subsection*{Approach 2}
This approach skims through the paper and answers the easy questions first. Then return to the start and complete the difficult questions. 
\subsection*{Approach 3}
This approach takes a equally divided time approach. In that the time allocated for the paper is divided between the number of questions equally and after the `timer' runs out for a questions, you move onto the next question.
\subsection*{Which Approach to Take?}
Knowing about these different approaches raises the question of which approach to take. This first year can be used as a testing ground to see which approach works well for an individual person. 

\section*{Revision}
Revision strategies are similar to exam strategies.

The most consistently helpful revision strategy is the backward planning discussed last week.

There are a number of different methods which can be used to revise content. 
\subsection*{Method 1}
This method revises the difficult subject first then revise the simple content last. 
\subsection*{Method 2}
This method revises the easy subject first then the difficult content last.
\subsection*{Method 3}
Generate test questions yourself. Then use those questions to test yourself. This method is particularly useful at University as there is a lack of questions available compared to A-Levels. 
\subsection*{Method 4}
Get Magic Whiteboard Paper which you can tear sheets off and stick on walls. Then draw on this with things you need to learn. Don't pay attention consciously to it then. You will subconsciously absorb the information which is on them. 