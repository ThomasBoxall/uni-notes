\lecture{PRACTICAL: Encryption}{2023-02-09}{14:00}{}{FTC 3}

Normally when we use encryption within a database, we pass the responsibility of encrypting the data to the front end service. This is to prevent the \textit{encryption seed} from being visible within the database logs where a `super-super admin' can see the insert statements and see the unencrypted data get inserted. 

\section*{Tutor Task}
Copy and run the following code.
\begin{sql}
-- create a new db for demo

create database secdb;

\c secdb
-- we can't copy and paste this next line of code at the same time as previous 2 lines!

-- Turn on encryption - It is not on by default.
CREATE EXTENSION IF NOT EXISTS pgcrypto;

-- bytea is a binary datatype
-- https://www.postgresql.org/docs/current/datatype-binary.html

CREATE TABLE secDemo(id serial PRIMARY KEY, pw bytea);

 -- insert into secdemo(pw) values ( encrypt( 'data', 'key', 'aes') );

INSERT INTO secdemo(pw)
VALUES (encrypt('Holiday!lips@', '56732', 'aes'));

select * from secdemo;

-- select decrypt(pw, 'key', 'aes') FROM secdemo;

select decrypt(pw, '56732', 'aes') as "decrypted version" FROM secdemo;

-- still bytea at this point

-- select convert_from(decrypt(pw, 'key', 'aes'), 'utf-8') FROM secdemo;
-- convert_from() converts from bytea to text

select convert_from(decrypt(pw, '56732', 'aes'), 'utf-8') as "converted from decrypted" FROM secdemo;
\end{sql}

\section*{Student Tasks}
T1. Make sure you are up to date with the practicals!

T2. Create a new database called \verb|sec3|
\begin{sql}
CREATE DATABASE sec3;
\c sec3
\end{sql}

T3. Turn encryption on in this new database.
\begin{sql}
CREATE EXTENSION IF NOT EXISTS pgcrypto;
\end{sql}

T4. Using the Tutor Task above, create a new table called \verb|member| and add 5 rows of data. This table must hold first and last names along with the member's date of birth, (stored as a \verb|date| datatype), a postcode and an encrypted password. Copy the code and data inserted below:
\begin{sql}
CREATE TABLE member(
    id serial PRIMARY KEY,
    fname VARCHAR(25) NOT NULL,
    lname VARCHAR(25) NOT NULL,
    dob date NOT NULL,
    postcode VARCHAR(8) NOT NULL,
    password bytea
);
-- insert values now
INSERT INTO member(fname, lname, dob, postcode, password) VALUES ('Dave', 'Davidson', '2022-01-01', 'NE1 4EQ', encrypt('cheese123', '1234', 'aes'));
INSERT INTO member(fname, lname, dob, postcode, password) VALUES ('Fred', 'Fredrikson', '2021-03-6', 'AB12 CDE', encrypt('mouse33', '1234', 'aes'));
INSERT INTO member(fname, lname, dob, postcode, password) VALUES ('Sue', 'Susan', '1972-05-02', 'BN35 7DQ', encrypt('secrue68', '1234', 'aes'));
INSERT INTO member(fname, lname, dob, postcode, password) VALUES ('Jane', 'Johnson', '2012-01-01', 'FE43 8GG', encrypt('cake43', '1234', 'aes'));
INSERT INTO member(fname, lname, dob, postcode, password) VALUES ('Sam', 'Sampson', '2016-01-03', 'HE8 0NH', encrypt('camping111', '1234', 'aes'));
\end{sql}

T5. Once stored, print out the data for all of the rows. Copy below
\begin{sql}
SELECT * FROM member;
\end{sql}
\begin{pseudo}
 id | fname |   lname    |    dob     | postcode |              password
----+-------+------------+------------+----------+------------------------------------
  1 | Dave  | Davidson   | 2022-01-01 | NE1 4EQ  | \x2054eed5d9908d4d0dbb1d11777b9f2f
  2 | Fred  | Fredrikson | 2021-03-06 | AB12 CDE | \x7669b0433719a86b2f2736f3ccee1757
  3 | Sue   | Susan      | 1972-05-02 | BN35 7DQ | \x05d04a89e6159a6d79f6e035a1ad1931
  4 | Jane  | Johnson    | 2012-01-01 | FE43 8GG | \xb10a802af936a09504f73b98a66e45ff
  5 | Sam   | Sampson    | 2016-01-03 | HE8 0NH  | \xa46293e279f31027034993007197c256
(5 rows)
\end{pseudo}

T6. Decrypt the values stored in the encrypted password attribute for all 5 rows. 
\begin{sql}
SELECT id, convert_from(decrypt(password, '1234', 'aes'), 'utf-8') FROM member;
\end{sql}
\begin{pseudo}
 id | convert_from
----+--------------
  1 | cheese123
  2 | mouse33
  3 | secrue68
  4 | cake43
  5 | camping111
(5 rows)
\end{pseudo}

\subsection*{From Last week's lecture}
T7. Connect to \verb|dsd_22| and add a unique constraint to the customer table on the town column. Copy the code and output below:
\begin{sql}
\c dsd_22

ALTER TABLE customer ADD CONSTRAINT table_unique UNIQUE (town);
\end{sql}
\begin{pseudo*}
ERROR:  could not create unique index "table_unique"
DETAIL:  Key (town)=(La Mohammedia) is duplicated.
\end{pseudo*}

T8. Using the manifest table, how many \verb|prod_id| are there?
\begin{sql}
SELECT count(DISTINCT prod_id) FROM manifest;
\end{sql}
\begin{pseudo}
 count
-------
    76
(1 row)
\end{pseudo}

T9. How many distinct \verb|prod_id| are there?
\begin{sql}
SELECT count(DISTINCT prod_id) FROM product; 
\end{sql}
\begin{pseudo}
 count
-------
   100
(1 row)
\end{pseudo}

T10. How many orders in the manifest table include the product with the id of \verb|24|?
\begin{sql}
SELECT count(manifest_id) FROM manifest
WHERE prod_id=24;
\end{sql}
\begin{pseudo}
 count
-------
     4
(1 row)
\end{pseudo}

T11. How many orders in the manifest table include the product with the id of \verb|2|?
\begin{sql}
SELECT count(manifest_id) FROM manifest
WHERE prod_id=2;
\end{sql}
\begin{pseudo}
 count
-------
     0
(1 row)
\end{pseudo}

T12. Again, in the manifest table, what code could be used to give the following output:
\begin{pseudo}
prod_id 
---------
     100
      99
      97
      95
      94
      93
      92
      91
\end{pseudo}
Copy your answer below:
\begin{sql}
SELECT DISTINCT prod_id FROM manifest
ORDER BY prod_id DESC
LIMIT 8;
\end{sql}

T13. Using alter table, add a check constraint to the \verb|dsd_22| staff table. The check must check that the length of a postcode is over 5 characters long. Hint: the \verb|length()| function will find out how long a value is. Copy the code below.
\begin{sql}
ALTER TABLE staff ADD CONSTRAINT postcode_length CHECK(length(postcode)> 5);
\end{sql}

T14. Now add a new staff member to the staff table using the insert code snippet below.
\begin{sql}
INSERT INTO staff (staff_fname, staff_lname, addr1, addr2, town, postcode, home_email, work_email, ROLE)
VALUES ('Tiny',
        'Smith',
        '85 Lilly Way',
        'Off Pole Lane',
        'Southsea',
        'PO98',
        'tsmith@smiths.com',
        'Tiny.Smith@dsd.com',
        5);
\end{sql}
T15. Copy the output below
\begin{pseudo}
ERROR:  new row for relation "staff" violates check constraint "postcode_length"
DETAIL:  Failing row contains (12, Tiny, Smith, 85 Lilly Way, Off Pole Lane, Southsea, PO98     , tsmith@smiths.com, Tiny.Smith@dsd.com, 5).
\end{pseudo}

\subsection*{Dates}
We can use dates in many ways.

Download the code from the folder code for practical  and run it in your NORMAL database - the one called \verb|upxxxxxx|.
\begin{sql}
create table date_check (
	id INT primary key,
	first_name VARCHAR(50) not null,
	last_name VARCHAR(50) not null,
	email VARCHAR(50) not null,
	joined DATE not null
);
insert into date_check (id, first_name, last_name, email, joined) values (1, 'Carie', 'Harling', 'charling0@yale.edu', '2022-04-28');
insert into date_check (id, first_name, last_name, email, joined) values (2, 'Deina', 'Brennans', 'dbrennans1@slashdot.org', '2022-04-08');
insert into date_check (id, first_name, last_name, email, joined) values (3, 'Devon', 'Matijasevic', 'dmatijasevic2@economist.com', '2022-09-25');
insert into date_check (id, first_name, last_name, email, joined) values (4, 'Wald', 'Kleinhausen', 'wkleinhausen3@trellian.com', '2022-08-13');
insert into date_check (id, first_name, last_name, email, joined) values (5, 'Cammie', 'Womack', 'cwomack4@who.int', '2022-06-19');
insert into date_check (id, first_name, last_name, email, joined) values (6, 'Cross', 'MacCallam', 'cmaccallam5@tuttocitta.it', '2023-02-05');
insert into date_check (id, first_name, last_name, email, joined) values (7, 'Maris', 'Flitcroft', 'mflitcroft6@clickbank.net', '2022-07-12');
insert into date_check (id, first_name, last_name, email, joined) values (8, 'Peggy', 'Gasquoine', 'pgasquoine7@ebay.com', '2022-07-22');
insert into date_check (id, first_name, last_name, email, joined) values (9, 'Kermit', 'Ninnoli', 'kninnoli8@smh.com.au', '2022-10-10');
insert into date_check (id, first_name, last_name, email, joined) values (10, 'Frieda', 'Glassford', 'fglassford9@wufoo.com', '2022-08-26');
insert into date_check (id, first_name, last_name, email, joined) values (11, 'Lanie', 'Boggish', 'lboggisha@comcast.net', '2022-03-31');
insert into date_check (id, first_name, last_name, email, joined) values (12, 'Amelie', 'Timmons', 'atimmonsb@wp.com', '2022-11-23');
insert into date_check (id, first_name, last_name, email, joined) values (13, 'Portia', 'Nielson', 'pnielsonc@wix.com', '2022-10-10');
insert into date_check (id, first_name, last_name, email, joined) values (14, 'Sara-ann', 'Ellens', 'sellensd@chronoengine.com', '2022-06-15');
insert into date_check (id, first_name, last_name, email, joined) values (15, 'Bob', 'Larcombe', 'blarcombee@dailymotion.com', '2022-06-28');
insert into date_check (id, first_name, last_name, email, joined) values (16, 'Celestyn', 'Wickenden', 'cwickendenf@prnewswire.com', '2022-06-15');
insert into date_check (id, first_name, last_name, email, joined) values (17, 'Rina', 'Dymoke', 'rdymokeg@discuz.net', '2022-07-19');
insert into date_check (id, first_name, last_name, email, joined) values (18, 'Isadora', 'Haughey', 'ihaugheyh@sfgate.com', '2022-07-31');
insert into date_check (id, first_name, last_name, email, joined) values (19, 'Demetria', 'Neem', 'dneemi@jiathis.com', '2022-05-08');
insert into date_check (id, first_name, last_name, email, joined) values (20, 'Feliza', 'Gras', 'fgrasj@printfriendly.com', '2022-03-19');

\end{sql}

T16. Select the last names and the date they joined and copy the results below. 
\begin{sql}
SELECT last_name, joined FROM date_check;
\end{sql}
\begin{pseudo}
  last_name  |   joined
-------------+------------
 Harling     | 2022-04-28
 Brennans    | 2022-04-08
 Matijasevic | 2022-09-25
 Kleinhausen | 2022-08-13
 Womack      | 2022-06-19
 MacCallam   | 2023-02-05
 Flitcroft   | 2022-07-12
 Gasquoine   | 2022-07-22
 Ninnoli     | 2022-10-10
 Glassford   | 2022-08-26
 Boggish     | 2022-03-31
 Timmons     | 2022-11-23
 Nielson     | 2022-10-10
 Ellens      | 2022-06-15
 Larcombe    | 2022-06-28
 Wickenden   | 2022-06-15
 Dymoke      | 2022-07-19
 Haughey     | 2022-07-31
 Neem        | 2022-05-08
 Gras        | 2022-03-19
(20 rows)
\end{pseudo}

T17. Now sort them into \verb|last_name| order. Copy the results below:
\begin{sql}
SELECT last_name, joined FROM date_check
ORDER BY last_name;
\end{sql}
\begin{pseudo}
  last_name  |   joined
-------------+------------
 Boggish     | 2022-03-31
 Brennans    | 2022-04-08
 Dymoke      | 2022-07-19
 Ellens      | 2022-06-15
 Flitcroft   | 2022-07-12
 Gasquoine   | 2022-07-22
 Glassford   | 2022-08-26
 Gras        | 2022-03-19
 Harling     | 2022-04-28
 Haughey     | 2022-07-31
 Kleinhausen | 2022-08-13
 Larcombe    | 2022-06-28
 MacCallam   | 2023-02-05
 Matijasevic | 2022-09-25
 Neem        | 2022-05-08
 Nielson     | 2022-10-10
 Ninnoli     | 2022-10-10
 Timmons     | 2022-11-23
 Wickenden   | 2022-06-15
 Womack      | 2022-06-19
(20 rows)
\end{pseudo}

T18. What happens if we sort by a column we are not displaying? Copy the output below:
\begin{sql}
SELECT last_name, joined FROM date_check
ORDER BY email;
\end{sql}
\begin{pseudo}
  last_name  |   joined
-------------+------------
 Timmons     | 2022-11-23
 Larcombe    | 2022-06-28
 Harling     | 2022-04-28
 MacCallam   | 2023-02-05
 Wickenden   | 2022-06-15
 Womack      | 2022-06-19
 Brennans    | 2022-04-08
 Matijasevic | 2022-09-25
 Neem        | 2022-05-08
 Glassford   | 2022-08-26
 Gras        | 2022-03-19
 Haughey     | 2022-07-31
 Ninnoli     | 2022-10-10
 Boggish     | 2022-03-31
 Flitcroft   | 2022-07-12
 Gasquoine   | 2022-07-22
 Nielson     | 2022-10-10
 Dymoke      | 2022-07-19
 Ellens      | 2022-06-15
 Kleinhausen | 2022-08-13
(20 rows)
\end{pseudo}

T19. How would you get a list of people who joined after October 1st 2022?
\begin{sql}
SELECT first_name, last_name, joined FROM date_check
WHERE joined > '2022-10-01';
\end{sql}
\begin{pseudo}
 first_name | last_name |   joined
------------+-----------+------------
 Cross      | MacCallam | 2023-02-05
 Kermit     | Ninnoli   | 2022-10-10
 Amelie     | Timmons   | 2022-11-23
 Portia     | Nielson   | 2022-10-10
(4 rows)
\end{pseudo}

T20. Order the output by joined date order. Copy this output below.
\begin{sql}
SELECT first_name, last_name, joined FROM date_check
WHERE joined > '2022-10-01'
ORDER BY joined ASC;
\end{sql}
\begin{pseudo}
 first_name | last_name |   joined
------------+-----------+------------
 Kermit     | Ninnoli   | 2022-10-10
 Portia     | Nielson   | 2022-10-10
 Amelie     | Timmons   | 2022-11-23
 Cross      | MacCallam | 2023-02-05
(4 rows)
\end{pseudo}

T21. Now order the output from 20 so that the joined date is the first order THEN try to order by the \verb|last_name|. Copy this code \& output below.
\begin{sql}
SELECT first_name, last_name, joined FROM date_check
WHERE joined > '2022-10-01'
ORDER BY joined, last_name ASC;
\end{sql}
\begin{pseudo}
 first_name | last_name |   joined
------------+-----------+------------
 Portia     | Nielson   | 2022-10-10
 Kermit     | Ninnoli   | 2022-10-10
 Amelie     | Timmons   | 2022-11-23
 Cross      | MacCallam | 2023-02-05
(4 rows)
\end{pseudo}

T22. We can use the \verb|between| keyword to find results that fall between two dates. Output all data for the people who joined between April 20th 2022 and November 30th 2022. Copy the output below.
\begin{sql}
SELECT first_name, last_name, joined FROM date_check
WHERE joined BETWEEN '2022-04-20' AND '2022-11-30'
ORDER BY joined, last_name ASC;
\end{sql}
\begin{pseudo}
 first_name |  last_name  |   joined
------------+-------------+------------
 Carie      | Harling     | 2022-04-28
 Demetria   | Neem        | 2022-05-08
 Sara-ann   | Ellens      | 2022-06-15
 Celestyn   | Wickenden   | 2022-06-15
 Cammie     | Womack      | 2022-06-19
 Bob        | Larcombe    | 2022-06-28
 Maris      | Flitcroft   | 2022-07-12
 Rina       | Dymoke      | 2022-07-19
 Peggy      | Gasquoine   | 2022-07-22
 Isadora    | Haughey     | 2022-07-31
 Wald       | Kleinhausen | 2022-08-13
 Frieda     | Glassford   | 2022-08-26
 Devon      | Matijasevic | 2022-09-25
 Portia     | Nielson     | 2022-10-10
 Kermit     | Ninnoli     | 2022-10-10
 Amelie     | Timmons     | 2022-11-23
(16 rows)
\end{pseudo}