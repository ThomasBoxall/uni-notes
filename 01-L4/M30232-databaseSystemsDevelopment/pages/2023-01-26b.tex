\lecture{PRACTICAL: Security One}{26-01-23}{14:00}{Val \& Co}{FTC 3}

T1. Create a new role. Call this new role your first name. It must be given a password and the ability to login. Copy your code and response below:
\begin{sql}
CREATE ROLE thomas WITH LOGIN PASSWORD 'highlySecure1!';
\end{sql}

T2. Try to use this new role by using the following code
\begin{pseudo}
psql -h localhost -p 5432 -U thomas
\end{pseudo}
Output:
\begin{pseudo}
Password for user thomas:
psql: FATAL:  database "thomas" does not exist
\end{pseudo}

T3. As your normal user, create a new database that has the same name as your new role. This needs to be owned by the new user. 
\begin{sql}
CREATE DATABASE thomas OWNER thomas;
\end{sql}
Outputs:
\begin{pseudo}
CREATE DATABASE
\end{pseudo}

T4. Try to use this new role by using the following code
\begin{pseudo}
psql -h localhost -p 5432 -U thomas
\end{pseudo}
Outputs
\begin{pseudo}
Password for user thomas:
\end{pseudo}
T5. What does the prompt look like when you log in with your new role? Copy it below.
\begin{pseudo}
thomas=>
\end{pseudo}

T6. List the databases available. Copy the output below.
\begin{pseudo}
thomas=> \l
                                   List of databases
      Name      |     Owner      | Encoding | Collate |  Ctype  |   Access privileges
----------------+----------------+----------+---------+---------+-----------------------
 code_test      | up2108121      | UTF8     | C.UTF-8 | C.UTF-8 |
 customer_db    | up2108121      | UTF8     | C.UTF-8 | C.UTF-8 |
 dsd_22         | up2108121      | UTF8     | C.UTF-8 | C.UTF-8 |
 mongo-2021-fix | mongo-2021-fix | UTF8     | C.UTF-8 | C.UTF-8 |
 postgres       | postgres       | UTF8     | C.UTF-8 | C.UTF-8 |
 template0      | postgres       | UTF8     | C.UTF-8 | C.UTF-8 | =c/postgres          +
                |                |          |         |         | postgres=CTc/postgres
 template1      | postgres       | UTF8     | C.UTF-8 | C.UTF-8 | =c/postgres          +
                |                |          |         |         | postgres=CTc/postgres
 thomas         | thomas         | UTF8     | C.UTF-8 | C.UTF-8 |
 up2108121      | up2108121      | UTF8     | C.UTF-8 | C.UTF-8 |
 up2108121_cw   | up2108121      | UTF8     | C.UTF-8 | C.UTF-8 |
 week02         | up2108121      | UTF8     | C.UTF-8 | C.UTF-8 |
(11 rows)
\end{pseudo}

T7. Connect to a different database and list the tables. Copy the output below.
\begin{pseudo}
thomas=> \c dsd_22
SSL connection (protocol: TLSv1.3, cipher: TLS_AES_256_GCM_SHA384, bits: 256, compression: off)
You are now connected to database "dsd_22" as user "thomas".
dsd_22=>
\end{pseudo}

T8. Select all of the data in one of the tables listed in T7. Copy the output below.
\begin{sql}
SELECT * FROM manifest;
\end{sql}
\begin{pseudo}
ERROR:  permission denied for table manifest
\end{pseudo}

T9. As your normal user, make the new role a superuser with the following code:
\begin{sql}
ALTER ROLE thomas WITH SUPERUSER;
\end{sql}

T10. Make sure your new role is logged out with \verb|\q| and then log in again. What does the prompt now look like? Copy this prompt below
\begin{pseudo}
up2108121@up2108121:~ psql -h localhost -p 5432 -U thomas
Password for user thomas:
psql (11.18 (Debian 11.18-0+deb10u1))
SSL connection (protocol: TLSv1.3, cipher: TLS_AES_256_GCM_SHA384, bits: 256, compression: off)
Type "help" for help.

thomas=#
\end{pseudo}

T11. List the databases available. Copy the output below.
\begin{pseudo}
thomas=# \l
                                   List of databases
      Name      |     Owner      | Encoding | Collate |  Ctype  |   Access privileges
----------------+----------------+----------+---------+---------+-----------------------
 code_test      | up2108121      | UTF8     | C.UTF-8 | C.UTF-8 |
 customer_db    | up2108121      | UTF8     | C.UTF-8 | C.UTF-8 |
 dsd_22         | up2108121      | UTF8     | C.UTF-8 | C.UTF-8 |
 mongo-2021-fix | mongo-2021-fix | UTF8     | C.UTF-8 | C.UTF-8 |
 postgres       | postgres       | UTF8     | C.UTF-8 | C.UTF-8 |
 template0      | postgres       | UTF8     | C.UTF-8 | C.UTF-8 | =c/postgres          +
                |                |          |         |         | postgres=CTc/postgres
 template1      | postgres       | UTF8     | C.UTF-8 | C.UTF-8 | =c/postgres          +
                |                |          |         |         | postgres=CTc/postgres
 thomas         | thomas         | UTF8     | C.UTF-8 | C.UTF-8 |
 up2108121      | up2108121      | UTF8     | C.UTF-8 | C.UTF-8 |
 up2108121_cw   | up2108121      | UTF8     | C.UTF-8 | C.UTF-8 |
 week02         | up2108121      | UTF8     | C.UTF-8 | C.UTF-8 |
(11 rows)
\end{pseudo}

T12. Connect to a different database and list the tables. Copy the output below.
\begin{pseudo}
\c up2108121_cw
\dt

List of relations
Schema |       Name       | Type  |   Owner
--------+------------------+-------+-----------
public | boat             | table | up2108121
public | boat_spec        | table | up2108121
public | boatyard         | table | up2108121
public | customer         | table | up2108121
public | role             | table | up2108121
public | service          | table | up2108121
public | service_contents | table | up2108121
public | service_item     | table | up2108121
public | service_staff    | table | up2108121
public | staff            | table | up2108121
public | staff_role       | table | up2108121
(11 rows)
\end{pseudo}

T12. Select all of the data in one of the tables listed in T7. Copy the output below.
\begin{sql}
\c dsd_22
SELECT * FROM manifest;
\end{sql}
\begin{pseudo}
 manifest_id | cust_ord_id | prod_id
-------------+-------------+---------
           1 |           1 |      84
           2 |           2 |       1
           3 |           3 |      91
           4 |           4 |       5
           5 |           5 |      97
           6 |           6 |      74
           7 |           7 |      88
           8 |           8 |      97
           9 |           9 |      66
          10 |          10 |      43
          11 |          11 |      78
          12 |          12 |      24
          13 |          13 |      69
          14 |          14 |      25
          15 |          15 |       4
          16 |          16 |      32
          17 |          17 |      66
          18 |          18 |      13
          19 |          19 |      83
          20 |          20 |       4
          21 |          21 |      45
          22 |          22 |       4
          23 |          23 |      93
          24 |          24 |      45
...
\end{pseudo}