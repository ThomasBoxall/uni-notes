\lecture{PRACTICAL: Normalisations and Joins}{24-11-22}{14:00}{Val}{FTC Floor 3}

\section*{Order of Execution}
\begin{enumerate}
    \item FROM \& JOIN (chose and join tables to get base data)
    \item WHERE \& SUBQUERY/ INTERSECTION/ UNION/ EXCEPT (filters the base data)
    \item GROUP BY (aggregates the base data)
    \item HAVING (filters the aggregated ata)
    \item SELECT (returns the final data, as functionality not displayed)
    \item ORDER BY (sort the final data)
    \item LIMIT (limits the returned data to a row count)
    \item display data
\end{enumerate}

\section*{Task 1}
See Google Doc and Lucid Chart.

\section*{Task 2}
1. Write a query to retrieve the first and last names of the customers in the customer table. Copy the query and the answer below.
\begin{sql}
SELECT cust_fname, cust_lname from customer;
\end{sql}
\begin{pseudo*}
   cust_fname    |    cust_lname    
-----------------+------------------
 Jobey           | Boeter
 York            | O'Deegan
 Penelope        | Hexter
 Chadd           | Franz-Schoninger
 Vikky           | Eke
 Marie-françoise | Currier
 Bénédicte       | Dozdill
 Görel           | Douthwaite
 Bérengère       | Menendez
 Pélagie         | Hachard
 Adaobi          | Musa
(11 rows)
\end{pseudo*}

2. Write a query to retrieve the first and last names and the towns they live in of the customers in the customer table. Copy the query and the answer below.
\begin{sql}
SELECT cust_fname, cust_lname, town FROM customer;
\end{sql}
\begin{pseudo*}
   cust_fname    |    cust_lname    |      town
-----------------+------------------+----------------
 Jobey           | Boeter           | La Mohammedia
 York            | O'Deegan         | Chemnitz
 Penelope        | Hexter           | Pingshan
 Chadd           | Franz-Schoninger | Baojia
 Vikky           | Eke              | Kamenný Přívoz
 Marie-françoise | Currier          | Waekolong
 Bénédicte       | Dozdill          | Dawuhan
 Görel           | Douthwaite       | Sunbu
 Bérengère       | Menendez         | Tsagaanders
 Pélagie         | Hachard          | Jiantou
 Adaobi          | Musa             | La Mohammedia
(11 rows)
\end{pseudo*}

3. Print out the first and last name of the customer / customers who live in La Mohammedia. Copy the query and the answer below.
\begin{sql}
SELECT cust_fname, cust_lname FROM customer WHERE town= 'La Mohammedia';
\end{sql}
\begin{pseudo*}
 cust_fname | cust_lname 
------------+------------
 Jobey      | Boeter
 Adaobi     | Musa
(2 rows)
\end{pseudo*}

4. Get the structure of the tables customer and cust\_order using the \verb|\d| command. Copy the code and the answer below.
\begin{pseudo*}
dsd_22=# \d customer
                                        Table "public.customer"
   Column   |          Type          | Collation | Nullable |                  Default
------------+------------------------+-----------+----------+-------------------------------------------
 cust_id    | integer                |           | not null | nextval('customer_cust_id_seq'::regclass)
 cust_fname | character varying(25)  |           | not null | 
 cust_lname | character varying(35)  |           | not null | 
 addr1      | character varying(50)  |           | not null | 
 addr2      | character varying(50)  |           |          | 
 town       | character varying(60)  |           | not null | 
 postcode   | character(9)           |           | not null | 
 email      | character varying(255) |           | not null | 
Indexes:
    "customer_pkey" PRIMARY KEY, btree (cust_id)
Referenced by:
    TABLE "cust_order" CONSTRAINT "cust_order_cust_id_fkey" FOREIGN KEY (cust_id) REFERENCES customer(cust_id)

dsd_22=# \d cust_order
                                   Table "public.cust_order"
   Column    |  Type   | Collation | Nullable |                     Default
-------------+---------+-----------+----------+-------------------------------------------------
 cust_ord_id | integer |           | not null | nextval('cust_order_cust_ord_id_seq'::regclass)
 staff_id    | integer |           |          | 
 cust_id     | integer |           |          | 
Indexes:
    "cust_order_pkey" PRIMARY KEY, btree (cust_ord_id)
Foreign-key constraints:
    "cust_order_cust_id_fkey" FOREIGN KEY (cust_id) REFERENCES customer(cust_id)
    "cust_order_staff_id_fkey" FOREIGN KEY (staff_id) REFERENCES staff(staff_id)
Referenced by:
    TABLE "manifest" CONSTRAINT "manifest_cust_ord_id_fkey" FOREIGN KEY (cust_ord_id) REFERENCES cust_order(cust_ord_id)
\end{pseudo*}

5. According to the answer from question 4, what are the names of the attributes in both tables that are the primary key and foreign keys? (hint - look at the section “Foreign-key constraints:” that appears in one of your outputs. Remember we are looking at customer and cust\_order)\\
customer pk - cust\_id\\
cust\_order pk - cust\_ord\_id\\
cust\_order fk - cust\_id\\
cust\_order fk - staff\_id

6. List all of the categories. Copy the query and the answer below.
\begin{sql}
SELECT * FROM category;
\end{sql}
\begin{pseudo*}
 cat_id |  cat_name   
--------+-------------
      1 | Men's Wear
      2 | Ladies Wear
      3 | Kid's Wear
      4 | Outdoor
      5 | Sport
      6 | Health
(6 rows)
\end{pseudo*}

7. What is the id number for the category Sport? Copy the query and the answer below.
\begin{sql}
SELECT cat_id from category where cat_name='Sport';
\end{sql}
\begin{pseudo*}
 cat_id
--------
      5
(1 row)
\end{pseudo*}

8. Write a query that joins the product table and the category table and prints out the prod\_name and the appropriate category. Copy the query and the answer below. (You can copy the just first screen of data if you want)
\begin{sql}
SELECT product.prod_name, category.cat_name FROM product
JOIN category ON category.cat_id = product.prod_cat;
\end{sql}
\begin{pseudo*}
                    prod_name                     |  cat_name   
--------------------------------------------------+-------------
 Multi-layered multi-tasking initiative           | Ladies Wear
 Operative analyzing task-force                   | Men's Wear
 Exclusive client-server array                    | Sport
 Balanced client-server product                   | Health
 Exclusive background website                     | Sport
 Pre-emptive holistic intranet                    | Health
 Re-engineered cohesive methodology               | Men's Wear
 Robust directional projection                    | Ladies Wear
 Inverse transitional infrastructure              | Outdoor
 Multi-tiered explicit paradigm                   | Health
...
(100 rows)
\end{pseudo*}

9. Write a query that will list each staff member's first and last name along with their work email and the role name that they hold. Copy the query and the answer below.
\begin{sql}
SELECT staff.staff_fname, staff.staff_lname, staff.work_email, role.role_name from staff
JOIN role ON staff.role = role.role_id;
\end{sql}
\begin{pseudo*}
 staff_fname | staff_lname |         work_email          |    role_name    
-------------+-------------+-----------------------------+-----------------
 Montgomery  | Housegoe    | Montgomery.Housegoe@dsd.com | Order Picker
 Niel        | Welsby      | Niel.Welsby@dsd.com         | Final Packer
 Jillene     | Revitt      | Jillene.Revitt@dsd.com      | Post Sales
 Harriette   | Fewster     | Harriette.Fewster@dsd.com   | Post Sales
 Aura        | Clewlowe    | Aura.Clewlowe@dsd.com       | Post Sales
 Hanan       | Gloster     | Hanan.Gloster@dsd.com       | Customer Retain
 Nikoletta   | Shrimpton   | Nikoletta.Shrimpton@dsd.com | Customer Retain
 Tim         | Illem       | Tim.Illem@dsd.com           | Misc
 Nell        | Olsson      | Nell.Olsson@dsd.com         | Misc
 Janeva      | Gillicuddy  | Janeva.Gillicuddy@dsd.com   | Misc
(10 rows)
\end{pseudo*}

10. Write a query that will show the last name and the role of staff members who put together orders from the customer whose last name is Eke. Include the cust\_order\_id and the customer's first and last names. Copy the query and the answer below.
\begin{sql}
SELECT staff.staff_lname, role.role_name, cust_order.cust_ord_id, customer.cust_fname, customer.cust_lname FROM staff
JOIN role ON staff.role = role.role_id
JOIN cust_order ON cust_order.staff_id = staff.staff_id
JOIN customer ON customer.cust_id = cust_order.cust_id
WHERE customer.cust_lname = 'Eke';
\end{sql}
\begin{pseudo*}
 staff_lname |    role_name    | cust_ord_id | cust_fname | cust_lname 
-------------+-----------------+-------------+------------+------------
 Welsby      | Final Packer    |          82 | Vikky      | Eke
 Clewlowe    | Post Sales      |          90 | Vikky      | Eke
 Welsby      | Final Packer    |         105 | Vikky      | Eke
 Housegoe    | Order Picker    |         115 | Vikky      | Eke
 Gillicuddy  | Misc            |         118 | Vikky      | Eke
 Welsby      | Final Packer    |         130 | Vikky      | Eke
 Shrimpton   | Customer Retain |         132 | Vikky      | Eke
 Welsby      | Final Packer    |         135 | Vikky      | Eke
 Housegoe    | Order Picker    |         145 | Vikky      | Eke
(9 rows)
\end{pseudo*}

11. Write a query that lists only the category names and the custome's last names for orders that have been placed by people who live in Sunbu. Copy the query and answer below.

\begin{sql}
SELECT customer.cust_lname, category.cat_name FROM customer
JOIN cust_order ON customer.cust_id = cust_order.cust_id
JOIN manifest ON cust_order.cust_ord_id = manifest.cust_ord_id
JOIN product ON product.prod_id = manifest.prod_id
JOIN category ON category.cat_id = product.prod_cat
WHERE customer.town = 'Sunbu';
\end{sql}
\begin{pseudo*}
 cust_lname |  cat_name   
------------+-------------
 Douthwaite | Outdoor
 Douthwaite | Sport
 Douthwaite | Kid's Wear
 Douthwaite | Outdoor
 Douthwaite | Sport
 Douthwaite | Sport
 Douthwaite | Ladies Wear
 Douthwaite | Outdoor
 Douthwaite | Sport
 Douthwaite | Ladies Wear
(10 rows)
\end{pseudo*}

12. This is a bit harder than the previous queries. Try to group the orders and count the number of orders in each category for the results from q11. (hint - this might be a bit difficult. Grouping does not allow a WHERE, use HAVING instead). Copy the query and answer below.
\begin{sql}
SELECT customer.cust_lname, count(category.cat_name), category.cat_name FROM customer
JOIN cust_order ON customer.cust_id = cust_order.cust_id
JOIN manifest ON cust_order.cust_ord_id = manifest.cust_ord_id
JOIN product ON product.prod_id = manifest.prod_id
JOIN category ON category.cat_id = product.prod_cat
GROUP BY customer.cust_lname, category.cat_name, customer.town
HAVING customer.town='Sunbu';
\end{sql}
\begin{pseudo*}
 cust_lname | count |  cat_name
------------+-------+-------------
 Douthwaite |     1 | Kid's Wear
 Douthwaite |     2 | Ladies Wear
 Douthwaite |     3 | Outdoor
 Douthwaite |     4 | Sport
\end{pseudo*}
