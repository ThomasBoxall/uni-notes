\taughtsession{Lecture}{Misconfigured File Permissions}{2024-02-26}{09:00}{Tobi}{}

\textit{Can bring in a single A4 sheet of paper (double sided) to the exam. It must be handwritten.}




\section{Files}
Within Linux, everything is a file. We can find files using OSINT (Google Dorking, etc) however locally on a Linux system - we have three commands we can use:
\begin{itemize}
    \item \verb|find| searches for files in a directory
    \item \verb|locate| finds files by name, quickly
    \item \verb|which| locates a command
\end{itemize}
We are also able to exploit the PATH variable by forcing programs to run our version of certain binary files. The command \verb|echo $PATH| shows what files are currently in the path.

\section{File Permissions}
Linux file permissions are split into three categories - Users, Groups and Others. It is possible to control the permissions of reading, writing to, and executing a file for each of these three groups. \\

The SUID (Set User Identity) bit gives special permission for the user access level and always executes as the user who owns the file. This is a thing to keep an eye out for as it is possible to execute a command as root if these permissions have been incorrectly configured. Similarly, the Set Group Identity allows a file to be executed as the group owner.\\

The sticky bit restricts file deletion at a directory level.