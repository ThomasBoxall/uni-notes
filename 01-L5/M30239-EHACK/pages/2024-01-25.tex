\taughtsession{Lab}{Introduction to Pentesting and Linux}{2024-01-25}{14:00}{Tobi}{}

\section{Introduction to the Lab Environment}
The lab environment in which we will complete practical sessions is a virtual machine within the Lion Gate 0.07 Digital Forensics Lab. We will use \textit{Virtual Box} as our Hypervisor and the machine is called VLABS.\\

Within the VLABS virtual machine, we will use a series of Docker Containers to ``virtualise'' the machines which we will penetrate. (Yes, that is a VM running in a VM). In the lab worksheets, there are a number of commands we use to start the Docker containers running.

\section{Planning \& Pre-Engagement}
The first stage of the penetration testing cycle, is planning how we will engage with the target machine. Commonly, this will begin with \textit{Information Gathering} - which is a process where we will gather information about the target device that we can utilise during the penetration process.\\

Much of the process we follow for each target device will differ, depending on what we know about it. In the following sections, we will use the hostname \verb|loan.atlas.local|

\subsection{DNS Enumeration}
The aim of \textit{Domain Name System (DNS) Enumeration} is to retrieve vital network information about the domain. For example, the IP addresses associated to it and if the target device is `up' or not.\\

The \verb|ping| command can be used to check if the target device is up:
\begin{verbatim}
    ping -c 1 loan.atlas.local
\end{verbatim}
(note the \verb|-c 1| flag instructs the computer to send 1 request)\\

We can then use the \verb|nslookup| command to identify the server's IP address from DNS.
\begin{verbatim}
    nslookup loan.atlas.local
\end{verbatim}

\subsection{Range Scanning}
Once we have identified the network which our target device resides on, we are able to \textit{range scan} that network, to identify what other devices are on it. We can use the tool \verb|nmap| to do this.
\begin{verbatim}
    nmap 192.168.1.12/24 -T5
\end{verbatim}
(note the \verb|-T5| flag is used to set \verb|nmap| to maximum speed; the \verb|-p-| flag could be added to specify to scan all possible ports rather than standard-used ports)

\subsection{Port Scanning / Banner Grabbing}
We can use a process called \textit{port scanning} to identify what ports are open on our target device, and therefore identify potential vulnerabilities. TCP port scanning works by sending TCP \verb|SYN| flags while expecting an acknowledgement / \verb|ACK| response; which if it comes - we know that the port is open and therefore something is running on it.\\

Commonly completed at the same time is a process called \textit{banner grabbing and service enumeration}, which is the process through which we can identify what service is running on each open port. When we send a message to the port, it will return it's `banner' which usually gives its service version. \\

We can use the \verb|nmap| tool to complete this task.
\begin{verbatim}
    nmap -sV loan.atlas.local -T5
\end{verbatim}
(note that the \verb|-sV| flag tells \verb|nmap| to identify the service version; we can also append the \verb|-p-| flag as above)

\section{Exploitation}
After we have gathered a suitable amount of information about the target device, we are able to start exploiting vulnerabilities.\\

There a masses of different ways in which you could exploit a device, this lab focused on a specific plugin within WordPresses vulnerabilities.

\subsection{HTTP Enumeration}
HTTP enumeration is the process in which we are looking for potential attack vectors within a HTTP service, most commonly a webserver. By default, HTTP traffic uses port 80 and HTTPS uses 443, these can be changed however.\\

One tool which is useful is \textit{Nikto}, which scans against webservers for vulnerabilities. This is used as shown below:
\begin{verbatim}
    nikto -h http://loan.atlas.local
\end{verbatim}

In this lab, we see from the output of Nikto that the webserver is running a version of Wordpress. From here, we can use \textit{WPscan} to explore that in greater detail.
\begin{verbatim}
    wpscan --url http://loan.atlas.local
\end{verbatim}

From here, we would search the names of the identified plugins on Exploit Databases, and we would see that a plugin is vulnerable then work to exploit it. There are well-documented instructions across the internet for the majority of vulnerabilities which we would come across in this module.

\subsection{Password Hash Cracking}
In sensibly designed systems, passwords are stored as hashes (the how and why of this covered in a different lecture). To crack these hashes, we use a tool - in this lab, we use \verb|hashcat|
\begin{verbatim}
    hashcat -m 400 -a 0 hashfile rockyou.txt
\end{verbatim}
Note that we have to provide a wordlist as the final parameter, this is a list of possible passwords which are known. 

\section{The Linux Environment}
\textit{NB this is abbreviated as compared to the Lab Worksheet, in the attempt as to not reproduce the lab worksheets word for word.}\\

As we have already seen, most of our exploitative work will take place in Linux, as it's better. There are a number of commands which can be used to retrieve information about the system we are working on:
\begin{description}[font=\ttfamily]
    \item[uname -r] gives the current operating systems kernel version
    \item[whoami] gives the username of the current user
    \item[id] gives the id of the current user, as well as what groups they are a member of
    \item[ps aux | wc -l] gives the number of currently running processes
    \item[service --status-all] gives the status of all the currently running services (note this is Ubuntu only)
    \item[free -h] gives the current RAM status, including how much is in use and how much is free
    \item[df -H] gives the current Disk status
    \item[ip a] gives the all the IP addresses of the current devices
\end{description}