\taughtsession{Lecture}{Introduction to Penetration Testing}{2024-01-22}{0900}{Tobi}{}

\textit{``If you start searching for Vulnerabilities in WordPress, you will find lots''}

\section{Introduction to Ethical Hacking}
Ethical Hacking is the process of finding vulnerabilities and reporting them to the correct people so that they can be rectified. Ethical hacking is a core component of the broader thing which is \textit{Cyber Security}, in which we are striving to protect the three core properties: Confidentiality (protecting information from being disclosed), Integrity (protecting information from being modified) and Availability (ensuring access to information when needed). 

\section{Penetration Testing}
\textit{Penetration Testing} is the continuous process of identifying, analysing, exploiting and making recommendations to vulnerabilities. Pen. Testing is often described as a cycle, which follows a very strict plan within a fixed timeframe. 
\begin{figure}[H]
    \centering
    \smartdiagramset{circular distance=4cm, 
    font=\footnotesize,
    uniform color list = black!40 for 8 items,
    text width=2.5cm }
    \smartdiagram[circular diagram:clockwise]{Pre-Engagement Interactions, Intelligence Gathering, Scanning, Threat Modelling, Vulnerability Analysis, Explotiation, Post Exploitation, Reporting}
    \caption{Pen. Testing Cycle}    
\end{figure}
There are three types of Pen. Testing:
\begin{description}
    \item[Black Box] where little or no knowledge is disclosed to the pen. tester
    \item[Grey Box] where some knowledge is disclosed to the pen. tester. They will not be provided full information on anything
    \item[White Box] where all knowledge is disclosed to the pen. tester
\end{description}

Through Pen. Testing, we actually exploit the vulnerabilities - not just look at them and go ``oh, that's a nice Vulnerability''. Vulnerability assessments can be carried out in a number of places:
\begin{description}
    \item[Human] through human errors, insider threats, social engineering, indifference
    \item[Application] Functions, storage, memory management, input validation
    \item[Host] Access Control, memory, malware, backdoor, OS / Kernel
    \item[Network] Map the network, services, leaks, intercept traffic
\end{description}

\section{Stages of Penetration Testing}
\subsection{Information Gathering}
In the \textit{information gathering} phase, the hacker strives to obtain as much information on the target service / device / company as possible. This could be done through passive methods such as:
\begin{itemize}
    \item Open Source Intelligence
    \item Google Dorking
    \item Social Media Analysis
    \item DNS Enumeration
\end{itemize}
Passive methods are methods where as much information as possible is gathered without establishing contact between the pen. tester and the target.\\

Alternatively, active information gathering techniques (where the pen. tester establishes contact with the target) could be used:
\begin{itemize}
    \item Open Ports and Service Enumeration
    \item Directory Scanning
    \item Common Weaknesses
\end{itemize}

\subsection{Exploitation}
After gathering information on the target, then next stage is to exploit and vulnerabilities which have been identified. Commonly this can be done through social engineering \& fishing, where illiterate users will handover compromising details unknowingly or through known exploits (such as the wp-google-maps exploit explored during the lecture and practical). The decision as to which exploit to use is quite complex and takes a number of factors into consideration including:
\begin{itemize}
    \item Reliability
    \item Complexity
    \item Detection
    \item Impact
    \item Environment
    \item Cost
\end{itemize}

\subsection{Post Exploitation}
After an exploit has been exploited, the next stage is to see what can be done with the access gained. Commonly this will be to attempt \textit{privilege escalation} through which a basic user account's permissions are escalated to be higher; or to maintain access - which could be done through keeping a SSH session alive or creating a start up service to open a backdoor. The pen. tester will need to cover their tracks, done through editing logs which in linux are found in the \verb|/var/log/| directory. Finally, the pen. tester will write a report detailing what they have found, how they exploited it and give recommendations on what can be done to close the exploit. 

\section{Defences}
There are a number of defences which can be used against hacking:
\begin{itemize}
    \item Firewalls
    \item Intrusion Detection Systems
    \item Intrusion Prevention Systems
    \item Regular Testing
    \item Effective Policies
    \item Regular Effective Training
    \item Patch Management
    \item Threat Intelligence
\end{itemize}