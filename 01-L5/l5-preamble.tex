% LEVEL 5 UNIVERSITY NOTES PREAMBLE
% THOMAS BOXALL <thomas@thomasboxall.net>
% 12 September 2023
%%%%%%%%%%%%%%%%%%%%%%%%%%%%%%%%%%%%%%%%%

% this preamble gets include in all university note documents. It contains custom stylling and useful features to speed up note taking.

%%%%%%%%%%%%%%%%%%%%%%%%
% Basic Document Setup %
%%%%%%%%%%%%%%%%%%%%%%%%

\usepackage{geometry}
\geometry{
a4paper,
total={170mm,257mm},
left=20mm,
top=20mm,
marginparsep=0mm,
}
\setlength\parindent{0pt} % get rid of the stupid indent

% add in capabilities for small notes in the left margin
\reversemarginpar
\newcommand{\marginnote}[1]{%
\marginpar{\footnotesize\texttt{#1}}
}

% set toc depth to only show chapters
\setcounter{tocdepth}{0}

% use section numbering for everything up too and including subsubsection
\setcounter{secnumdepth}{3}

% BASIC PACKAGES
\usepackage[dvipsnames]{xcolor}
\usepackage{graphicx}
\usepackage{fontawesome}
\usepackage[colorlinks=true, linkcolor=myBlue]{hyperref}
\usepackage{lastpage}
\usepackage{datetime2} % changes \today format to be yyyy-mm-dd
\usepackage{enumitem}
\usepackage{amsmath}
\usepackage{amssymb}

\usepackage{smartdiagram} % used for fancy diagrams.
\usepackage{pgf, tikz}
\usepackage{pgfplots}
\pgfplotsset{compat=1.11}
\usetikzlibrary{arrows, automata, arrows.meta, bending, positioning, trees, patterns, calc}
\tikzset{%
  graphnode/.style={
    circle,
    minimum width=0.3cm,
    inner sep=0pt,
    text width=4mm,
    align=center
    },
  emphline/.style={
    line width=0.2cm,
    gray!50
  },
  blueln/.style={
    blue!50
  }
}

% CUSTOM COLOURS
\definecolor{myGrey}{HTML}{121212}
\definecolor{myBlue}{HTML}{192A3D}


%%%
% DOC PROPERTIES SETUP
%%%

% module name
% module code
% when from - to
% credits
% Year

\def\university{University of Portsmouth}
\def\course{BSc (Hons) Computer Science}

\def\notesAuthor{Thomas Boxall}
\def\notesAuthorContact{up2108121@myport.ac.uk}

\newcommand{\documentsetup}[6]{%
    \def\moduleName{#1}
    \def\moduleNameShort{\texttt{#2}}
    \def\moduleCode{#3}
    \def\moduleDates{#4}
    \def\moduleCredits{#5}
    \def\courseYear{#6}
}

%%%
% TABLES & FLOATS
%%% 
\usepackage{float}
\usepackage{tabularx}
\usepackage{ragged2e}
\usepackage{multirow}
\usepackage{multicol}
\usepackage{array}
\usepackage{svg}
\usepackage{longtable}

\renewcommand{\arraystretch}{1.6} % make cells vertically bigger

\newcolumntype{C}[1]{>{\centering\let\newline\\\arraybackslash\hspace{0pt}}p{#1}} % create custom column type with definable width

% custom table env

% \newenvironment{nicetable}[1]
%     {\begin{table}[H]
        
%         \begin{center}
%         \raggedright
%         \begin{tabular}{#1}
%     }%
%     {   \end{tabular}
%         \end{center}
%      % end of ragged right
%     % \caption{#2}
%     \end{table}
% } % end of nicetable


%%%
% Code Blocks etc
%%%

\usepackage{upquote}
\usepackage{listings}

\lstset{
  basicstyle=\ttfamily,
  mathescape,
}

\lstdefinestyle{haskellTrace}{
  moredelim=[is][\underbar]{_}{_},
  keepspaces=true,
  escapechar=^
}

%%%
% MATHEMATICAL ENVIRONMENTS
%%%
\usepackage{amsthm}
\renewcommand\qedsymbol{$\blacksquare$}
\newtheorem{theorem}{Theorem}[section]

\usepackage[framemethod=tikz]{mdframed}
% new mdframed style that places the edges at the corners:
\mdfdefinestyle{proof}{
  skipabove         = .5\baselineskip ,
  skipbelow         = .5\baselineskip ,
  leftmargin        = 0pt ,
  rightmargin       = 0pt ,
  innermargin       = 0pt ,
  innertopmargin    = .5em ,
  innerleftmargin   = .5em ,
  innerrightmargin  = 0pt ,
  innerbottommargin = .5em ,
  hidealllines      = true ,
  singleextra       = {
    \draw (O) -- ++(0,.675em) (O) -- ++(.675em,0) ;
    \draw (P-|O) -- ++(0,-.675em) (P-|O) -- ++(.675em,0) ;
  },
  firstextra        = {
    \draw (P-|O) -- ++(0,-.675em) (P-|O) -- ++(.675em,0) ;
  },
  secondextra       = {
    \draw (O) -- ++(0,.675em) (O) -- ++(.675em,0) ;
  },
}
% put the new mdframed style around the proof environment:
\surroundwithmdframed[style=proof]{proof}


%%%
% CHAPTER & HEADERS
%%%
\usepackage[raggedright,bf]{titlesec}
\usepackage{fancyhdr}

% reformat & rename chapter heading
\titlespacing*{\chapter}{0pt}{5pt}{10pt}
\renewcommand*{\chaptername}{Page}

% redefine \chapter so it uses pagestyle=fancy
\makeatletter
\renewcommand\chapter{\if@openright\cleardoublepage\else\clearpage\fi
\thispagestyle{fancy}%
\global\@topnum\z@
\@afterindentfalse
\secdef\@chapter\@schapter}
\makeatother

% design headers and footers
\pagestyle{fancy}
\fancyhf{}
\fancyhead[L]{\moduleCode{} (\moduleNameShort)}
\fancyhead[R]{\leftmark} %put project title here
\fancyfoot[L]{\footnotesize \texttt{compiled at} \\ \texttt{\DTMnow}}
\fancyfoot[C]{\textbf{\thepage}\ of \pageref*{LastPage}}
\fancyfoot[R]{\notesAuthor}
\renewcommand{\footrulewidth}{0.4pt}
\addtolength{\topmargin}{-1.59999pt}
\setlength{\headheight}{13.59999pt}

% custom headers for taught session
\newcommand{\taughtsession}[6]{%
    \chapter[#1 - #2 (#3)]{#1 - #2}
    \chaptermark{#2}
    \begin{figure}[H]
        \begin{minipage}[t]{0.2\textwidth}
            \faCalendar\ #3
        \end{minipage}\hfill
        \begin{minipage}[t]{0.2\textwidth}
            \faClockO\ #4
        \end{minipage}\hfill
        \begin{minipage}[t]{0.2\textwidth}
            \faMortarBoard\ #5
        \end{minipage}\hfill
    \end{figure}
    \color{myBlue}
    \rule[1em]{\textwidth}{0.25px}
    \color{black}
    } % end of new command

% custom notes document title page

\newcommand{\makedocumenttitlepage}{\begin{titlepage}
    \pagecolor{myBlue}
    \color{white}
    \rule{\textwidth}{1px}
    \vspace{0.025\textheight}

    \huge{\university}\\
    \huge{\course}\\
    \huge{\courseYear{} Year}

    \vfill

    \LARGE{\textbf{\moduleName} (\moduleNameShort)}\\
    \Large{\moduleCode}\\
    \large{\moduleDates}\\
    \large{\moduleCredits{} Credits}

    \vfill

    \begin{flushright}
        \large{\notesAuthor}\\
        \texttt{\notesAuthorContact}
    \end{flushright}
    \vspace{0.2\textheight}
    \rule{\textwidth}{1px}
\end{titlepage}
\nopagecolor
}
