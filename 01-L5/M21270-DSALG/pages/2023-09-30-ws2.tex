\taughtsession{Async lecture}{Tools of the Trade I: Efficiency \& BigO}{2023-09-30}{}{}{}

BigO Notation is used to define the efficiency of an algorithm quantitatively. This is a very helpful tool to have when designing algorithms as it allows us to compare multiple algorithms to and understand which is the best one to use. 
\begin{table}[H]
    \centering
    {\RaggedRight
    \begin{tabular}{p{0.2\textwidth}p{0.2\textwidth}p{0.4\textwidth}}
    \textbf{Name} & \textbf{Notation} & \textbf{Description}\\
    \hline
    \hline
    Constant & $O(1)$ & Algorithm always executes in the same amount of time regardless of the size of the dataset.\\
    \hline
    Logarithmic & $O(\log_n)$ & Algorithm which halves the dataset with each pass, efficient with large datasets, increases execution time at a slower rate than that at which the dataset size increases.\\
    \hline
    Linear & $O(n)$ & Algorithm whose performance declines as the data set grows, reduces efficiency with increasingly large dataset.\\
    \hline
    Loglinear & $O(N\log_n)$ & Algorithm that divides a dataset but can be solved using concurrency on independent divided lists.\\
    \hline
    Polynomial & $O(N^2)$ & Algorithm whose performance is proportional to the size of the dataset, efficiency significantly reduces with increasingly large datasets.\\
    \hline
    Exponential & $O(2^n)$ & Algorithm that doubles with each addition to the dataset in each pass, very inefficient.\\
    \hline
    \end{tabular}
    } % end of rr     
    \caption{BigO Notation complexity values, listed best to worse}
\end{table}

BigO doesn't look at the exact number of operations, it looks at when the size of a problem approximates infinity therefore two very similar algorithms which are slightly different may have the same Big O despite one having double the number of operations.

\section{Calculating The BigO Value}
\begin{enumerate}
    \item Determine the basic operations (including: assignment, multiplication, addition, subtraction, division, etc)
    \item Count how many basic operations there are in the algorithm (some basic algebraic addition required here!)
    \item Convert the total number of operations to BigO (done by: ignoring the less dominant terms; and ignoring the constant coefficient - i.e. $2n+1$ becomes $n$).
\end{enumerate}
