\taughtsession{Async Lecture}{Self-Organising Trees}{2023-11-09}{}{}{}

Another type of tree is a \textit{splay tree}. These are self-organising binary trees which organise themself such that recently accessed elements are always quick to access again. 

\section{Basic Operations}
All basic operations (insertion, search, removal) are performed in $O(log_2n)$ time and they all include an extra step called \textit{splaying}.\\

Searching:
\begin{enumerate}
    \item Locate node
    \item Splay the node to the root
\end{enumerate}

Insertion:
\begin{enumerate}
    \item Insert node
    \item Splay inserted node to the root
\end{enumerate}

Deletion
\begin{enumerate}
    \item Locate the node to be deleted
    \item Replace node to be deleted with its in-order predecessor or its in-order successor (same process as deletion in a BST)
    \item Splay the parent of the removed node to the root
\end{enumerate}

\section{Splaying}
Splaying is the movement of a node to the root, which is achieved through a series of tree rotations which are similar to those used in AVL trees. Splaying works from the bottom to the top (root) of the tree, which is often called bottom-up splaying. Splaying involves a series of double rotations, until the accessed node reaches either the root or the child of the root when a single rotation is performed.\\

No single access operation on the Splay tree is guaranteed to be efficient, this is by design. The worst case will be $O(n)$, and the average across a series of operations tends to be $O(log_2n)$. 

\section{Rotations}

\subsection{Single Rotations}
Single rotations are used when the accessed node \verb|S| is the child of the root node \verb|P|. The effect of this rotation is to move the node up one level in the tree.

\subsubsection{Zig Rotation}
There are two variations of the \textit{Zig} rotation.
\begin{figure}[H]
    \centering
    \includesvg[width=0.8\textwidth]{assets/splay-zig.svg}
    \caption{Splay tree rotations (Zig)}
\end{figure}

Rotation \textit{(a)} is the equivalent of the LL rotation used with AVL trees; rotation \textit{(b)} is the equivalent of the RR rotation used with AVL trees.

\subsection{Double Rotations}
Double rotations involve the accessed node \verb|S|, the parent \verb|P| and the grandparent \verb|G|. The effect of these rotations is to move the node up two levels in the tree.

\subsubsection{ZigZag Rotation}
There are two variations of the \textit{zigzag} rotation. 
\begin{figure}[H]
    \centering
    \includesvg[width=0.8\textwidth]{assets/splay-zigzag.svg}
    \caption{Splay tree rotations (ZigZag)}    
\end{figure}

Rotation \textit{(c)} is the equivalent of the RL rotation used with AVL trees; rotation \textit{(d)} is the equivalent of LR rotation used with AVL trees. The ZigZag rotations tend to make the trees more balanced, and they will reduce the height of the tree by 1. 

\subsubsection{ZigZig Rotation}
There are two variants of the \textit{ZigZig} rotation.
\begin{figure}[H]
    \centering
    \includesvg[width=0.8\textwidth]{assets/splay-zigzig.svg}
    \caption{Splay tree rotations (ZigZig)}
\end{figure}

Pseudocode to the effect of rotation \textit{(e)} is shown below:
\begin{verbatim}
    G.left = P.right
    P.right = G

    P.left = S.right
    S.right = P
\end{verbatim}

Pseudocode to the effect of rotation \textit{(f)} is shown below:
\begin{verbatim}
    G.right = P.left
    P.left = G

    P.right = S.left
    S.left = P
\end{verbatim}