\taughtsession{Async lecture}{Introduction to Data Structures and ADT}{2023-09-30}{}{}{}

\section{Data Structures}
A \textit{Data Structure} is a way to store and organise data in order to facilitate access and modification. There is no single data structure which is perfect for every application - we need to choose the best for whatever we are creating. \\

There are two parts to a data structure: a collection of elements, each of which is either a data type of another data structure; and a set of associations or relationships (the structure) involving the collection of elements.

\subsection{Classification of Data Structures}
Data structures can be classified based on their predecessor and successor.

\begin{table}[H]
    \centering
    {\RaggedRight
    \begin{tabular}{p{0.2\textwidth} p{0.2\textwidth} p{0.2\textwidth} p{0.3\textwidth}}
    \textbf{Name} & \textbf{Predecessor} & \textbf{Successor} & \textbf{Examples}\\
    \hline
    \hline
    Linear & unique & unique & stack, queue\\
    \hline
    Hierarchical & unique & many & family tree, management structure\\
    \hline
    Graph & many & many & railway map, social network\\
    \hline
    Set Structure & no & no & DSALG class\\
    \hline
    \end{tabular}
    } % end of rr     
    \caption{Classifications of Data Structures}
    \end{table}

\subsection{Choosing the right Data Structure}
When choosing a data structure, it is important to analyse the problem, determine the basic operations needed and select the most efficient data structure. Choosing the right data structure will make the operations simple \& efficient and choosing the wrong data structure will make your operations cumbersome and inefficient.

\subsection{CRUD}
\textit{CRUD Operations}: Create, Read, Update and Delete are the basic operations which all data structures must be able to do. It is common for a data structure to use a different name to refer to the operation, however. 

\section{Abstract Data Type}
An Abstract Data Type (ADT) is a collection of data and associated methods stored as a single module. The data within an ADT cannot be accessed directly, it must be accessed indirectly through its methods. An ADT consists of: the data structure itself; methods to access the data structure; methods to modify the data structure; and internal methods (which are not accessible from outside the ADT).

\section{Algorithms}
An \textit{Algorithm} is any well-defined computational procedure that takes some data, or set of data as input and produces some data or set of data as output. It is the sequence of computational steps which are gone through that transforms the input into the output which is the algorithm. The algorithm must process the data efficiently (both in terms of time and space).

\subsection{Classifications of Algorithms}
There are a number of different classifications of algorithms - four are shown below. Definitions are from 
