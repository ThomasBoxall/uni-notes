\taughtsession{Lecture}{IP Addresses \& Subnetting}{2023-10-02}{09:00}{Thanos}{}

\textit{NB: This page also covers this lecture and the following week's (2023-10-09) as the same slide deck \& topic was split across two weeks.}

\section{Layer 3 Functionalities}
Layer 3, in the OSI model, handles the routing of the data by delivering it to the correct destination. It is the layer which allows networks to communicate with each other.\\

The functionalities of layer 3 are spread all over the network - in ad hoc hardware (routers) and in PCs (through routing software by the operating systems)

\subsection{Internet Protocol, a reminder}
The Internet Protocol, IP, is a connectionless protocol which delivers datagrams through best effort delivery. This means it's not 100\% efficient at delivering data however it will try its best to deliver the data its supposed to deliver. Naturally, this introduces a level of unreliability - as there is no guarantee of orderly delivery. However, there is an error checking algorithm used whereby if the buffer is full or the error check fails, the packet is discarded and another protocol may issue the send again command.\\

The Internet Protocol also has a number of functions when used in data transmission and receiving. In transmission: encapsulates data from the transport layer into datagrams and prepares headers (the source and destination addresses, etc) as well as applying routing algorithms at routers and forwarding the datagram to the Network Interface Card of the device which is transmitting the datagram. When receiving, IP: checks the validity of incoming datagrams then reads the header; it then checks if forwarding is required and if it is, then it will send to the appropriate network interface to forward the packet and if not requried then it will pass the payload to the next upper layer of the OSI model.\\

The Internet Protocol also provides us with IP addresses, its this which we will focus on for the majority of the lecture.

\section{IP Addresses}
An IP address is a unique identifier used to identify different devices on the network. In regular operation, there are two types - \textit{IPv4} and the newer \textit{IPv6}. We will primarily be focusing on IPv4.\\

IPv4 uses a 32-bit string which has two notations.
\begin{description}
    \item[System Notation] uses a 32-bit string of binary.\\ For example \verb|10010011101000110001010000001001|
    \item[Dotted Notation (bin)] uses a 32-bit string of binary, with the bits divided into bytes.\\ For example \verb|10010011.10100011.00010100.00001001|
    \item[Dotted Notation (dec)] uses decimal representtion of the binary numbers, this is the most common to see as it is the most human friendly. As each section of the IP address is a byte, the range of decimal values is 0 to 255 inclusive..\\ For example, \verb|147.162.20.9|
\end{description}

\subsection{IP Address Structure}
Any IPv4 address is portioned into two fields. The first being the \textit{network address} and the second being the \textit{host address}. The network address is the same for every device on the network (e.g. \verb|192.168.xxx.xxx|) and the host address is the part which uniquely identifies that device on the network (e.g. \verb|xxx.xxx.101.236|). 

\subsection{Classful IP Addresses}
There are two ways to use IP Addresses, \textit{classful} and \textit{classless}. Classful is the older method which is being used less however we will cover this first the conver classless later in the module.\\

In classful IP addressing, the network ID can either be 8, 12 or 24 bits in length (this is either 1, 2, or 3 blocks). The first bits of the NetworkID, as shown in the diagram below, indicate which class a IP address belongs to.


\begin{figure}[H]
    \centering
    \includesvg[width=0.9\textwidth]{assets/ip-address-classes-primary.svg}
    \caption{Primary IP address classes and structure}
\end{figure}

There are also two additional classes, these are shown below.

\begin{figure}[H]
    \centering
    \includesvg[width=0.9\textwidth]{assets/ip-address-classes-secondary.svg}
    \caption{Primary IP address classes and structure}
\end{figure}

The table below shows the dotted decimal ranges which are allocated for the different classes.

\begin{table}[H]
    \centering
    {\RaggedRight
    \begin{tabular}{p{0.2\textwidth} p{0.2\textwidth} p{0.2\textwidth}}
    \textbf{Address Class} & \textbf{Start IP range} & \textbf{End IP range}\\
    \hline
    \hline
    Class A & 1.xxx.xxx.xxx & 126.xxx.xxx.xxx\\
    \hline
    Class B & 128.0.xxx.xxx & 191.255.xxx.xxx\\
    \hline
    Class C & 192.0.0.xxx & 223.255.255.xxx\\
    \hline
    Class D & 224.xxx.xxx.xxx & 239.xxx.xxx.xxx\\
    \hline
    Class E & 240.xxx.xxx.xxx & 255.xxx.xxx.xxx\\
    \hline
    \end{tabular}
    }
    \caption{Dotted Decimal ranges for classful IP addresses}
\end{table}

Despite the ranges shown above, there are some reserved IP address ranges for different purposes. These are shown below.
\begin{table}[H]
    \centering
    {\RaggedRight
    \begin{tabular}{p{0.2\textwidth} p{0.2\textwidth} p{0.2\textwidth} p{0.1\textwidth}}
    \textbf{Start IP range} & \textbf{End IP range} & \textbf{Purpose} & \textbf{Class}\\
    \hline
    \hline
    \verb|10.0.0.0| & \verb|10.255.255.255| & Non-Internet Routable LAN use & A\\
    \hline
    \verb|127.0.0.0| & \verb|127.255.255.255| & Localhost loopback address & -\\
    \hline
    \verb|172.16.0.0| & \verb|172.31.255.255| & Non-Internet Routable LAN use & B\\
    \hline
    \verb|192.168.0.0| & \verb|192.168.255.255| & Non-Internet Routable LAN use & C\\
    \hline
    \end{tabular}
    }
    \caption{Dotted Decimal ranges for classful IP addresses}
\end{table}

When referring to a network address of a given IP address, then all the HostID bits should be set to \verb|0|. For example, the IP address \verb|12.25.89.124| has the HostID of \verb|12.0.0.0|.

\section{Subnetting}
IPv4 provides us a theoretical maximum of 4,294,967,296 unique IP addresses. These are broken into three classes
\begin{description}
    \item[Class C] provides 254 assignable host addresses ($2^8 - 2$)
    \item[Class B] provides 65534 assignable host addresses ($2^{16} - 2$)
    \item[Class A] provides 16,777,214 assignable host addresses ($2^{24} - 2$)
\end{description} 

This is a very inflexible system, as there are only three boxes which everyone must fit into.
\subsection{Usable Host Addresses}
The number of usable host addresses for a given IP range can be calculated from the formula: total number of host addresses minus 2.\\

The following example will show this:
\begin{itemize}
    \item You have been assigned a class B network address (\verb|1928.147.0.0|)
    \item This gives the IP range \verb|128.147.0.0| - \verb|128.147.255.255|
    \item However! The first assignable address of it is \verb|128.147.0.1| as \verb|128.147.0.0| is the network address which is not assignable
    \item The last assignable address is \verb|128.147.255.254| as \verb|128.147.255.255| is the network's broadcast address - which is not assignable.
\end{itemize}

\subsection{Introduction to Subnetting}
Subnetting is the process of dividing one big network into several \textit{subnetworks}. Each subnet behaves as a physical network however they are not physically separated, just logically separated.\\

We use subnetting because despite the fact, for a class B network, we can accommodate 65534 hosts - its inefficient to do this and is a pain to manage. There are also performance drawbacks to not subnetting.\\

When subnetting, we introduce a new component of the IP address. n-bits of the HostID now become a SubnetID. This is used to identify the subnet. Commonly, for class B IP addresses, this is the third byte.

\subsection{Subnet Address and Mask}
In this example, we use the host IP address of \verb|148.197.9.18|\\ (\verb|10010100.11000101.00001001.00010010|). As this is a Class B IP address, it has the default subnet mask of \verb|255.255.0.0| (\verb|11111111.1111111.00000000.00000000|).\\

We now create a \textit{Custom Subnet Mask}, which is decided by the network admin and will be longer than the default class mask. It tells us where the new boundary between the NetworkID and HostID is. We will set the custom subnet mask to \verb|/21|, the subnet mask now reads as \verb|255.255.248.0|\\ (\verb|11111111.11111111.11111000.00000000|).\\

Ultimately, this gives us a new (sub)network ID of \verb|148.197.8.0/21|\\ (\verb|10010100.11000101.00001000.00000000|)

\subsection{How Many Subnets and Hosts?}
The number of subnets you can create is calculated from the formula $2^n$ where $n$ is the number of bits used to create the SubnetID. For example, if the SubnetID is \verb|255|, this uses 8-bits therefore $2^8=256$ subnets.\\

As the SubnetID is 8 bits long, this leaves the HostID with 8-bits. The number usable hosts per subnet can be calculated with the formula $2^n - 2$ where $n$ is the number of bits in the HostID. Using the above example, where the SubnetID is 8 bits therefore the HostID is 8 bits, we get $2^8 - 2 = 254$ usable host addresses. But why do we have to subtract 2. We have to subtract 2 from the total number of Host addresses because when the HostID bits are all 1s, this is the broadcast address for that network and where the HostID bits are all 0's is reserved for \textit{that} device on the network.