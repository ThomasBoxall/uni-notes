\taughtsession{Lecture}{Synchronisation \& Deadlock}{2023-10-17}{13:00}{Tamer}{}

\section{Synchronisation}
Beyond Mutual Exclusion, there are other kinds of synchronisation. \textit{Join Synchronisation} is used between parent and child where the \verb|join| operation in the parent can only complete when the child thread terminates. \textit{Barrier Synchronisation} takes effect across a group of \verb|N| processes and it works as such that no single thread can progress until all threads have reached their barrier operation. The final type of synchronisation is where thread \verb|i| sends a message to thread \verb|j|; this delays \verb|j|'s progression as naturally thread \verb|j| can't receive the message until thread \verb|i| has sent it.\\

Generally, synchronisation consists in a particular thread having to wait until some condition is created by one or more threads. The Semaphores which we used last week are a general mechanism used to achieve synchronisation.

\section{Resource Deadlock}
\subsection{Resources}
Computer Systems have many kinds of \textit{resource}. A single resource can be accessed by either a single process or single thready at a time. An example of this would be a shared data structure in the operation system (where we use MutEx to manage access to it for different threads) or a physical device such as a printer. 

\subsection{Deadlock}
\textit{Resource Deadlock} can occur when processes (or threads) need to acquire access to more than one exclusive resource. For example, a program might need to use the scanner and printer therefore it would require exclusive access of both of these resources.\\

The classic example of this is when you have two threads \verb|A| and \verb|B|, and two shared resources \verb|P| and \verb|S|. In this example \verb|A| already has exclusive access to \verb|P| and \verb|B| already has access to \verb|S|. However, \verb|A| also needs access to \verb|S| and \verb|B| also needs access to \verb|P|. This has caused a deadlock as both threads are waiting on access to a resource which is currently in use while neither realise they are in deadlock. 

\subsection{What is Deadlock?}
Deadlock is a situation where a process or a set of processes wait indefinitely for an event that can never occur.\\

In practice, a set of threads is in a resource deadlock state when every thread in the set is waiting on a resource which is being held by another thread in the set.\\

Resource deadlock can be modelled using a \textit{Resource Allocation Graph}, which shows the processes are requesting which resources and which resources have been granted to which processes.

\subsection{Resource Allocation Graph}
