\taughtsession{Lecture}{Synchronisation \& Deadlock}{2023-10-17}{13:00}{Tamer}{}

\section{Synchronisation}
Beyond Mutual Exclusion, there are other kinds of synchronisation. \textit{Join Synchronisation} is used between parent and child where the \verb|join| operation in the parent can only complete when the child thread terminates. \textit{Barrier Synchronisation} takes effect across a group of \verb|N| processes and it works as such that no single thread can progress until all threads have reached their barrier operation. The final type of synchronisation is where thread \verb|i| sends a message to thread \verb|j|; this delays \verb|j|'s progression as naturally thread \verb|j| can't receive the message until thread \verb|i| has sent it.\\

Generally, synchronisation consists in a particular thread having to wait until some condition is created by one or more threads. The Semaphores which we used last week are a general mechanism used to achieve synchronisation.

\section{Resource Deadlock}
\subsection{Resources}
Computer Systems have many kinds of \textit{resource}. A single resource can be accessed by either a single process or single thready at a time. An example of this would be a shared data structure in the operation system (where we use MutEx to manage access to it for different threads) or a physical device such as a printer. 

\subsection{Deadlock}
\textit{Resource Deadlock} can occur when processes (or threads) need to acquire access to more than one exclusive resource. For example, a program might need to use the scanner and printer therefore it would require exclusive access of both of these resources.\\

The classic example of this is when you have two threads \verb|A| and \verb|B|, and two shared resources \verb|P| and \verb|S|. In this example \verb|A| already has exclusive access to \verb|P| and \verb|B| already has access to \verb|S|. However, \verb|A| also needs access to \verb|S| and \verb|B| also needs access to \verb|P|. This has caused a deadlock as both threads are waiting on access to a resource which is currently in use while neither realise they are in deadlock. 

\subsection{What is Deadlock?}
Deadlock is a situation where a process or a set of processes wait indefinitely for an event that can never occur.\\

In practice, a set of threads is in a resource deadlock state when every thread in the set is waiting on a resource which is being held by another thread in the set.\\

Resource deadlock can be modelled using a \textit{Resource Allocation Graph}, which shows the processes are requesting which resources and which resources have been granted to which processes.

\subsection{Resource Allocation Graph}
\begin{figure}[H]
    \centering
    \includesvg[width=0.7\textwidth]{assets/resource-allocation-graphs.svg}
    \caption{Three resource allocation graphs}
\end{figure}
The above figure shows three different examples of the a resource allocation graph (RAG). In RAGs, a circle indicates a process and a square indicates a resource which can be used by the processes. The arrows between the processes and resources are important, as an arrow pointing from a process to a resource indicates that the process is waiting for that resource to become available and an arrow pointing from a resource to a process indicates that the process is holding that resource.\\

In the above example, resource R is assigned to process A; process B is waiting for resource S; and processes C \& D are in deadlock over resources T and U.\\

In a RAG, anytime there is a loop (or cycle), deadlock has occurred.

\section{Dealing with Deadlock}
One method used to deal with Deadlock is called \textit{deadlock detection and recovery}. In this method, you allow the system to enter the deadlock state then run detection algorithms periodically to check if the system has entered deadlock or not; if deadlock is detected, it performs a recovery scheme to get out of the deadlock. To detect the deadlock, it searches for cycles in the Resource Allocation Graph.

\subsection{Deadlock Recovery}
The most efficient way to recover from a deadlock is to kill processes until the deadlock cycle is eliminated. This then means that the surviving processes get access to the resources and that they can continue; the killed processes must attempt to access the resources again which hopefully won't result in a deadlock this time!\\

This process of killing off processed when deadlock occurs is commonly used in \textit{Relational Database Management Systems} where multiple transactions attempting to gain access to the same record causes a deadlock. The changes made can be ``rolled back'' which means the clients accessing the RDBMS can try again. 

\subsection{Deadlock Avoidance}
When designing systems and writing code, it is much better to keep the system \textit{safe} which means to avoid entering \textit{unsafe} states which may turn into a deadlock later.\\

Modern devices come with built in deadlock avoidance mechanisms - these delay acquisition of any resource if acquiring it would allow the system into an unsafe region.\\

Dijkstra's \textit{Banker's Algorithm} can be used to avoid deadlocks. This works by requiring all processes to declare the maximum number of resource units that they may request. It then keeps track of the current allocation for each process and their currents needs. When it receives a request, it pretends to honour the request and tries to fulfill the needs of all the other processes in some order so it can check what state will occur (safe or unsafe) if it grants the request - if it leads to a safe state then the request is granted and if not, then the request is denied. 
