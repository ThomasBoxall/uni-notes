\taughtsession{Lecture}{Agile}{2023-10-13}{11:00}{Claudia}{}

\section{History of Agile}
The agile development methodology came about when developers were getting tired of the constraints and paperwork requirements of Waterfall and other development methodologies of the day. The methodology was published in February 2001 by a number of extremely experienced developers who said that they value the following during software development
\begin{itemize}
    \item Individuals and interactions (over processes and tools)
    \item Working software (over comprehensive documentation)
    \item Customer collaboration (over contract negotiation)
    \item Responding to change (over following a plan)
\end{itemize}
Whilst the \textit{Agile Manifesto} doesn't hate on the bracketed alternatives (above) too much, it does justify why it prefers the alternatives:
\begin{description}
    \item[Responding to change] ``make a detailed plan for the next week, rough plan for the next 3 months and extremely crude plans beyond that''
    \item[Customer Collaboration] ``the best contracts are those that govern the way the development team and the customer will work together''
    \item[Individuals and interactions] ``a good process will not save a project from failure if the team doesn't have strong players, but a bad process can make even the strongest of players ineffective''
    \item[working Software] ``software without documentation is a disaster. [...] however, too much documentation is worse than too little''
\end{description}

\section{Agile Development Terminology}
\begin{description}
    \item[Customer] the person / group that defines and prioritises features
    \item[Release Plan] maps out the next six (or so) iterations which has details of the next release
    \item[Refactoring] rewriting code to change structure but not behaviour
    \item[User Stories] are tokens of a conversation about a requirement
    \item[Iteration Plan] the developer selects the number of stories to make into tasks
    \item[Pair Programming] where pairs of programmers co-author code
\end{description}
\subsection{Pair Programming}
Pair Programming is a commonly used technique used in Agile Development as it allows for quicker peer-reviewing of code to take place rather than having to wait for the code to be submitted and reviewed formally when a feature is complete. In Pair Programming, one programmer types the code and the other watches for errors, these roles change often and the code is marked as being authored by both programmers. The membership of pairs is changed very frequently - sometimes with pairs changing during a single working day (therefore, each programmer is in two different pairs in a day), this leads to every member of the team working with every other member at different points.

\section{Scrum}
Scrum is one of the Agile Software Development methods. Another is Extreme Programming (not covered in this lecture).\\

Scrum is a efficiency focused method whereby teams meet daily and work in a series of sprints. A sprint will have a small, achievable, goal. Daily ``stand up'' meetings are held to ensure that the entire team is on the same page with each other. This is lead by a \textit{Scrum Master}. There will be a longer backlog of tasks which get selected from to build the sprints. The \textit{Product Owner} represents the customer and prioritises the requirements. The practices of Scrum are shown below.
\begin{description}
    \item[Sprint Planning Meeting] is attended by all, and the Product Owner (PO) selects tasks from the backlog and defines the goal for the sprint
    \item[Sprint Review Meeting] is where the team demos the product increments to the PO at the end of the sprint
    \item[Sprint Retrospective] is where the team and Scrum Master (SM) review the just-finished sprint at the end of it, analysing what went well and what can be improved
    \item[Daily Scrum] is the 15-minute meeting attended by the team and SM
    \item[Product Backlog] is the list of all functionality desired for the system; it is written and prioritised by the PO and updated by the SM
    \item[Sprint Backlog] is the list of functionality planned for the sprint. It is a subset of the product backlog
    \item[Sprint burndown chart] is a chart which shows progress on a daily basis 
\end{description}

\section{When to use Agile}
Due to the nature of Agile development, there are times when it is a good tool to use and times when it is a bad tool to use.
\begin{itemize}
    \item When the requirements change often
    \item When the product you are building is small-to-medium sized
    \item When the team building the product is small-to-medium sized
    \item When the customer is prepared to be involved as a stakeholder
    \item When there are a limited number of external rules and regulations which the software has to comply with
\end{itemize}

\section{When not to use Agile}
\begin{itemize}
    \item When developing safety critical systems
    \item When the team is larger or when the team is distributed geographically
    \item When the customer isn't experienced with Agile projects
    \item When the customer needs to be involved all the way through as it can be difficult to keep the customer involved in the process
    \item When it is not possible to get a strict contract drawn-up up front
\end{itemize}