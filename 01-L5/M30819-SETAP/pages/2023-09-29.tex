\taughtsession{Lecture}{A brief introduction}{2023-09-29}{11:00}{Claudia}{}

\section{What is Software Engineering Theory and Practice About?}
"How to engineer software."\\

Software Engineering Theory and Practice (SETAP) teaches us how to engineer software, which is different from hacking software. Until now - the majority of development we have done has probably been through hacking software. Engineering implies a process, a set of steps we follow every time in order to be able to replicate what we are doing.\\

SETAP is not just another programming module. We need to know how to program and be familiar (or willing to learn) a language chosen in our groups but we won't be given programming stuff to learn. The majority of the programming will come in Teaching Block 2, when we implement our applications. TB1 will primarily be for analysis and design of the solution. While a portion of our final mark will come from our programming ability, it will not be the entirety of our final mark.\\

The module is designed to make us work as part of a team, and improve our skills at that. The thing that is important to remember is that when working in a team, things take longer to complete.\\

As part of our project, we will be writing documentation, more information to come about this at some point.

\subsection{Assessments}
Working in groups of 5-6 students, we will develop a medium size application. There are two submission points (Friday 15 December 2023 and Friday 10 May 2024) at which different things will need to be handed in. Each submission is equally weighted at 50\%.

\section{Rules}
A number of rules have been designed to make it clear how this coursework is assessed.
\begin{enumerate}
    \item Both submissions and all their components (e.g. code repository and demo) are team submissions
    \item An overall mark will be assigned to each submission based on the merits of its content. We will call this \verb|MarkOverall|.
    \item Each submission must be accompanied by a \textit{Contributions Table}
    \item All team members must agree with and sign off the information provided in the Contributions Table. If someones name does not appear in the table, the assumption is you did not contribute anything.
    \item Individual marks will be decided based on the percentages included in the contributions table as follows:
        \[ \mathrm{IndividualMark} = \mathrm{MarkOverall} - ( \mathrm{MaximumContribution} - \mathrm{IndividualContribution}) \]
    \item If everyone on the team made an equal contribution, all team members should be assigned the same percentage, i.e. 
        \[\frac{100}{\mathrm{NoOfTeamMembers}}\]
    \item Submitting a Contributions Table where the percentages do not add up to 100\% will lead to all members of the team being assigned the MarkOverall.
    \item Failure to submit a Contributions Table will lead to all members of the team being assigned MarkOverall.
    \item Mediation support is available for completing the Contributions Table in cases where any (or more) members of a team does not agree with the information provided in the table. (\textit{Please make sure you contact the module coordinator in good time - at least 10 days before the deadline}).
    \item Disagreements in relation to the Contributions Table brought up 2 days before the deadline or later will not be considered for mediation support.
\end{enumerate}
