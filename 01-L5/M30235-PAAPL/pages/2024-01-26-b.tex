\taughtsession{Lecture}{Overview and Evaluation of Programming Languages}{2024-01-26}{14:00}{Jaicheng}{}

\section{The `TPK' Algorithm}
The TPK algorithm was designed by \textit{Trabb, Pardo} and \textit{Knuth} in the 1970s for illustration purposes. It is designed to:
\begin{enumerate}
    \item read 11 numbers (entered by the user using their keyboard) into an array,
    \item process the array in reverse order, applying a mathematical function to each value
    \item then for each value - reporting the value or a message saying that the value is too large
\end{enumerate}
The algorithm includes all the basic constructs which would be expected to exist in a modern language therefore making it useful to use when understanding how languages work. A pseudocode implementation of the TPK algorithm is below:
\begin{verbatim}
    input 11 numbers into a sequence A
    reverse sequence A
    for each item in sequence A
        call a function to do an operation
        if result overflows
            alert user
        else
            print result
\end{verbatim}

\section{Fortran}
Fortran (\textit{For}mula \textit{Tran}slation) is the first well-known high-level programming language. It was developed by a team at IBM led b John Backus with the goals: to lower the costs involved with programming and debugging; and to compete with ``hand coded'' assembly language programs in terms of execution speed. The first Fortran compiler, built for the IBM 704 mainframe, was completed in 1957. \\

Early source code had a strict, specific, format which was in part due to it being a punched-card program where the column and row position of the punch is important.\\

The TPK algorithm in Fortran is shown below:
\begin{verbatim}
    C THE TPK ALGORITHM IN FORTRAN
      FUNF(T)=SQRTF(ABSF(T))+5.0*T**3
      DIMENSION A(11)
    1 FORMAT(11F12.4)
      READ 1, A
      DO 10 J=1,11
      I=11-J
      Y=FUNF(A(I+1))
      IF(400.0-Y) 4,8,8
    4 PRINT 5,I
    5 FORMAT(I10,10H TOO LARGE)
      GOTO 10
    8 PRINT 9,I,Y
    9 FORMAT(I10,F12.7)
    10 CONTINUE
      STOP
\end{verbatim}

A letter \verb|C| in the first column indicated that the card was a comment and as such it should be ignored by the compiler. Non-Compiler cards were divided into four fields:
\begin{description}
    \item[1-5] is the label field; a sequence of digits here were taken as a label for the purpose of 
\end{description}

%continue with above