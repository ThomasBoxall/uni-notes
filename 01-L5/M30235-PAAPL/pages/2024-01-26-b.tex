\taughtsession{Lecture}{Overview and Evaluation of Programming Languages}{2024-01-26}{14:00}{Jaicheng}{}

\section{The `TPK' Algorithm}
The TPK algorithm was designed by \textit{Trabb, Pardo} and \textit{Knuth} in the 1970s for illustration purposes. It is designed to:
\begin{enumerate}
    \item read 11 numbers (entered by the user using their keyboard) into an array,
    \item process the array in reverse order, applying a mathematical function to each value
    \item then for each value - reporting the value or a message saying that the value is too large
\end{enumerate}
The algorithm includes all the basic constructs which would be expected to exist in a modern language therefore making it useful to use when understanding how languages work. A pseudocode implementation of the TPK algorithm is below:
\begin{verbatim}
    input 11 numbers into a sequence A
    reverse sequence A
    for each item in sequence A
        call a function to do an operation
        if result overflows
            alert user
        else
            print result
\end{verbatim}

\section{Fortran}
Fortran (\textit{For}mula \textit{Tran}slation) is the first well-known high-level programming language. It was developed by a team at IBM led by John Backus with the goals: to lower the costs involved with programming and debugging; and to compete with ``hand coded'' assembly language programs in terms of execution speed. The first Fortran compiler, built for the IBM 704 mainframe, was completed in 1957. \\

Early source code had a strict, specific, format which was in part due to it being a punched-card program where the column and row position of the punch is important.\\

The TPK algorithm in Fortran is shown below:
\begin{verbatim}
    C THE TPK ALGORITHM IN FORTRAN
      FUNF(T)=SQRTF(ABSF(T))+5.0*T**3
      DIMENSION A(11)
    1 FORMAT(11F12.4)
      READ 1, A
      DO 10 J=1,11
      I=11-J
      Y=FUNF(A(I+1))
      IF(400.0-Y) 4,8,8
    4 PRINT 5,I
    5 FORMAT(I10,10H TOO LARGE)
      GOTO 10
    8 PRINT 9,I,Y
    9 FORMAT(I10,F12.7)
    10 CONTINUE
      STOP
\end{verbatim}

A letter \verb|C| in the first column indicated that the card was a comment and as such it should be ignored by the compiler. Non-Compiler cards were divided into four fields:
\begin{description}
    \item[1-5] is the label field; a sequence of digits here indiciates the purpoose of the card and therefore the instruction.
    \item[6] is a continuation field whereby a non-blank character here caused the card to be taken as a continuation of the statement on the previous card. 
    \item[7-72] is the statement field
    \item[73-80] are ignored by the compiler. This means that they can be used for card identification purposes in the event that the cards were dropped.
\end{description}

The restrictions on the structure of the code were removed in Fortran 90, where it became a Free-Form language. 

\section{COBOL}
COBOL (\textit{CO}mmon \textit{B}usiness \textit{O}riented \textit{L}anguage) was created at the end of the 1950s by the US Department of Defence. It was initially developed as a language for business data processing from which comes its verbose syntax that was designed with the ability for managers to be able to read it in mind. COBOL was never designed to be used as a scientific language and has many critics, where programmers felt that the verbosity of the language increased program length, not readability. 

\section{Algol}
Algol (\textit{ALGO}righmic \textit{L}anguage) was originally designed to overcome the problems with Fortran in the late 1950s. Arguably, it is one of the most successful high level programming languages of the time  because it was influential over the design of subsequent high level languages.\\

Algol 60 introduced the use of formal notation for syntax, block structure (with locally defined variables, whoop whoop), supported recursive procedures (until this point, you could do it however the languages didn't like it) and readable if and for statements. Ultimately, Algol died out with the rise in Fortran's popularity. The TPK algorithm in Algol is shown below.
\begin{verbatim}
  begin
    comment TPK algorithm in Algol 60;
    integer i; real y; real array a[0:10];
    real procedure f(t); real t; value t;
      f := sqrt(abs(t))+5*t^3;
    for i := 0 step 1 until 10 do
      read(a[i]);
    for i := 10 step -1 until 0 do begin
      y := f(a[i]);
      if y > 400 then
        write(i, "TOO LARGE")
      else
        write(i, y);
    end
  end
\end{verbatim}

\section{Pascal}
Pascal is a direct descendant of Algol, which was intended to be more efficient in order to compete with Fortran as a general purpose language. An early Pascal compiler was designed to be portable, compiling the source code to a virtual machine (\textit{P-Code}). Pascal was popularised in the late 1970s as a good teaching language as it enforced a ``good'' programming style, it was especially popular amongst Universities. Pascal is still in development, with more recent versions adding modules and classes (for example, the Object Pascal Language, which is sometimes known as Delphi). The TPK algorithm in Pascal is shown below:
\begin{verbatim}
  program example(input, output); (* TPK alg in Pascal *)
  var i : integer; y : real; a : array [0..10] of real;
  function f(t : real) : real;
  begin
    f := sqrt(abs(t)) + 5*t*t*t
  end;
  begin
    for i := 0 to 10 do read(a[i]);
      for i := 10 downto 0 do
        begin
          y := f(a[i]);
          if y > 400 then
            writeln(i,' TOO LARGE')
          else
            writeln(i, y)
        end
  end.
\end{verbatim}

\section{Systems Programming: C}
In the early 1970s, it was decided that a more efficient language would be required for systems programming (such as compilers, operating systems, etc) as the languages created during the 1960s (Fortran and COBOL) weren't sufficient. C was developed alongside the UNIX operating system at Bell Labs; and has proven to be a very successful in systems programming and as a general-purpose programming language. It combines high-level features, such as functions \& loops with low-level operations, such as arithmetic on memory addresses. Critics argue that C code is less readable and it's weak typing causes problems. Shown below is the TPK algorithm written in C.
\begin{verbatim}
  #include <stdio.h>/* TPK algorithm in ANSI C */
  #include <math.h>
  double f(double t) {
    return sqrt(fabs(t)) + 5*pow(t,3);
  }
  main() {
    int i; double y; double a[11];
    for (i = 0; i <= 10; i++)
      scanf("%lf", &a[i]); /* %lf means long double*/
    for (i = 10; i >= 0; i--) {
      y = f(a[i]);
      if (y > 400)
        printf("%d TOO LARGE\n", i); /* %d means double*/
      else
        printf("%d %lf\n", i, y);
    }
  }
\end{verbatim}

\section{Object Oriented Languages}
A major application of computers is the simulation of real-world systems (for example, hospital waiting lists, industrial production lines, etc). Early programming languages were developed specifically for simulation which include GPSS and Simula 1. The designers of Simula (Dahl and Nygaard) introduced the concept of a \textit{class} to their programs to represent simulated entities. This was the first `object-oriented language' as we know them, and therefore Simula 67 can be considered the parent OOL. Smalltalk in 1980 and Eiffel in 1986 were influential in the further development of OOLs; which was furthered in 1985 by Bell Labs who developed C++, the OOL cousin to C. Java and C\# are considered descendants of C++. 

\section{Scripting Language}
When Scripting Languages were first introduced, they were brought in to automated the task which a human operator could do, for example shell scripts. However nowerdays a scripting language loosely refers to high-level general-purpose programming languages such as: Pearl, Python, PHP, Ruby, JavaScript, and Matlab. Scripting languages are generally typeless with relatively simple syntax and semantics; they are usually interpreted and are, by design, fast to learn and write in. Shown below is the TPK algorithm in Python:
\begin{verbatim}
  from math import sqrt
  def f(t):
    return sqrt(abs(t))+5*t**3
  a = [input() for i in range(11)]
  for i in range(10,-1,-1): # range() is equiv to [10,-1)
    y = f(a[i])
    if y > 400:
      print i, "TOO LARGE"
    else:
      print i, y
\end{verbatim}

\section{Logical Programming (Prolog)}
Logical Programming is a type of programming which is built upon logical statements which are comprised of variables, constants and structures. In Prolog, variables begin with capital letters, constants are either atoms or integers and structures consist of a functor and arguments.\\

A Prolog program consists of facts and rules. Prolog programs are used to answer queries, although simple arithmetic operations are possible. A query is a fact or rule that initiates a search for success in a Prolog program. It specifies a search goal by naming variables that are of interests. 

\section{Language Evaluation Criteria}
There are a number of criteria on which the readability, writability and reliability of a programming language can be evaluated.
\subsection{Simplicity}
The simplicity of a language is defined based on the simplicity of the syntax and the number of constructs (small number of constructs is simple). Programmers will tend to only learn the part of a large language which suits them and therefore probably not use the best construct / syntax for what they are trying to achieve. It is, obviously, possible to make a language too simple for example Assembly Language. 

\subsection{Lexical Elements}
The form that the individual lexical elements (i.e. words or symbols) or a language take can affect the languages' readability. The meaning of a symbol / keyword should ideally be obvious from it's name. 

\subsection{Orthogonality}
Orthogonality in a programming language means that it has a relatively small number of control and data constructs which can be combined in a relatively small number of ways and every possible combination is legal and meaningful. When using an orthogonal language, a programmer is not required to remember a lot of ``special cases'' in the use of it's constructs. The orthogonality of a language influences both the readability and the writability of software; if a language's rules contain fewer special cases, it is easier to learn. 

\subsection{Data Types}
It is also important for a language to have a rich set of data types; as well as having adequate mechanisms for combining types. 

\subsection{Expressiveness}
The expressiveness of a programming language relates to how much code is required to implement computations. 