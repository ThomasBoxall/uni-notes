\taughtsession{Lecture}{Functions}{2024-02-06}{17:00}{Janka}{}

A \textit{function} can be described in two ways. The mathematical definition is that ``a function is a special type of relation in which a single input will have at most one output''. The alternative definition of a function is that it is a mysterious black box which takes an input and returns and output. The same function, with the same input will always return the same output.\\

If we take $A$ and $B$ to be nonempty sets, then: $A$ is a (total) function $f$ from $A$ to $B$, $f: A \rightarrow B$, is a relation from $A$ to $B$ such that
\begin{center}
    for all $x \in A$ there is exactly on element in $B, f(x)$
\end{center}
associated with $x$ by relation $f$. Note that the word `total' is used to describe the above function which means that every input has a defined output. The function $f: \mathbb{Z} \rightarrow \mathbb{Z}$ which is defined by $f(x) = 2x$ is also an example of a total function. \\

It is also possible to have a `partial' function, this is where some of the inputs do not have defined outputs. For example the function $f(\frac{1}{x})$ where $x=0$, would be undefined therefore the function is classed as `partial'. 

\section{Describing A Function}
There are a few different ways in which a function can be described.
\subsection{By A Formula}
This is the most common method used. The function $f$ from $\mathbb{N} \rightarrow \mathbb{N}$ that maps every natural number $x$ to its cube $x^3$ can be described as:
\[f(x) = x^3\]
\subsection{By All Possible Associations}
Whilst this is a valid method, it will generally not be used for efficiency reasons. The function $g$ from $A = \{a, b, c\}$ to $B=\{1, 2, 3\}$ would be shown as:
\[g(a) = 1, g(b) = 1, g(c) = 2\]

\section{Domain, Co-Domain \& Range}
The domain of a function is the set of all input values for which there is a defined output. For example if we let $f: A \rightarrow B$ then the subset $D \subset A$ of all elements for which $f$ is defined is the domain. In the case of a total function, $D = A$ and in the case of a partial function, $ D \subseteq A$\\

The co-domain of a function is the set of all possible output values; not just the ones which map to an input. For example, if we let $f: A \rightarrow B$ then the set $B$ is the co-domain. \\

The range (also sometimes known as the image) of a function is the set of elements in the co-domain which map to an input. For example, if we let $f: A \rightarrow B$ then the range is denoted by $range(f)$. The range can also be expressed as:
\[range(f) = \{f (x) | x \in A\}\]

\section{Properties of Functions}
Functions have a number of properties. 
\subsection{Injective}
The function $f: A \rightarrow B$ is injective (or one-to-one) if there is only one input that maps to each output. It can mathematically be defined as:
\[\mathrm{for\ all\ } x,y \in A \mathrm{\ if \ } x \neq y \Rightarrow f(x) \neq \Rightarrow f(y)\]

\subsection{Surjective}
A function $f: A \rightarrow B$ is surjective (or onto) if the $range(f)$ is the co-domain $B$. It can mathematically be defined as
\[\mathrm{for\ all\ } y \in B \mathrm{\ there\ exists\ } x \in A \mathrm{\ such\ that\ } f(x) = y \]
A function which is not \textit{onto} is \textit{into}.

\subsection{Bijective}
A function $f: A \rightarrow B$ is bijective (or one-to-one correspondence) if it is both injective and surjective.

\section{Composite Functions}
A new function can be constructed by combining other simpler functions in some way. If we let $f: A \rightarrow B$ and $g B \rightarrow C$ be functions. The composition of $g$ with $f$ is the function denoted by $g \circ f: A \rightarrow C$ and defined by:
\[(g\circ f) (x) = g(f(x)) \mathrm{\ for\ all\ } x\in A\]
$(g\circ f) (x) = g(f(x))$ is read as $g$ of $f$, which means do $f$ first then do $g$. 

\section{Inverse Functions}
An inverse function is where the output of function $f$ can be fed into the input of function $f^{-1}$ to get the original input of $f$. This is mathematically defined as: $f: X \rightarrow Y$ is a bijective function, then there is an inverse function $f^{-1}: X \rightarrow Y$ that is defined as:
\[f^{-1} (y) = x \mathrm{\ if\ and\ only\ if\ } f(x) = y \]
