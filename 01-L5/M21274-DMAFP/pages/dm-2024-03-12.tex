\taughtsession{Lecture}{Graphs: An Introduction}{2024-03-12}{1700}{Janka}{}

\section{History of Graphs}
Euler introduced the concept of \textit{graph theory} in 1736 when he abstracted the Bridges of K\"{o}nigsberg problem into a series of \textit{vertices} and \textit{edges}. Graphs were further developed with the Travelling Salesman Problem whereby adding \textit{weighting} values to an edge allows the most efficient route to be calculated.\\

Graphs can be a useful tool in solving a number of different mathematical problems, including those which are abstractions of a real world thing - for example the cities in which the travelling salesman needs to travel through in their problem. There are four main applications of graph theory today:
\begin{description}
    \item[Existence Problems] are problems in which we want to prove the existence (or lack thereof) of something. For example the K\"{o}nigsberg bridge problem.
    \item[Optimisatoin Problems] are problems in which we are looking to find the best (most efficient) solution (this can either be a maximisation or minimisation). For example - the Travelling Salesman Problem.
    \item[Construction Problems] are problems in which we are trying to prove if a solution exists and if it does - how it can be constructed. For example - finding a walk in the K\"{o}ingsberg bridge problem
    \item[Enumeration Problems] are problems in which we are looking for how many objects have a given property. For example - how many optimal routes can be found in the Travelling Salesman Problem.   
\end{description}

\section{Basic Terminology}
A graph, $G$ is a pair $(V, E)$ of sets:
\begin{itemize}
    \item[$V$] is a non-empty set of vertices (nodes)
    \item[$E$] is a set of edges. Each element of $E$ is a set of two distinct elements of $V$
\end{itemize}

% oh boy, here comez the tikz

\begin{minipage}{0.4\textwidth}
    \begin{figure}[H]
        \centering
        \begin{tikzpicture}[node distance= 1.5cm,]
            \node[graphnode, draw] (A) {$u$};
            \node[graphnode, draw] (B) [right of=A] {$v$};

            \path (A) edge[above] node[align=center] {$e$} (B);
        \end{tikzpicture}
        \caption{Example Graph}
    \end{figure}
\end{minipage} \hfill
\begin{minipage}{0.55\textwidth}
If $e \in E$, then $e = \{u, v\}$, $u$ and $v$ are different elements of $V$ called the end vertices of $e$ (where $e$ joins vertices $u$ and $v$). 
\end{minipage}\vspace{0.5em}

The notation for $v$ and $u$ changes too; we can use $uv$ instead of $e = \{u, v\}$ for the edge $e$, meaning $uv$ and $vu$ are the same edge. The vertices $u$ and $v$ are said to be \textit{incident} with the edge $uv$ or that they are \textit{adjacent} because they are the end vertices of an edge.\\

We are only considering Finite Graphs in this module. These are graphs in which all their elements are defined and are capped in a certain place.

\begin{minipage}{0.3\textwidth}
    \begin{figure}[H]
        \centering
        \begin{tikzpicture}[node distance= 1.5cm,]
            \node[graphnode, draw] (A) {$v_1$};
            \node[graphnode, draw] (B) [right of=A] {$v_2$};
            \node[graphnode, draw] (C) [above of=A] {$v_3$};
            \node[graphnode, draw] (D) [above of=B] {$v_4$};
            \node[graphnode, draw, yshift=-0.75cm] (E) [above=of $(C)!0.5!(D)$] {$v_5$};
            \node[graphnode, draw, yshift=-1.2cm] (F) [above=of $(A)!0.5!(B)$] {$v_6$};

            \path (A) edge[above] node[align=center] {} (B);
            \path (B) edge[above] node[align=center] {} (D);
            \path (C) edge[above] node[align=center] {} (E);
            \path (C) edge[above] node[align=center] {} (A);
            \path (D) edge[above] node[align=center] {} (C);
            \path (D) edge[above] node[align=center] {} (E);
        \end{tikzpicture}
        \caption{Finite Graph}
    \end{figure}
\end{minipage} \hfill
\begin{minipage}{0.65\textwidth}
\begin{align*}
    V &= \{v_1, v_2, v_3, v_4, v_5, v_6\} \\
    E &= \{\{v_1, v_2\}, \{v_2, v_4\}, \{v_1, v_3\}, \{v_3, v_4\}, \{v_3, v_5\}, \{v_4, v_5\}\} \\
    \mathrm{or\ } E &= \{v_1v_2, v_2v_4, v_1v_3, v_3v_4, v_3v_5, v_4v_5\}
\end{align*}
\end{minipage}\vspace{0.5em}

Note that although the diagrams of the graph can look nice, and help us to visualise them - they are not actually important for most of our problems.

\subsection{Multigraph}
A multigraph / pseudograph is like a graph, however it may contain loops and/or multiple edges. In this module, there is not a formal definition of one and we won't study them too much.

\begin{minipage}{0.45\textwidth}
\begin{figure}[H]
    \centering
    \begin{tikzpicture}[node distance= 1.5cm,]
        \node[graphnode, draw] (A) {$v_1$};
        \node[graphnode, draw] (B) [above of=A] {$v_2$};
        \node[graphnode, draw] (C) [above of=B] {$v_3$};
        \node[graphnode, draw] (D) [right of=B] {$v_4$};

        \path (A) edge[above] node[align=center] {} (D);
        \path (C) edge[above] node[align=center] {} (D);
        \path (B) edge[above] node[align=center] {} (D);

        \path (A) edge[out=110, in=250, looseness=2] (B);
        \path (A) edge[out=70, in=290, looseness=2] (B);
        \path (B) edge[out=110, in=250, looseness=2] (C);
        \path (B) edge[out=70, in=290, looseness=2] (C);

    \end{tikzpicture}
    \caption{Graph with multiple edges}
\end{figure}
\end{minipage} \hfill
\begin{minipage}{0.45\textwidth}
\begin{figure}[H]
    \centering
    \begin{tikzpicture}[node distance= 1.5cm,]
        \node[graphnode, draw] (A) {$v_1$};
        \node[graphnode, draw] (B) [above of=A] {$v_2$};
        \node[graphnode, draw] (D) [right of=B] {$v_3$};

        \path (A) edge[above] node[align=center] {} (D);
        \path (B) edge[above] node[align=center] {} (D);

        \path (A) edge[out=110, in=250, looseness=2] (B);
        \path (A) edge[out=70, in=290, looseness=2] (B);

        \path (B) edge[out=70, in=110, looseness=8] (B);
        \path (D) edge[out=70, in=110, looseness=8] (D);

    \end{tikzpicture}
    \caption{Graph with loops}
\end{figure}
\end{minipage}

\subsection{Degree of a Vertex}
The number of edges incident with a vertex, $v$, is called the degree of $v$ and is denoted by $\deg v$. A vertex of degree $0$ is said to be isolated.

\begin{minipage}{0.45\textwidth}
\begin{figure}[H]
    \centering
    \begin{tikzpicture}[node distance= 1.5cm,]
        \node[graphnode, draw] (A) {$v_1$};
        \node[graphnode, draw] (B) [above of=A] {$v_2$};
        \node[graphnode, draw] (C) [right of=B] {$v_3$};
        \node[graphnode, draw] (D) [right of=C] {$v_4$};
        \node[graphnode, draw] (E) [below of=D] {$v_5$};
        \node[graphnode, draw] (F) [right of=E] {$v_6$};
        \node[graphnode] (X) [right of=D] {};
        \node[graphnode, draw] (G) [right of=X] {$v_7$};

        \path (B) edge[above] node[align=center] {} (C);
        \path (A) edge[above] node[align=center] {} (C);
        \path (C) edge[above] node[align=center] {} (D);
        \path (D) edge[above] node[align=center] {} (F);
        \path (F) edge[above] node[align=center] {} (G);

        \path (A) edge[out=110, in=250, looseness=2] (B);
        \path (A) edge[out=70, in=290, looseness=2] (B);

        \path (B) edge[out=70, in=110, looseness=8] (B);
        \path (C) edge[out=70, in=110, looseness=8] (C);

    \end{tikzpicture}
    \caption{Degree of Vertex example graph}
\end{figure}
\end{minipage} \hfill
\begin{minipage}{0.45\textwidth}
\begin{align*}
    & \deg V_1 = 2, && \deg V_6 = 2, \\
    & \deg V_7 = 1, && \deg V_5 = 0, \\
    & \deg V_1 = 3, && \deg V_2 = 5, \\
    & \deg V_3 = 5 && 
\end{align*}
\end{minipage}\vspace{0.5em}

Note that $V_5$ is an isolated vertex in the above example. 

\subsection{Degree Sequence}
When $d_1$, $d_2$, \ldots, $d_n$ are the degrees of the vertices of a graph (or multigraph), $G$, ordered so that $d_1 \leq d_2 \leq \cdots d_n$ Then $(d_1, d_2, \ldots, d_n)$ is called the \textit{degree sequence} of $G$. 

\begin{minipage}{0.5\textwidth}
    \begin{figure}[H]
        \centering
        \begin{tikzpicture}[node distance= 1.5cm,]
            \node[graphnode, draw] (A) {$v_1$};
            \node[graphnode, draw] (B) [right of=A] {$v_2$};
            \node[graphnode, draw] (C) [above of=A] {$v_3$};
            \node[graphnode, draw] (D) [above of=B] {$v_4$};
            \node[graphnode, draw, yshift=-0.75cm] (E) [above=of $(C)!0.5!(D)$] {$v_5$};
            \node[graphnode, draw, yshift=-1.2cm] (F) [above=of $(A)!0.5!(B)$] {$v_6$};

            \path (A) edge[above] node[align=center] {} (B);
            \path (B) edge[above] node[align=center] {} (D);
            \path (C) edge[above] node[align=center] {} (E);
            \path (C) edge[above] node[align=center] {} (A);
            \path (D) edge[above] node[align=center] {} (C);
            \path (D) edge[above] node[align=center] {} (E);
        \end{tikzpicture}
        \caption{Example Graph For Degree Sequence}
    \end{figure}
\end{minipage} \hfill
\begin{minipage}{0.45\textwidth}
The degree sequence: $(0, 2, 2, 2, 3, 3)$
\end{minipage}

\section{Euler Theorem}
Euler's Theorem (or Handshaking Lemma) states that: In any graph $G = (V,E)$, the sum of all the vertex-degrees is equal to twice the number of edges,
\[\sum_{v \in V} \deg v = 2|E|\]
This is a great revelation, however it lead to two more revelations:
\begin{enumerate}
    \item In any graph, the sum of all the vertex-degrees is an even number
    \item In any graph, the number of vertices of odd degree is even
\end{enumerate}

\section{Special Types of Graphs}
We will now see that there are special types of graphs that have their own names.
\subsection{Complete Graphs}
For any positive integer $n$, the complete graph on $n$ vertices, denoted $K_n$, is that graph with $n$ vertices every two of which are adjacent.

\begin{minipage}[b][][b]{0.23\textwidth}
    \begin{figure}[H]
        \centering
        \begin{tikzpicture}[node distance= 1.5cm,]
            \node[graphnode, draw] (A) {$v_1$};
        \end{tikzpicture}
        \caption{$K_1$}
    \end{figure}
\end{minipage} \hfill
\begin{minipage}[b][][b]{0.23\textwidth}
    \begin{figure}[H]
        \centering
        \begin{tikzpicture}[node distance= 1.5cm,]
            \node[graphnode, draw] (A) {$v_1$};
            \node[graphnode, draw] (B) [right of=A] {$v_2$};
            \path (A) edge[above] node[align=center] {} (B);
        \end{tikzpicture}
        \caption{$K_2$}
    \end{figure}
\end{minipage} \hfill
\begin{minipage}[b][][b]{0.23\textwidth}
    \begin{figure}[H]
        \centering
        \begin{tikzpicture}[node distance= 1.5cm,]
            \node[graphnode, draw] (A) {$v_1$};
            \node[graphnode, draw] (B) [right of=A] {$v_2$};
            \node[graphnode, draw, yshift=-0.75cm] (C) [above=of $(A)!0.5!(B)$] {$v_5$};
            \path (A) edge[above] node[align=center] {} (B);
            \path (C) edge[above] node[align=center] {} (B);
            \path (A) edge[above] node[align=center] {} (C);
        \end{tikzpicture}
        \caption{$K_3$}
    \end{figure}
\end{minipage} \hfill
\begin{minipage}[b][][b]{0.23\textwidth}
    \begin{figure}[H]
        \centering
        \begin{tikzpicture}[node distance= 1.5cm,]
            \node[graphnode, draw] (A) {$v_1$};
            \node[graphnode, draw] (B) [right of=A] {$v_2$};
            \node[graphnode, draw] (C) [above of=B] {$v_3$};
            \node[graphnode, draw] (D) [left of=C] {$v_4$};
            \path (A) edge[above] node[align=center] {} (B);
            \path (B) edge[above] node[align=center] {} (C);
            \path (C) edge[above] node[align=center] {} (D);
            \path (A) edge[above] node[align=center] {} (D);
            \path (A) edge[above] node[align=center] {} (C);
            \path (B) edge[above] node[align=center] {} (D);
        \end{tikzpicture}
        \caption{$K_4$}
    \end{figure}
\end{minipage} \vspace{0.5em}
% could do K_5 if I feel the urge in the future.

The complete graph on $n$ has $\displaystyle \frac{n(n-1)}{2}$ edges because Eulers theorem. 

\subsection{Bipartite Graphs}
A bipartite graph is one whose vertices can be partitioned into disjoint sets $V_1$ and $V_2$ in such a way that every edge joins in a vertex in $V_1$ and a vertex in $V_2$ (no edges within $V_1$ nor within $V_2$). 

\begin{minipage}{0.5\textwidth}
    \begin{figure}[H]
        \centering
        \begin{tikzpicture}[node distance= 1.5cm,]
            \node[graphnode, draw] (A) {$v_1$};
            \node[graphnode, draw] (B) [right of= A] {$v_2$};
            \node[graphnode, draw] (C) [above of= A]{$v_3$};
            \node[graphnode, draw] (D) [right of= C]{$v_4$};
            \node[graphnode, draw] (E) [right of= D]{$v_5$};
            \node[graphnode, draw] (F) [right of= E]{$v_6$};
            
            \path (A) edge[above] node[align=center] {} (C);
            \path (A) edge[above] node[align=center] {} (E);

            \path (B) edge[above] node[align=center] {} (C);
            \path (B) edge[above] node[align=center] {} (D);
            \path (B) edge[above] node[align=center] {} (F);

        \end{tikzpicture}
        \caption{Incomplete bipartite graph}
    \end{figure}
\end{minipage} \hfill
\begin{minipage}{0.45\textwidth}
    \begin{figure}[H]
        \centering
        \begin{tikzpicture}[node distance= 1.5cm,]
            \node[graphnode, draw] (A) {$v_1$};
            \node[graphnode, draw] (B) [right of= A] {$v_2$};
            \node[graphnode, draw] (C) [above of= A]{$v_3$};
            \node[graphnode, draw] (D) [right of= C]{$v_4$};
            \node[graphnode, draw] (E) [right of= D]{$v_5$};
            \node[graphnode, draw] (F) [right of= E]{$v_6$};
            
            \path (A) edge[above] node[align=center] {} (C);
            \path (A) edge[above] node[align=center] {} (D);
            \path (A) edge[above] node[align=center] {} (E);
            \path (A) edge[above] node[align=center] {} (F);

            \path (B) edge[above] node[align=center] {} (C);
            \path (B) edge[above] node[align=center] {} (D);
            \path (B) edge[above] node[align=center] {} (E);
            \path (B) edge[above] node[align=center] {} (F);

        \end{tikzpicture}
        \caption{Complete bipartite graph}
    \end{figure}
\end{minipage}\vspace{0.5em}

A complete bipartite graph is a bipartite graph in which every vertex in $V_1$ is joined to every vertex in $V_2$. 

\section{Subgraphs}
A graph $H$ is a subgraph of $G$ if and only if the vertex and edge set of $H$ are respectively subsets of the vertex and edge set of $G$. 

\begin{minipage}{0.3\textwidth}
    \begin{figure}[H]
        \centering
        \begin{tikzpicture}[node distance= 1.5cm,]
            \node[graphnode, draw] (A) {$v_1$};
            \node[graphnode, draw] (E) [right of= A]{$v_5$};
            \node[graphnode, draw] (C) [right of= E]{$v_3$};
            \node[graphnode, draw] (D) [above of= E]{$v_4$};
            \node[graphnode, draw] (B) [below of= E]{$v_2$};

            \path (A) edge[above] node[align=center] {} (B);
            \path (B) edge[above] node[align=center] {} (C);
            \path (C) edge[above] node[align=center] {} (D);
            \path (D) edge[above] node[align=center] {} (A);
            \path (A) edge[above] node[align=center] {} (E);
            \path (E) edge[above] node[align=center] {} (B);
        \end{tikzpicture}
        \caption{$G$}
    \end{figure}
\end{minipage} \hfill
\begin{minipage}{0.3\textwidth}
    \begin{figure}[H]
        \centering
        \begin{tikzpicture}[node distance= 1.5cm,]
            \node[graphnode, draw] (A) {$v_1$};
            \node[graphnode, draw] (E) [left of= A]{$v_5$};
            \node[graphnode, draw] (C) [left of= E]{$v_3$};
            \node[graphnode, draw] (D) [above of= E]{$v_4$};
            \node[graphnode, draw] (B) [below of= E]{$v_2$};

            \path (B) edge[above] node[align=center] {} (C);
            \path (C) edge[above] node[align=center] {} (D);
            \path (D) edge[above] node[align=center] {} (A);
            \path (E) edge[above] node[align=center] {} (B);
        \end{tikzpicture}
        \caption{$H_1$}
    \end{figure}
\end{minipage} \hfill
\begin{minipage}{0.3\textwidth}
    \begin{figure}[H]
        \centering
        \begin{tikzpicture}[node distance= 1.5cm,]
            \node[graphnode, draw] (A) {$v_1$};
            \node[graphnode, draw] (E) [right of= A]{$v_5$};
            \node[graphnode, draw] (C) [right of= E]{$v_3$};
            \node[graphnode, draw] (D) [below of= E]{$v_4$};
            \node[graphnode, draw] (B) [above of= E]{$v_2$};

            \path (A) edge[above] node[align=center] {} (B);
            \path (B) edge[above] node[align=center] {} (C);
            \path (C) edge[above] node[align=center] {} (D);
            \path (D) edge[above] node[align=center] {} (A);
            \path (A) edge[above] node[align=center] {} (E);
            \path (E) edge[above] node[align=center] {} (B);
        \end{tikzpicture}
        \caption{$H_2$}
    \end{figure}
\end{minipage}

\section{Isomorphic Graphs}
Two graphs, $G$ and $H$ are said to be isomorphic if $H$ can be obtained from $G$ by re-labelling the vertices. 

\begin{minipage}{0.45\textwidth}
\begin{figure}[H]
    \centering
    \begin{tikzpicture}[node distance= 1.5cm,]
        \node[graphnode, draw] (A) {$v_1$};
        \node[graphnode, draw] (B) [right of= A]{$v_2$};
        \node[graphnode, draw] (C) [above of= B]{$v_3$};
        \node[graphnode, draw] (D) [left of= C]{$v_4$};
        
        \path (A) edge[above] node[align=center] {} (B);
        \path (B) edge[above] node[align=center] {} (C);
        \path (C) edge[above] node[align=center] {} (D);
        \path (D) edge[above] node[align=center] {} (A);
    \end{tikzpicture}
    \caption{$G$}
\end{figure}
\end{minipage} \hfill
\begin{minipage}{0.45\textwidth}
\begin{figure}[H]
    \centering
    \begin{tikzpicture}[node distance= 1.5cm,]
        \node[graphnode, draw] (A) {$u_1$};
        \node[graphnode, draw] (B) [right of= A]{$u_2$};
        \node[graphnode, draw] (C) [above of= A]{$u_3$};
        \node[graphnode, draw] (D) [right of= C]{$u_4$};
        
        \path (A) edge[above] node[align=center] {} (B);
        \path (B) edge[above] node[align=center] {} (C);
        \path (C) edge[above] node[align=center] {} (D);
        \path (D) edge[above] node[align=center] {} (A);
    \end{tikzpicture}
    \caption{$H$}
\end{figure}
\end{minipage}\vspace{0.5em}

In the above example, $v_1$ maps to $u_1$, $v_2$ to $u_2$ and so on. What this means is that if the $u$ labelling was to be replaced with the $v$ labelling - the same graph would be obtained.\\

$G$ is isomorphic to $H$ is there is a bijective function $f : V(G) \rightarrow V(H)$ such that
\begin{itemize}
    \item if $u$ and $v$ are adjacent in $G$ then $f(u)$ and $f(v)$ are adjacent in $H$
    \item if $u$ and $v$ are not adjacent in $G$ then $f(u)$ and $f(v)$ are not adjacent in $H$.
\end{itemize}

It is difficult to prove that two graphs are isomorphic, as we have to try all the bijections between vertex sets and check them. It can be shown, that if $G$ and $H$ are isomorphic graphs then $G$ and $H$:
\begin{itemize}
    \item have the same number of vertices
    \item have the same number of edges
    \item have the same degree sequence
    \item either both are connected or both are not connected (to be covered next week)
\end{itemize}
So it can be easier to show that graphs are not isomorphic, which is shown by proving that one of the properties above is broken.