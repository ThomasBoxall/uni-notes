\taughtsession{Lecture}{Functional Programming in Python}{2024-04-22}{12:00}{Matthew}{}

\textit{This lecture is for information only; the content will not be assessed}.\\

In recent years, functional programming concepts have become increasingly common in mainstream imperative languages. These include languages such as: Python, JavaScript, C\#, Java (from v8). This lecture will introduce the concept of Functional Programming in Python. 

\section{Functions are Data in Python}
When defining a function in Python, we are in fact introducing a new variable with the function as it's value. This can be seen below:
\begin{verbatim}
    def f(arg):
        return arg + 1
    print(f) # prints <function f at 0xb56b11ec>
    g = f
    g(3) # returns 4
    f = 7
    f, g # returns (7, <function f at 0xb56b11ec>)
\end{verbatim}
In the above example we can see the function, \verb|f|, be defined then used. Then the data value of \verb|f| (this is the function) gets assigned to \verb|g| and we re-assign \verb|f| to \verb|7|. We finally return the values of \verb|f| and \verb|g| as a tuple.

\section{List Comprehension}
Python's list comprehension is very similar to that of Haskell's. They work by forming a new list from an old list. 
\begin{verbatim}
    >>> aList = [1, 2, 3, 4, 5]
    >>> [2 * i for i in aList]
    [2, 4, 6, 8, 10]
\end{verbatim}
In the above example, we can see how we can use the keyword \verb|for| and \verb|in| to iterate through every item in the list and multiply it by 2. The below example takes this further, discarding all values where they are less than or equal to 3.
\begin{verbatim}
    >>> [i * 2 for i in aList if i > 3]
    [8, 10]
\end{verbatim}

In Python, tuples are indexed like lists, however they are immutable. In the following example, note how you can iterate through a tuple containing list, using a \verb|for| loop. This example is written in an imperative way.
\begin{verbatim}
    testData = [("Sam",67), ("Kate",35), ("Jill",75),
                ("Fred", 45), ("Alice",50), ("Bob", 38)]
    def passed(lst):
        names = []
        for (name, mark) in lst:
            if mark >= 40:
                names.append(name)
        return names 
\end{verbatim}

This is all well and good, however it's a bit too imperative - we want to make it more functional, as below:
\begin{verbatim}
    def passed(lst):
        return [name for (name, mark) in lst if mark >= 40]
\end{verbatim}

\section{Lambda Expressions}
Python includes, in a similar way to Haskell, anonymous functions - called lambda expressions; for example:
\begin{verbatim}
    lambda x : x * 10
\end{verbatim}

A lambda expression represents a function that multiplies its arguments by 10 (and is therefore similar to the \verb|(*10)| in Haskell). 

Python also supports higher-order functions, for example the following function returns a function which can then be called. 
\begin{verbatim}
    def multiplier(n):
        return lambda x: x * n
\end{verbatim}
Which can therefore be used:
\begin{verbatim}
    >>> m10 = multiplier(10)
    >>> m10(6)
    60
\end{verbatim}

\subsection{Higher Order Functions}
We recall that Haskell has the higher order functions \verb|map|, \verb|filter| and \verb|foldr|. Python includes built-in equivalents functions \verb|map|, \verb|filter| and \verb|reduce| (which is almost equivalent to \verb|foldr|). \\

Python's \verb|map| function takes a function and a list and gives a new list:
\begin{verbatim}
    >>> aList = [2, 3, 4, 9, 4, 7]
    >>> map(lambda x : x * 10, aList)
    [20, 30, 60, 90, 40, 70]
\end{verbatim}

The \verb|filter| function gives a new list of only those elements that have a given property:
\begin{verbatim}
    >>> filter(lambda x : x % 2 == 0, aList)
    [2, 6, 4]
\end{verbatim}

The \verb|reduce| function takes takes a binary function (which is a function of two arguments), and a list and then returns a single value. It applies the function to the first two elements of the list then on the result and the next item and so on. 
\begin{verbatim}
    >>> reduce(lambda x, y : x + y, aList)
    31
    >>> reduce(lambda x, y : x * y, aList)
    9072
\end{verbatim}

There is an optional third argument which can be passed as a `starting' value:
\begin{verbatim}
    >>> aList = []
    >>> reduce(lambda x, y: x * y, aList, 1)
    1
\end{verbatim}

\section{Benefits to Functions as Data}
If we consider the following simple function which maps the number of days, based on the month number passed as parameter:
\begin{verbatim}
    def daysInMonth(month):
        numDays =  [31,28,31,30,31,30,31,31,30,31,30,31]
        return numDays[month - 1]
\end{verbatim}

We here have an \textit{inefficiency} (no, not just the fact we're using Python), in that for every time the function is called - a new instance of \verb|numDays| is stored in memory. So a possible solution to this problem could be to define \verb|numDays| as a global variable:
\begin{verbatim}
    numDays = [31,28,31,30,31,30,31,31,30,31,30,31]
    def daysInMonth(month):
        return numDays[month - 1]
\end{verbatim}
However, this is another bad idea - as it's using a global variable. Another potential solution could be to encapsulate it as a class and write the method as a static method, however this is also a bad use of an OO concept and is overly complicated.\\

So, instead we use a \textit{closure}:
\begin{verbatim}
    def makeDaysInMonth():
        numDays =  [31,28,31,30,31,30,31,31,30,31,30,31]
        def f(month):
            return numDays[month-1]
        return f
    daysInMonth = makeDaysInMonth()
\end{verbatim}
The returned function (which we've assigned to \verb|daysInMonth|) has access to a local variable of another function that has exited. 

\section{Recursion}
Recursion fanboys will argue that using recursion instead of iteration will lead to simpler, more readable functions. For example:
\begin{verbatim}
    def fibonacci(n):
        if n < 2:
            return n
        else:
            return fibonacci(n-2) + fibonacci(n-1)
\end{verbatim}

However, this function will suffer from a major issue of efficiency caused by the repeated computation of intermediate results. We can attempt to solve this using a higher-order function. Our solution will make use of a Python Dictionary.

\subsection{Dictionaries}
Dictionaries are unordered collections of data, whose values are indexed by key. Dictionary literals are written as a sequence of \verb|key:value| paris within braces (\verb|{| and \verb|}|). For example:
\begin{verbatim}
    >>> shopping  = {"eggs" : 2, "ham" : 4}
\end{verbatim}

\textit{Further notes on Dictionaries in Python are available in 1st Year Programming Module}.

\subsection{Memoization}
Our solution to the recursive problem is to use a technique called memoization for remembering results of calls to a function such as fibonacci. We can use a general purpose memoization function:
\begin{verbatim}
    def memoize(f):
        cache = {}
        def g(arg):
            if arg in cache:
                return cache[arg]
            else:
                cache[arg] = f(arg)
                return cache[arg]
        return g
\end{verbatim}
Which we can then give to the function we want to optimise:
\begin{verbatim}
    fibonacci = memoize(fibonacci)
\end{verbatim}