\taughtsession{Lecture}{Sets}{2024-01-23}{17:00}{Janka}{}

\section{Introduction}
\textit{Sets} underpin maths and Computer Science. A set is a collection of objects, which are called the elements (also known as members of the set). For example, a set of the numbers 1, 3, 8; or the collection of students in a class born in March. There are two characteristics of sets:
\begin{enumerate}
    \item There are no repeated occurrences of elements
    \item There is no particular order of the elements
\end{enumerate}

\section{Set Notation}
The elements of a set are enclosed in braces with their names being denoted by a \textit{letter}, for example:
\[A = \{1, 2, 3\}, \quad C=\{Portsmouth, Brighton, London\}\]

There are two ways that we can describe the members of a set. We can \textit{list the elements} which is mainly used for finite sets, for example:
\[A = \{3, 6, 9, 12\}\]
We can \textit{specify a property} that all the elements in the set have in common. The `$|$' character is read `such that', sometimes `$:$' is used in it's place. For example:
\[B = \{x | x \mathrm{\ is\ a\ multiple\ of\ 3\ and\ } 0 < x < 15\}\]
We can also use \textit{three dots} to informally denote a sequence of elements that we don't wish to write down, for example:
\[C = \{1, \ldots, 10\}\]

\subsection{Sets of Numbers}
There are some reserved letters to denote specific sets of numbers in maths. These are shown below:
\begin{itemize}
    \item $\mathbb{N}$ (or $N$) is used for the set of natural numbers (integers $>=$ 0). $\mathbb{N} = \{0, 1, 2, 3, 4, \ldots \}$
    \item $\mathbb{Z}$ (or $Z$) is used for the set of integers. $\mathbb{N} = \{\ldots, -1, 0, 1, \ldots \}$
    \item $\mathbb{Q}$ (or $Q$) is used for the set of rational numbers (number which can be expressed as a quotient or fraction). $\displaystyle \mathbb{Q} = \{0, \frac{1}{2}, \frac{1}{3}, \frac{1}{4}, \ldots \}$
    \item $\mathbb{R}$ (or $R$) is used for the set of real numbers. $\mathbb{R} = \{\displaystyle \ldots, -1, 0, \frac{1}{2}, \ldots \}$
\end{itemize}

\subsection{Elements of a Set}
We can use the $\in$ symbol to denote if an element is a member of a given set. For example, if $x$ is a member of $S$ - then we can say:
\[x \in S\]
The symbol $\notin$ denotes an element is not a member of a given set. For example, if $y$ is \textbf{not} a member of $S$ - then we can say:
\[y \notin S\]

\subsection{Many Ways to Say The Same Thing}
There are several ways of describing the same set, for example for the set $S$ of \textit{odd integers}:
\begin{align*}
    S &= \{\ldots, -5, -3, -1, 1, 3, 5, \ldots\}\\
    &= \{x | x \mathrm{\ is\ an\ odd\ integer\ } \}\\
    &= \{x | x=2k+1 \mathrm{\ for\ some\ integer\ } k \}\\
    &= \{x | x = 2k+ 1 \mathrm{\ for\ some\ } k \in \mathbb{Z}\}\\
    &= \{2k+1|k \in \mathbb{Z}\}
\end{align*}

The phrase ``for some [integers $K$]'', means ``for all [integers $k$]''

\subsection{Empty Sets}
Where a set has \textit{no elements}, it is called an empty set or null set. It's denoted with the $\emptyset$ symbol, for example:
\[\emptyset = \{ \}\]

\subsection{Finite \& Infinite Sets}
If the number of elements in the set is fixed (for example when counting the elements at a fixed rate for a set amount of time), then the set is \textit{finite}. If the set $X$ is finite, then we call $|X|$ the \textit{cardinality} of $X$ therefore:
\[|X| = \mathrm{number\ of\ elements\ in\ }X \]
If the counting never stops then $X$ is an infinite set.

\subsection{Subsets}
A subset is where one set's elements are entirely present in another set. There are three conditions we need to know about:
\begin{itemize}
    \item $A \subseteq B$: $A$ is a subset of $B$ therefore every element in $A$ is also in $B$.
    \item $A \nsubseteq B$: $A$ is not a subset of $B$.
    \item $A \subset B$: $A$ is a proper subset of $B$, therefore $B$ has at least one additional element which is not in $A$.
\end{itemize}
