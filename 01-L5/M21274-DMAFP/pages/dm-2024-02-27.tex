\taughtsession{Lecture}{Methods of Proof}{2024-02-27}{1700}{Janka}{}

\section{What is a Proof?}
A Mathematical Proof is a deductive argument for a mathematical statement, showing that the stated assumptions logically guarantee the conclusion. Mathematical Proofs are carefully reasoned and there are a number of different ways we can deduce one. 

\subsection{Formal Example}

\begin{theorem}
    Prove that for all integers $m$ and $n$, if $m$ is odd and $n$ is even, then $m + n$ is odd.
\end{theorem}

An \textit{argument} (theorem) is a finite collection of statements ($p_1, p_2, \ldots, p_n$) called \textit{premises} (hypothesis) followed by a statement $q$ called the \textit{conclusion}.
\[(p_1 \wedge p_2 \wedge \ldots \wedge p_n) \Rightarrow q\]
In the above, the premises are ``$m$, $n$, integers, $m$ is odd, $n$ is even'' and the conclusion is ``$m+n$ is odd''. The argument is valid (or the theorem holds) if, whenever $p_1$, $p_2$, $\ldots$, $p_n$ are all true, then $q$ is also true.

\subsection{Methods of Proof}
To ``prove a theorem'' means to show that if all premises are true then the conclusion is also true. 
\begin{itemize}
    \item[] $m,n$ are integers
    \item[] $m$ is odd, $n$ is even
    \item[] $\ldots$
    \item[] Can we prove that $m+n$ is odd?
\end{itemize}

We have a number of different techniques which we can use. Which of the following to choose depends on the problem and experience:
\begin{enumerate}
    \item A direct proof
    \item A proof by contradiction
    \item A proof by contrapositive
    \item A proof by mathematical induction
\end{enumerate}

\section{A Direct Proof}
In a direct proof, we start with the hypothesis of a statement (premises) and make one deduction after another until we reach the conclusion. Theorems which are able to be proved are often of the form:
\[p\ (hypothesis) \Rightarrow q\ (conclusion)\]

While proving the proof, we can use:
\begin{itemize}
    \item Previously Proven Facts
    \item Definitions
    \item Known basic properties
\end{itemize}

Our task is to prove that then $q$ is also true.

\subsection{Our First Theorem}
It's time to prove our first theorem, how exciting!

\begin{theorem}
    For all integers $m$ and $n$, if $m$ is odd and $n$ is even, then $m+n$ is odd.
\end{theorem}
As we are proving this using a direct proof, we assume that the hypothesis is true and derive the conclusion. We start doing this by realising some properties of odd and even numbers.

\begin{proof}
An integer $r$ is even if and only if there exists an integer $k$ such that $r=2k$. (This is the basic definition of an even number, no room for disagreement here!)\\

Similarly, an integer $r$ is odd if and only if there exists an integer $k$ such that $r = 2k + 1$\\

To apply these derivations to our theorem:
\begin{itemize}
    \item[] $m$ is odd $\Rightarrow$ there exists an integer $k$ such that $m=2k + 1$
    \item[] $n$ is even $\Rightarrow$ there exists an integer $l$ such that $n=2l$
\end{itemize}

We can use maths to show that the sum is:
\begin{align*}
    m + n &= (2k+l) +2l\\
    &= 2(k+l)+1
\end{align*}
$\therefore$ $m+n$ is odd.
\end{proof}

Note the black square ($\blacksquare$) on the right above at the end of the proof. This is stands for QED which stands for ``Quod Erat Demonstrandum'', which stands for ``what was to be demonstrated'' which stands for ``boom done, there I proved it, I'm leaving now''. 

\section{Proof by Contradiction (indirect proof)}
As we know there are only, and can only, be two options for the truth value of a conclusion: $True$ or $False$. If supporting that the premises are true and the conclusion is false, we are able to arrive at a contradiction (a conclusion that is contradictory to our assumptions or something obviously untrue like $1=0$) $\Rightarrow$ our conclusions must be true!\\

While proving the proof, we can use:
\begin{itemize}
    \item Previously known facts
    \item Definitions
    \item Known basic properties
\end{itemize}

Our task is to prove a contradiction, which is done through proving that $q$ is true (assuming that $p$ is true).

\subsection{Our Second Theorem}
\begin{theorem}
    For every $n \in \mathbb{N}$, if $n^2$ is even, then $n$ is even
\end{theorem}
We can rewrite this as:
\begin{center}
    $n^2$ is even, $n \in \mathbb{N} \Rightarrow n$ is even 
\end{center}
We can then take our hypothesis to be to be that $n^2$ is even while $n$ is a natural number; and a false conclusion to be $n$ is not even, hence $n$ is odd. We want to prove any contradiction (which could be $r \wedge \neg r$ for a proposition $r$). 

\begin{proof}
Since we take that $n$ is odd, there exists $k \in \mathbb{N}$ such that $n=2k+1$. We can now derive the following:
\begin{align*}
    n^2 &= (2k+1)^2\\
    &= 4k^2 + 4k + 1\\
    &= 2(2k^2+2k) + 1
\end{align*}
This means that $n^2$ is odd.\\

Now $n^2$ is even (the hypotheses) and $n^2$ is odd (which we have just proved). This is a \textit{crazy} situation, an impossible contradiction! As we have found a contradiction $\Rightarrow$ the conclusion must be true!\\

$\therefore$ for every $n \in \mathbb{N}$ if $n^2$ is even, then $n$ is even.
\end{proof}

\section{Proof by Contrapositive (indirect proof, again)}
The contrapositive of the condition proposition $p \rightarrow q$ is the proposition $\neg q \rightarrow \neg p$. Note that the conditional proposition and its contrapositive are logically equivalent
\[p \rightarrow q  \equiv \neg q \rightarrow \neg p \]
To prove a statement by contrapositive, we prove that the contrapositive statement is a direct proof and conclude that the original statement is true. This means that instead of the original theorem, $p \rightarrow q$, we prove by a direct proof the contrapositive theorem $\neg q \rightarrow \neg p$.\\

While proving the proof, we can use
\begin{itemize}
    \item previously proven facts
    \item definitions
    \item known basic properties
\end{itemize}

Our task is to prove that $\neg p$ is true. In this way we prove that $\neg q \rightarrow \neg p$ and because of $p \rightarrow q \equiv \neg q \rightarrow \neg p $ necessarily $p \rightarrow q$ must be true as well (meaning that the theorem $p \rightarrow q$ is valid). 
\subsection{Our Second Theorem (again)}
\begin{theorem}
    For every $n \in \mathbb{N}$, if $n^2$ is even then $n$ is even.
\end{theorem}
Our contrapositive statement is: ``For every $n \in \mathbb{N}$ if $n$ is not even, then $n^2$ is not even''. We can use the fact ``An integer is not even, if and only if, it is odd'' to derive our contrapositive:\\

\textbf{Contrapositive.} For every $n \in \mathbb{N}$, if $n$ is odd then $n^2$ is odd. \\

We are now able to prove the contrapositive statement using a direct proof by proving that $n^2$ is odd. 

\begin{proof}
Since $n$ is odd, there exists $k \in \mathbb{N}$ such that $n=2k+1$ We can now derive the following:
\begin{align*}
    n^2 &= (2k+1)^2\\
    &= 4k^2 + 4k + 1\\
    &= 2(2k^2+2k) + 1
\end{align*}
Therefore $n^2$ is odd.\\

This means that the contrapositive statement is true and by logical equivalence also the theorem:
\begin{center}
    ``For every $n \in \mathbb{N}$, if $n^2$ is even, then $n$ is even''
\end{center}
is valid
\end{proof}

\section{Mathematical Induction}
Mathematical Induction is one of the most basic methods of proof. It is a useful technique to use to establish the truth of a statement about natural natural numbers. It is a method for proving a statement (given in the form of a proposition) $P(n)$ is true for every natural number, $n$, and that the infinitely many cases $P(0), P(1), P(2), P(3), \ldots$ all hold. This is done in two stages.
\begin{enumerate}
    \item Proving a simple case (the base case), which proves the statement for $n=x$ (where $x$ is any number) without assuming any knowledges of other cases.
    \item Having proven that when $n=x$, $P(n)$ is true; we assume that $P(n)$ is true for all $n\geq x$. (the inductive step) (this is \textit{just} the case, which works through magic, don't look to deep into it.)
\end{enumerate}

\subsection{The Third Theorem}
\begin{theorem}
    Prove that for any integer $n \geq 1$, the sum of the first $n$ natural numbers is $\displaystyle S(n) = \frac{n(n+1)}{2}$
\end{theorem}

\begin{proof}
    We will start by looking at the basic step, where we will take $n=1$.
    \begin{align*}
        S(1) &= 1\\
        &= \frac{1(1+1)}{2}\\
        &= 1
    \end{align*}

    We will now complete the inductive step.
    \begin{itemize}
        \item[(a)] We can assume that $S(n)$ of the first $n$ natural numbers is $\displaystyle \frac{n(n+1)}{2}$
        \item[(b)] We need to prove that the statement is true also for the first $n+1$ natural numbers, this means:
    \end{itemize}
    \[S(n+1) = \frac{(n+1) (n+2)}{2}\]
    
    Which  leaves us with the following assumption:
    \[S(n) = 1 + 2 + \ldots + n = \frac{n(n+1)}{2}\]
    Now we can derive the following:
    \begin{align*}
        S(n+1) &= 1 + 2 + \ldots + n + (n+1)\\
        &= S(n) + (n+1)\\
        &= \frac{n(n+1)}{2} + (n+1)\\
        &= \frac{n(n+1)}{2} + \frac{2(n+1)}{2}\\
        &= \frac{(n+1)(n+2)}{2}
    \end{align*}
    Therefore, assuming that the formula is true for $n$, we have proved the formula is true for $n+1$.
\end{proof}

\section{Disproving a Universally Quantified Statement}
As we saw in the previous lecture, a universally quantified statement is one which applies ``for all'' or ``for some''.\\

To disprove
\[\forall x \in D\ P(x)\]
we have to find one $x \in D$ that makes $P(x)$ false. This value of $x$ is called the `counterexample'.\\

For example, the statement:
\[\forall n \in \mathbb{N} (2^n+1\ \mathrm{is\ prime})\]
is false because, where $n=3$, $2^3+1=9$ which is not prime.