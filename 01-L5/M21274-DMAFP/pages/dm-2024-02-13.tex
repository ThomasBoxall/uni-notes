\taughtsession{Lecture}{Logic I: Introduction to Propositions \& Logic}{2024-02-13}{1700}{Janka}{}

\section{Reasoning}
Reasoning is something that we are introduced to doing from a young age. Younger children will continually ask ``why?'' as they attempt to make sense of the world while growing up; as they grow older they will generally only want the facts however. \textit{Logic} is the discipline which deals with the method of reasoning:
\begin{itemize}
    \item in mathematics to prove theorems
    \item in computer science to verify the correctness of programs and to prove some theorems
    \item in the natural and physical sciences to draw conclusions from experiments
    \item and in our everyday lives to solve a multitude of problems!
\end{itemize}
Over time, people come to understand that the following statement is how logic \& reasoning works:
\begin{center}
    ``If $X$ then $Y$'' is true and ``$X$'' is true $\Rightarrow$ so ``$Y$'' must be true
\end{center}

\section{Propositions}
A \textit{proposition} it a \textit{statement} (which is a declarative sentence) that can either be true or false; however not both. A proposition will be exact, not wishy-washy. For example - $3-x=5$ is not a proposition as it has an unknown value of $x$; however ``The earth is flat'' is a proposition as it can, and only can, categorically be True or False.\\

\subsection{Propositional Variables}
Propositions can be quite long. Mathematicians like efficiency, therefore they apply a single letter variable to a propositional statement to make their lives easier. The letters $p, q, r, \ldots$ are used to denote propositional variables.\\

Statement can be combined with logical connectives to obtain compound statements. For example $p\ and\ q$

\section{Logical Connectives}
Logical Connectives are used to combine propositional statements together; they are very similar to Boolean Algebra\footnote{Covered in A-Level Electronics, A-Level Computer Science, 1st Year ArchOS Module}. The truth value of a compound statement depends only on:
\begin{itemize}
    \item the truth values of the statements being combined
    \item the types of connectives being used
\end{itemize}

\subsection{Negation (not)}
If $p$ is a statement, the negation of $p$ is the statement $not\ p$, denoted by $\neg p$. The truth table of negation is shown below:
\begin{table}[H]
    \centering
    \begin{tabular}{C{0.1\textwidth} C{0.1\textwidth}}
        \textbf{$p$} & \textbf{$\neg p$}\\
        \hline
        \hline
        T & F \\
        \hline
        F & T \\
        \hline
    \end{tabular}
\end{table}

\subsection{Conjunction (and)}
If $p$ and $q$ are statement, the conjunction of $p$ and $q$ is the compound statement $p\ and\ q$, as denoted by $p \wedge q$
\begin{table}[H]
    \centering
    \begin{tabular}{C{0.1\textwidth} C{0.1\textwidth} C{0.1\textwidth}}
        $p$ & $q$ & $p \wedge q$\\
        \hline
        \hline
        T & T & T\\
        \hline
        T & F & F \\
        \hline
        F & T & F \\
        \hline
        F & F & F\\
        \hline
    \end{tabular}
\end{table}

\subsection{Disjunction (or)}
If $p$ and $q$ are statements, the (inclusive) disjunction of $p$ and $q$ is the compound statement $p\ or\ q$ as $p \vee q$
\begin{table}[H]
    \centering
    \begin{tabular}{C{0.1\textwidth} C{0.1\textwidth} C{0.1\textwidth}}
        $p$ & $q$ & $p \vee q$\\
        \hline
        \hline
        T & T & T\\
        \hline
        T & F & T \\
        \hline
        F & T & T \\
        \hline
        F & F & F\\
        \hline
    \end{tabular}
\end{table}

\subsection{Conditional Proposition (implication)}
If $p$ and $q$ are statements then the compound statement ``$if\ p\ then\ q$'' dented $p \rightarrow q$ (or $p \Rightarrow q$) is called implication. The hypotheses in the statement is $p$ and the conclusion is denoted by $q$.

\begin{table}[H]
    \centering
    \begin{tabular}{C{0.1\textwidth} C{0.1\textwidth} C{0.1\textwidth}}
        $p$ & $q$ & $p \rightarrow q$\\
        \hline
        \hline
        T & T & T\\
        \hline
        T & F & F \\
        \hline
        F & T & T \\
        \hline
        F & F & T\\
        \hline
    \end{tabular}
\end{table}

\subsection{Conditional Proposition (bidirectional)}
If $p$ and $q$ are statement, the compound statement ``if and only if'' (abbreviated to \textit{iff}), denoted by $p \Leftrightarrow q$, is called the biconditional of $p$ and $q$. 
\begin{table}[H]
    \centering
    \begin{tabular}{C{0.1\textwidth} C{0.1\textwidth} C{0.1\textwidth}}
        $p$ & $q$ & $p \Leftrightarrow q$\\
        \hline
        \hline
        T & T & T\\
        \hline
        T & F & F \\
        \hline
        F & T & F \\
        \hline
        F & F & T\\
        \hline
    \end{tabular}
\end{table}

\section{Truth of Compound Properties}
It is possible to combine the truth tables for single logical connectives to make \textit{more complicated truth tables}. In similar fashion to Algebraic \& standard numeric expression evaluation - there is a hierarchy of evaluation, as seen below (listed highest to lowest):
\begin{enumerate}
    \item brackets
    \item negation ($\neg $)
    \item conjunction ($\wedge$)
    \item disjunction ($\vee$)
    \item implication ($\rightarrow$)
    \item bidirectional ($\leftrightarrow$)
\end{enumerate}
For connectives of equal priority - work from left-to-right through them.

\section{Special Conditions}
\subsection{Tautology}
A statement that is true for all possible values of its propositional variables is called a tautology. For example $p: p \vee \neg  p$ is a tautology.
\subsection{Contradiction}
A statement that is false for all possible values of its propositional variables is called a contradiction. For example, any proposition $p: p \wedge \neg p$ is a contradiction. In a truth table, the final column (where the output is) will always be false. 
\subsection{Contingency}
A statement that can be either true or false depending on the truth values of its propositional variables is called a contingency. For example the proposition ``Murray will win the Wimbledon next year'' us a contingency because the truth of the statement is dependent on the propositional variable. 
\subsection{Contrapositive}
The contrapositive of a conditional statement $p \rightarrow q$ is $\neg q \rightarrow \neg p$. The conditional statement is logically equivalent to its contrapositive.

\section{Logical Equivalence}
Two statements are said to be \textit{logically equivalent}, $\equiv$, if and only if they have identical truth values for each possible value of their statement variables. Logical equivalence corresponds to $=$ with numbers.\\

Shown below are the rules of Logical Equivalence. There is not a requirement to memorise these as it should be possible to derive them in the exam when required. Note that they are the same as Boolean algebra's laws, except with funky symbols.
\begin{itemize}
    \item $p \wedge p \equiv p$
    \item $p \vee p \equiv p$
    \item $p \wedge T \equiv p$
    \item $p \wedge F \equiv F$
    \item $p \vee T \equiv T$
    \item $p \vee F \equiv p$
    \item $\neg (\neg p) \equiv p$
    \item $p \vee (\neg p) \equiv T$
    \item $p \wedge (\neg p) \equiv F$
    \item $p \wedge q \equiv q \wedge p$ (commutativity)
    \item $p \vee q \equiv q \vee p$ (commutativity)
    \item $(p \vee (q \wedge r)) \equiv ((p \vee q) \wedge (p \vee r))$ (distributivity)
    \item $(p \wedge (q \vee r)) \equiv ((p \wedge q) \vee (p \wedge r))$ (distributivity)
    \item $p \rightarrow q \equiv (\neg p \vee q)$
    \item $\neg (p \vee q) \equiv ((\neg p)\wedge(\neg q))$ (De Morgan's Law)
    \item $\neg (p \wedge q) \equiv ((\neg p)\vee(\neg q))$ (De Morgan's Law)
\end{itemize}

\subsection{Example of logical equivalence proof}
\textit{Prove that $(p\rightarrow q) \vee (p \rightarrow r) \equiv p \rightarrow (p \vee r)$}
\begin{align*}
    (p\rightarrow q) \vee (p \rightarrow r) &\equiv  &&\\
    & \equiv (\neg p \vee q) \vee (\neg p \vee r) && \mathrm{logical\ equivalence\ law}\\
    & \equiv \neg p \vee q \vee \neg p \vee r && \mathrm{all\ are\ `or'\ so\ remove\ brackets}\\
    & \equiv \neg p \vee q \vee r && \mathrm{get\ rid\ of\ second} \neg p\\
    & \equiv \neg p \vee (q \vee r)&& \mathrm{add\ brackets\ in} \\
    & \equiv p \rightarrow (q \vee r) && \mathrm{logical\ equivalence\ law\ again}
\end{align*}

\section{Necessary and Sufficient Condition}
A necessary condition is a condition such that statement $B$ cannot be true without statement $A$ being true. However it is possible for statement $A$ to be true to even if statement $B$ is not true.\\

A sufficient condition is a condition such that knowing statement $A$ is true guarantees that statement $B$ is true.\\

A statement ($A$) is said to be ``necessary and sufficient'' for the statement $B$ when $B$ is true if and only if $A$ is also true. 