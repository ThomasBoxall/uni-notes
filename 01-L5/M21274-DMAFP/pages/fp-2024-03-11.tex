\taughtsession{Lecture}{Algebraic Types}{2024-03-11}{12:00}{Matthew}{}

\textit{``There's nothing in the blob, it's a blob''}

\section{Types in Haskell: A Recap}
As we have already seen, there are a number of built-in types in Haskell:
\begin{itemize}
    \item \verb|Int|, \verb|Float|, \verb|Bool| and \verb|Char| (the basic types)
    \item \verb|(Int, Int, Char)| (tuple types)
    \item \verb|[Int]| or \verb|[(Int, Char)]| (the list type)
\end{itemize}

It is also possible to give convenient names to types using synonyms, for example:
\begin{itemize}
    \item \verb|type HouseNumber = Int|
    \item \verb|type StreetName = String|
    \item \verb|type Address = (HouseName, StreetName)|
\end{itemize}
These types are only so well and good, as they do not provide a convenient way to model more complex structures (such as Binary Trees). Within Haskell, we are able to use \textit{Algebraic Types} to define arbitrarily complex types.

\section{Algebraic Types}
We declare the algebraic type using the keyword \verb|data|, followed by:
\begin{itemize}
    \item the name of the type being defined
    \item a list of constructors
\end{itemize}
Note that both have to begin with a capital letter. For example:
\begin{verbatim}
    data Day = Mon | Tue | Wed | Thur | Fri | Sat | Sun
\end{verbatim}
As we can see above, the simplest algebraic types are those where the constructors don't take any arguments. Here, the constructors are the data values (or members of the type). We call a type which has been defined in this way an \textit{enumerated type}.

\subsection{Enumerated Type}
Another example of an enumerated type is Haskell's Boolean type:
\begin{verbatim}
    data Bool = False | True
\end{verbatim}

The simplest way to define a function on an Enumerated Type is by pattern matching. Using the \verb|Day| data type an example is:
\begin{verbatim}
    isWeekend :: Day -> Bool
    isWeekend Sat = True
    isWeekend Sun = True
    isWeekend _   = False
\end{verbatim}

This might look rather obtuse, and that would be a correct observation. This is because our data type (\verb|Day|) doesn't include any operators such as \verb|==|. It is possible to define \verb|==| ourselves, however this would be tedious.

\subsection{Algebraic Types \& Type Classes}
We can get Haskell to provide a \verb|==| operator by declaring that we want our type (\verb|Day|) to be a member of the \verb|Eq| type class. (Note that \verb|Eq| includes all types that include \verb|==| and \verb|/=|). To do this, we would need to define \verb|Day| as follows:
\begin{verbatim}
    data Day = Mon | Tue | Wed | Thur | Fri | Sat | Sun
               deriving (Eq)
\end{verbatim}

Now that we have the \verb|==| operator implemented for \verb|Day|, we can redefine \verb|isWeekend| as:
\begin{verbatim}
    isWeekend :: Day -> Bool
    isWeekend day = day == Sat || day == Sun
\end{verbatim}

However, \verb|Eq| only includes \textit{Eq}uality operators - what happens if we want to compare two values and see if one is greater than the other.\\

Fortunately, Haskell includes more type classes which can provide us with the functionality that we require for this:
\begin{itemize}
    \item \verb|Ord| provides us with \verb|<|, \verb|<=|, \verb|>| and \verb|>=|
    \item \verb|Show| to provide a function
    \begin{verbatim}
        show :: Day -> String
    \end{verbatim}
    which converts \verb|Day| values into strings
    \item \verb|Read| which provides a function
    \begin{verbatim}
        read :: String -> Day
    \end{verbatim}
    which converts strings (for example ``Mon'') into \verb|Day| values.
\end{itemize}

When using all these additional type classes, we may end up with the following definition of \verb|Day|:
\begin{verbatim}
    data Day = Mon | Tue | Wed | Thur | Fri | Sat | Sun
               deriving (Eq,Ord,Show,Read)
\end{verbatim}

This therefore means we can redefine \verb|isWeekend| as follows:
\begin{verbatim}
    isWeekend :: Day -> Bool
    isWeekend day = day >= Sat
\end{verbatim}

\subsection{Product Types}
As we saw in an earlier lecture, we can define student records (their name and a mark) as a type synonym giving a name to a tuple:
\begin{verbatim}
    type StudentMark = (String, Int)
\end{verbatim}
However, it is also possible for us to use an algebraic type with a constructor that has two arguments - one for the name and one for the mark:
\begin{verbatim}
    data StudentMark = Student String int
\end{verbatim}
Here, the constructor \verb|Student| is followed by it's argument types. For example:
\begin{verbatim}
    Student "Sam" 44
    Student "Jill" 64
\end{verbatim}

As could be expected, it is also possible to define functions which use the product data types:
\begin{verbatim}
    betterStu :: StudentMark -> StudentMark -> String
    betterStu (Student s1 m1) (Student s2 m2)
        | m1 >= m2  = s1
        | otherwise = s2
\end{verbatim}

Which would be executed with:
\begin{verbatim}
    ghci> betterStu (Student "Sam" 44) (Student "Jo" 73)
    "Jo"
\end{verbatim}

The main advantage of using an algebraic type is that every data value has an explicit label of its purpose (for example, \verb|Student|)

\subsection{Sum Types}
It is possible for us to combine the ideas of enumerated types and product types, this gives us types whose elements can be built in different ways. For example, a shape might be specified by it's radius or a rectangle specified by its height and width. It is possible to represent this using a \textit{sum} type:
\begin{verbatim}
    data Shape = Circle Float |
                 Rectangle Float Float
\end{verbatim}
Which would be declared as:
\begin{verbatim}
    Circle 9.0
    Rectangle 4.5 6.0
\end{verbatim}

This is a bit like an Abstract Class, where the parent class has no types of it's own and that the child classes are Concrete classes. \verb|Shape| would be the parent, abstract class, and \verb|Circle| \& \verb|Rectangle| would be the child, concrete classes. 

As expected, it is possible to write a single function for a sum type (i.e \verb|Shape|) which runs different code depending on the type of \verb|Shape| used. This is done through Pattern Matching:
\begin{verbatim}
    area :: Shape -> Float
    area (Circle r)      = pi * r * r
    area (Rectangle h w) = h * w
\end{verbatim}

This can be extremely useful, for example in the problem of representing addresses. Some buildings have numbers, and some buildings have names. We can define a data type for representing the first line of an address as:
\begin{verbatim}
    data Address = Address Building String
    data Buildings = Number Int | Name String
\end{verbatim}
Which we can use as:
\begin{verbatim}
    Address (Number 42) "High Street"
    Address (Name "Seaview") "Uplands Road"
\end{verbatim}

\subsection{Recursive Types}

\begin{figure}[ht]
    \centering
    \begin{tikzpicture}
        [
            level 1/.style={sibling distance=25mm},
            level 2/.style={sibling distance=15mm},
        ]   
    \node{5}   
        child {node {1}
            child {node {\bullet}}
            child {node {\bullet}}
        }
        child {node {8}
            child {node {7}
                child {node {\bullet}}
                child {node {\bullet}}
            }
            child {node {\bullet}}
        };
    \end{tikzpicture}
    \caption{Binary Tree Example}
\end{figure}

Types can be described in terms of themselves. If we take the example of a binary tree, which we know is defined recursively as:
\begin{itemize}
    \item a null node; or
    \item a node with a value, left sub-tree and a right-sub-tree
\end{itemize}

We can define a data type directly from this definition:
\begin{verbatim}
    data Tree = Null |
                Node Int Tree Tree
\end{verbatim}

Which we can see in-use below (note that the final value is represented by the diagram on the previous page):
\begin{verbatim}
    Null
    Node 7 Null Null
    Node 5 (Node 1 Null Null)
           (Node 8 (Node 7 Null Null) Null)
\end{verbatim}

It is possible to define functions which interact with recursive types. Generally, functions on trees will mirror the recursive structure of the type (meaning they use \verb|Null| as the base case \& \verb|Node| for the recursion). For example, the following function returns the height of a binary tree:
\begin{verbatim}
    height :: Tree -> Int
    height Null = 0
    height (Node _ st1 st2) = 1 + max (height st1) (height st2)
\end{verbatim}
