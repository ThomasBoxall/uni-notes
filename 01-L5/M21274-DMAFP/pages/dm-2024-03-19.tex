\taughtsession{Lecture}{Walks, Trails, Paths}{2024-03-19}{1700}{Janka}{}

\section{Walks}
A \textit{walk} in a multigraph is an alternating sequence of vertices and edges (beginning and ending with a vertex), where each edge is incident with the vertex immediately preceding and following it. The length of a walk is the number of edges in it.\\

Many real problems, when translated to graph theory - enquire about the possibility of walking through a graph. Most of the definitions and results about walks are valid for graphs and multigraphs, even if we don't always specific this.\\

\begin{minipage}{0.5\textwidth}
    \begin{figure}[H]
        \centering
        \begin{tikzpicture}[node distance= 1.5cm,]
            \node[graphnode, draw] (a) {$a$};
            \node[graphnode] (s1) [below of=a] {};
            \node[graphnode, draw] (d) [below of=s1] {$d$};
            \node[graphnode] (s2) [right of=d] {};
            \node[graphnode, draw] (b) [above of=s2] {$b$};
            \node[graphnode, draw] (f) [below of=s2] {$f$};
            \node[graphnode, draw] (e) [right of=s2] {$e$};
            \node[graphnode] (s3) [right of=e] {};
            \node[graphnode, draw] (c) [above of=s3] {$c$};
            \node[graphnode, draw] (g) [below of=s3] {$g$};

            \draw[emphline] (a) edge[above] node[] {} (b);
            \draw[emphline] (a) edge[above] node[] {} (d);
            \draw[emphline] (b) edge[above] node[] {} (c);
            \draw[emphline] (b) edge[above] node[] {} (d);
            \draw[emphline] (b) edge[above] node[] {} (e);
            \draw[emphline] (e) edge[above] node[] {} (d);
            \draw[emphline] (c) edge[above] node[] {} (e);
            \draw[emphline] (d) edge[above] node[] {} (f);
            

            \draw (a) edge[right] node[align=center] {$1$} (d);
            \draw (a) edge[above] node[align=center] {$2$} (b);
            \draw (b) edge[above] node[align=center] {$5$} (d);
            \draw (b) edge[above] node[align=center] {$6$} (c);
            \draw (b) edge[above] node[align=center] {$3$} (e);
            \draw (b) edge[above] node[align=center] {} (e);
            \draw (c) edge[right] node[align=center] {$7$} (e);
            \draw (d) edge[below] node[align=center] {$9$} (f);
            \draw (e) edge[below] node[align=center] {$4,8$} (d);
            \draw (e) edge[above] node[align=center] {} (f);
            \draw (e) edge[above] node[align=center] {} (g);
            \draw (f) edge[above] node[align=center] {} (g);

        \end{tikzpicture}
        \caption{Example of a Walk}
    \end{figure}
\end{minipage} \hfill
\begin{minipage}{0.45\textwidth}
A walk of length 9: d - da - a - ab - b - be - e - ed - d - db - b - bc - c - ce - e - ed - d - df\\

Alternatively represented as:\\
(d, a, b, e, d, b, c, e, d, f)
\end{minipage}

A walk is \textit{closed} if the first vertex is the same as the last, for example the walk (d, a, b, e, d, b, c, e, d), and is otherwise said to be an \textit{open} walk.

\subsection{Trails and Paths}
A \textit{trail} is a walk in which all edges are distinct. A \textit{path} is a walk in which all vertices are distinct.
\begin{minipage}{0.5\textwidth}
    \begin{figure}[H]
        \centering
        \begin{tikzpicture}[node distance= 1.5cm,]
            \node[graphnode, draw] (a) {$a$};
            \node[graphnode] (s1) [below of=a] {};
            \node[graphnode, draw] (d) [below of=s1] {$d$};
            \node[graphnode] (s2) [right of=d] {};
            \node[graphnode, draw] (b) [above of=s2] {$b$};
            \node[graphnode, draw] (f) [below of=s2] {$f$};
            \node[graphnode, draw] (e) [right of=s2] {$e$};
            \node[graphnode] (s3) [right of=e] {};
            \node[graphnode, draw] (c) [above of=s3] {$c$};
            \node[graphnode, draw] (g) [below of=s3] {$g$};

            \draw[emphline] (a) edge[above] node[] {} (b);
            \draw[emphline] (a) edge[above] node[] {} (d);
            \draw[emphline] (b) edge[above] node[] {} (e);
            \draw[emphline] (e) edge[above] node[] {} (d);
            \draw[emphline] (d) edge[above] node[] {} (f);
            

            \draw (a) edge[right] node[align=center] {$1$} (d);
            \draw (a) edge[above] node[align=center] {$2$} (b);
            \draw (b) edge[above] node[align=center] {} (d);
            \draw (b) edge[above] node[align=center] {} (c);
            \draw (b) edge[above] node[align=center] {$3$} (e);
            \draw (b) edge[above] node[align=center] {} (e);
            \draw (c) edge[right] node[align=center] {} (e);
            \draw (d) edge[below] node[align=center] {$5$} (f);
            \draw (e) edge[below] node[align=center] {$4$} (d);
            \draw (e) edge[above] node[align=center] {} (f);
            \draw (e) edge[above] node[align=center] {} (g);
            \draw (f) edge[above] node[align=center] {} (g);

        \end{tikzpicture}
        \caption{Example of a Trail}
    \end{figure}
\end{minipage} \hfill
\begin{minipage}{0.45\textwidth}
A trail (d, a, b, e, d, f) of length 5.\\

Not all the vertices of a trail are necessarily different.
\end{minipage}

\begin{minipage}{0.5\textwidth}
    \begin{figure}[H]
        \centering
        \begin{tikzpicture}[node distance= 1.5cm,]
            \node[graphnode, draw] (a) {$a$};
            \node[graphnode] (s1) [below of=a] {};
            \node[graphnode, draw] (d) [below of=s1] {$d$};
            \node[graphnode] (s2) [right of=d] {};
            \node[graphnode, draw] (b) [above of=s2] {$b$};
            \node[graphnode, draw] (f) [below of=s2] {$f$};
            \node[graphnode, draw] (e) [right of=s2] {$e$};
            \node[graphnode] (s3) [right of=e] {};
            \node[graphnode, draw] (c) [above of=s3] {$c$};
            \node[graphnode, draw] (g) [below of=s3] {$g$};

            \draw[emphline] (d) edge[above] node[] {} (a);
            \draw[emphline] (a) edge[above] node[] {} (b);
            \draw[emphline] (b) edge[above] node[] {} (e);
            \draw[emphline] (e) edge[above] node[] {} (f);
            \draw[emphline] (f) edge[above] node[] {} (g);
            

            \draw (a) edge[right] node[align=center] {$1$} (d);
            \draw (a) edge[above] node[align=center] {$2$} (b);
            \draw (b) edge[above] node[align=center] {} (d);
            \draw (b) edge[above] node[align=center] {} (c);
            \draw (b) edge[above] node[align=center] {$3$} (e);
            \draw (b) edge[above] node[align=center] {} (e);
            \draw (c) edge[right] node[align=center] {} (e);
            \draw (d) edge[below] node[align=center] {} (f);
            \draw (e) edge[below] node[align=center] {} (d);
            \draw (e) edge[left] node[align=center] {$4$} (f);
            \draw (e) edge[above] node[align=center] {} (g);
            \draw (f) edge[below] node[align=center] {$5$} (g);

        \end{tikzpicture}
        \caption{Example of a Path}
    \end{figure}
\end{minipage} \hfill
\begin{minipage}{0.45\textwidth}
A path of length 5: (d, a, b, e, f, g).\\

All the vertices and edges of a path are different. 
\end{minipage}

\subsection{Circuits and Cycles}
A closed walk in which all edges are different is called a \textit{circuit} (this is a closed trail). A closed walk in which all vertices (except are the first and the last vertex) are different is called a \textit{cycle} (this is a closed path). 

\begin{minipage}{0.5\textwidth}
    \begin{figure}[H]
        \centering
        \begin{tikzpicture}[node distance= 1.5cm,]
            %nodes
            \node[graphnode, draw] (a) {$a$};
            \node[graphnode] (s1) [below of=a] {};
            \node[graphnode, draw] (d) [below of=s1] {$d$};
            \node[graphnode, draw] (b) [right of=s1] {$b$};
            \node[graphnode] (s3) [below of=b] {};
            \node[graphnode, draw] (f) [below of=s3] {$f$};
            \node[graphnode] (s2) [right of=b] {};
            \node[graphnode, draw] (h) [above of=s2] {$h$};
            \node[graphnode, draw] (c) [right of=s2] {$c$};
            \node[graphnode, draw] (e) [below of=s2] {$e$};
            \node[graphnode] (s4) [right of=e] {};
            \node[graphnode, draw] (g) [below of=s4] {$g$};
            \node[graphnode, draw] (i) [right of=s4] {$i$};

            %circuit emphlines
            \draw[emphline] (d) edge[above] node[] {} (a);
            \draw[emphline] (a) edge[above] node[] {} (b);
            \draw[emphline] (b) edge[above] node[] {} (c);
            \draw[emphline] (c) edge[above] node[] {} (h);
            \draw[emphline] (h) edge[above] node[] {} (b);
            \draw[emphline] (b) edge[above] node[] {} (d);

            %cycle emphlines
            \draw[emphline, blueln] (f) edge[above] node[] {} (e);
            \draw[emphline, blueln] (e) edge[above] node[] {} (i);
            \draw[emphline, blueln] (i) edge[above] node[] {} (g);
            \draw[emphline, blueln] (g) edge[above] node[] {} (f);

            % lines
            \draw (a) edge[above] node[align=center] {} (d);
            \draw (b) edge[above] node[align=center] {} (a);
            \draw (b) edge[above] node[align=center] {} (d);
            \draw (b) edge[above] node[align=center] {} (h);
            \draw (b) edge[above] node[align=center] {} (c);
            \draw (b) edge[above] node[align=center] {} (e);
            \draw (h) edge[above] node[align=center] {} (c);
            \draw (e) edge[above] node[align=center] {} (c);
            \draw (e) edge[above] node[align=center] {} (i);
            \draw (e) edge[above] node[align=center] {} (g);
            \draw (e) edge[above] node[align=center] {} (f);
            \draw (e) edge[above] node[align=center] {} (d);
            \draw (g) edge[above] node[align=center] {} (i);
            \draw (g) edge[above] node[align=center] {} (f);
            \draw (f) edge[above] node[align=center] {} (d);

        \end{tikzpicture}
        \caption{Examples of Circuits and Cycles}
    \end{figure}
\end{minipage} \hfill
\begin{minipage}{0.45\textwidth}
A circuit of length 6: (d, a, b, c, h, b, d)\\

A cycle of length 4: (f, e, i, g, f)
\end{minipage}