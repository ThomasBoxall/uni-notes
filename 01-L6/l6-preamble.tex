% LEVEL 6 UNIVERSITY NOTES PREAMBLE
% THOMAS BOXALL <thomas@thomasboxall.net>
% 21 September 2025
%%%%%%%%%%%%%%%%%%%%%%%%%%%%%%%%%%%%%%%%%

% this preamble gets include in all university note documents. It contains custom styling and useful features to speed up note taking.

% heavily inspired from l5 preamble but slightly different for added spice

%%%%%%%%%%%%%%%%%%%%%%%%
% Basic Document Setup %
%%%%%%%%%%%%%%%%%%%%%%%%

% BASIC PACKAGES
\usepackage[dvipsnames, table]{xcolor}
\usepackage{graphicx}
\usepackage{fontawesome5}
\usepackage[colorlinks=true, linkcolor=black]{hyperref}
\usepackage{lastpage}
\usepackage{datetime2} % changes \today format to be yyyy-mm-dd
\usepackage{enumitem}
\usepackage{amsmath}
\usepackage{amssymb}
\usepackage{fancyvrb}
\usepackage[skip=0.7em]{parskip} % change space between paragraphs so no need for \\ between each para

% Margins
\usepackage{geometry}
\geometry{
a4paper,
total={170mm,257mm},
left=20mm,
top=20mm,
marginparsep=0mm,
}
\setlength\parindent{0pt} % get rid of the stupid indent

% add in capabilities for small notes in the left margin
\reversemarginpar
\newcommand{\marginnote}[1]{%
\marginpar{\footnotesize\texttt{#1}}
}

% set toc depth to only show chapters
\setcounter{tocdepth}{0}

% use section numbering for everything up too and including subsubsection
\setcounter{secnumdepth}{3}


% CUSTOM COLOURS
\definecolor{themeGrey}{HTML}{121212}
\definecolor{themeBlue}{HTML}{0c243d}
\definecolor{themeRed}{HTML}{ba3636}
\definecolor{themePurple}{HTML}{80347C}
\definecolor{themeGreen}{HTML}{506641}

%%%
% Coloured Boxes
%%%

\usepackage[skins, most]{tcolorbox}
\newtcolorbox{example}[1]{
    enhanced,
    colback=white,
    colframe=themeBlue,
    coltext=black,
    coltitle=white,
    attach boxed title to top left={yshift*=-\tcboxedtitleheight}, 
    title={Example: #1},
    parbox=false,
    breakable,
    arc=2pt,
    boxed title size=title,
    boxed title style={%
        rounded corners=northeast, 
        rounded corners=northwest, 
        colback=tcbcolframe, 
        boxrule=0pt,
    },
    underlay boxed title={%
        \path[fill=tcbcolframe] (title.south west)--(title.south east) 
            to[out=0, in=180] ([xshift=5mm]title.east)--
            (title.center-|frame.east)
            [rounded corners=2pt] |- 
            (frame.north) -| cycle; 
    },
}

\newlist{worddefs}{description}{1}
\setlist[worddefs]{font=\bfseries, labelindent=\parindent, leftmargin=2em, style=sameline}

\newtcolorbox{define}{
    enhanced,
    colback=white,
    colframe=themeGreen,
    coltext=black,
    coltitle=white,
    attach boxed title to top left={yshift*=-\tcboxedtitleheight}, 
    title=Definitions,
    parbox=false,
    breakable,
    arc=2pt,
    boxed title size=title,
    boxed title style={%
        rounded corners=northeast, 
        rounded corners=northwest, 
        colback=tcbcolframe, 
        boxrule=0pt,
    },
    underlay boxed title={%
        \path[fill=tcbcolframe] (title.south west)--(title.south east) 
            to[out=0, in=180] ([xshift=5mm]title.east)--
            (title.center-|frame.east)
            [rounded corners=2pt] |- 
            (frame.north) -| cycle; 
    },
    before upper=\begin{worddefs},
    after upper=\end{worddefs}
}

\newtcolorbox{todo}{
    enhanced,
    arc=2pt,
    colback=white,
    colframe=themeRed,
    leftrule=10mm,
    coltext=black,
    overlay={
        \node[
            anchor=west,
            outer sep=5pt
        ] at (frame.west) {
            % Replace the image with the Font Awesome icon
            \color{white}\faExclamationTriangle
        };
}
}

\newtcolorbox{extlink}{
    enhanced,
    arc=2pt,
    colback=white,
    colframe=themePurple,
    leftrule=10mm,
    coltext=black,
    overlay={
        \node[
            anchor=west,
            outer sep=5pt
        ] at (frame.west) {
            \color{white}\faExternalLink*
        };
}
}

%%%
% TIKZ
%%%

\usepackage{tikz}
\usetikzlibrary{positioning, arrows.meta, decorations.pathreplacing, automata, shapes}

%%%
% DOC PROPERTIES SETUP
%%%

\def\university{University of Portsmouth}
\def\course{BSc (Hons) Computer Science}

\def\notesAuthor{Thomas Boxall}
\def\notesAuthorContact{thomas.boxall1@myport.ac.uk}

\newcommand{\documentsetup}[6]{%
    \def\moduleName{#1}
    \def\moduleNameShort{#2}
    \def\moduleCode{#3}
    \def\moduleDates{#4}
    \def\moduleCredits{#5}
    \def\courseYear{#6}
}

\makeatletter
\AtBeginDocument{
  \hypersetup{
    pdftitle={ \moduleName{} (\moduleCode) Notes },
    pdfauthor={Thomas Boxall},
    pdfsubject={ \courseYear{} Year BSc Computer Science Notes },
    pdfcreator={Blood, Sweat \& Tears}
  }
}
\makeatother

%%%
% TABLES & FLOATS
%%% 
\usepackage{float}
\usepackage{tabularx}
\usepackage{ragged2e}
\usepackage{multirow}
\usepackage{multicol}
\usepackage{array}
\usepackage{svg}
\usepackage{longtable}

\renewcommand{\arraystretch}{1.6} % make cells vertically bigger

\newcolumntype{C}[1]{>{\centering\let\newline\\\arraybackslash\hspace{0pt}}p{#1}} % create custom column type with definable width

\newcommand{\thead}[1]{\centering\arraybackslash \cellcolor{themeGrey}\leavevmode\color{white}\textbf{#1}}

%%%
% Code Blocks etc
%%%

\usepackage{upquote}
\usepackage{listings}

\lstset{
  basicstyle=\ttfamily,
  mathescape,
}

\lstdefinestyle{haskellTrace}{
  moredelim=[is][\underbar]{_}{_},
  keepspaces=true,
  escapechar=^
}


%%%
% MATHEMATICAL ENVIRONMENTS
%%%
\usepackage{amsthm}
\renewcommand\qedsymbol{$\blacksquare$}
\newtheorem{theorem}{Theorem}[section]

\usepackage[framemethod=tikz]{mdframed}
% new mdframed style that places the edges at the corners:
\mdfdefinestyle{proof}{
  skipabove         = .5\baselineskip ,
  skipbelow         = .5\baselineskip ,
  leftmargin        = 0pt ,
  rightmargin       = 0pt ,
  innermargin       = 0pt ,
  innertopmargin    = .5em ,
  innerleftmargin   = .5em ,
  innerrightmargin  = 0pt ,
  innerbottommargin = .5em ,
  hidealllines      = true ,
  singleextra       = {
    \draw (O) -- ++(0,.675em) (O) -- ++(.675em,0) ;
    \draw (P-|O) -- ++(0,-.675em) (P-|O) -- ++(.675em,0) ;
  },
  firstextra        = {
    \draw (P-|O) -- ++(0,-.675em) (P-|O) -- ++(.675em,0) ;
  },
  secondextra       = {
    \draw (O) -- ++(0,.675em) (O) -- ++(.675em,0) ;
  },
}
% put the new mdframed style around the proof environment:
\surroundwithmdframed[style=proof]{proof}


%%%
% CHAPTER & HEADERS
%%%
\usepackage[raggedright,bf]{titlesec}
\usepackage{fancyhdr}

% reformat & rename chapter heading
\titlespacing*{\chapter}{0pt}{-20pt}{0pt}
\renewcommand*{\chaptername}{Page}

% resize the chapter title
\titleformat{\chapter}[display]
  {\normalfont\Huge\bfseries} % <format>
  {\chaptertitlename\ \thechapter} % <label>
  {0.25em} % <sep>
  {\Huge} % <before-code>

% redefine \chapter so it uses pagestyle=fancy
\makeatletter
\renewcommand\chapter{\if@openright\cleardoublepage\else\clearpage\fi
\thispagestyle{fancy}%
\global\@topnum\z@
\@afterindentfalse
\secdef\@chapter\@schapter}
\makeatother

% design headers and footers
\pagestyle{fancy}
\fancyhf{}
\fancyhead[L]{\moduleCode{} (\moduleNameShort)}
\fancyhead[R]{\leftmark}
\fancyfoot[L]{\footnotesize \texttt{\DTMnow}}
\fancyfoot[C]{\thepage{} of \pageref*{LastPage}}
\fancyfoot[R]{\notesAuthor}
\renewcommand{\footrulewidth}{0.4pt}
\addtolength{\topmargin}{-1.59999pt}
\setlength{\headheight}{13.59999pt}

% custom headers for taught session
\newcommand{\taughtsession}[6]{%
    \chapter[#1 - #2 (#3)]{#1 - #2}
    \chaptermark{#2}
    \vspace{0.25em}
    \begin{figure}[H]
        \begin{minipage}[t]{0.2\textwidth}
            \faCalendar*[regular]{} #3
        \end{minipage}\hfill
        \begin{minipage}[t]{0.2\textwidth}
            \faClock[regular]{} #4
        \end{minipage}\hfill
        \begin{minipage}[t]{0.2\textwidth}
            \faChalkboardTeacher{} #5
        \end{minipage}\hfill
    \end{figure}
    \vspace{-1.5em}
    \rule[0em]{\textwidth}{0.25px}
    
    } % end of new command

% custom notes document title page

\newcommand{\makedocumenttitlepage}{\begin{titlepage}
    \pagecolor{themeBlue}
    \color{white}
    \rule{\textwidth}{1px}
    \vspace{0.025\textheight}

    \huge{\university}\\
    \huge{\course}\\
    \huge{\courseYear{} Year}

    \vfill

    \LARGE{\textbf{\moduleName} (\moduleNameShort)}\\[0.25em]
    \Large{\moduleCode}\\[0.25em]
    \large{\moduleDates}\\[0.25em]
    \large{\moduleCredits{} Credits}

    \vfill

    \begin{flushright}
        \large{\notesAuthor}\\[0.25em]
        \texttt{\notesAuthorContact}
    \end{flushright}
    \vspace{0.2\textheight}
    \rule{\textwidth}{1px}
\end{titlepage}
\nopagecolor
}
