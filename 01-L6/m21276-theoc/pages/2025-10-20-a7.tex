\taughtsession{Lecture}{A7: Pushdown Automata}{2025-10-20}{14:00}{Janka}{}

We have already seen how \textit{Deterministic Finite Automata} and \textit{Non-Deterministic Finite Automata} can be used to identify regular languages. 

Today we will explore the wonderful world of \textit{(Non-Deterministic) Pushdown Automata} (PDA) and how these can be used to identify context-free languages. When we talk about a PDA, we are talking about a Non-Deterministic Pushdown Automata (NPDA) by default.

\section{Non-Deterministic Pushdown Automata}
NPDAs are like finite automata, in that they read a single input character from the input tape at a time and may or may not perform a transition based on this. However, NPDAs also have a stack memory where they can store an arbitrary amount of information.

\subsection{The Stack Memory}
As we have seen in modules of years gone past, the stack operates in \textit{Last In First Out} where only the top element can be operated on in a single operation. 

There are three operations we can do with the stack:
\begin{description}
    \item[pop] reads the top symbol and removes it from the stack
    \item[push] writes a designed symbol onto the top of the stack; for example $push(X)$ means put $X$ on the top of the stack
    \item[nop] does nothing to the stack 
\end{description}

The symbols put onto the stack are different from the language's alphabet which is used on the input tape.

At the start of processing a fresh input on the input tape - the stack starts with only the initial stack symbol (\$) on the stack. The automaton starts in it's initial state, as we'd expect.

\subsection{Transitions}
For each step, there are three inputs used to determine the transition:
\begin{itemize}
    \item The Current State,
    \item The Input Element,
    \item The Top Symbol of The Stack
\end{itemize}

One transition step includes:
\begin{itemize}
    \item changing the state (as with FAs)
    \item (optional) reading a symbol from the input tape and moving to the next right symbol (as with FA) 
    \item change the stack - push a symbol onto the stack, pop a symbol off the stack, no change to the stack
\end{itemize}

Transition steps are formally defined by transition functions, although often represented in the form of transition instructions. 

For each transition function, there are three inputs: the state, the input character (which can be $\Lambda$), and the character which must be at the top of the stack for the input condition to hold. There are two outputs of the transition functions: the ``new'' state, and a stack operation.

There are three different ways we can represent the transition.

Firstly, we can see the transition on a transition diagram. The label on the edge is vital here - as it gives us the input character, stack character, and output operation on the stack. The two states show the initial state, and the output state, as with a FA.

\begin{figure}[H]
    \centering
    \begin{tikzpicture}[node distance=5cm, on grid, shorten >= 2pt, shorten <= 2pt]
        \node[state] (p) {$p$};
        \node[state] (q) [right of=p] {$q$};
        
        \path[->] (p) edge [above] node [align=center] {$a,A/push(B)$} (q);
        
    \end{tikzpicture}
    \caption{Transition Diagram representation of a transition}
\end{figure}

Alternatively, we can write this as a transition function:
\[T(p, a, A) = (push(B), q)\]
Where here we can see the left hand side are the three inputs: the input state, the input character, and the stack input character; and the right hand side shows us the output operation on the stack and the state to transition to. 

Finally, we can re-write this as a transition instruction:
\[(p, a, A, push(B), q)\]
Here we can see that there is a single set of brackets containing, in order: the initial state; the input character; the stack input character; the stack output function; and the state to transition to. 

\subsection{Formal Definition}
We can formally describe NPDA as:
\begin{itemize}
    \item a finite set $Q$ of states (and the start states, and the set of accepting / final states)
    \item a finite set $\Sigma$ which is called the input alphabet
    \item a finite set $\Gamma$ which is called the stack alphabet (and the initial stack symbol \$)
    \item a finite set of transition instructions, or a transition functions where
    \[T:Q \times \Sigma \cup \{\Lambda\} \times \Gamma \rightarrow \Gamma^* \times Q\]
    or represented by a `transition' diagram.
\end{itemize}

An input string is accepted by an NPDA if there is some path (i.e. a sequence of instructions) from the start state to a final state that consumes all letters of the string; otherwise the string is rejected. The language of an NPDA is the set of all strings that it accepts.

There are a few reasons that an NPDA may reject a string:
\begin{itemize}
    \item If reading an input string finishes without reaching a final state 
    \item If for a current state / symbol on the stack / input symbol there is no transition
    \item If it attempts to pop the empty stack
\end{itemize}

\begin{example}{$\{a^nb^n|n\geq 0\}$}
Yes, $\{a^nb^n|n\geq 0\}$ is back again. 

To tackle this problem we first come up with a plan on how to solve it:
\begin{enumerate}
    \item Begin reading the string, and for each $a$ read push a $Y$ onto the stack
    \item On the first $b$ change states, and begin removing one $Y$ fro the stack
    \item If you reach the end of the input and have just cleared the stack, accept the string
    \item Otherwise reject (e.g. if the stack runs out before the input - more $b$'s than $a$'s)
\end{enumerate}

Now we have figured out how we need our NPDA to behave - we can design it. 

It will have 3 states: 0 (start), 1, 2 (final); the input alphabet will contain two elements: $\{a, b\}$ and the stack alphabet will also contain 2 elements: $\{Y, \$\}$.

Note here that $\Lambda$ is not in the input alphabet as we can safely assume it always will be; and that the stack alphabet contains the initial stack symbol \$. 

From here we can draw our NPDA.
\begin{figure}[H]
    \centering
    \begin{tikzpicture}[node distance=5cm, on grid, shorten >= 2pt, shorten <= 2pt]
        \node[state, initial] (0) {$0$};
        \node[state] (1) [right of=0] {$1$};
        \node[state, accepting] (2) [right of=1] {$2$};
        
        \path[->] (0) edge [above] node [align=center] {$b,Y/pop$} (1);
        \path[->] (1) edge [above] node [align=center] {$\Lambda,\$/nop$} (2);
        \path[->] (0) edge [loop above] node [align=center] {$a,Y/push(Y)$\\$a,\$/push(Y)$} (0);
        \path[->] (1) edge [loop above] node [align=center] {$b,Y/pop$} (1);
        \path[->, bend right=30] (0) edge [below] node [align=center] {$\Lambda,\$/nop$} (2);
        
    \end{tikzpicture}
    \caption{NPDA Diagram}
\end{figure}

From this NPDA - we can learn there are 6 allowed instructions:
\begin{align*}
T_1: & (0, a \$, push(Y), 0)\\
T_2: & (0, a, Y, push(Y), 0)\\
T_3: & (0, \Lambda, \$, nop, 2)\\
T_4: & (0, b, Y, pop, 1)\\
T_5: & (1, b, Y, pop, 1)\\
T_6: & (1, \Lambda, \$, nop, 2)
\end{align*}
\end{example}

\subsection{Instantaneous Description}
We are able to use \textit{Instantaneous Descriptions} to describe the process the NPDA is going through at any instant. Three are three things we need to keep track of:
\begin{itemize}
    \item The current state
    \item What input characters are left
    \item What is on the stack
\end{itemize}

The instantaneous descriptions take the form:
\begin{center}
    (current state, unconsumed input, stack contents)
\end{center}

We can see this in action if we take an NPDA which has the instantaneous description:
\[(0, abba, YZ\$)\]
and the NPDA includes an instruction of the following form:
\[(0, a, Y, pop, 1)\]
After the transition, the instantaneous description is changed to:
\[1, bba, Z\$\]

From this we can see that the transition function has changed the state from 0 to 1, used up the letter $a$ from the input tape and popped $Y$ from the top of the stack.

\begin{example}{Instantaneous Descriptions}
If we take the string $aabb$, we can represent it's journey through the NPDA with instantaneous descriptions.
\begin{align*}
    Start &\rightarrow (0, aabb, \$)\\
    T_1 &\rightarrow (0, abb, Y\$)\\
    T_2 &\rightarrow (0, bb, YY\$)\\
    T_4 &\rightarrow (1, b, Y\$)\\
    T_5 &\rightarrow (1, \Lambda, \$)\\
    T_6 &\rightarrow (2, \Lambda, \$)
\end{align*}    
We have reached the final state, 2, therefore we accept the input string.

Note the transition marker $T_n$ denotes which transition function was used for a given instantaneous description. 
\end{example}

\subsection{NPDAs and Context-Free Languages}
As we have seen in previous lectures, the class of context-free languages are generated by context-free grammars which have all their production rules of the form $N \rightarrow \alpha$ where $N$ is a non-terminal $\alpha$ is any string over the alphabet of terminals and non-terminals. 

This looks familiar - as we have seen that this $N \rightarrow \alpha$ is the form that the productions take for NPDAs. Therefore we can theorise that the Context-Free languages are exactly the languages that that are accepted by non-deterministic pushdown automata.

This theorem can be proven in two steps (similar to NFA $\Leftrightarrow$ regular languages):
\begin{enumerate}
    \item If we take a NPDA: we can find a context-free grammar which generates the language accepted by the given NPDA.
    \item If we take a context-free language: we can find an NPDA that accepts the given context-free language.
\end{enumerate}

\begin{example}{Finding a NPDA for a Context-Free Language}
If we take a language containing all strings over $a$ and $b$ with exactly the same number of $a$'s as $b$'s. We can show that this is context-free. 

We can plan for this as follows:
\begin{itemize}
    \item Keep track of the difference between the number of $a$'s and $b$'s we've read by changing the symbol in the stack.
    \item Use one symbol, $X$, if we've seen more $a$s and another, $Y$, if we've seen more $b$s.
\end{itemize}

For our NPDA, we use two states: 0 (start) and 1 (final). We use the input alphabet $\{a,b\}$ and our stack alphabet as $\{X, Y, \$\}$.

\begin{figure}[H]
    \centering
    \begin{tikzpicture}[node distance=5cm, on grid, shorten >= 2pt, shorten <= 2pt]
        \node[state, initial] (0) {$0$};
        \node[state, accepting] (1) [right of=0] {$1$};
        
        \path[->] (0) edge [above] node [align=center] {$\Lambda, \$ / nop$} (1);
        \path[->] (0) edge [loop above] node [align=center] {$a,\$/push(X)$ \\ $a,X/push(X)$ \\ $a,Y/pop$ \\ $b,\$/push(Y)$ \\ $b,Y/push(Y)$ \\ $b,X/pop$} (0);
        
    \end{tikzpicture}
    \caption{NPDA Solution}
\end{figure}

Form here we can define the 7 instructions for this NPDA in instruction form:
\begin{align*}
T_1: & (0,a,\$, push(X), 0)\\
T_2: & (0, a, X, push(X), 0)\\
T_3: & (0, a, Y, pop, 0)\\
T_4: & (0, b, \$, push(Y), 0)\\
T_5: & (0, b, Y, push(Y), 0)\\
T_6: & (0, b, X, pop, 0)\\
T_7: & (0, \Lambda, \$, nop, 1)
\end{align*}

Alternatively, this can be represented using the Instantaneous Descriptions method:
\begin{align*}
Start \rightarrow & (0, abbbaa, \$)\\
T_1 \rightarrow & (0, bbbaa, X\$)\\
T_6 \rightarrow & (0, bbaa, \$)\\
T_4 \rightarrow & (0, baa, Y\$)\\
T_5 \rightarrow & (0, aa, YY\$)\\
T_5 \rightarrow & (0, a, Y\$)\\
T_5 \rightarrow & (0,\Lambda, \$)\\
T_6 \rightarrow & (1, \Lambda, \$)
\end{align*}

\end{example}

\section{Determinism vs Non-Determinism}
In a similar way to that of the finite automata, push-down automata can either be \textit{deterministic} or \textit{non-deterministic}. 

A deterministic PDA never has a choice of the next step. It has, at most, one possible output for every combination of state, input and character. This is in a similar way to that of the DFA. 

For every combination of state \& stack character, only one of the transactions is allowed. This is either for the empty symbol, $\Lambda$, or for an input symbol, or there can be no transaction at all. 

\begin{example}{NPDA vs DPDA}
The following instructions cannot be contained by a deterministic push-down automata while they can be contained by a non-deterministic push-down automata.
\[(0, a, \$, push(Y), 0); \quad (0, a, \$, pop, 1)\]   
\end{example}

In a different way to that of the DFA and NFA (where both recognise the same languages) - NPDA and DPDA accept different languages. Deterministic push-down finite automata cannot recognise the whole family of context-free languages. 

\begin{example}{NPDA works where DPDA fails}
If we take the language $L=\{ww^R | w \in \{a, b\}^+\}$ which recognises even length palindromes, we can design a PDA which recognises this.

We start with our plan:
\begin{itemize}
    \item Read in each string and save it to the stack
    \item At each step, consider the possibility that we might have reached the middle
    \item Once reaching the midpoint - start working backwards, removing things from the stack if they match what was saved
\end{itemize}
Except this leaves us with a problem; how do we know we've reached the midpoint? A Non-deterministic PDA can help.

\begin{figure}[H]
    \centering
    \begin{tikzpicture}[node distance=5cm, on grid, shorten >= 2pt, shorten <= 2pt]
        \node[state, initial] (0) {$0$};
        \node[state] (1) [right of=0] {$1$};
        \node[state, accepting] (2) [right of=1] {$2$};
        
        \path[->] (0) edge [above] node [align=center] {$\Lambda,X/nop$ \\ $\Lambda,Y/nop$} (1);
        \path[->] (1) edge [above] node [align=center] {$\Lambda,\$/nop$} (2);
        \path[->] (0) edge [loop below] node [align=center] {$a,*/push(X)$ \\ $b,*/push(Y)$} (0);
        \path[->] (1) edge [loop below] node [align=center] {$a,X/pop$ \\ $b,Y/pop$} (1);
        
    \end{tikzpicture}
    \caption{NPDA Solution}
\end{figure}
In the above figure, $*$ stands for $X, Y, \$$. 

We can then see the instantaneous description derivation for an example palindrome: $aabbaa$.
\begin{align*}
\textrm{Start} & \rightarrow (0,aabbaa,\$)\\
\textrm{Load\ stack} & \rightarrow (0, abbaa, X\$)\\
\textrm{Load\ stack} & \rightarrow (0, bbaa, XX\$)\\
\textrm{Load\ stack} & \rightarrow (0, baa, YXX\$)\\
\textrm{Try:\ is\ this\ the\ middle?} & \rightarrow (1, baa, YXX\$)\\
\textrm{Pop\ stack} & \rightarrow (1, aa, XX\$)\\
\textrm{Pop\ stack} & \rightarrow (1, a, X\$)\\
\textrm{Pop\ stack} & \rightarrow (1, \Lambda, \$)\\
\textrm{Done} & \rightarrow (2, \Lambda, \$)\\
\end{align*}

\end{example}

The above example was an example of a non-deterministic PDA. This is because, from state 0, it branches either loading another letter on or trying to take letters off. This could only be done non-deterministically. A deterministic PDA would need to know when to start removing letters from the stack. Therefore a NPDA can recognise the language of the even palindromes, but a DPDA cannot. 

Deterministic push-down automata recognise regular languages and also some which are not regular, but not all of the context free languages. 