\taughtsession{Lecture}{B1: Computability and Equivalent Models Pt. 1}{2025-11-17}{14:00}{Janka}

We are now halfway through this module, exam on Wednesday to celebrate this. Happy international day of students, although no time to celebrate until after the exam on Wednesday - must spend all time revising for Wednesday, Turing Machines dancing in sleep and all...

Today's lecture will begin the ``B'' part of the module - this will be examined in the exam in the January Exam period, and is weighted 50\% (same as the exam on Wednesday for Part A topics).

\begin{extlink}
In the slides on Moodle - there is some history about what will be covered in this half of the module, including a reference to a video on Bob National. 
\end{extlink}

\section{Computability}
Something is \textit{computable} if there is some computation that computes it, or if it can be described by an algorithm. We can take this to mean that a computation is an execution of an algorithm. 

The ``computable'' property has something to do with a formal process (execution) and a formal description (algorithm). For example:
\begin{itemize}
    \item the derivation process associated with grammars
    \item the evaluation process associated with functions
    \item the state transition process associated with machines
    \item the execution process associated with programs and programming languages
\end{itemize}

A \textit{model} is a formalization of an idea. This means we have several ways to model the idea of computability.

We know that one computational model is more powerful than other computational models. For example, using what we've learnt in Part A, the Turing Machine is more powerful than a pushdown automata.

There is a most powerful model. There are many models equivalent to that. Therefore, as proven by the Church-Turing Thesis, anything that is intuitively computable can be computed by a Turing machine. 

This is called a `thesis' rather than a `theorem' as we only have an informal idea of what being computable is, as mathematicians have not yet been able to define this. 

\begin{todo}
finish from slides. 

TIL it doesn't work doing this live as Janka moves too fast.
\end{todo}