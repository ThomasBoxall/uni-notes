\taughtsession{Lecture}{Signal Encoding Techniques}{2025-09-29}{11:00}{Asim}

\begin{extlink}
There is a deck of slides on Moodle introducing this module's structure \& assessments, etc.
\end{extlink}

\section{Introduction to Concepts}

\subsection{Computer Networks}
\begin{define}
    \item[Computer Network] A system that connects two or more computing devices for transmitting and sharing information (data)
\end{define}

There are a number of different activities which can be done on a network:
\begin{itemize}
    \item Watching Videos
    \item Playing Games
    \item Sending and Receiving Messages (including not just text)
    \item Paying Bills
\end{itemize}

The core function of the network is to exchange data between interconnected devices.

\subsection{Data \& Signals}
Data is the information which is transmitted between devices on the network. The type of the data depends on the context and may include:
\begin{itemize}
    \item Video
    \item Audio
    \item Text
    \item Images
\end{itemize}

For data to be able to travel on the network - it has to be converted into digital or analog format. Once in either of these formats, the resultant data is known as \textit{signals}.

\begin{define}
\item[Signal] Electromagnetic Waves that carry data 
\item[Analog Signal]  are signals which vary smoothly over time and have no fixed point to change at and don't have fixed levels
\item[Discreet Signal] are signals which maintain constant level for some time then then change to another constant level
\item[Digital Signal] are signals which have only two levels - one high to indicate on, or 1, and one low to indicate off, or 0
\end{define}

\begin{figure}[H]
    \centering
    \begin{minipage}[h]{0.45\textwidth}
        \begin{figure}[H]
        \centering
        \begin{tikzpicture}[>=stealth]

            \draw[step=1cm,gray,very thin,dashed] (0,-1) grid (6,1);

            \draw[->] (0, 0) -- (6, 0) node[right] {\tiny time};
            
            \draw[->] (0, -1.5) -- (0, 1.5) node[above] {\tiny amplitude};
            
            \draw[themeBlue, thick, smooth, samples=100, domain=0:6] plot (\x, {sin(\x * 90)}); 

        \end{tikzpicture}
        \caption{Analog Signal}
        \end{figure}
    \end{minipage}\hfill
    \begin{minipage}[h]{0.45\textwidth}
        \begin{figure}[H]
        \centering
        \begin{tikzpicture}[>=stealth]

            \draw[step=1cm,gray,very thin,dashed] (0,-1) grid (6,1);

            \draw[->] (0, 0) -- (6, 0) node[right] {\tiny time};
            
            \draw[->] (0, -1.5) -- (0, 1.5) node[above] {\tiny amplitude};

            \draw[themeBlue, thick] (0,1) -- (1,1) -- (1,0.7) -- (2,0.7) -- (2,0.2) -- (3,0.2) -- (3,0.5) -- (4,0.5) -- (4,1) -- (5,1) -- (5,0.1) -- (6,0.1);

        \end{tikzpicture}
        \caption{Discrete Signal}
        \end{figure}    
    \end{minipage}
\end{figure}

\begin{figure}[H]
    \centering
    \begin{minipage}[h]{0.45\textwidth}
        \begin{figure}[H]
        \centering
        \begin{tikzpicture}[>=stealth]

            \draw[step=1cm,gray,very thin,dashed] (0,-1) grid (6,1);

            \draw[->] (0, 0) -- (6, 0) node[right] {\tiny time};
            
            \draw[->] (0, -1.5) -- (0, 1.5) node[above] {\tiny amplitude};
            
            \draw[themeBlue, thick] (0,1) -- (1,1) -- (1,0) -- (2,0) -- (2,1) -- (3,1) -- (3,1) -- (4,1) -- (4,0) -- (5,0) -- (5,1) -- (6,1);

        \end{tikzpicture}
        \caption{Digital Signal}
        \end{figure}
    \end{minipage}\hfill
    \begin{minipage}[h]{0.45\textwidth}
        \ 
    \end{minipage}
\end{figure}

\section{Data Communications}
Within a Data Communications Network - the data will convert between Digital and Analog data at various points. A \textit{modem} may be used to complete this conversion.

\begin{figure}[H]
    \centering
    \begin{tikzpicture}[
        block/.style={
            draw,
            rectangle,
            minimum height=1cm,
            text width=2cm,
            align=center,
        },
        >=stealth,
        node distance=1cm
    ]

        \node[block] (source) {Source (\textit{Workstation})}; 
        \node[block, right=of source] (transmitter) {Transmitter (\textit{Modem})};
        \node[block, right=of transmitter] (transsys) {Transmission System (\textit{Public telephone network})};
        \node[block, right=of transsys] (modemrec) {Receiver (\textit{Modem})}; 
        \node[block, right=of modemrec] (dest) {Destination (\textit{System})}; 

        \draw[->, thick] (source) -- (transmitter);
        \draw[->, thick] (transmitter) -- (transsys);
        \draw[->, thick] (transsys) -- (modemrec);
        \draw[->, thick] (modemrec) -- (dest);

        \coordinate (startFirstBrace) at ([yshift=0.7cm]source.north west);
        \coordinate (endFirstBrace) at ([yshift=0.7cm]transmitter.north east);

        \draw [decorate, decoration={brace, amplitude=8pt, raise=4pt}] 
            (startFirstBrace) -- (endFirstBrace)
            node [midway, above=12pt] {Source System};

        \coordinate (startSecondBrace) at ([yshift=0.7cm]modemrec.north west);
        \coordinate (endSecondBrace) at ([yshift=0.7cm]dest.north east);

        \draw [decorate, decoration={brace, amplitude=8pt, raise=4pt}] 
            (startSecondBrace) -- (endSecondBrace)
            node [midway, above=12pt] {Destination System};

    \end{tikzpicture}
    \caption{Block Diagram of Example Data Communications Signal Chain}
\end{figure}


In the above diagram, the \textit{workstation} provides the \textit{modem} with a digital signal. The modem then converts this digital signal into an analog signal which can be transmitted across the \textit{Public Telephone Network} that uses analog signals. The receiving \textit{modem} converts the signal to a digital format which the \textit{server} receives. 

In saying this, however, there are some devices which still work entirely on analog or digital signals. For example, the analog telephone network including the switching and transmission is entirely analog. There is also the inverse whereby there are entirely digital signals are processed. 

\section{Digital-To-Digital Signal Encoding}
Digital signals are transmitted such that an individual value is transmitted for a defined period of time called the \textit{bit duration}. The \textit{bit duration} is known to both the sender and receiver, allowing the receiver to correctly interpret the transmitted signal which the sender will always transmit at the beginning of the bit duration. We assume the sender and receiver are synchronised and therefore the clock or is not transmitted.

The move between two different defined voltages within the transmission is called the \textit{transition}.

As digital signals are discrete (meaning there are defined, absolute data levels) encoding a digital signal as another digital signal is considerably simpler as we don't have to convert one data level to another. 

\subsection{Nonreturn to Zero Level (NRZ-L)}
This method only has two levels of data transmitted. It works through mapping the high data level (1) and low data level (0) to a signal level:
\begin{itemize}
    \item 0 - represented by a high level signal
    \item 1 - represented by a low level signal
\end{itemize}

\begin{figure}[H]
\centering
\begin{tikzpicture}[>=stealth]

    \draw[step=1cm,gray,very thin,dashed] (0,-1.5) grid (6,1.5);
    \draw[->] (0, 0) -- (6, 0) node[right] {\tiny time};
    \draw[->] (0, -1.5) -- (0, 1.5) node[above] {\tiny amplitude};

    \node at (0.5, 1.25) {1};
    \node at (1.5, 1.25) {1};
    \node at (2.5, 1.25) {0};
    \node at (3.5, 1.25) {0};
    \node at (4.5, 1.25) {1};
    \node at (5.5, 1.25) {0};

    \node at (-0.5, 1) { high};
    \node at (-0.5, -1) {low};
    \node at (-0.5, 0) {zero};

    \draw[themeBlue, thick] (0,-1) -- (1,-1) -- (2,-1) -- (2,1) -- (3,1) -- (4,1) -- (4,-1) -- (5,-1) -- (5,1) -- (6,1);

\end{tikzpicture}
\caption{Example of NRZ-L Encoding using 110010}
\end{figure}

\subsection{Nonreturn to Zero Inverted (NRZ-I)}
This method is similar to NRZ-L in that it uses two levels of data transmitted: high and low. However it uses the transitions to define the data being transmitted. Looking at the bit following a transition point, even if there is no transition present:
\begin{itemize}
    \item Where there is a transition (regardless of 1 to 0, or 0 to 1): the following signal bit is transmitted as is high
    \item Where there is not a transition (regardless of 1 to 1, or 0 to 0): the following signal bit is transmitted as low
\end{itemize}

\begin{figure}[H]
\centering
\begin{tikzpicture}[>=stealth]

    \draw[step=1cm,gray,very thin,dashed] (0,-1.5) grid (6,1.5);
    \draw[->] (0, 0) -- (6, 0) node[right] {\tiny time};
    \draw[->] (0, -1.5) -- (0, 1.5) node[above] {\tiny amplitude};

    \node at (0.5, 1.25) {1};
    \node at (1.5, 1.25) {1};
    \node at (2.5, 1.25) {0};
    \node at (3.5, 1.25) {0};
    \node at (4.5, 1.25) {1};
    \node at (5.5, 1.25) {0};

    \node at (-0.5, 1) { high};
    \node at (-0.5, -1) {low};
    \node at (-0.5, 0) {zero};

    \draw[themeBlue, thick] (0,-1) -- (0,1) -- (1,1) -- (2,1) -- (2,-1) -- (3,-1) -- (4,-1) -- (4,1) -- (5,1) -- (5,-1) -- (6,-1);

\end{tikzpicture}
\caption{Example of NRZ-I Encoding using 110010}
\end{figure}

\subsection{Bipolar-AMI}
This has three signal levels: high, zero, and low. It works through setting the signal based on the bit to be transitioned to:
\begin{itemize}
    \item 0 - represented by transmitting zero
    \item 1 - represented by alternating high and low values, regardless of it there is a 0 value in the middle
\end{itemize}
\begin{figure}[H]
\centering
\begin{tikzpicture}[>=stealth]

    \draw[step=1cm,gray,very thin,dashed] (0,-1.5) grid (6,1.5);
    \draw[->] (0, 0) -- (6, 0) node[right] {\tiny time};
    \draw[->] (0, -1.5) -- (0, 1.5) node[above] {\tiny amplitude};

    \node at (0.5, 1.25) {1};
    \node at (1.5, 1.25) {1};
    \node at (2.5, 1.25) {0};
    \node at (3.5, 1.25) {0};
    \node at (4.5, 1.25) {1};
    \node at (5.5, 1.25) {0};

    \node at (-0.5, 1) { high};
    \node at (-0.5, -1) {low};
    \node at (-0.5, 0) {zero};

    \draw[themeBlue, thick] (0,0) -- (0,1) -- (1,1) -- (1,-1) -- (2,-1) -- (2,0) -- (4,0) -- (4,1) -- (5,1) -- (5,0) -- (6,0);

\end{tikzpicture}
\caption{Example of Bipolar-AMI Encoding using 110010}
\end{figure}


\subsection{Pseudoternary}
This has three signal levels: high, zero, low. It works through setting the signal based on the bit to be transmitted:
\begin{itemize}
    \item 0 - represented by alternating high and low values, regardless of if there is a 1 value in the middle of them
    \item 1 - represented by transmitting zero value
\end{itemize}

This is the inverse of Bipolar-AMI.

\begin{figure}[H]
\centering
\begin{tikzpicture}[>=stealth]

    \draw[step=1cm,gray,very thin,dashed] (0,-1.5) grid (6,1.5);
    \draw[->] (0, 0) -- (6, 0) node[right] {\tiny time};
    \draw[->] (0, -1.5) -- (0, 1.5) node[above] {\tiny amplitude};

    \node at (0.5, 1.25) {1};
    \node at (1.5, 1.25) {1};
    \node at (2.5, 1.25) {0};
    \node at (3.5, 1.25) {0};
    \node at (4.5, 1.25) {1};
    \node at (5.5, 1.25) {0};

    \node at (-0.5, 1) { high};
    \node at (-0.5, -1) {low};
    \node at (-0.5, 0) {zero};

    \draw[themeBlue, thick] (0,0) -- (1,0) -- (2,0) -- (2,1) -- (3,1) -- (3,-1) -- (4,-1) -- (4,0) -- (5,0) -- (5,1) -- (6,1);

\end{tikzpicture}
\caption{Example of Pseudoternary Encoding using 110010}
\end{figure}

\subsection{Manchester}
This has two signal levels: high and low. It works through having up-to two transitions per interval:
\begin{itemize}
    \item 0 - represented by a transition from high to low in the middle of the interval (which for successive zeros will require a low-to-high transition at the start of the interval)
    \item 1 - represented by a transition from low to high in the middle of the interval (which for successive ones will require a high-to-low transition at the start of the interval)
\end{itemize}

\begin{figure}[H]
\centering
\begin{tikzpicture}[>=stealth]

    \draw[step=1cm,gray,very thin,dashed] (0,-1.5) grid (6,1.5);
    \draw[->] (0, 0) -- (6, 0) node[right] {\tiny time};
    \draw[->] (0, -1.5) -- (0, 1.5) node[above] {\tiny amplitude};

    \node at (0.5, 1.25) {1};
    \node at (1.5, 1.25) {1};
    \node at (2.5, 1.25) {0};
    \node at (3.5, 1.25) {0};
    \node at (4.5, 1.25) {1};
    \node at (5.5, 1.25) {0};

    \node at (-0.5, 1) { high};
    \node at (-0.5, -1) {low};
    \node at (-0.5, 0) {zero};

    \draw[themeBlue, thick] (0,-1) -- (0.5,-1) -- (0.5,1) -- (1,1) -- (1,-1) -- (1.5,-1) -- (1.5,1) -- (2.5,1) -- (2.5,-1) -- (3,-1) -- (3,1) -- (3.5,1) -- (3.5,-1) -- (4.5,-1) -- (4.5,1) -- (5,1) -- (5.5,1) -- (5.5,-1) -- (6,-1);

\end{tikzpicture}
\caption{Example of Manchester Encoding using 110010}
\end{figure}

\subsection{Differential Manchester}
This has two signal levels: high and low. It works by always transitioning in the middle of the interval and then examining the transition at the beginning of the interval:
\begin{itemize}
    \item 0 - represented by a transition at the beginning of the interval
    \item 1 - represented by no transition at the beginning of the interval
\end{itemize}

\begin{figure}[H]
\centering
\begin{tikzpicture}[>=stealth]

    \draw[step=1cm,gray,very thin,dashed] (0,-1.5) grid (6,1.5);
    \draw[->] (0, 0) -- (6, 0) node[right] {\tiny time};
    \draw[->] (0, -1.5) -- (0, 1.5) node[above] {\tiny amplitude};

    \node at (0.5, 1.25) {1};
    \node at (1.5, 1.25) {1};
    \node at (2.5, 1.25) {0};
    \node at (3.5, 1.25) {0};
    \node at (4.5, 1.25) {1};
    \node at (5.5, 1.25) {0};

    \node at (-0.5, 1) { high};
    \node at (-0.5, -1) {low};
    \node at (-0.5, 0) {zero};

    \draw[themeBlue, thick] (0,-1) -- (0.5,-1) -- (0.5,1) -- (1.5,1) -- (1.5,-1) -- (2,-1) -- (2,1) -- (2.5,1) -- (2.5,-1) -- (3,-1) -- (3,1) -- (3.5,1)-- (3.5,-1) -- (4.5,-1) -- (4.5,1) -- (5,1) -- (5,-1) -- (5.5, -1) -- (5.5, 1) -- (6,1);

\end{tikzpicture}
\caption{Example of Differential Manchester Encoding using 110010}
\end{figure}

\subsection{Multi-Level Transmit 3 (MLT-3)}
This uses three different signal levels: low, zero and high. It works by cycling through these states based on the bit to be transmitted:
\begin{itemize}
    \item 0 - remain on the current signal level (regardless of signal level value)
    \item 1 - move to the next state in the cycle of signal levels (high - zero - low - zero repeat)
\end{itemize}

\begin{figure}[H]
\centering
\begin{tikzpicture}[>=stealth]

    \draw[step=1cm,gray,very thin,dashed] (0,-1.5) grid (6,1.5);
    \draw[->] (0, 0) -- (6, 0) node[right] {\tiny time};
    \draw[->] (0, -1.5) -- (0, 1.5) node[above] {\tiny amplitude};

    \node at (0.5, 1.25) {1};
    \node at (1.5, 1.25) {1};
    \node at (2.5, 1.25) {0};
    \node at (3.5, 1.25) {0};
    \node at (4.5, 1.25) {1};
    \node at (5.5, 1.25) {0};

    \node at (-0.5, 1) { high};
    \node at (-0.5, -1) {low};
    \node at (-0.5, 0) {zero};

    \draw[themeBlue, thick] (0,0) -- (0,1) -- (1,1) -- (1,0) -- (4,0) -- (4,-1) -- (6,-1);

\end{tikzpicture}
\caption{Example of MLT-3 Encoding using 110010}
\end{figure}


\section{Electromagnetic Waves}
Electromagnetic waves are the digital representation of an analog signal. They are considered to be smooth as they don't have fixed values. The key properties of any EM wave are as follows: Amplitude, Phase, Wavelength \& Frequency.

\begin{figure}[H]
    \centering
    \begin{minipage}[h]{0.45\textwidth}
        \begin{figure}[H]
        \centering
        \begin{tikzpicture}[>=stealth]

            % \draw[step=1cm,gray,very thin,dashed] (0,-1) grid (6,1);

            \draw[->] (0, 0) -- (6, 0) node[right] {\tiny time};
            
            \draw[->] (0, -1.5) -- (0, 1.5) node[above] {\tiny amplitude};
            
            \draw[themeBlue, thick, smooth, samples=100, domain=0:6] plot (\x, {sin(\x * 90)}); 

            % amplitude
            \draw[dashed, <->] (1,0) -- (1, 0.95) node[midway, left] {\footnotesize $A$};

        \end{tikzpicture}
        \caption{Electromagnetic Wave showing Amplitude ($A$)}
        \end{figure}
    \end{minipage}\hfill
    \begin{minipage}[h]{0.45\textwidth}
        \begin{figure}[H]
        \centering
        \begin{tikzpicture}[>=stealth]

            % \draw[step=1cm,gray,very thin,dashed] (0,-1) grid (6,1);

            \draw[->] (0, 0) -- (6, 0) node[right] {\tiny time};
            
            \draw[->] (0, -1.5) -- (0, 1.5) node[above] {\tiny amplitude};
            
            \draw[themeBlue, thick, smooth, samples=100, domain=0:6] plot (\x, {sin(\x * 90)}); 

            % phase shift by +90
            \draw[themeBlue, dashed, smooth, samples=100, domain=0:6] plot (\x, {sin(\x * 90 + 90)});

        \end{tikzpicture}
        \caption{Electromagnetic Wave showing Phase (dashed)}
        \end{figure}
    \end{minipage}
\end{figure}

\begin{figure}[H]
    \centering
    \begin{minipage}[h]{0.45\textwidth}
        \begin{figure}[H]
        \centering
        \begin{tikzpicture}[>=stealth]

            % \draw[step=1cm,gray,very thin,dashed] (0,-1) grid (6,1);

            \draw[->] (0, 0) -- (6, 0) node[right] {\tiny time};
            
            \draw[->] (0, -1.5) -- (0, 1.5) node[above] {\tiny amplitude};
            
            \draw[themeBlue, thick, smooth, samples=100, domain=0:6] plot (\x, {sin(\x * 90)}); 

            % wavelength / cycle
            \draw[dashed, <->] (1, 1.1) --  (5,1.1) node[midway, above] {\footnotesize $\lambda$};

        \end{tikzpicture}
        \caption{Electromagnetic Wave showing Wavelength ($\lambda$)}
        \end{figure}
    \end{minipage}\hfill
    \begin{minipage}[h]{0.45\textwidth}
        \begin{figure}[H]
        \centering
        \begin{tikzpicture}[>=stealth]

            % \draw[step=1cm,gray,very thin,dashed] (0,-1) grid (6,1);

            \draw[->] (0, 0) -- (6, 0) node[right] {\tiny time};
            
            \draw[->] (0, -1.5) -- (0, 1.5) node[above] {\tiny amplitude};
            
            \draw[themeBlue, thick, smooth, samples=100, domain=0:6] plot (\x, {sin(\x * 90)}); 

            \draw[themeBlue, dashed, smooth, samples=100, domain=0:6] plot (\x, {sin(\x * 180)});

        \end{tikzpicture}
        \caption{Electromagnetic Wave showing increased frequency (dashed)}
        \end{figure}
    \end{minipage}
\end{figure}

\subsection{Carrier Waves \& Modulation}
\begin{define}
    \item[Carrier Wave] a continuous, periodic waveform that carries no information
\end{define}

A \textit{carrier wave} is modified by an information-bearing signal to convey information. The modification can be by either changing: it's amplitude, frequency, phase, or some combination of the three. The process of modifying a carrier wave is called \textit{modulation}. 

\section{Digital Data, Analog Signals}
There are many examples of where digital data has to be transmitted through an analog medium. The most well-known of which being the public telephone network. This is designed to transmit voices, within the frequency of 300 to 3400 Hz; therefore a problem is presented when a digital device is connected to the network. This problem is overcome by connecting the digital device to a Modem (Modulator-Demodulator) which converts digital data to analog signals and the other way around. 

There are a number of different methods which can be used to convert digital data onto a analog signal. These methods employ a carrier wave for the digital data to be modulated onto. For the subsequent examples - it's assumed the carrier wave is a sine wave. 

\subsection{Amplitude-Shift Keying}
Amplitude-Shift Keying (ASK) modulation works by modulating the digital signal onto the amplitude of the carrier wave. Meaning that the amplitude of the carrier wave is increased for a digital 0 and decreased for a digital 1. 

\begin{figure}[H]
\centering
\begin{tikzpicture}[>=stealth]

    \draw[step=1cm,gray,very thin,dashed] (0,-1.5) grid (6,4.5);
    % analog x axis
    \draw[->] (0, 0) -- (6, 0);
    % y axis
    \draw[->] (0, -1.5) -- (0, 1.25);
    \draw[->] (0, 1.75) -- (0, 4.5);
    % digital x axis
    \draw[->] (0,3) -- (6,3);

    %digital y axis markers
    \node at (-0.5, 2) {0};
    \node at (-0.5, 4) {1};
    \draw [decorate, decoration={brace, amplitude=8pt, raise=4pt}] 
            (-0.7, 1.75) -- (-0.7, 4.25)
            node [midway, left=12pt] {Digital};
    % analog y axis markers
    \node at (-0.5, 1) {high};
    \node at (-0.5, -1) {low};
    \draw [decorate, decoration={brace, amplitude=8pt, raise=4pt}] 
            (-0.7, -1.25) -- (-0.7, 1.25)
            node [midway, left=12pt] {ASK};

    % digital signal
    \draw[themeBlue, thick] (0, 2) -- (1, 2) -- (1, 4) -- (3, 4) -- (3, 2) -- (5, 2) -- (5, 4) -- (6, 4);

    % ask signal
    \draw[themeBlue, thick, smooth, samples=100, domain=0:1] plot (\x, {sin(\x * 360)}); 
    \draw[themeBlue, thick, smooth, samples=100, domain=1:3] plot (\x, {0.5 * sin(\x * 360)}); 
    \draw[themeBlue, thick, smooth, samples=100, domain=3:5] plot (\x, {sin(\x * 360)}); 
    \draw[themeBlue, thick, smooth, samples=100, domain=5:6] plot (\x, {0.5 * sin(\x * 360)}); 


\end{tikzpicture}
\caption{Example of ASK Modulation}
\end{figure}

\subsection{Frequency-Shift Keying}
Frequency-Shift Keying (FSK) modulation works by modulating the digital signal onto the frequency of the carrier wave. This means that the frequency of the carrier wave is increased for a digital 1 and decreased for a digital 0. 

The sender and receiver may use different frequencies to allow full-duplex transmissions on the same channel. It is less susceptible to errors than ASK modulation is. FSK can be used for higher frequencies (3 - 30 MHz), radio and Local Area Network transmissions. It can also support Multiple Levels in MFSK.

The \textit{bandwidth} of the transmission is the difference between the frequency of the high frequency (representing a 1) and the low frequency (representing a 0). Different frequency carrier waves will be used for sending and receiving on the same line. 

The most common form of FSK is Binary FSK (BFSK) in which the two binary values are represented by two different frequencies near the carrier frequency:
\begin{itemize}
    \item 0 - represented by $A \cos(2 \pi f_1 t)$
    \item 1 - represented by $A \cos(2 \pi f_2 t)$
\end{itemize}

\begin{figure}[H]
\centering
\begin{tikzpicture}[>=stealth]

    \draw[step=1cm,gray,very thin,dashed] (0,-1.5) grid (6,4.5);
    % analog x axis
    \draw[->] (0, 0) -- (6, 0);
    % y axis
    \draw[->] (0, -1.5) -- (0, 1.25);
    \draw[->] (0, 1.75) -- (0, 4.5);
    % digital x axis
    \draw[->] (0,3) -- (6,3);

    %digital y axis markers
    \node at (-0.5, 2) {0};
    \node at (-0.5, 4) {1};
    \draw [decorate, decoration={brace, amplitude=8pt, raise=4pt}] 
            (-0.7, 1.75) -- (-0.7, 4.25)
            node [midway, left=12pt] {Digital};
    % analog y axis markers
    \node at (-0.5, 1) {high};
    \node at (-0.5, -1) {low};
    \draw [decorate, decoration={brace, amplitude=8pt, raise=4pt}] 
            (-0.7, -1.25) -- (-0.7, 1.25)
            node [midway, left=12pt] {BFSK};

    % digital signal
    \draw[themeBlue, thick] (0, 2) -- (1, 2) -- (1, 4) -- (3, 4) -- (3, 2) -- (5, 2) -- (5, 4) -- (6, 4);

    % bfsk signal
    \draw[themeBlue, thick, smooth, samples=100, domain=0:1] plot (\x, {sin(\x * 360)}); 
    \draw[themeBlue, thick, smooth, samples=100, domain=1:3] plot (\x, {sin(\x * 720)}); 
    \draw[themeBlue, thick, smooth, samples=100, domain=3:5] plot (\x, {sin(\x * 360)}); 
    \draw[themeBlue, thick, smooth, samples=100, domain=5:6] plot (\x, {sin(\x * 720)}); 


\end{tikzpicture}
\caption{Example of BFSK Modulation}
\end{figure}

\subsection{Phase Shift Keying}
Phase Shift Keying (PSK) Modulation works by modulating the digital signal onto the phase of the carrier wave. This means that the phase of the carrier wave is adjusted to represent digital 0 and digital 1, respectively. There are two different types of PSK studied here.

\subsubsection{Binary Phase Shift Keying}
Binary PSK (BPSK) works by using two phases to represent two different binary digits (0 and 1) and shifting between them:
\begin{itemize}
    \item 0 - represented by the sine wave $A \cos (2 \pi f t + 180)$ which equals the sine wave $-A \cos (2 \pi ft)$
    \item 1 - represented by the sine wave $A \cos (2 \pi f t)$
\end{itemize}

\begin{figure}[H]
\centering
\begin{tikzpicture}[>=stealth]

    \draw[step=1cm,gray,very thin,dashed] (0,-1.5) grid (6,4.5);
    % analog x axis
    \draw[->] (0, 0) -- (6, 0);
    % y axis
    \draw[->] (0, -1.5) -- (0, 1.25);
    \draw[->] (0, 1.75) -- (0, 4.5);
    % digital x axis
    \draw[->] (0,3) -- (6,3);

    %digital y axis markers
    \node at (-0.5, 2) {0};
    \node at (-0.5, 4) {1};
    \draw [decorate, decoration={brace, amplitude=8pt, raise=4pt}] 
            (-0.7, 1.75) -- (-0.7, 4.25)
            node [midway, left=12pt] {Digital};
    % analog y axis markers
    \node at (-0.5, 1) {high};
    \node at (-0.5, -1) {low};
    \draw [decorate, decoration={brace, amplitude=8pt, raise=4pt}] 
            (-0.7, -1.25) -- (-0.7, 1.25)
            node [midway, left=12pt] {BPSK};

    % digital signal
    \draw[themeBlue, thick] (0, 2) -- (1, 2) -- (1, 4) -- (3, 4) -- (3, 2) -- (5, 2) -- (5, 4) -- (6, 4);

    % bpsk signal
    \draw[themeBlue, thick, smooth, samples=100, domain=0:1] plot (\x, {-sin(\x * 360)}); 
    \draw[themeBlue, thick, smooth, samples=100, domain=1:3] plot (\x, {sin(\x * 360)}); 
    \draw[themeBlue, thick, smooth, samples=100, domain=3:5] plot (\x, {-sin(\x * 360)}); 
    \draw[themeBlue, thick, smooth, samples=100, domain=5:6] plot (\x, {sin(\x * 360)}); 


\end{tikzpicture}
\caption{Example of BPSK Modulation}
\end{figure}

\subsubsection{Differential Phase Shift Keying}
Differential PSK (DPSK) works by referencing the previous bit in the current bit transmitted. DPSK removes the need for an accurate local oscillator phase at the receiver which is matched with the transmitter, because so long as the preceding phase is received correctly - the phase reference is accurate. 
\begin{itemize}
    \item 0 - send a signal similar to the previous one
    \item 1 - send a signal with phase shift as compared to the previous one
\end{itemize}

\begin{figure}[H]
\centering
\begin{tikzpicture}[>=stealth]

    \draw[step=1cm,gray,very thin,dashed] (0,-1.5) grid (6,4.5);
    % analog x axis
    \draw[->] (0, 0) -- (6, 0);
    % y axis
    \draw[->] (0, -1.5) -- (0, 1.25);
    \draw[->] (0, 1.75) -- (0, 4.5);
    % digital x axis
    \draw[->] (0,3) -- (6,3);

    %digital y axis markers
    \node at (-0.5, 2) {0};
    \node at (-0.5, 4) {1};
    \draw [decorate, decoration={brace, amplitude=8pt, raise=4pt}] 
            (-0.7, 1.75) -- (-0.7, 4.25)
            node [midway, left=12pt] {Digital};
    % analog y axis markers
    \node at (-0.5, 1) {high};
    \node at (-0.5, -1) {low};
    \draw [decorate, decoration={brace, amplitude=8pt, raise=4pt}] 
            (-0.7, -1.25) -- (-0.7, 1.25)
            node [midway, left=12pt] {DPSK};

    % digital signal
    \draw[themeBlue, thick] (0, 2) -- (1, 2) -- (1, 4) -- (3, 4) -- (3, 2) -- (5, 2) -- (5, 4) -- (6, 4);

    % dpsk signal
    \draw[themeBlue, thick, smooth, samples=100, domain=0:1] plot (\x, {-sin(\x * 360)}); 
    \draw[themeBlue, thick, smooth, samples=100, domain=1:2] plot (\x, {sin(\x * 360)}); 
    \draw[themeBlue, thick, smooth, samples=100, domain=2:5] plot (\x, {-sin(\x * 360)}); 
    \draw[themeBlue, thick, smooth, samples=100, domain=5:6] plot (\x, {sin(\x * 360)}); 


\end{tikzpicture}
\caption{Example of DPSK Modulation}
\end{figure}