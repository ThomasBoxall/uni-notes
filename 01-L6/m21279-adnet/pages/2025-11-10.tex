\taughtsession{Lecture}{Wireless LANs}{2025-11-10}{11:00}{Asim}

\section{What is a Wireless LAN?}

Wireless LANs (WLAN) are an extension of a wired LAN, as we saw last week. Wireless LANs allow the client devices to connect wirelessly to the LAN as the name would suggest.

Within wireless LAN architecture - there is a backbone wired LAN which supports the wired components within the LAN and provides one or more bridges or routers to link with other networks. There is also a \textit{control module} (CM) which acts as an interface to a WLAN, often seen as the ``Router'' in a small domestic LAN or a Wireless Access Point (AP) in a larger LAN.

The CM includes either bridge or router functionality to link the WLAN to the backbone. They also include some sort of control logic, such as a polling or token-passing scheme to regulate the access from the end-systems. 

When discussing WLANs, we refer to \textit{user modules} (UM), which are the end-user devices. They can be stand-alone devices such as Laptops or Smartphones, etc. Alternatively, UMs can also be wireless receiver devices which output a wired LAN signal which allows devices to have a wired connection.

\subsection{Large WLANs}
In larger networks - there can be multiple control modules, each interconnected by a wired LAN. Each control module supports a number of wireless end systems within their transmission systems (often between 100 and 300m for enterprise APs). Where there are multiple control modules on a network - they each have to have their own frequencies assigned, to prevent interference. When moving from one cell, the coverage area of a CM, to another cell and staying connected to the same network - the CMs will hand off the device providing a seamless experience for the end-user. 

In multi-cell networks - a single cell is referred to as a \textit{basic service set}. The entire wireless network including the LAN backbone is referred to as the \textit{extended service set}.

\subsection{Ad-Hoc WLANs}
The final type of WLAN we may come across is an \textit{Ad-Hoc Wireless LAN}. In this topology, there is no fixed infrastructure (i.e. CMs, APs, etc). Rather, a collection of stations within a range of each other may dynamically configure themselves into a temporary network of peer-to-peer communications.

\section{WLAN Requirements}
There are a number of requirements for a WLAN to be effective and efficient.

\begin{description}
    \item[Throughput] The MAC protocol should make efficient use of the wireless medium to maximise capacity
    \item[Number of Nodes] WLANs may need to support hundreds of nodes across multiple cells
    \item[Connection to backbone LAN] There should be a good quality connection to the backbone LAN. For infrastructure WLANs - this is accomplished through CMs
    \item[Service Area] Typically, the coverage area for a WLAN is between 100 and 300m
    \item[Battery Power Consumption] Often the devices connecting to a WLAN are battery powered - it is important that the battery isn't drained through connecting to a WLAN. Typical WLAN implementations have features to reduce power consumption when not using the network (i.e. a sleep mode)
    \item[Transmission Robustness and Security] WLANs can be vulnerable to interference and network eavesdropping. The design of WLAN must permit reliable transmission even in a noisy environment and should provide some level of security from eavesdropping. 
    \item[License Free Operation] WLANs should operate within available frequency-bands which do not require the purchase of special licenses
    \item[Handoff / Roaming] The MAC protocol used in the WLAN should enable mobile stations to move from one cell to another
    \item[Dynamic Configuration] The MAC addressing and network management aspects of the WLAN should permit dynamic and automated addition, deletion and relocation of end systems without disruption to other users.
\end{description}

\section{IEEE 802.11}
The IEEE 802.11 protocol is the main protocol used for communications within the wireless network. This is the protocol which is commonly known as Wi-Fi. There have been many variations and subsequent discoveries, which are represented by a letter, or multiple letters on the end of the name - such as 802.11ax. 

\begin{definition}
    \item[Associated Stations] Connected devices to a WLAN
    \item[Access Point] Provides access to the distribution system for associated stations
    \item[Basic Service Set (BSS)] Set of stations controlled by a single coordinator
    \item[Extended Service Set (ESS)] A set of one or more connected BSSs
    \item[Distribution Systems (DS)] A system used ot interconnect a set of BSSs and integrated LANs to create an ESS
    \item[Frame] MAC protocol data uint 
\end{definition}

\subsection{Architecture}

\begin{todo}
diagram of IEE 802.11 architecture

add explanation from slides on what this is and how it works.
\end{todo}

\subsection{Services}
The IEEE 802.11 protocol specifies a number of services. 

\begin{todo}
add from table on slides - turn into sentences. include explanation about what each one does.
\end{todo}

\section{Access Control}
There is a need within WLANs to control who can communicate on the network and when. This is to prevent collisions in the wireless network. 

There are two modes in which WLANs can operate. Both of these operate at the MAC layer. 

\begin{todo}
add diag from slides
\end{todo}

\subsection{Point Coordination Function}
Point Coordination Function (PCF) is a centralised control which provides a contention free service (for example base stations to a backbone). PCF sits on top of the contentious DCF function.

PCF works by the base station polling the other stations asking them if they hve frames to transmit, guaranteeing no collision. The base station sends a beacon frame (between 10 and 100 times a second) which invites stations to sign in. The frame contains information such as hopping frequencies, dwell time, clock synchronisation, etc. When a station is signed in, it is guaranteed a fraction of the bandwidth (therefore making it possible to get quality of service) in a round-robin tims-share style. The base station also manages the power; through putting some stations in a standby mode until awakened by a reception. 

When polling, the Point Coordinator makes use of \textit{PCF Interframe Space} (PIFS) which is smaller than \textit{DCF Interframe Space} (DIFS). The point coordinator can seize the medium and lock out all asynchronous traffic while it issues polls and receives responses.

When a station is polled, it may respond with a \textit{Short Interframe Space} (SIFS) which are smaller than PIFS or DIFS and are designed for this situation. 

\begin{todo}
sifs? 

notes on slides talk about this.
\end{todo}

\subsection{Distributed Coordination Frame}

\begin{todo}
finish lecture from slides.
\end{todo}