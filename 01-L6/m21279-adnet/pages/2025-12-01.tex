\taughtsession{Lecture}{Satellite Networks}{2025-12-01}{11:00}{Asim}

A \textit{satellite} is a large reflector which orbits in the sky, around the earth. Satellites have several transponders, which are electronics that listen to different portions of the spectrum - each amplifying the incoming signal (the uplink), changing its frequency then broadcasting it to earth (the downlink). 

The sky-based satellites are combined with components on the ground which are called \textit{gateway stations}. These gateway stations are equipped with antennas directed towards the satellite, and some will have network control centres and operation control centres which deal with satellite management and orbit control. 

A satellite communication system works where two or more stations on or near the earth (gateway stations) communicate via one or more satellites that serve as relay stations in space. 

There are a number of different ways of categorising communication satellites:
\begin{description}
    \item[Coverage Area] which could be global, regional or national. The larger the coverage area, the more satellites must be involved in a single networked system
    \item[Service Type] which could be fixed service satellite (FSS), broadcast service satellite (BSS) and mobile service satellite (MSS).
    \item[General Usage] which could involve commercial, military, amateur, or experimental
\end{description}

The area of coverage for a satellite communication system exceeds that of a terrestrial system. In the case of a geostationary satellite - a single antenna is visible to about one quarter of the earths surface. Spacecraft power and allocated bandwidth are limited resources that call for careful tradeoffs in gateway stations / satellite design parameters. The conditions between communication between satellites are more time invariant than those between satellite and earth station or between two terrestrial wireless antennas. Therefore, satellite-to-satellite communication links can be designed with great precision. Transmission cost is independent of distance, within the satellites area of coverage. Satellites can transmit in broadcast, multicast and point-to-point mode. Satellites provide very high bandwidth and the quality of transmission is usually extremely high. For a geostationary satellite - there is an earth-satellite-earth propagation delay of about a quarter of a second. However, in many cases - a transmitting earth station can receive its own transmissions.

Satellite can be used for any of the following: telephone communications; television, radio or digital cinema broadcasting; amateur radio; providing internet access; military activities; and disaster management.

\section{Topology of Satellite Networks}
There are two common topologies which satellite networks may adopt.

Firstly, it is possible for a satellite to provide a point-to-point link between two distant ground-based antennas.

Alternatively, the satellite provides communication abilities between one ground-based transmitter and a number of ground based receivers. There exists a variation of this configuration wherein there is two-way communications among earth stations with one central hub and many remote stations.

\section{Satellite Orbits}
\begin{todo}
Add diagram from slides
\end{todo}

Satellite orbits may be classified in a number of ways:
\begin{itemize}
    \item The orbit may be circular, with the centre of the circle at the centre of the earth, or elliptical - with the earths centre at one of the two foci of the ellipse.
    \item A satellite may orbit around the earth in different planes. An equatorial orbit is directly above the earth's equator. A polar orbit passes over both poles; while other orbits are referred to as inclined orbits.
    \item The altitude of communications satellites is classified as geostationary orbit (GEO), medium earth orbit (MEO), and low earth orbit (LEO) - more on this in a moment...
\end{itemize}

The \textit{orbital period} is the time it takes for the satellite to go around the earth and can be calculated as follows:
\begin{center}
    Orbital Period = $\displaystyle \frac{1}{1000} \times$ (Radius of the earth + Altitude of a satellite)$^{1.5}$
\end{center}

\begin{example}{Orbital Periods}
\textbf{Ex. 1: The Moon}
\begin{center}
    Orbital Period = $\displaystyle \frac{1}{1000} \times (6378Km + 384000Km)^{1.5} = 2439090sec \approx 1 month$
\end{center}

\textbf{Ex. 2: GEO Satellite}
\begin{center}
    Orbital Period = $\displaystyle \frac{1}{1000} \times (6378Km + 35786Km)^{1.5} = 86579sec \approx 24 hours$
\end{center}
\end{example}

\begin{todo}
check if table from slide 6 needed or if this is represented elsewhere
\end{todo}

\subsection{Geosynchronous (Geostationary) Satellites}
Geosynchronous, or Geostationary, (GEO) satellites orbit at about 35786Km above the earth's surface moving at the same speed as the earth. 

\begin{todo}
Finish notes from slides.
\end{todo}